%     1.3.1. Ingeniería de proteínas
%         1.3.1.1. Ingeniería de proteínas modulares (a.k.a. quiméricas) (Qué es, 1 página)
%         1.3.1.2. Ejemplos (1 página y 1 figura)
%     1.3.2. Ingeniería de linkers.
%         1.3.2.1. Diseño positivo (propiedades deseadas, 1 página, 1 figura)
%             1.3.2.1.1. Propiedades conformacionales
%             1.3.2.1.2. Carga
%         1.3.2.2. Diseño negativo (propiedades no deseadas, 2 páginas y 1 figura)
%             1.3.2.1.1. Propiedades conformacionales
%             1.3.2.1.2. Propiedades espectroscópicas
%             1.3.2.1.3. Actividades biológicas
%             1.3.2.1.4. Carga metabólica
%         1.3.2.3. Linkers naturales
%             1.3.2.3.1 Características (1 página y 1 figura)
%             1.3.2.3.2 Uso en ingeniería de proteínas (ventajas y desventajas, 1 página)
%         1.3.2.4. Diseño racional
%             1.3.2.4.1 Diseños y conceptos comunes (1 página y 1 figura)
%             1.3.2.4.2 Algoritmos existentes (cómo funcionan, ventajas y desventajas, 2 páginas y 2 figuras)
%             
       
\section{Ingeniería de proteínas}\label{proteinEngineering}
\subsection{Ingeniería de proteínas modulares}    
\subsubsection{Conceptos generales}
Como producto de los avances logrados en las tecnologías asociadas al ADN recombinante, se ha desarrollado una nueva generación de proteínas compuestas por la integración
de diferentes módulos secuenciales.

La idea de armar proteínas a partir de la unión de módulos no es algo nuevo sino que sigue la lógica presentada previamente de que las proteínas naturales son usualmente modulares. 
Es decir, se está simulando el proceso evolutivo natural que desarrolla nuevas proteínas mediante la combinación de dominios preexistentes.

La utilización de la técnica de ADN recombinante para construir nuevas proteínas abre toda una gama de posibilidades que van desde inserción de pequeñas secuencias en extremos de proteína naturales, 
con el fin de poder identificarlas o separarlas, hasta el diseño de construcciones proteicas que buscan obtener proteínas con nuevas funcionalidades o propiedades diferentes.
% construidas mediante la combinación de distintos dominios.
% En los casos más simples el proceso puede ser ...
% En el caso de nuevos diseños, donde se busca combinar módulos para dar nuevas funcionalidades o hacerlas más eficientes, el proceso experimental puede ser mas complejo.

% s primeros ejemplos de construcciones artificiales probablemente sean la inserción de epítopes o tags(pequeñas secuencias?) en los extremos de alguna proteína para poder localizarlas y/o separarlas.
% en solución o en el entorno celular.
% Más adelante se desarrollaron nuevas construcciones combinando unidades estructurales y/o funcionales en una sola molécula. 
% Esta fusión de dos o más dominios abre toda una gama de posibilidades para construir proteínas con nuevas o mejores funcionalidades.
% 

% Las distintas bases de datos permiten obtener, entonces, una gran cantidad de módulos estructurales y funcionales y gran parte del diseño de proteínas quiméricas 
% se basa en utilizar estos conocimientos y anotaciones para crear nuevas proteínas con estructuras/funciones/propiedades combinadas.
En los primeros casos el proceso experimental suele ser bastante directo y basta con fusionar un segmento en el extremo deseado de la proteína. 
En los últimos casos de uso, la implementación del proceso experimental tiende a ser más complejo, requiriendo una aproximación de ingeniería. 
En ésta se pueden distinguire varios aspectos de diseño a considerar, los cuales no siempre son totalmente independientes entre si:
% requiriendo un diseño   o un proceso iterativo.
por un lado está el proceso de diseño de la nueva proteína, por otro lado están los aspectos asociados con la técnica experimental de ADN recombinante y expresión de proteinas heterólogas. 
En el gráfico \ref{esquemaProcesoFusion} se representan los pasos generales que pueden formar parte de este proceso.

\begin{figure}[htbp]
\centering
\includegraphics[width=0.7\textwidth]{img/esquemaProcesoFusion.jpg} 
\caption{Figura obtenida de \cite{yu2015synthetic}}
\label{esquemaProcesoFusion}
\end{figure}

El diseño de la proteína de fusión(quimérica) requiere de dos elementos indispensables:
los dominios o proteínas a fusionar, y la secuencia linker que los va a unir.
La elección de los dominios/proteínas está fuertemente ligada al producto final que se desea obtener y, normalmente, es una decisión directa alrededor de la cual se diseña el resto del experimento.
Por otro lado, la selección de un linker adecuado para unir los dominios/proteinas de acuerdo al objetivo que se busca puede ser un paso complicado y es generalmente ignorado cuando se piensa en el diseño de una proteína quimérica.
La unión directa de los dominios/proteínas o el uso de un linker inadecuado puede resultar en resultados indeseables, por ej. puede restringirse la capacidad de plegado de algun dominio globular, 
disminuir el rendimiento de la proteína resultante o la actividad biológica de alguno de los módulos.
La correcta selección, o mejor aún, el diseño racional de las secuencias linker para unir los módulos es un aspecto importante, aunque poco desarrollado, del diseño de proteínas quiméricas. 
Esta falta de desarrollo en el tema se debe quizás a la falta de conocimiento sobre los factores estructurales que gobiernan la flexibilidad entre los dominios, y resulta ser un factor claramente 
limitante en el diseño \textit{de-novo} de proteínas quiméricas. 
Los conocimientos relevantes sobre estos aspectos estructurales de las secuencias linker tema pueden estar disponibles actualmente a partir de la gran cantidad de secuencias que se disponen, los avances en 
la identificación automática de dominios y las nuevas técnicas para predecir propiedades conformacionales a partir de la información secuencial.



\subsubsection{Ejemplos}

% EJEMPLOS

%

% Examples of this approach include green fluores-
% cent protein (GFP) 1 fusion proteins used in cellular localiza-
% tion studies (1), new antibody types such as multivalent
% antibodies and single-chain antibodies (2-7), artificial
% restriction enzymes consisting of zinc-finger and nuclease
% domains (8, 9),


La gran cantidad de posibilidades de aplicación que tiene ingeniería de proteínas quiméricas alienta a seguir investigando 
los aspectos básicos de las proteínas asociados a este proceso y a desarrollar técnicas que ayuden en el prceso de diseño.
Algunos ejemplos de aplicación son: 

% AGREGAR ALGO DE TECNICAS FRET
\begin{itemize}
 \item Creación de enzimas multifuncionales\cite{ljungcrantz1989construction,fan2009engineering}:
% Cómo se dijo en la sección anterior, se poseen cada vez mas dominios anotados con una gran cantidad de funciones correspondientes.
La forma mas común de crear proteinas quiméricas es fusionar genéticamente dos o más dominios con funcionalidades distintas, obteniendo una nueva funcion global para la proteína, resultante de la suma de éstas.
En el caso de enzimas el objetivo generalmente es que una misma construcción pueda tener afinidad por distintos sustratos, o que las distintas enzimas formen una cadena metabólica, donde el producto de una sea a la vez el sustrato de otra.
Es importante que el linker que una a estas distintas enzimas permita la correcta actividad de cada una por separado.

\item Facilitar el estudio de interacciones proteina-proteina\cite{reddy2013linkers}: 
Tradicionamlente, la forma más simple para estudiar la interacción entre dos proteinas es expresarlas conjuntamente para obtener el complejo formado.
Sin embargo, si la afinidad es muy baja, no es tan simple que este complejo se forme de manera estable.
La creación de una proteína quimérica que une a los dominios/proteínas que forman el complejo permite mantenerlas unidas covalentemente mediante una secuencia linker.
La flexibilidad de esta secuencia debería permitir la interacción libre entre los componentes generando la formación del complejo. Al estar unidas covalentemente, habrá una posibilidad mucho mayor de interacción y aumentará la cinética de 
formación, permitiendo realizar los estudios biofísicos necesarios sobre éste.
% The characterization of protein–protein interactions is often required to gain an understanding of various biological processes. 
% Yet, the study of protein–protein interactions for many complexes is hampered when one or more partners of the complex are unfolded or unstable. 
% Traditionally, this problem has been addressed by the co-expression and/or copurification of both proteins. 
% However, for weakly interacting or unstable complexes, the co-expression and/or co-purification often results in a single pro-
% tein. 
% Protein engineering techniques were another
% option to address unfolded or unstable proteins,
% using a single polypeptide chain chimera to link the
% two binding partners via a flexible amino acid linker
% 13
% . With these chimeric proteins, it was then possible
% to maintain both the intramolecular and intermolec-
% ular protein–protein interactions, 14 and chimeric
% proteins have been used to generate stable, soluble
% binary complexes for structural studies, as well as
% functional dimers.
% Linking binding partners using an artificial
% linker will increase the proximity between the inter-
% acting partners and preserve the natural interaction.
% In cases where the interacting partners are not
% linked, it is possible that the binding partners might
% dissociate due to their low affinity and/or due to the
% crystallization conditions.
\item Construcción de sensores FRET:
La técnica FRET se basa en utilizar el efecto de transferencia de energia entre dos cromóforos(un donor y un aceptor).
Dado que la eficiencia en la transferencia, y por lo tanto de la señal de salida, depende de la distancia, este efecto puede ser utilizado en la técnica para evaluar la distancia entre dos 
cromóforos cuya distancia depende de algun proceso molecular de interés. \\
La construcción del sensor generalmente implica utilizar un par de dominios de unión cuya interacción estable depende de algun evento molecular de interés(ej. unión de algún elementos o la ocurrencia de algún proceso).
Cada dominio de unión es unido por separado a uno de los cromóforos(donor o aceptor) y estas construcciones se unen entre si por linkers flexibles que les permiten interaccionar libremente.
En su forma libre, esta construcción permite a los cromóforos moverse de manera independiente por lo tanto no ocurrirá una transferencia significativa de energía. 
Cuando los dominios fusionados a cada cromóforo se unén (lo cual depende de la ocurrencia del evento de interés), la distancia quedará fijada y esto cambiará el perfil de transferencia de energía.
Este esquema se muestra gráficamente en la imagen \ref{fretSensor}

\begin{figure}[htbp]
\centering
\includegraphics[width=0.7\textwidth]{img/fret.png} 
\caption{Figura obtenida de \cite{crone2013gfp}}
\label{fretSensor}
\end{figure}

Esta técnica es muy versátil y por lo tanto existen una gran cantidad de ejemplos y modificaciones \cite{vinkenborg2009genetically,qiao2006zinc}. %REFERENCIAS
En muchos casos la dependencia del proceso molecular con la distancia esta directamente relacionada con la secuencia linker, 
por ejemplo si esta secuencia contiene elementos target de modificaciones post-traduccionales que pueden cambiar las propiedades conformacionales del linker e, indirectamente, el perfil de distancias entre los extremos. 
En otros casos la conformación en la que los dominios están unidos fija una distancia importante entre los cromóforos, entonces el sensor mide la disminución en la transferencia de energía como evento resultado del evento de interés.
En todos los casos es importante el proceso de ingenieria aplicado sobre el linker para obtener el correcto funcionamiento del sensor.



% ESTE EJEMPLO LO SACO PARA QUE NO QUEDE TAN LARGA LA SECCION
% 
% \item Diseño de métodos de detección para técnicas inmunoquímicas \cite{arai1998construction,arai2000fluorolabeling,suzuki1999open}:
% % La unión de fragmentos de anticuerpos a enzimas o proteínas que permitan la detección de la unión es la base de las técnicas inmunoquímicas
% Los anticuerpos poseen la capacidad de unirse a proteínas que tengan expuestos epítopes específicos. 
% Esta especificidad permite que se usen para identificar la presencia de proteínas de interés en el contexto de diversos métodos experimentales.
% % Para poder revelar esta proceso de unión específica es necesario fusionar el anticuerpo con algun elemento que permita la detección, generalmente de forma visual.  
% Este proceso, entonces, requiere la utilización de ingeniería de proteínas quiméricas para fusionar el anticuerpo con un dominio cromóforo, una enzima específica, u otro elemento que permita revelar la unión, generalmente de forma visual.


\end{itemize}












%     1.3.2. Ingeniería de linkers.
%         1.3.2.1. Diseño positivo (propiedades deseadas, 1 página, 1 figura)
%             1.3.2.1.1. Propiedades conformacionales
%             1.3.2.1.2. Carga
%         1.3.2.2. Diseño negativo (propiedades no deseadas, 2 páginas y 1 figura)
%             1.3.2.1.1. Propiedades conformacionales
%             1.3.2.1.2. Propiedades espectroscópicas
%             1.3.2.1.3. Actividades biológicas
%             1.3.2.1.4. Carga metabólica
     
      
\subsection{Ingeniería de secuencias linker}

\subsubsection{Aspectos de diseño}\label{aspectosDiseno}
% 
% *******************************************************************
% *******************************************************************% 
%  		DIVIDIR ESTA SECCION IGUAL QUE EL CAPITULO 3 :   PROPIEDADES CONFORMACIONALES , ELEMENTOS BIOLOGICAMENTE FUNCIONALES,  OTRAS PROPIEDADES
% % *******************************************************************
% *******************************************************************
% 
% 


% EN ESTA  SECCION PONER LA LISTA COMPLETA DE LOS REQUERIMIENTOS POSITIVOS Y NEGATIVOS DE UN LINKER:
% LONGITUD, COMPOSICION, CONFORMACION DESORDENADA INTRINSECA, AUSENCIA DE ESTRUCTURA SECUNDARIA, AUSENCIA DE ELEMENTOS FUNCIONALES

% \subsubsection{Diseño positivo}

La función de la secuencia linker es unir los dominios covalentemente a la vez que estos actúan como unidades independientes, manteniendo sus capacidades de plegado, funciones individuales y 
permitiendo que se muevan e interaccionen libremente. 
Esta funcionalidad depende directamente de la longitud y propiedades conformacionales del linker, por lo tanto, son los primeros aspectos a tener en cuenta en el diseño de esta.

A pesar de ser un factor directo de la eficiencia\cite{robinson1998optimizing}, la funcionalidad del linker no suele ser tan sensible a la longitud y un amplio rango de longitudes pueden resultar efectivas, siempre 
que se eviten las secuencias muy cortas, que no logran obtener una separación suficiente, o las secuencias extremadamente largas, que harían que los dominios funcionen de forma totalmente independiente entre si.
% harían muy ineficaz la interacción entre proteínas.

Por su parte, los requerimientos conformacionales suelen ser más estrictos e incluso pequeños cambios en la estructura pueden limitar la funcionalidad del diseño.

\paragraph{Propiedades conformacionales} \hspace{0pt} \\
En primer lugar buscamos que la secuencia linker adopte una conformación desordenada, es decir, que no se pliegue formando una estructura compacta con los dominios que une.
Esta conformación provee la flexibilidad necesaria para que los dominios puedan moverse libremente y las interacciones puedan ocurrir sin ninguna restricción estructural.
% Para las arquitecturas más comunes, compuestas de dominios globulares unidos por linkers, esta flexibilidad es el requerimiento mas importante ya que permite a los distintos dominios interaccionar libremente.
Para lograr esto es importante controlar que la secuencia no sea propensa a formar estructuras secundarias rígidas tales como $\alpha$-hélices o hebras-$\beta$, ya que esto
limitaría la flexibilidad, afectando directamente la interacción entre los dominios que une.
Estas propiedades conformacionales deseadas se corresponden con las características vistas para las proteínas intrínsecamente desordenadas. 

En la figura \ref{conformacionLinker} se muestran como influyen estos requerimientos en la construcción de una proteína quimérica que une dos dominios EBFP y EGFP. 
Los gráficos A,B y E muestran conformaciones globales de la proteína que son posibles gracias a la flexibilidad del linker intrínsecamente desordenado con una longitud apropiada. 
Las interacciones que permiten estas conformaciones no se pueden obtener cuando se usa una secuencia con estructura de estructura de hélice-$\alpha$ como se ve en las figuras C y D, donde la rigidez de ésta
hace que las conformaciones globales sean mucho más limitadas y ciertas interacciones no puedan ocurrir.


\begin{figure}[htbp]
\centering
\includegraphics[width=0.3\textwidth]{img/conformacionLinker.png} 
\caption{Figura obtenida de \cite{arai2004conformations}}
\label{conformacionLinker}
\end{figure}



Las características conformacionales buscadas no sólo serán resultado de interacciones intramoleculares sino que dependen del contexto en el cual la proteína diseñada se encuentre. 
La construcción quimérica se expresa en un sistema biológico determinado a partir del cual puede ser luego aislada y purificada para estudiar. 
Esto impone ciertos requerimientos al diseño del linker ya que las interacciones estables con otros componentes del sistema pueden limitar seriamente la flexibilidad intrínseca de este.

Uno de los procesos que pueden afectar más severamente la funcionalidad del linker es la formación de agregados proteicos durante la producción de la proteína\cite{lebendiker2014production}.
En este caso la unión a otras proteínas no solo limita la flexibilidad del linker sino también arrastra a la proteína completa inhibiendo totalmente la actividad de esta.
Además, la agregación no es sólo un problema de la expresión en el sistema biológico, donde la proteína debe mantenerse en solución,
sino que también puede ocurrir durante las etapas posteriores de separación y purificación.
Es importante que el diseño del linker tenga en cuenta estos aspectos(y otros similares), evitando que la secuencia posea tendencias a formar uniones estables con otros componentes del sistema.


\paragraph{Elementos biológicamente funcionales} \hspace{0pt} \\
% 
% 
Los aspectos conformacionales no son los únicos relevantes. 
Es importante que, además, la secuencia se mantenga biológicamente inerte frente al sistema en el cual se expresa de manera que el linker funcione únicamente como conector entre los dominios.
Incluso cuando se trate de secuencias con longitud corta, es importante que no posea ningún elemento secuencial que pueda funcionar como módulo de interacción iniciando un proceso biológico. 

% Las interacciones desfavorables durante el proceso de ingeniería no se limitan a estas uniones estables, como vimos previamente las IDPs pueden poseer actividades biológicas, 
% principalmente a través de módulos de interacción en la secuencia. 
% Al no adoptar una estructura plegada, estos elementos quedan expuestos al entorno y permiten mediar diferentes interacciones funcionales.
La gama de actividades que involucra a los módulos funcionales, aún aquellos que puedan encontrarse en secuencias cortas (como suelen ser los linkers), es muy diversa y por lo tanto pueden tener implicancias diferentes 
tanto sobre el linker y la proteína quimérica como en el normal funcionamiento del sistema biológico donde se expresa esta. %sobre el linker en particular y la proteína quimérica en general.
% Entre las actividades biológicas asociadas a estos segmentos están las modificaciones post-traduccionales, unión a ligandos, sitios de clivaje, etc.
Ejemplo de esto son los sitios target de clivaje, cuya ocurrencia dentro de la secuencia linker tendría consecuencias obvias que afectarían directamente a la unión covalente que genera el linker.
Otros ejemplos son las modificaciones post-traduccionales sobre la secuencia, las cuales pueden modificar las características de esta(por ejemplo la carga neta) impactando directamente en las propiedades conformacionales asociadas.
% si el linker es target de PTM puede sufrir cambios que afecten la carga de esta secuencia, y como la carga esta directamente relacionada con las propiedades conformacionales, estas modificaciones pueden afectar la flexibilidad
% Por lo tanto, es importante que la secuencia linker este libre de cualquier elemento funcional en su secuencia. 

% La idea del diseño uniendo distintos módulos es que la funcionalidad global se origine exclusivamente a partir de la unión de éstos, por lo tanto la secuencia que los une no debe proveer ninguna funcionalidad adicional, es decir,
% debe permanecer inerte ante el entorno en el cual la nueva proteína desarrolla su actividad. 


% ESTO LO PUEDO SACAR
Muchas veces un mismo proceso representa efectos biológicos y conformacionales negativos conjuntamente, tal es el caso de los MoREs, los cuales usualmente señalizan 
el desarrollo de un proceso biológico mediante un mecanismo conocido como plegamiento y unión, en el cual se estabilizan estructuras secundarias donde antes existia una conformación desordenada.

% Otro ejemplo es la unión a ligandos mediante el proceso de plegado y unión(\textit{binding \& folding}), un proceso mediante el cual se estabilizan estructuras secundarias que de otro modo sólo ocurren en forma transitiva.  
% Esta es una propiedad que describimos como característica de las proteínas intrínsecamente desordenadas y, por lo tanto, es posible que ocurra en el linker dada su conformación intrínsecamente desordenada.


\paragraph{Otras propiedades} \hspace{0pt} \\

% COMPOSICION DE AMINOACIDOS: Carga metabolica y propiedades espectroscópicas
% Los requerimientos para la secuencia linker pueden originarse en otros pasos del proceso de producción, no solo en la funcionalidad que aporta a la construcción. 

% por el proceso para producir la proteína quimérica no están limitados a aquellos que apuntan a mantener las propiedades deseadas de la construcción, 
% sino que pueden estar asociados a otros aspectos del proceso.
Como parte del proceso de producción se requiere que el sistema exprese la construcción diseñada, lo cual, al ser una proteína foránea, estará directamente ligado a la carga
metabólica que la secuencia imponga sobre el sistema\cite{glick1995metabolic}. Esta carga metabólica es el resultado de la utilización de recursos propios de la célula para la expresión de esta nueva proteína y depende, entre otros aspectos, 
de la composición de aminoácidos de esta. 
Diferentes aminoácidos representan diferentes requerimientos metabólicos, por lo tanto, un aspecto relevante de la secuencia es que esta tenga una composición que imponga mínimos requerimientos metabólicos al sistema. 
La composición raramente puede ser modificada en los módulos que se están fusionando, pero si es posible tener estos aspectos en cuenta al momento de diseñar la secuencia linker la cual deberá imponer, idealmente, 
la mínima carga posible al sistema de expresión. La importancia de este requerimientos depende de las consideraciones del usuario que diseña el experimento, del sistema de expresión que se va a usar, el nivel de expresión requerida, etc.

En otros casos, los pasos posteriores a la expresión pueden imponer requerimientos sobre el linker que provienen de la manipulación, detección o procesamiento de la proteína construida.
Por ejemplo, dado que las técnicas espectroscópicas se usan de forma rutinaria en el trabajo con proteínas, es posible que el usuario 
quiera realizar la construcción utilizando un linker cuya composición específica no posea aminoácidos con absorbancia en el rango del UV, 
de forma que se elimine cualquier interferencia con este tipo de técnicas.

Todas estas situaciones están asociadas a la composición del linker diseñado. 
La composición ideal de esta secuencia linker dependerá, entonces, no sólo de las propiedades secuenciales asociadas a la funcionalidad dentro de la construcción, sino también de aquellos aspectos 
que el usuario considere relevantes como parte del proceso experimental.






% 
% 
% 
% % carga neta
% % RESIDUOS CARGADOS
% % Una ventaja importante del proceso de diseño es que se pueda definir la carga neta que tendrá la secuencia linker. 
% % Este requerimiento puede tener varios fundamentos, principalmente 
% Distintas técnicas de laboratorio para identificación y purificación de proteínas se basan en la carga neta de la secuencia.
% 
% 
% Por otro lado, la carga de la secuencia puede mediar interacciones en el entorno celular, por ejemplo, al construir proteínas cuya función requiera unirse al DNA es 
% requerir que la secuencia linker no tenga residos carga ya que 
% 
% Otro ejemplo que afecta a la composición está asociada con interacciones iónicas no deseadas. 
% Por ejemplo al construir una proteína cuya funcionalidad esté asociada a la unión a DNA, será deseable que la composición final no contenga aminoácidos que puedan encontrarse en estados con carga positiva, debido a que 
% la formación de interacciones con la estructura de fosfatos propia del DNA podría interferir en la función de la proteína.
% 
% 
% 
% 


























  1.3.2.3. Linkers naturales
%             1.3.2.3.1 Características (1 página y 1 figura)
%             1.3.2.3.2 Uso en ingeniería de proteínas (ventajas y desventajas, 1 página)

\subsubsection{Linkers naturales}


\paragraph{Características} \hspace{0pt} \\
% El 'descubrimiento' de las regiones/secuencias linkers esta ligado a las teorias de structura-funcion(desarrolladas en la parte de conformacion) que se dieron durante casi 100 años
% En un principio se comenzo a pensar en una estructura rigida asociada a la proteina, luego se fueron revelendo propiedades dinamicas que le permitian cumplir la funcion. 
% Todo esto esta muy asociado a las tecnicas experimentales que se fueron desarrollando.
Como ya se mencionó, la modularidad en proteínas naturales es algo muy común y existen muchos ejemplos de proteinas compuestas de dos o mas dominios funcionales unidas por linkers.
Estas secuencias linker sirven, principalmente, para mantener unidos los distintos dominios y proveer la flexibilidad necesaria para que puedan ocurrir las interacciones asociadas a la actividad biológica de la proteína
(ver figura \ref{multidomainRb}). Esta es la funcionalidad básica inherente a todos los linkers naturales, y está directamente asociada a sus propiedades conformacionales. 
Sin embargo, en muchos casos también proveen otras funciones que forman parte de la actividad biológica de la proteína, tales como intervenir en las interacciones cooperativas entre los dominios, 
contener elementos secuenciales que median interacciones con otras proteínas o ligandos, etc.

\begin{figure}[ht]
\centering
\includegraphics[width=0.8\textwidth]{img/multidomainRb.png} 
\caption{Ejemplo de linkers naturales de la proteina multidominio RB (Retinoblastoma). Los linkers y loops se muestran en \textbf{a} mediante lineas entrecortadas. En \textbf{b} se muestra una representación más abstracta de la misma estructura multidominio.
REferencia..................Molecular mechanisms underlying RB protein function} 
\label{multidomainRb}
\end{figure}


El primer estudio realizado sobre secuencias linkers registra la preferencia intrínseca por las conformaciones desplegadas\cite{argos1990investigation}.   
Sin embargo este estudio está bastante desactualizado y los resultados no son representativos ya que se realizó sobre un total de 51 linkers detectados manualmente a partir de 32 proteínas.
En \cite{george2002analysis} se encuentra un análisis más actual sobre un total de 638 proteínas multidominio, cuyas estructuras fueron analizadas mediante métodos automatizados 
para determinar los dominios y luego las regiones linkers(un total de 1280).
En \cite{chen2013fusion} se revisan los resultados obtenidos en ambos con respecto a diversas propiedades como son longitud, hidrofobicidad, enriquecimiento de ciertos aminoacidos y estructura secundaria adoptada.


En general, los residuos encontrados pertenecían principalmente al grupo de aminoácidos polares(con o sin carga), 
y puntualmente los preferidos eran Pro, Thr y Gln.
La ocurrencia de Prolina es bastante esperable, su cadena lateral cíclica y la falta de un hidrógeno disponible 
en su función amida previene la formación de puentes de hidrógeno con otros aminoácidos, esto le da una baja 
tendencia a formar estructuras secundarias típicas (hélices-$\alpha$ y hojas-$\beta$).
Por lo tanto, las secuencias ricas en prolina resultan en una estructura más bien rígida que, como parte de un linker, permitirían 
prevenir la formación de una estructura plegada e interacciones desfavorables entre/con los dominios. 



% PROPIEDADS ESTRUCTURALES

% ESTRUCTURA SECUNDARIA
% Natural linkers adopt various conformations in secondary structure, such as helical, β-strand, coil/bend and turns, to exert their functions.
% the largest proportion of linker residues, 38.3\%,
% adopt the α-helical secondary structure, 13.6\% are in
% β-strands, 8.4\% are in turns and the rest, 37.6\%, are in coil
% or bend secondary structures.

En general, se encontro que los linkers naturales adoptaban principalmente conformaciones desplegadas y tenían estructuras independientes sin interacciones con los dominios adyacentes.
En términos de estructura, se encontraron tanto linkers flexibles como relativamente rígidos, estos últimos formados principalmente por $\alpha$-hélices en su estructura secundaria. 
% A través de una mayor rigidez, ciertos linkers pueden ayudar a reducir interacciones no funcionales entre los dominios que unen. 
% Por otro lado, una estructura más flexible, como es usual para conformaciones extendidas, provee una mayor flexibilidad y libertad de movimiento a los dominios.
Basándose en la frecuencia de este tipo de estructura secundaria, en \cite{george2002analysis} se esboza una clasificación de los linkers dividiéndolos en dos categorías: helicoidales y no helicoidales.
% Los linkers con estructuras de $\alpha$-hélice pueden tener funciones como, por ejemplo, actuar como espaciadores rígidos impidiendo interacciones no funcionales entre los dominios.
% Aunque no sea exhaustiva, esta clasificación permite ver diferencias en los linkers también a nivel de estructura que adquieren, existiendo linkers que, a través de una mayor rigidez, impiden . 
Todas las propiedades analizadas(longitud, composición, hidrofobicidad y estructura secundaria.) resultaron importantes para alcanzar las funciones deseadas.



% AL FINAL DE TODO PONGO ESTOS EJEMPLOS, DEMOSTRANDO QUE LA FUNCIONALIDAD DE LOS LINKERS NO ESTÁ LIMITADA A PROVEER FLEXIBLIDAD Y QUE PUEDEN CUMPLIR OTRAS FUNCIONES

%EJEMPLOS DE PROPIEDADES CONFORMACIONALES
En terminos de flexibilidad, y por lo tanto de propiedades conformacionales en general, los linkers muestran un rango muy amplio de características las cuales se pueden apreciar mejor a partir de algunos ejemplos.
En algunos casos la flexibilidad provista se origina a nivel de estructura terciaria, y esta limitada a una región corta que funciona como bisagra.% EJEMPLO!!!!
En otros casos la flexibilidad es ``completa'' y la región del linker es intrínsecamente desordenada\cite{luo2010flexibility}, 
encontrándose incluso que la funcionalidad de la proteína se pierde cuando se reemplaza por un linker que adopta una estructura de hélice-$\alpha$ dándole una mayor rigidez estructural\cite{hrycyna1998structural}
% EJEMPLO !
Ademas, existen linkers que, a pesar de mantener una conformación extendida en solución, pueden plegarse (o fijar una estructura secundaria transitiva) en presencia de ligandos.
Por ejemplo en el caso de ciertas proteínas que se unen a DNA\cite{laity2000dna}
%ESTE ES UN EJEMPLO DE FUNCIONALIDAD 


En algunos casos, las secuencias linker permiten mediar la propagacion eficiente de los efectos originados por la unión de ligandos o modificaciones post-traduccionales en uno de los dominios que conecta(alosterismo).
Un ejemplo es el caso de la miosina del musculo liso, en la cual un dominio que comprende la funcion motora es activado mediante fosforilacion  de una cadena regulatoria unida a este mediante un linker\cite{ikebe1998hinge}.
% OTRO EJEMPLO
% An example is the intramolecular interaction between the Src homology domains (SH2 and SH3) and the catalytic domains of Src family kinases, which results in repression of catalytic
% activity. Repression by the regulatory domain is nullified upon mutation of Trp260 to Ala within the linker separating the SH2 and kinase domain, which proves that the linker plays a crucial role in the coupling of
% the regulatory domains to the catalytic domain



%  ejemplo de la importancia de la secuencia linker en 
% otro ejemplo de funcionalidad esta en  \cite{tsutsumi2012charged}  ...



% CONCLUSIONES
Por lo tanto, a pesar que la flexibilidad es un aspecto que se sabe esta ligado a las secuencias linker en la naturaleza\cite{wriggers2005control}, 
y que estás regiones permiten los grandes cambios conformacionales en las estructuras de las proteínas, 
no son simplemente secuencias que evolucionan hacia conformaciones totalmente flexibles, ya que cada proteína impone requerimientos conformacionales específicos. 
% esto puede no estar directamente ligado al cumplimiento de la función requerida, o no hacerlo de forma eficiente.

De esta forma, las secuencias linker son una parte mas de las proteínas y sus propiedades conformacionales no pueden definirse en forma generalizada, sino que están 
% De esta forma, las propiedades conformacionales de las regiones linkers naturales están usualmente 
asociadas a restricciones funcionales de la proteína global buscando balancear la flexibilidad requerida 
para que los dominios puedan explorar el ensamble de conformaciones asociado a su funcionalidad, con la rigidez requerida para que no existan interacciones desfavorables entre estos, 
resultando en perfiles estructurales muy variados.




\paragraph{Utilización en diseños de proteínas quiméricas} \hspace{0pt} \\

% USO EN INGENIERIA DE PROTEINAS
En primer lugar, si bien en todos los casos los linkers naturales permiten mantener covalentemente unidos a los dominios, 
las propiedades conformacionales pueden ser muy distintas a las que buscamos para un linker en el contexto de una proteína quimérica.
Por otro lado, en cada proteína natural,   y por lo tanto, al utilizar este linker estaríamos agregando un módulo biológicamente funcional a la proteína quimérica.

% Sin embargo, en cada proteína, la secuencia linker puede tener propiedades conformacionales independientes, 
% En cada proteina, la secuencia linker puede tener una estructura y una función que haya sido seleccionada para la actividad biológica de la proteina como un todo, 
% y esta función del linker en este contexto puede no ser solamente la unión covalente de los dominios, tal como se busca al diseñar un linker. 



% UTILIZACION DE LINKERS NATURALES PARA 
Dado este contexto, el uso de secuencias linkers naturales en el diseño de proteínas quiméricas debe realizarse con mucho cuidado ya que no todos cumplen con los requerimientos conformacionales y funcionales buscados.
% dado que los linkers en la naturaleza funcionan muchas veces como una parte integrada de la proteína, no sólo en términos conformacionales sino también en cuanto a aspectos funcionales.
% DIVERSIDAD
Por último, aún en el caso de encontrar secuencias naturales que cumplan los requerimientos buscados, la diversidad de secuencias existente, principalmente en cuanto a composición y longitud, puede no ajustarse a la diversidad de 
preferencias que se presentan en el contexto de cada experimento posible que requiera la utilización de proteínas quiméricas. %, los cuales están asociados, como vimos, a los pasos del proceso de ingeniería.

% Por otro lado, si bien existe una gran cantidad de secuencias linkers naturales con propiedades conformacionales y funcionales tan diferentes, no es fácil encontrar secuencias que se adapten correctamente a los requerimientos del proceso
% que requiere la ingenieria de proteínas quiméricas.

Poder usar secuencias naturales presenta, sin embargo, varias ventajas. 
En primer lugar, generalmente se conocen las propiedades conformacionales y funcionales asociadas a los linkers naturales en el contexto de un sistema biológico, o estas puede inferirse a partir de análisis secuenciales comparativos. 
Además, estas secuencias han evolucionado como parte de sistemas biológicos, de forma que no suelen plantear ningún desafío para la expresión.
% ACÁ PUEDO PONER QUE LOS LINKERS NATURALES TAMBIEN PUEDEN SERVIR COMO BASE PARA UN PROCESO DE INGENIERIA?? O LO PASO A LA PROX. SECCION???









% Los primeros estudios de analisis (estructura y composicion) de secuencias linkers\cite{argos1990investigation} se comenzaron a hacer a partir del analisis estadistico de secuencias que podian ser clasificadas 
% como linkers a partir de las primeras estructuras de proteinas multidominio almacenadas en bases de datos.
% % Estos estudios estaban sesgados por todo el proceso historico de descubrimiento marcado por 
% Los resultados indicaban que la mayoria adquiría una conformación desplegada(tipo \textit{coil}) sin estructura secundaria. 
% % many studies of linker peptides in various protein families have come to the conclusion that linkers lack regular secondary structure (la mayoria se encontraba en estructuraas tipo coil), 
% % they display varying degrees of flexibility to match their particular biological purpose and are rich in Ala, Pro and charged residues
% Asi surgio la idea que los linkers eran secuencias cortas cuya unica funcion era proveer la conexion covalente, lo que estaba de acuerdo con el concepto de hinge-bending.
% Este concepto indicaba que la flexibilidad de estas regiones cortas dentro de un polipéptido permitía el suficiente movimiento a los dominios estructurales.
% % The concept of hinge-bending, whereby the relative flexibility of these short regions of the polypeptide chain allows significant movement of structural domains, gained widespread acceptance in
% % the 1980s and early 1990s, after evidence for conformational transitions in identical or homologous proteins became known.



% A lo largo de los años los conocimientos sobre la composición y propiedades de estas secuencias ha ido cambiando, 
% a medida que mayor cantidad de estructuras se resolvian y mayor conocimiento se obtenia acerca de los dominios que componen 
% las proteinas. Tambien influyeron otras cosas como tecnicas biofisicas que permiten obtener información del ensamble conformacional en solución,
% o algoritmos para automatizar la identificación de los dominios y las secuencias que actúan como linker en una proteina.
% 

% Although the role of linker sequences is likely to be primarily topological, allowing distant parts of the polypeptide chain to interact with diverse partner sequences that might be far apart or close together, 
% linkers and unstructured tail sequences play quite specific roles in a number of systems.








% A FUTURO
% Con el incremento del número de estructuras almacenadas en la PDB que ocurrió en los últimos años, sería posible realizar un estudio actualizado de las propiedades de los linkers naturales.
% Además, sería interesante extender el número de propiedades analizadas agregando categorías asociadas a función y estructura de la proteína, e identificando la relación entre estas y las propiedades del linker.
% With the rapid increase of the number of protein structures deposited in the PDB database, an updated study of natural linkers could be conducted. 
% In addition to the properties analyzed in previous studies (e.g., amino acid composition, structure classification), 
% it would be interesting to categorize the multi-domain proteins by their functions and structures, and identify the relationship between them and the linker properties


% 
% 
% Based on From George and
% Heringa’s secondary structure analysis, linkers were grouped into two categories: helical and
% non-helical. The $\alpha$-helix was a rigid and stable structure, with intra-segment hydrogen bonds
% and a closely packed backbone [28]. Some $\alpha$-helical conformations form rapidly during
% folding [28], allowing the correct folding of connecting protein domains without non-native
% interactions with the linker. Linkers in an $\alpha$-helix structure might also serve as rigid spacers
% to effectively separate protein domains, and to reduce their unfavorable interactions.
% Therefore, this conformation was commonly adopted by many natural and empirical linkers
% (to be discussed later). On the other hand, without an inherent rigid structure, the non-helical
% linkers tended to be rich in Pro, which could increase the stiffness of the linker as mentioned
% previously [25]. As a result, non-helical linkers with Pro-rich sequence could exhibit
% relatively rigid structures and serve to reduce inter-domain interference.
% 
% 
% 
% Both flexible and relatively rigid peptide linkers are found in many multidomain proteins. 
% Linkers are thought to control favorable and unfavorable interactions between adjacent domains by means of variable softness
% furnished by their primary sequence. Large-scale structural heterogeneity of multidomain proteins
% and their complexes, facilitated by soft peptide linkers, is now seen as the norm rather than the
% exception. Biophysical discoveries as well as computational algorithms and databases have
% reshaped our understanding of the often spectacular biomolecular dynamics enabled by soft linkers.
% Absence of such motion, as in so-called molecular rulers, also has desirable functional effects in
% protein architecture.







% 
% 
% \subsubsection{Molecular rulers}
% These linkers are more defined by their ability to reliably predict and maintain end-to-end distances between attached domains. 
% Such structurally rigid peptides have been conjugated to molecules to serve a metric function.
% These linkers are rich in Proline. 
% Proline is common to many naturally derived interdomain linkers, and structural studies indicate that proline-rich sequences form relatively rigid extended structures to prevent unfavorable interactions between the domains.
% The probable reason why proline is favored over other residues in linking different domains is the inability of proline to donate hydrogen bonds or participate comfortably in any regular secondary structure conformation. This ensures a relatively rigid separation of the domains, thereby preventing unfavorable contacts between them.
% 
% Although short stretches of hard linker sequences are located between functionally relevant regions of protein structure, mutations within such sequences may have no effect on the function.  
% Such linkers are therefore necessary to keep the other amino acid interactions in register, but the nature of the side chain is often unimportant.
% 
% The observed natural tendency to form rigid linkers might also
% be related to avoiding proteolytic cleavage, as linkers are likely
% targets for protease degradation
% 
% Linker
% sequences vary greatly in length and composition, but
% many are rich in polar, uncharged amino acids (such as
% Ser, Thr, Gln and Asn), in the small residues Ala and Gly,
% and in Pro residues. Many of these residues tend to bias
% the polypeptide chain towards the polyproline-II region
% of the RAMACHANDRAN PLOT 27,28 .This means that such
% linkers, although flexible, have a propensity to be highly
% extended. Compositionally biased linker sequences of
% significant length are found mainly in eukaryotic pro-
% teins 1,29 , but short linker sequences of similar composi-
% tion, known as Q-linkers, are also found in a number of
% bacterial regulatory proteins 30 .
% In the absence of their targets, modular proteins
% often behave as ‘beads on a flexible string’, where the
% function of the linker is, primarily, to enable a relatively
% unhindered spatial search by the attached domains 31 .
% However, binding can induce structure formation in
% linkers, which can have significant functional conse-
% quences. For example, the sequence-specific binding of
% CYS HIS ZINC-FINGER PROTEINS to DNA causes the linker to
% fold, cap and thereby stabilize the preceding helix in the
% protein, and to orientate the next zinc finger correctly
% for binding in the major groove of DNA
% 
% 













%          1.3.2.4. Diseño racional
%             1.3.2.4.1 Diseños y conceptos comunes (1 página y 1 figura)
%             1.3.2.4.2 Algoritmos existentes (cómo funcionan, ventajas y desventajas, 2 páginas y 2 figuras)
%             
\subsubsection{Diseño racional}
% The general properties of linkers derived from naturally-occurring multi-domain proteins(que se vieron en la seccion anterior) can be considered as the foundation in linker design. 

\paragraph{Diseños y conceptos comunes} \hspace{0pt} \\

% PRIMERO HABLAR DE LINKERS USADOS EMPIRICAMENTE 
Con tantos y tan distintos requerimientos positivos y negativos para el diseño, el problema de encontrar un linker adecuado puede llegar a ser un proceso complejo.
Los diseños encontrados en la literatura están basados en la intuición y en las propiedades generales de los linkers naturales encontrados en proteínas multi-dominio.
En \cite{chen2013fusion} se revisan los diseños resultantes de este proceso artesanal que han sido útiles al evaluarse experimentalmente en varios contextos.


% A partir de esta recopilación se intenta hacer una clasificación general que resulta en 3 categorias:
% linkers flexibles, linkers rígidos, y linkers que pueden experimentar clivaje \textit{in-vivo}. 
% Como se puede ver, esta clasificación esta basada, principalmente, en propiedades estructurales/conformacionales. 
% Para obtener secuencias que posean otras propiedades de interés deberán analizarse cada uno de los linkers que se detallan en este trabajo.

Basándose en los aspectos conformacionales, un linker poli-G parece ser la opción evidente para proveer a la construcción de máxima flexibilidad.
Como vimos antes, la flexibilidad no es todo, una secuencia nucleotídica con alto contenido de Guanina(el codón que codifica para Glicina contiene Guanina en 2 de las 3 posiciones) 
puede ser difícil de manejar experimentalmente y de expresar para el huésped\cite{trinh2004optimization}.
% y linkers poli-G, por ejemplo, pueden ser target de actividad proteolítica, a través del patrón Gly-Gly-X, donde X es un residuo con cadena lateral hidrofóbica.
Además, la cadena poli-G puede resultar en una conformación altamente inestable debido a que no contiene ningún residuo polar, afectando la actividad de los dominios que une\cite{robinson1998optimizing}. 
La secuencia poli-G por si sola, puede no resultar el mejor linker, sin embargo se ha utilizado en algunos casos\cite{iwakura1998effects,de2012characterization,sabourin2007flexible} y es la base de la mayoría de los linkers flexibles 
encontrados, sobre los cuales se agregand modificaciones usando otros aminoácidos pequeños como Serina.% y Thr?

Probablemente, el linker más común encontrado en la literatura es el compuesto por distinto número de repeticiones del motivo $(G_4S_n)$, %PONER REFERENCIAS AL USO DE ESTE.
en estos casos la sustitución de algunas posiciones por residuos polares (Ser) busca reducir estabilizar el linker permitiendo interacciones entre la cadena lateral polar y el medio acuoso,
lo que previene interacciones no deseadas entre el linker y los dominios que une de forma tal que no interfieran en la funcionalidad.
Una aproximación de diseño más avanzada se puede ver en \cite{bird1988single}, donde se utilizan residuos Gly y Ser para proveer flexibilidad pero se agregan Glu y Lys para incrementar la solubilidad,
el linker resultante tiene la secuencia \textit{}.
% COMENTAR EL METODO USADO MAS EN DETALLE



% % ACA EMPIEZO A HABLAR DE OTROS ASPECTOS NO ESTRUCTURALES

% En primer lugar, un péptido poli-G sería extremadamente inestable y por lo tanto podría actuar como una carga energética, estructural o interferir en procesos de catálisis de los dominios que une, 
% especialmente si tiene una longitud excesiva.
% Se conoce además que el patrón Gly-Gly-X, donde X es un residuo con cadena lateral hidrofóbica, es un sitio target de actividad proteolítica. 
%% Como se ve, entonces, existe una gran variedad de aspectos a considerar que exigen distintos requerimientos además de las propiedades conformacionales.
% Por ejemplo, es importante que los linkers should be invulnerable to host proteases, as they are often the targets for degradation. 
% 
% 
% El linker poli-G es sólo un ejemplo. 
% Como vimos en secciones anteriores, distintos elementos funcionales(que actuan principalmente mediante mecanismos de reconocimiento) pueden encontrarse en regiones con distintas propiedades conformacionales. 
% La existencia de estos elementos debe tenerse en cuenta cuando se esta usando la secuencia linker diseñada, y/o su eliminación debe formar parte de la etapa de diseño del linker.
% La longitud del \textit{loop} creado por el linker puede tener un profundo efecto sobre la actividad del linker en una proteína quimérica\cite{nagi1997inverse}.
% Además del ejemplo simple de resistencia proteolítica, las regiones linker también pueden afectar la estabilidad, solubilidad, formación de complejos.
% linker regions can affect the stability, solubility, oligomeric state, and proteolytic resistance ofthe fused protein
% Como se muestra en \cite{robinson1998optimizing}, en algunos casos se tienen efectos importantes variando la longitud y composición de la secuencia
% En base a esto, es esperable que se tengan requerimientos específicos relacionadas con la longitud y la composición. 
% % The stable linkage between functional domains provides many advantages such as a prolonged plasma half-life (e.g. albumin or Fc-fusions). 
% % However, it also has several potential drawbacks including steric hindrance between functional domains, decreased bioactivity, and altered biodistribution and metabolism of the protein moieties due to the interference between domains 
% % In other systems, however, linker regions can affect the stability, solubility, oligomeric state, and proteolytic resistance of the fused proteins
% % Thus, it is important that the length and amino acid composition of a potential linker is optimized in order to preserve the biological activity of the individual proteins in the fused complex.
% 
% 


% peptide sequences consisting of flexible and hydrophilic residues (arbitrary repeats of glycine and serine residues) are used because they are assumed to form a random coil and do not interact with (the folding of)
% the protein domains


En todos estos casos, los péptidos con motivos repetidos de Gly y Ser son usados porque se asume que adoptan una conformacion similar a random coil y no interfieren en el plegado y funcionamiento de los dominios que unen,
la funcionalidad provista por estos se evalúa en cada caso en particular.
% En \cite{evers2006quantitative} se hace una evaluación de estos linkers tan usados, en el contexto de la unión de dos proteínas fluorescentes, evaluando cuantitativamente las propiedades conformacionales y el efecto de la longitud del linker 
% a partir de la transferencia de energía entre estas.
Estos linkers poseen conformaciones desestructuradas (Gly suele considerarse como capaz de romper la estructura ordenada de las hélices), y esta flexibilidad puede, en algunos casos, impedir que se logre una separación 
suficiente entre los dominios \cite{evers2006quantitative}.
En estos casos, donde se requiere mantener la flexibilidad, pero al mismo tiempo mantener suficiente distancia entre los dominios, se han utilizado linkers que adquieren una estructura completa de $\alpha$-hélice.
El ejemplo más común de linkers con estas propiedades es $(EAAAK)_n$ \cite{arai2001design}. 
En otros casos, la rigidez se obtiene a partir de polipéptidos poli-prolina \cite{schuler2005polyproline}




% LINKERS RIGIDOS
% ref 34 = \cite{arai2001design}
% ref 35 = \cite{arai2004conformations}
% 
% An empirical rigid linker with the sequence of A(EAAAK) n A (n = 2-5) was first designed
% by Arai et al. [34, 35]. The linker displayed α-helical conformation, which was stabilized by
% the Glu − -Lys + salt bridges within segments. To test whether they could effectively separate
% the protein domains, these helical linkers were inserted between enhanced blue fluorescent
% protein (EBFP) and enhanced green fluorescent protein (EGFP), and the fluorescent
% resonance energy transfer (FRET) efficiency between EBFP and EGFP was measured [34].
% The FRET efficiency decreased as the length of helical peptides increased, indicating that
% helical linkers can control the distance between domains by changing repetitions of the
% EAAAK motif. Compared to flexible linkers with the same length, the helical linkers
% induced much less FRET efficiency when inserted into EBFP-EGFP fusion proteins,
% suggesting that helical linkers can separate functional domains more effectively.







% PONER LOS EJEMPLOS QUE USAN UNA COMBINACION DE ESTOS LINKERS RIGIDOS Y FLEXIBLES PARA UN DISEÑO DE FRET:
% el linker totalmente flexible permite todo tipo de interacciones entre los dominios, pero
% en este caso lo que se busca es que en la conformacion normal se evite lo mas que se pueda las interacciones entre dominios, ya que estas generan resonancia energetica no deseada(solo se quiere).
% primero: Antibody Detection by Using a FRET-Based Protein Conformational Switch: 
% aca lo que buscan es un linker suficientemente flexible para permitir la interaccion de las dos partes del sensor en su forma no unida al anticuerpo(teniendo asi una alta tasa de emision),
% pero que a la vez permita un 'efficient bridging'(mantenga cierta distancia) como para que, al unirse ambos extremos al anticuerpo, se pierda la interaccion disminuyendo la tasa de emision.
% primero usan un linker super flexible (17 Gly-SerGly repeat), despues prueban uno que incorpora motivos alfa-helice
% El segundo caso donde se usa este nuevo linker es en: a sensor for quantification of macromolecular crowding in living cells
% El linker que terminan usando en ambos es $A(EAAAK)_6A(GSG)_6A(EAAAK)_6A$





\paragraph{Algoritmos existentes} \hspace{0pt} \\


% DESPUES EMPIEZO CON DISEÑO RACIONAL
A pesar que algunos de los ejemplos mencionados hasta acá muestran cierta metodologia con alto grado de racionalidad en el proceso de diseño, 
la creación de un método sistemático de diseño racional para secuencias linkers basado en los requerimientos vistos está aún en los principios del desarrollo. 


% Although many examples of various types of linkers have been developed in the past, the rational design of linkers for the construction of fusion proteins is still in its infancy. 

% En algunos casos se ha 
% Existen pocos ejemplos concretos donde se haya utilizado una aproximación racional para el diseño de linkers \cite{arai2001design,arai2004conformations}.

Los estudios detallados de la composición, estructura y función de linkers naturales son un claro punto de inicio.
Estos abrieron la posibilidad de crear bases de datos conteniendo las secuencias encontradas y sus propiedades asociadas.


% Systematic, strategic scientific endeavors are in demand to greatly advance the science of linker design and application.
% Many technology platforms may be investigated in more depth towards understanding the connection between linker composition and structure, and ultimately tie them to linker function.
% The study of linker composition and structure, and the investigation of linker function should go hand in hand when designing a novel linker.
% With the rapid increase of the number of protein structures deposited in the PDB database, an updated study of natural linkers could be conducted.
% The establishment of more databases and searching programs for linkers would be another fruitful direction. 
% As discussed earlier, only two studies have been performed to analyze the characteristics of the linkers in natural multi-domain proteins.
% ESTO ESTA CASI IGUAL EN LA SECCION ANTERIOR

% The extensive studies on the structures of empirical linkers have provided us with useful information for optimal linker design. 
% Ultimately, more searching algorithms for linker databases could be developed, and provide more linker candidates for protein fusion based on user specifications.










% FINALMENTE EXPLICAR LAS APROXIMACIONES RACIONALES QUE ENCONTRE, QUE CONSISTEN BASICAMENTE EN BUSCAR EN BBDD


Los estudios realizados sobre linkers naturales que mencionados en la sección anterior) 
Asociados a estas bases de datos, se han desarrollado distintos métodos que permiten extraer  ,  que simula un mecanismo de diseño.
% The extensive studies about linkers in natural multi-domain proteins and recombinant fusion proteins fostered the idea of building databases and coming up with linker (designing??) tools 
% to aid the (rational???) design of linkers based on the desired characteristics of fusion proteins.
% Es decir, actualmente la metodologia esta centrada en crear bases de datos de linkers y hacer consultas sobre esta en base a las propiedades que se buscan.

Los resultados del estudio desarrollado en \cite{george2002analysis} son el primer ejemplo de este tipo de metodologías.
En este trabajo se estudian diversos aspectos de secuencias linkers, los cuales son extraídos mediante un método automatizado a partir de una base de datos de estructuras proteicas. 
los linkers y se desarrolla una base de datos asociada a un algoritmo de búsqueda que puede ser utilizado mediante un servidor web\cite{linkerdbIBIVU}.
El algoritmo implementado acepta distintos parámetros de búsqueda tales como longitud del linker, accesibilidad del solvente, estructura secundaria adoptada, similitud secuencial con una secuencia input, etc.
El programa devuelve las secuencias linker que contienen los criterios solicitados y, además, provee información del contexto en el que se encuentra el linker, con información del ID en PDB, descripciones de la proteína, etc.,
de forma que el usuario pueda inferir otras propiedades del linker que no pueden ser extraídas automáticamente.

% Otro ejemplo de búsqueda sobre bases de datos es \url{http://bioinf.modares.ac.ir/software/linda/}
% NO HAY NINGUNA REFERENCIA A ESTE SERVIDOR, EN NINGUN LADO SE DESCRIBE EL FUNCIONAMIENTO

Un ejemplo más reciente de este tipo de aproximación mediante bases de datos da origen a la herramienta LINKER \cite{crasto2000linker,xue2004linker}.
Al margen de la falta de creatividad en el nombre de la aplicación, esta posee un método de búsqueda que brinda una gran cantidad de opciones al usuario incluyendo aspectos experimentales como la sensibilidad a la actividad de proteasas.
En este caso, el centro del método es base de datos conteniendo secuencias loop extraidas de PDB y que son luego levemente procesadas removiento secuencias idénticas, hairpin loops y secuencias de menos de 4 residuos. 
El programa de búsqueda/diseño construido sobre esta base de datos asume que la conformación de loop adoptada por la estructura cristalizada que se encuentra en la PDB dará una conformación extendida 
si se utiliza esta secuencia como linker en una proteína quimérica
Desafortunadamente, el servidor web asociado a este programa no se encuentra más disponible.



La versión más actual de este tipo de soluciones se encuentra en SynLinker\cite{liu2015synlinker}. 
Este nuevo programa incorpora en su base de datos no sólo secuencias linker naturales, sino también algunos linkers empíricos extraídos de la literatura, 
que siguen los principios de diseño que vimos hasta ahora en esta sección. Si bien representan solo una parte de los linkers totales y principalmente se trata de simples variaciones de los modelos comunes vistos hasta acá,
estas secuencias tienen la particularidad de haber sido evaluados experimentalmente como linkers en proteínas quiméricas lo que les da ciertas propiedades particulares que los distiguen de los linkers naturales.
En la figura \ref{SynLinker} se muestran de manera gráfica todos los elementos secuenciales que se integran en la base de datos sobre la cual corre el método de búsqueda de SynLinker.

\begin{figure}[htbp,centered]
\centering
\includegraphics[width=0.8\textwidth]{img/synLink.png} 
\caption{ } 
\label{SynLinker}
\end{figure}




Esta herramienta va aún mas allá en el proceso de diseño y permite obtener un modelo computacional asociado a la construcción que estamos armando. 
Este modelo representa 1 o más linkers resultantes de la búsqueda, unidos a estructuras de dominios seleccionados por el usuario. 
Este modelo puede ser usado directamente como input en el próximo paso(siguiendo el esquema de la figura \ref{esquemaProcesoFusion}) para evaluar mediante simulaciones 
de dinámica molecular algunas propiedades conformacionales de la construccion obtenida.


En todos estos ejemplos, el proceso de diseño asume que la secuencia obtenida se comportará de la misma forma en la proteína natural que en una nueva construcción uniendo dominios completamente diferentes.
Por otro lado, la selección está siempre basa en propiedades conformacionales aunque, como vimos antes, estas no son los únicos requerimientos de las secuencias linker.
De esta forma ,a pesar que las bases de datos no proveen una solución total al problema de diseño, la construcción de estas y los métodos de búsqueda asociados 
ayuda a la utilización de los conocimientos adquiridos a partir de estudios sobre secuencias naturales y sirve como primer paso para un proceso posterior de ingeniería que debe involucrar siempre la evaluación experimental de la construcción.
El desarrollo de métodos de búsqueda más abarcativos junto con nuevos estudios para encontrar secuencias linker naturales podría generar un avance en este tipo de metodologías.
% building an empirical linker database could help summarize the knowledge and facilitate the future linker design.
% The extensive studies on the structures of empirical linkers have provided us with useful information for optimal linker design. 
% Con el rápido incremento de los conocimientos sobre la secuencia y estructuras de proteínas, nuevos estudios podrían aportar conocimientos relevantes en esta dirección.
% Ultimately, more searching algorithms for linker databases could be developed, and provide more linker candidates for protein fusion based on user specifications.
% Lo bueno de las BBDD es que los elementos que contienen suelen haber sido probados experimentalmente, lo cual es fundamental.








































% In summary, linkers can adopt various structures and exert diverse functions to fulfill the  application of fusion proteins (Table 2). 
% The flexible linkers are often rich in small or hydrophilic amino acids such as Gly or Ser to provide the structural flexibility and have  been applied to connect functional domains that favor interdomain interactions or
% movements. In cases where sufficient separation of protein domains is required, rigid linkers may be preferable. 
% By adopting α-helical structures or incorporating Pro, the rigid linkers can efficiently keep protein moieties at a distance. 
% Both flexible and rigid linkers are stable in vivo, and do not allow the separation of joined proteins. Cleavable linkers, on the other
% hand, permit the release of free functional domain in vivo via reduction or proteolytic cleavage. They can be utilized to improve the bioactivity of chimeric proteins, or to  specifically deliver prodrugs to target sites where the linkers are processed to activate bioactivity. The rational choice of linkers should be based on the properties of the linkers
% and the desired fusion proteins.

% 
% % FLEXIBLE LINKERS
% Flexible linkers are usually applied when the joined domains require a certain degree of movement or interaction. They are generally composed of small, non-polar (e.g. Gly) or polar (e.g. Ser or Thr) amino acids
% Este tipo de polipeptidos do not affect the function of the individual proteins to which they attach. 
% 
% The small size of these amino acids provides flexibility, and allows for mobility of the connecting functional domains. 
% The incorporation of Ser or Thr can maintain the stability of the linker in aqueous solutions by forming hydrogen bonds with the water molecules, and therefore reduces the unfavorable interaction between the linker and the protein moieties.
% The most commonly used flexible linkers have sequences consisting primarily of stretches of Gly and Ser residues (“GS” linker). 
% By adjusting the copy number “n”, the length of this GS linker can be optimized to achieve appropriate separation of the functional domains, or to maintain necessary inter-domain interactions.
% The loop length created by the linker can have a profound effect on the action of the linker in the fused complex
% 
% Many other flexible linkers have been designed for recombinant fusion proteins. As suggested by Argos [23], these flexible linkers are also rich in small or polar amino acids such as Gly and Ser, but can contain additional amino acids such as Thr and Ala to maintain flexibility, as
% well as polar amino acids such as Lys and Glu to improve solubility.
% 
% 
% 
% % LINKERS RIGIDOS (MOLECULAR RULERS)
% While flexible linkers have the advantage to connect the functional domains passively and
% permitting certain degree of movements, the lack of rigidity of these linkers can be a
% limitation. There are several examples in the literature where the use of flexible linkers
% resulted in poor expression yields or loss of biological activity.
% 
% The ineffectiveness of flexible linkers in these
% instances was attributed to an inefficient separation of the protein domains or insufficient
% reduction of their interference with each other. Under these situations, rigid linkers have
% been successfully applied to keep a fixed distance between the domains and to maintain their
% independent functions
% 
% The major concern in the design of a molecular ruler is the possibility of softening and structural failure that arises when the ruler is unable to provide a predictable separation distance between its bound
% moieties. An adequate cushion distance is often required when designing the linkers.
% 
% Alpha helix-forming linkers with the sequence of (EAAAK) n have been applied to the
% construction of many recombinant fusion proteins [18, 20]. As suggested by George and
% Heringa [24], many natural linkers exhibited $\alpha$-helical structures. The $\alpha$-helical structure
% was rigid and stable, with intra-segment hydrogen bonds and a closely packed backbone
% [28]. Therefore, the stiff $\alpha$-helical linkers may act as rigid spacers between protein domains.
% 
% 
% Another type of rigid linkers has a Pro-rich sequence, (XP) n , with X designating any amino
% acid, preferably Ala, Lys, or Glu. As suggested by George and Heringa [24], the presence of
% Pro in non-helical linkers can increase the stiffness, and allows for effective separation of
% the protein domains. The structure of proline-rich sequences was extensively investigated by
% several groups
% 
% Un ejemplo interesante, relacionado con la aplicacion que motivó este trabajo(FRET) se puede ver en (ref Design of the linkers which effectively separate domains of a bifunctional fusion protein - Ryoichi Arai,): 
% En este trabajo.....
% An empirical rigid linker with the sequence of A(EAAAK) n A (n = 2-5) was first designed.
% The linker displayed  $\alpha$-helical conformation, which was stabilized by
% the Glu Lys salt bridges within segments. To test whether they could effectively separate
% the protein domains, these helical linkers were inserted between enhanced blue fluorescent
% protein (EBFP) and enhanced green fluorescent protein (EGFP), and the fluorescent
% resonance energy transfer (FRET) efficiency between EBFP and EGFP was measured [34].
% The FRET efficiency decreased as the length of helical peptides increased, indicating that
% helical linkers can control the distance between domains by changing repetitions of the
% EAAAK motif. Compared to flexible linkers with the same length, the helical linkers
% induced much less FRET efficiency when inserted into EBFP-EGFP fusion proteins,
% suggesting that helical linkers can separate functional domains more effectively.
% 
% 
% % IN-VIVO CLEAVABLE LINKERS
% Under these circumstances, cleavable linkers are introduced to release free functional
% domains in vivo . The design of in vivo cleavable linker in recombinant fusion proteins is
% quite challenging. Unlike the versatility of crosslinking agents available for chemical
% conjugation methods, linkers in recombinant fusion proteins are required to be
% oligopeptides. The linkers introduced in this section take advantage of the unique in vivo
% processes, and are cleaved under specific conditions such as the presence of reducing
% reagents or proteases. This type of linker may reduce steric hindrance, improve bioactivity,
% or achieve independent actions/metabolism of individual domains of recombinant fusion
% proteins after linker cleavage
% 



