\chapter{Análisis de las secuencias}
\label{tools}

En este capítulo se describen las evaluaciones que la herramienta permite realizar sobre las secuencias para detectar características deseadas y no deseadas durante el proceso iterativo de diseño. 
Se explica en general cuál es el objetivo de aplicar cada uno de los métodos y los fundamentos en los que se basan.
Se detalla cómo se integran los distintos recursos en el contexto de nuestra aplicación, representando sus resultados en el esquema de puntajes propio de nuestro método.

El conjunto de evaluaciones que se describen corresponde a esta primera versión de nuestra herramienta. Por lo tanto no representa un conjunto exhaustivo ni definitivo, 
sino una primera etapa que deberá ser analizada y modificada de forma iterativa en función de los resultados obtenidos.
% hipotesis subyacente 

La definición del conjunto de evaluaciones utilizadas está directamente asociado a los fundamentos del método vistos en la sección \ref{fundamentos}.
Por un lado, se tiene en cuenta la hipótesis general sobre el espacio de secuencias buscadas, que nos permite asumir que este espacio es relativamente grande con respecto al conjunto total de posibles soluciones.
Por otro lado tenemos un conjunto de herramientas bioinformáticas que utilizan mecanismos y aproximaciones muy distintas y cuyos resultados pueden complementarse. 
Basándonos en estas propiedades, podemos pensar que la utilización de diferentes herramientas para la detección de las características de interés, aun cuando algunos resultados de estas puedan solaparse entre si,
es una buena práctica al momento de definir el conjunto de herramientas utilizadas. 
% Como principio general para definir este conjunto, se tuvo en cuenta la hipótesis general sobre el espacio de secuencias buscadas, vista en la sección \ref{espacioSecuencial} . 
% De esta forma, dado que el espacio de soluciones buscadas se asume como considerablemente grande con respecto al espacio total de posibles secuencias, al hacer   
% una evaluacion sobrerestrictiva con respecto a las caracteristicas buscadas no se impediría alcanzar un resultado.

Por lo tanto, un aspecto que se repite es el solapamiento entre las características detectadas por las herramientas de evaluación, pensado con el fin de hacer mas exhaustiva la detección,
además de políticas considerablemente abarcativas en cuanto a los métodos, con el fin de asegurar una mayor cobertura de las propiedades buscadas.
Esta decisión permite dar mejores diseños finales, sin tener un aumento considerable en el tiempo de búsqueda.
Como complemento de esta política, parte del trabajo realizado durante el desarrollo de la herramienta consistió en implementar un código versátil que permita modificar fácilmente el conjunto de métodos de evaluación.
% La implementación está centrada en el esquema de mutaciones(visto en el capitulo previo) y es posible añadir herramientas de forma modularizada.
% Para esto, se debe agregar en la implementación una función independiente que tome como parámetro la secuencia correspondiente a evaluar y devuelva el puntaje asociado a cada posición. 
% Cualquier herramienta que pueda ser adaptada a este esquema podrá ser agregada a la implementación.



% Como complemento a esta idea de tener un amplio conjunto de propiedades/métodos a evaluar, se provee al usuario la posibilidad de seleccionar al momento de la ejecución 
% permite parametrizar la definición de las evaluaciones realizadas(\ref{evaluacion}). 
% Es decir, en cada ejecución, el usuario puede redefinir mediante parámentros el conjunto de evaluaciones realizadas seleccionando un subconjunto del total disponible, de acuerdo a sus propios objetivos específicos.


\section{Propiedades conformacionales} \label{propiedadesConformacionales}

Uno de los objetivos fundamentales de esta herramienta consiste en la restricción de elementos estructurales ordenados en la secuencia. 
Esto brinda al diseño resultante la flexibilidad requerida para la función de linker, adoptando una conformación intrínsecamente desordenada. 

En primer lugar vamos a utilizar la herramienta IUPred, descrita en la sección \ref{iupred}, para intentar detectar aquellas posiciones que puedan generar interacciones 
favorables en el contexto de la secuencia, indicando una tendencia a adoptar estructuras plegadas, caracteristicas de las proteínas globulares.


% %ESTO LO PASO DIRECTAMENTE A LA PARTE DE TMHMM 
% Sin embargo, la estabilización de estructuras tridimensionales puede deberse a otras interacciones con el contexto, que van mas allá de las interacciones intramoleculares.
% Las proteínas de membrana son módulos proteícos con propiedades secuenciales distintivas y que, a pesar de su diferencia con las proteínas globulares, 
% pueden adoptar una estructura tridimensional determinada, estabilizada por interacciones con el contexto hidrofóbico en el que normalmente se encuentran.
% Teniendo en cuenta este panorama más amplio, creemos conveniente agregar a la evaluación un predictor para identificar proteínas de membrana,
% principalmente de los segmentos transmembrana ya que son estos los que poseen propiedades particulares y, probablemente, no hayan sido detectados por el método anterior.
% Utilizamos la herramienta TMHMM(\ref{tmhmm}) para predecir su ocurrencia, profundizando nuestra capacidad de encontrar regiones propensas a formar estructuras ordenadas.

Por otro lado, utilizamos TMHMM (ver sección \ref{tmhmm}) para predecir la ocurrencia de segmentos transmembrana, 
los cuales presentan estructuras ordenadas con propiedades particulares que permiten identificarlos como módulos proteicos independientes.
% , probablemente no hayan sido detectados por el método anterior.




% *******************************************************
% ESTA PARTE QUE SIGUE LA BORRE PARA REDUCIR AL MAXIMO LA SECCION 3.1
% *************************************************************

% 
% % 
% % QUE METODOS EXISTEN
% Muchas de las propiedades vistas en la introducción sobre IDRs/IDPs(propiedades fisicoquimicas, composición, etc) pueden trasladarse fácilmente en herramientas de predicción.
% Además, se han desarrollado aproximaciones y métodos que hacen uso de distintos conceptos y conocimientos de IDPs estudiadas experimentalmente.
% Actualmente existen una gran cantidad de metodos diversos para predecir desorden estructural\cite{he2009predicting}, entre los que se incluyen
% % *** AGREGAR REFERENCIAS
% GlobBplot \cite{linding2003globplot}, PONDR, FoldIndex, DisEMBL, DISOPRED y DISOPRED2, IUPred, 
% FoldUnfold, RONN, DISpro, DisPSSMP y DisPSSMP2, Spritz , PrDOS , etc.
% 
% Para evitar detallar todos los métodos(ya que son muchos) usaremos la clasificación desarrollada en \cite{habchi2014introducing}, donde se analizan distintas 
% aproximaciones y se esboza una clasificación, concluyendo que el problema de la detección de IDPs  se puede atacar desde tres direcciones distintas:
% \begin{enumerate}
% 
% \item Los primeros predictores se basaron en propiedades de la composición y propiedades fisicoquímicas de la secuencia como la relación carga/hidrofobicidad. 
% Estas características(detalladas en la introducción ) se obtuvieron haciendo análisis estadísticos sobre conjuntos reducidos de IDPs/IDRs descubiertas inicialmente.
% % Como resultado de estos estudios se encontro que the ID proteins differ dramatically from the ordered proteins in their amino acid sequences. 
% 
% \item Distintas aproximaciones utilizan algoritmos de tipo \textit{machine-learning}. Estos y otros algoritmos de aprendizaje automático son ejecutado primero sobre datos de entrenamiento 
% (en este caso disintos conjuntos de secuencias que, se sabe, adquieren conformaciones no-plegadas) para intentar extraer un patrón propio del conjunto de datos.
% El algoritmo permite aplicar luego este patrón encontrado para predecir las mismas propiedades sobre un conjunto de datos nuevos. 
% La calidad del predictor resultante es muy variable ya que dependera del método aplicado, la representatividad del conjunto de entrenamiento usado y la complejidad del patrón a predecir.
% En particular, el número de IDPs determinadas experimentalmente es todavia pequeño y esto afecta considerablemente la calidad de los predictores desarrllados.
% Además se debe tener en cuenta que ciertas regiones pueden igualmente plegarse mediante procesos de binding and folding, por lo que la calidad de los datos de entrenamiento no está asegurada.
% 
% 
% \item Una gran cantidad de métodos siguen una aproximación común en el análisis de propiedades conformacionales que utiliza conocimientos sobre la propensión a formar interacciones entre cada par de AAs.
% La idea subyacente es que las IDPs no están plegadas porque no tienen la posibilidad de tener suficientes contactos entre sus residuos para superar la pérdida de entropía que ocurre durante el plegamiento.
% Es decir, en este tipo de conformaciones lo que se espera es que las interacciones posibles entre residuos no sean tan favorables.
% Esta hipótesis es aplicada de distinta forma para dar diversos métodos.
% A diferencia de los métodos mencionados en el punto anterior, los parámetros usados para hacer las evaluaciones son extraídos de datos conocidos sobre proteínas plegadas y/o IDPs ya que las 
% tendencias de interacción son las mismas para cualquier par de residuos en un mismo contexto.
% El conjunto de datos desde donde se extraen los conocimientos previos es mucho más númeroso(porque incluye bases de datos de estructuras ordenadas) y, por lo tanto, se espera que la predicción sea más acertada y menos sesgada.
% Se debe tener en cuenta que, como se vió en secciones previas, muchas IDRs pueden plegarse temporalmente en presencia de ciertos ligandos, por lo tanto cuando se extraen propiedades de bases de bases de datos de estructuras ordenadas 
% es posible que algunos datos tampoco sean totalmente válidos.   
% \end{enumerate}
% 
% 
% % It should be stressed that it is difficult and maybe impractical to establish the “best” predictor at the moment. 
% % Some predictors perform better on short disordered regions (i.e., DISOPRED2 and PreLink), 
% % while other predictors (IUPred for instance) perform well in predicting long disordered segments, and finally some predictors, such as PONDR, GlobPlot, and FoldIndex, 
% % have been trained on both short and long disorder and provide a balanced performance. 
% % Therefore, to avoid pitfalls, different predictors should be combined, as performed by metapredictors that seek a consensus of the scores of different predictors relying on different
% % principles (PONDR-FIT for instance).
% % To handle limitations inherent in prediction accuracy due to distinct flavors of disorder, however, different predictors are recently combined into metapredictors, such as metaPrDOS [9] or PONDR-FIT[10]. 
% % These combined predictors do show improved performance over their composite ones.
% 









% PEGAR EL TEMA DE AGREGACION DICIENDO COMO QUE ES OTRA FORMA DE INTERACCIONES QUE NO SON INTRAMOLECULARES PERO DAN UNA CONFORMACION TRIDIMENSIONAL NO-FLEXIBLE
% Otra forma de alcanzar una estructura tridimensional , como ya se describió, 
Los agregados proteicos son otro tipo de módulos identificables que involucran la formación de una estructura tridimensional estable, en este caso mediante interacciones con otras unidades proteicas.
% Una estructura tridimensional rígida puede ser alcanzada por interacciones con otras proteínas, como se vió en la formación de agregados proteicos. 
Estas conformaciones agregadas pueden o no tener una estructura tridimensional ordenada pero, en todos los casos, limitan la flexibilidad requerida para una secuencia linker, generando, además,
módulos proteicos insolubles que afectan a todo el sistema biológico.
% Por lo tanto, a continuación nos centraremos en evitar la formación de agregados a partir del análisis secuencial.
Debido a estas implicancias altamente negativas se evaluará detalladamente la tendencia a formar agregados dentro de la secuencia que estamos diseñando.
 

Existe actualmente una gran variedad de métodos existentes para predecir agregación amorfa y formación de fibras amiloides \cite{hamodrakas2011protein,redler2014computational,agrawal2011aggregation}.
Cada método hace sus propias hipótesis e implementa predictores independientes, los cuales varían desde análisis muy simples (por ejemplo, análisis de la composición de aminoácidos) a métodos específicos más complejos, siendo 
la capacidad para formar hojas $\beta$ una característica central de las evaluaciones, ya que es un denominador común de la formación de agregados.

En primer lugar, en nuestra herramienta, evaluaremos la tendencia a formar agregados utilizando TANGO (sección \ref{tango}). 
Para evaluar específicamente la formación de fibrillas amiloides utilizaremos Waltz (descrito en la sección \ref{waltz}) y PASTA (sección \ref{pasta}).
Por último, evaluamos la presencia de determinantes secuenciales que pueden indicar la formación de fibrillas amiloides, descrito en la sección \ref{determinantesSecuenciales}.



















% En el capítulo 1 nos centramos en la diferenciación entre proteínas plegadas y desordenadas, dado que esta clasificación era suficiente para introducir las conceptos de flexibilidad y desorden.
% A pesar que ya hemos mencionado las ideas de dominios/proteínas globulares, no detallamos la clasificación completa que contiene estos plegamientos, así es como, en general,
% las proteínas pueden clasificarse como globulares, de membrana, fibrosas o desordenadas, de acuerdo a las propiedades del plegamiento(o la falta de éste).


% *************************************************************
% FUNDAMENTOS DE LOS PREDICTORES DE SECUENCIAS TRANSMEMBRANA 
%    	PUEDE IR EN LA PARTE DE TMHMM   PERO LO SACO MOMENTANEAMENTE  ***  
% El conjunto de proteínas conocidas como de membrana posee una topología particular compuesta por segmentos que se encuentran insertados dentro del medio lipídico de la membrana y segmentos extra- e intra- celulares.
% Claramente, su conformación y composición está altamente marcada por estos dos tipos de segmentos ya que las interacciones que ocurren con el medio son totalmente distintas. 
% Estas propiedades hacen que se puedan diferenciar lo suficiente de otros tipos de proteínas plegadas como las globulares, que se mantienen normalmente solubles en el medio acuoso de la célula. 
% De hecho, los métodos mencionados para predecir conformaciones desordenadas están pensados en general como métodos para diferenciar entre éstas conformaciones desordenadas y las estructuras de proteínas globulares ya que,
% por ejemplo, cuando se evalúan las interacciones con el medio se asume que la proteína se encuentra en el medio acuoso, o cuando se utilizan datos de proteínas se filtran primero las proteínas de membrana(o los segmentos transmembrana) 
% para que sus propiedades secuenciales no provoquen una desviación de los datos. 
% Por lo tanto, para varios predictores no se espera que puedan diferenciar claramente el plegamiento que adquieren regiones transmembrana de una conformación con desorden intrínseco.


% La particularidad que presentan esta clase de proteínas es que las hélices transmembrana son considerablemente mas fáciles de predecir que las hélices en dominios globulares.
% La rason de esta mejora en la precisión se debe a que la mayoria de las helices transmembrana estan codificadas por una secuencia inusualmente larga de residuos hidrofóbicos, 
% que le permiten mantenerse estables en el nucleo de las membranas lipídicas.
% The hydrophobic signal is so strong that a straightforward approach of calculating a propensity scale for residues in
% transmembrane helices and applying a sliding window with a cutoff already performs quite well.


















% **********************************************************************
%    ESTE TEXTO QUE SIGUE LO BORRO COMPLETAMENTE, ANTES ESTABA PARA EXPLICAR la PERDIDAS DE FLEXIBILIDAD POR INTERACCIONES, FOLDING-AND-BINDING, CHAPERONAS, AGREGADOS..
%   LA PARTE DE CHAPERONAS Y MoREs  LA PASE A LA SECCION 3.2
% ***********************************************************************

% Hasta acá, intentamos asegurarnos que la secuencia no pueda adquirir una estructura ordenada de forma estable en su forma aislada.
% % Es decir, la perdidad de flexiblidad ocurria por interacciones intramoleculares.
% Sabemos, sin embargo, que la situación en el entorno celular es claramente diferente a esta condición ideal de aislamiento, 
% cualquier interacción con otros elementos que implique una reducción del ensamble conformacional, o lo que es lo mismo, 
% un aumento en las restricciones conformacionales, afectará directamente a la flexibilidad. 
% Para que la flexibilidad se mantenga en el contexto celular, intentaremos predecir perdidas de flexibilidad por interacciones con ligandos(otras proteínas iguales o distintas, DNA, membranas, etc).
% % y/u otras proteínas(iguales o distintas).
% 
% 
% % INTERACCIONES QUE ESTABILIZAN ESTRUCTURAS ORDENADAS --------- binding and folding
% En la sección \ref{proteinLandscape} se vió que, para las IDRs/IDPs, si bien su secuencia no tiene las caracteristicas necesarias para plegarse en una estructura ordenada, puede ocurrir que, ante la interacción con un ligando, 
% segmentos con estructuras ordenadas transitivas se estabilicen u otros segmentos experimenten un proceso de plegamiento, dando regiones con estructura ordenada estable como parte del complejo de interacción. 
% Si bien estos segmentos son muy importantes para la función de reconocimiento, en esta sección nos enfocamos en quitarlas por motivos estructurales, es decir, porque la adquisición de una estructura secundaria estable
% generaría una pérdida de flexibilidad en el linker.
% En nuestra evaluación utilizaremos ANCHOR\ref{anchor}, que busca identificar segmentos dentro de IDRs/IDPs que puedan adquirir una conformación rigida en estados de unión a ligandos.
% 
% % QUE OTRA HERRAMIENTAS EXISTEN!! !!! *****************
% % METODO DE DETECCION DE MoREs
% % Although the information content of short sites is very limited, due to their enrichment in hydrophobic residues, indirect techniques have
% % some success in delineating them. As mentioned, for example, the HCA plot 43 (see section 3.2) can indeed unveil such binding
% % sites. 25 The PONDR VL-XT, 23 ANCHOR (http://anchor.enzim.hu), 207 and DynaMine 208 disorder predictors are sensitive
% % to local tendency of ordering, and are thus informative in highlighting potential induced folding regions.
% 
% 
% 
% 
% 
% 
% 
% 

% 
% 
% % ACA CONECTO CON LA FORMACION DE AGREGADOS
% Existen chaperonas cuya función implica unirse a proteínas que no estén totalmente(correctamente) plegadas para evitar la agregación de estas. 
% El objetivo de esto es intervenir en situaciones ``anormales'', como por ejemplo estress térmico, donde la pérdida de la estructura nativa puede llevar a uniones entre distintas proteínas dando un agregado no funcional.
% Si bien uno de los objetivos de este trabajo es lograr una tendencia reducida a la agregación en la secuencia generada y 
% ciertas chaperonas podrían facilitar esto, la unión implica, como se menciono anteriormente, una pérdida de flexibilidad que es necesaria para la función de linker. 
% Por lo tanto, a continuación nos centraremos en evitar la formación de agregados a partir del análisis secuencial.
% % De esta forma buscamos evitar la agregacion mediante analisis sobre la secuencia que tienen este objetivo particular.
% 
% 
% 
% 
% % ***************************
% % ***********  AGREGACION !! 
% % POR QUE QUEREMOS/DEBEMOS PREVENIR LA FORMACION DE AGREGADOS
% % La formación de agregados es una forma especial de interacciones que involucra una gran cantidad de proteínas.  
% En primer lugar, la formación de agregados que involucren a las secuencias linker implicaría una clara reducción en la flexibilidad debido a la interacción con otras proteínas.
% Sin embargo, la interacción entre las secuencias linker y su consecuente pérdida de flexibilidad sería solo el principio del problema.
% Dadas las propiedades de los agregados, principalmente de la formación de fibras amiloides(estructura que se asume genérica de los polipeptidicos), 
% la formación de esta estructura abarcará a la secuencia completa de la proteína.
% En nuestro caso, la formación de fibrillas amiloides dentro de la secuencias linkers ``arrastraria'' a los dominios que únen a formar parte del agregado, interfiriendo directamente en la funcionalidad de estos. 
% La funcionalidad de cualquier dominio proteico se podria ver limitada si esta se encuentra formando algún tipo de agregado.
% En la sección \ref{agregados} se detallan los métodos usados para evaluar la tendencia a formar agregados sobe las secuencias.
% 























\subsection{IUPred: Análisis de tendencia al desorden} \label{iupred}

La herramienta IUPred \cite{dosztanyi2005pairwise} permite diferenciar aquellas secuencias que pueden formar suficientes interacciones favorables entre sus residuos, y por lo tanto podrían plegarse en una estructura globular, 
de aquellas no tienen esta capacidad y, por si solas, permanecen en una conformación desordenada.
Esta diferencia nos permite, dentro de nuestro método, identificar las posiciones de la secuencia evaluada que tienen una mayor tendencia a formar conformaciones globulares, 
una propiedad que describimos como no deseada para secuencias linker.

% La herramienta IUPred \cite{dosztanyi2005pairwise} permite diferenciar secuencias con capacidad para formar plegamientos globulares de aquellas que no y, por lo tanto, están destinadas a permanecer en una conformación desordenada.
% En nuestro método de evaluación, utilizamos esta herramienta para identificar aquellas posiciones que tienen una tendencia a formar conformaciones globulares en el contexto de la secuencia evaluada,
% una propiedad que describimos como no deseada para secuencias linker.



% ***********************
% DESCRIPCION DEL METODO
% ***********************
El primer paso en el desarrollo de IUPred es definir un modelo simple para calcular la energía total de una estructura nativa de una proteína, a partir de los contactos que se encuentran en esta:

% FORMULA 1
\begin{equation}\label{modelo1}
E = \sum_{ij=1}^{20} M_{ij}C_{ij}
\end{equation}

\noindent donde $M_{ij}$ es el potencial de interacción entre dos residuos de tipos $i,j$ (sólo depende del tipo de residuos), y $C_{ij}$ es el número de este par de residuos que se encuentran en contacto en la estructura.
% de contactos entre residuos de tipos $i,j$ encontrados en la estructura.

Sin embargo, el método IUPred busca poder evaluar la contribución energética de cada aminoácido usando, únicamente, la composición de aminoácidos como parámetro. 
% Sin embargo, obviamente, se requiere conocer una estructura a partir de la cual extraer los pares que interaccionan y poder así evaluar el aporte energético de éstos.
% El centro de IUPred es poder evaluar la contribución energética promedio por AA en una secuencia usando como parámetro solamente su composición.
Esta evaluación tendrá la forma: 
% PONER FORMULA 2
\begin{equation}\label{modelo2}
\frac{E_{estimada}}{L} = \sum_{ij=1}^{20} n_{i}P_{ij}n_{j}
\end{equation}

\noindent donde $n_i$ y $n_j$ representan la frecuencia de residuos de tipo $i$ y $j$, respectivamente, en la secuencia.
P es la matriz de predicción de energia, que indica cómo la energía de un residuo de tipo $i$ depende de la existencia de residuos de tipo $j$ en el contexto.
% is the energy predictor matrix, which tells how the energy of amino acid i depends on the jth element of the amino acid composition vector.


% la idea es que la existencia de contactos(interacciones) favorables es resultado de las potenciales parejas de interacción en la secuencia.
% La idea es, entonces, obtener un potencial de interacción estadístico para cada par de AAs(Pij), el cual es totalmente independiente de la posicion que ocupan estos en la secuencia.
% Este valor representa la energia de interaccion entre cada ocurrencia del par ij en cualquier proteína.
% Es decir, para cada posible aminoacido i , saber como depende la contribucion energetica con respecto a la presencia de un aminoacido de tipo j en la secuencia.
% Conociendo este valor promedio, usando la ecuacion anterior(que solo depende de la composición) se podría obtener el valor de E por residuo (E/L).
% Lo que se requiere ahora es poder derivar la matriz P.

De esta forma, sin conocer la estructura de la proteína, basamos el cálculo de la energía en valores estadísticos (matriz P) que pueden ser extraídos de una base de datos de proteínas globulares.
Para calcular P, primero se desglosa la energía total de cada proteína en las contribuciones que hace cada tipo de aminoácido.
Esto puede hacerse reusando el modelo de la ecuación \ref{modelo1}, ya que conocemos las estructuras. 
La energía total por tipo de residuo se obtiene como:
% ecuacion de ek
\begin{equation}\label{ek1}
  e_i(evaluada)= \sum_{j=1}^{20} M_{ij}C_{ij}
\end{equation}
% Este valor depende de los contactos(interacciones) que hacen todos los aminoacidos de tipo i dentro de esa proteína. 
% En algunas estructuras, un par ij de aminoacidos estaran formando contactos , en otro no , en otro solo un par, etc...

Usando una aproximación similar a la de la ecuación \ref{modelo2}, esta energía por tipo de residuo se puede estimar mediante:
% la contribución de cada tipo de aminoacido a la energia de una proteína específica se puede estimar como:
% ecuacion ek estimado
\begin{equation}\label{ek2}
e_i(estimada) = N_i\sum_{j=1}^{20} P_{ij}n_{j}
\end{equation}
\noindent donde $N_i$ representa el número total de residuos de tipo $i$ en la secuencia ($L*n_i$).

% Utilizando las ecuaciones \ref{ek1} y \ref{ek2} tenemos la base para estimar cada valor $P_{ij}$ de la matriz a partir de los valores $M_{ij}$ y una base de datos de estructuras globulares conocidas.
Cada valor $P_{ij}$ de la matriz puede estimarse minimizando la diferencia entre los $e_i(evaluada)$ y los $e_i(estimada)$ para todas las estructuras de una base de datos de estructuras globulares.
% Dadas las propiedades de la ecuacion \ref{ek2}, la minimizacion se hace para cada fila de la matriz por separado.
La función a minimizar es, entonces: 
% funcion de Zi
\begin{equation}\label{z}
Z_i = \sum_{k} (e_i^k - N_i^k\sum_{j=1}^{20} P_{ij}n_{j}^k)^2   
\end{equation}
\noindent donde $k$ indica el índice de la estructura en la base de datos. 
% Es decir, se minimiza esta diferencia cuadrática para los valores de $e_i$ sobre todas las estructuras.


% Se tienen ahora los valores obtenidos de $P_{ij}$ los cuales son, luego, evaluados mediante la comparación de los cálculos de energía para un set de estructuras usando la ecuación \ref{modelo1} y la ecuación \ref{modelo2}, 
% obteniéndose una correlación que indica, de acuerdo a los análisis estadísticos aplicados, un nivel razonable de concordancia entre ambos cálculos.

Una vez obtenidos los valores de la matriz $P$, tenemos un modelo completo que nos permite estimar la energía miníma de la estructura asociada a una secuencia, sin asumir ninguna conformación (\ref{modelo2}).
Este modelo primero se puso a prueba sobre dos conjuntos de proteínas globulares e IDPs obteniéndose que la energía estimada para el conjunto de IDPs son menos favorables que las correspondientes al set de proteínas globulares,
lo que está de acuerdo con la hipótesis que indica que las proteínas globulares tienen secuencias específicas con potencial para formar un gran número de interacciones favorables, mientras que las IDPs no.
Utilizando esta separación significativa, lo que resta es transformar esta aproximación en un método para predecir el desorden a partir de la secuencia, 
para lo cual se transforma el valor de energía en un valor de probabilidad (\textit{score} resultante, $s_k$).



% *********************************************
% % SACO ESTA ULTIMA PARTE  EXPLICA COMO SE ADAPTA P PARA TENER EN CUENTA SOLO EL CONTEXTO CERCANO DE CADA POSICION
% *********************************************
% Sin embargo, el modelo se debe adaptar ya que es mas realista si se consideran solo la composición de la secuencia mas próxima a la posición que se está analizando, de manera que se puedan analizar por regiones desordenadas/ordenadas.
% Para esto se recalculan los valores de la matriz $P_{ij}$ pero cada posición se trata de forma separada teniendo en cuenta, solamente, la secuencia del contexto
% (sólo se tienen en cuenta posibles interacciones con residuos a distancias entre 2 y 100 posiciones alrededor).

% % SACO ESTA ULTIMA ECUACION, ME PARECE QUE QUEDA BASTANTE CLARO EN EL TEXTO
% La energía de a pares asociada a cada posición $k$ de una secuencia se calcula ahora según la ecuación \ref{modelofinal}.
% 
% \begin{equation}\label{modelofinal}
% E_i^k = \sum_{j=1}^{20} P_{ij}f_{j}^k(w_o) 
% \end{equation}
% 
% donde $f_{j}^k(w_o)$ es la fracción de residuos de tipo $j$ en el entorno de la posición $k$

% % ESTO LO PUSE ANTES, PORQUE SAQUE LA ULTIMA ECUACION
% El resultado final de la predicción consiste en la transformación del valor de energía en un valor de probabilidad(\textit{score} resultante, $s_k$).








% *************************
% UTILIZACION DEL METODO 
% ***********************
La herramienta para el cálculo del score a partir de una secuencia está disponible a través de un servidor web \cite{iupredWeb,dosztanyi2005iupred} o descargando la implementación y ejecutándola localmente \cite{iupredDownload}.   
En nuestro caso utilizaremos la segunda opción. 
Tal como se menciona en la información del servidor \cite{dosztanyi2005iupred}, los residuos que tengan un valor asociado de \textit{score} mayor a 0.5 pueden ser tomados como desordenados.
Por lo tanto, en nuestro método, los residuos que posean un \textit{score} resultante menor a 0.5 tendrán un valor de 1 en el puntaje asociado a la posición.
Por ejemplo, la evaluación de la secuencia \texttt{VLKQTKGVGASGSFR} con IUPred devuelve los valores de score que se ven en la figura \ref{iupredResults}
% PONER RESULTADOS DE IUPred EN GRAFICO O TABLA

\begin{figure}[h!]
% {\linewidth}
\centering
\includegraphics[width=0.4\textwidth]{img/iupredTabla.png} 
\caption{}
\label{iupredResults}
\end{figure}

Utilizando nuestro esquema de evaluación, estos valores resultan en los siguientes puntajes:

\vspace{0.5cm}
\begin{adjustbox}{width=\textwidth}
\begin{tabular}{lllllllllllllllll} 
\hline
Secuencia & \textbf{V} & \textbf{L} & \textbf{K} & \textbf{Q} & \textbf{T} & \textbf{K} & \textbf{G} & \textbf{V} & \textbf{G} & \textbf{A} & \textbf{S} & \textbf{G} & \textbf{S} & \textbf{F} & \textbf{R} \\ \hline
Evaluación con IUPred & 1 & 1 & 1 & 1 & 1 & 1 & 1 & 1 & 1 & 1 & 1 & 1 & 0 & 0 & 0\\ \hline
\end{tabular}
\end{adjustbox}







% 
% 
% 
% 
% 
% % RESUMEN SACADO DEL REVIEW
% % 
% % Dosztanyi et al. suggested that a large number of interresidue interactions is responsible for structure stabilization of proteins [50, 51]. In contrast, IDPs don’t have sufficient numbers of stabilizing inter-residue interac-
% % tions. Based on this reasoning, an IUPred algorithm estimating the inter-residue interactions was designed. First, the interaction energy between each pair of amino acids
% % based on their C β positions was estimated. This was done by calculating the potential mutual contact energies for
% % all amino acid pairs in a dataset of globular proteins with known structure. This is a fairly standard approach in
% % computational biology, and in this work Dosztanyi et al. compared several such mutual contact energies estimated
% % previously by other researchers, with the set developed by Thomas and Dill, found to be the best in this particu-
% % lar application [92]. The various pairwise energies were assembled into a 20x20 energy matrix, which was used
% % in the next step, the estimation of the mutual interaction energies for any given protein. The prediction utilizes
% % this energy prediction matrix and the amino acid compositions put into a quadratic expression. These statistical values represent the ability to form stabilization contacts
% % between amino acids in polypeptide chains. The potential mutual interactions were estimated using amino acid
% % compositions, not three-dimensional structures. These composition-based energies were compared with three-
% % dimensional structure-based energies of the proteins for which the actual side chain interactions are known. The
% % composition-based potential mutual interaction energies and the structure-based energies were found to be highly
% % correlated, thus the former can be used to estimate the latter even when the structures are not known. To use
% % this approach to predict structure or disorder, composition-based calculations for a set of proteins that fold
% % into three-dimensional structures were compared with composition-based calculations for a set of disordered
% % proteins. The estimated potential interaction energies for the structured proteins were much greater than the same
% % energies for the unstructured proteins, and from these results the energy boundary between ordered and disor-
% % dered proteins as a function of length was determined. This boundary allows the recognition of intrinsic disor-
% % der. In brief, if a sequence contains too few hydrophobic residues, then the composition-based potential mutual in-
% % teraction energy will necessarily be too small and thereby indicate the lack of potential for folding
% 



% Partiendo del concepto general que la conformación nativa está determinada por la estructura primaria de la proteína, y que esta conformación se corresponde con el mínimo global del
% espacio conformacional, es posible parametrizar un modelo que permita predecir este mínimo de energía sin asumir ninguna conformación estructural. (Ref. 2)
% Esta aproximación es posible ya que la contribución energética de un residuo depende, no solo del tipo de aminoácido, sino también de los potenciales parejas de interacción en la secuencia. 
% El aporte de un residuo será más favorable si la secuencia en la que se encuentra contiene más residuos que pueden formar interacciones favorables con este. 
% La forma de plantear este modelo es mediante una expresión cuadrática sobre la composición de aminoácidos de la secuencia:
% Los valores de n representan las frecuencias de aminoácidos i y j en la secuencia.
% El valor de P es el parámetro a estimar, el cual se deriva a partir del modelo mencionado previamente, que evalúa la energía a partir de las interacciones que ocurren en la estructura. 
% El ajuste se realiza minimizando la diferencia entre ambas ecuaciones.
% De esta forma, mediante un ajuste de mínimos cuadrados se puede parametrizar el modelo a partir de datos de estructuras pertenecientes a proteínas globulares.
% Dado que las proteínas globulares forman un gran número de interacciones entre los residuos (lo que les provee la energía estabilizante para superar la pérdida de entropía), 
% y las proteínas IU/desordenadas tienen secuencias especiales que no poseen esta capacidad de formación de interacciones, la estimación del potencial de interacción permite diferenciar entre 
% regiones de proteínas ordenadas y desordenadas. Esto transforma el modelo de predicción en un eficiente método para diferenciar secciones desordenadas de secciones con estructura definida.
% 
% Como se mencionó previamente, la parametrización de este modelo se realiza a partir de estructuras contenidas en una base de datos de proteínas globulares, 
% por lo que son datos suficientes para realizar una buena parametrización, además de ser consistentes y curadas. Esto diferencia el método de otros, que se basan en adaptar 
% un modelo a datos de estructuras correspondientes a proteínas intrínsecamente desordenadas, agrupados en bases de datos chicas, con datos obtenidos usando diversas técnicas 
% y con distintos significados del término desordenado.
%   

%   

  
  
  
  

  
  
  
  
  
\subsection{TMHMM: Secuencias transmembrana} \label{tmhmm}

% QUE QUEREMOS DETECTAR
% La estabilización de estructuras tridimensionales puede deberse a interacciones con el contexto, que van mas allá de las interacciones intramoleculares.
Las proteínas de membrana son módulos proteícos con propiedades secuenciales distintivas y que, a pesar de su diferencia con las proteínas globulares, 
pueden adoptar una estructura tridimensional determinada, estabilizada por interacciones con el contexto hidrofóbico en el que normalmente se encuentran.
Teniendo en cuenta este panorama más amplio, utilizamos el predictor TMHMM \cite{krogh2001predicting} para identificar, dentro de la secuencia que estamos evaluando, segmentos transmembrana ya que probablemente no hayan sido detectados por otros métodos de nuestra evaluación.

% EL MÉTODO
TMHMM es un método para detección de segmentos transmembrana que utiliza una aproximación mediante modelos ocultos de Markov (Hidden Markov Models, o HMM).
Un HMM representa un modelo de Markov con estados no visibles. Es decir, el sistema es descrito por un modelo estocástico representado por estados y transiciones entre estos, las cuales están determinadas únicamente por el estado actual.

Describiendo el modelo de Markov que represente a la arquitectura de una proteína transmembrana se puede conocer si una secuencia desconocida pertenece a esta clase, evaluando si se adapta a este modelo.
Para poder describir una proteína mediante un HMM se debe definir un conjunto de estados, cada uno correspondiente a una región o sitio específico de la proteína que se está modelando.
Cada estado tendrá un valor de probabilidad asociado a cada uno de los 20 aminoácidos posibles. Además, se deben definir las transiciones posibles(y las probabilidades asociadas) entre los estados según la arquitecura que se está describiendo, por ejemplo si a partir 
de un estado pueden ocurrir nuevas instancias de este o si debe pasarse a un nuevo estado determinado.
La distribución de probabilidades de los aminoácidos en cada estado y la probabilidad de transición se derivan a partir de frecuencias observadas en un conjunto de proteínas conocidas que se adaptan al modelo

El modelo utilizado para describir la arquitectura de las proteínas en el método TMHMM (proteínas transmembrana) es cíclico y está compuesto por 7 estados distintos que representan: el núcleo de la hélice transmembrana, 
los dos extremos de ésta, el loop que se encuentra en el lado citoplasmático, dos loops en el lado no-citoplasmático, y un dominio globular en el medio de estos.
En la figura \ref{tmhmmModel} se ve la descripción gráfica de este.
En \cite{sonnhammer1998hidden} se describe como se procede al entrenamiento del HMM para poder obtener todos los parámetros asociados.


\begin{figure}[h!,centered]
\centering
\includegraphics[width=0.66\textwidth]{img/tmhmmModel.png} 
\caption{\textbf{A:} esquema general de la arquitectura asociada al modelo. En \textbf{B} y \textbf{C} se muestran en detalle algunos de los estados y cómo las
transiciones entre estos permite definir los limites en las longitudes de las distintas regiones y sus propiedades. Figura extraida de \cite{sonnhammer1998hidden}}
\label{tmhmmModel}
\end{figure}


Para evaluar secuencias se utiliza el algoritmo de Viterbi, el cual permite calcular cual es el camino más probable del modelo que adopta la secuencia a evaluar. Es decir, cual es la secuencia mas probable de estados del HMM 
(entre todas las posibles), que produce la secuencia de estados observados (aminoácidos de la secuencia).
El resultado de la ejecución muestra cómo los residuos de la secuencia se ajustan a los diferentes estados del modelo de acuerdo a este recorrido más probable, clasificándolos según pertenezcan a regiones intracelulares, hélices transmembrana, 
o regiones extracelulares. 
% MODO DE UTILIZACION
Este método puede ser ejecutado como servicio web o descargado como paquete de software en \cite{tmhmmServer}.
En nuestro caso realizamos la ejecución de manera local.

Nuestro interés está en detectar las regiones que forman los segmentos transmembrana, por lo tanto, el puntaje resultante será igual a 1 para aquellas posiciones que pertenezcan a estos segmentos y 0 para el resto. 
Por ejemplo, la ejecución de TMHMM a partir de la secuencia \texttt{NFVLIGSFVAFFVITYFLE} devuelve los siguientes resultados: 

% \noindent
\texttt{inside         1-1}\\
\indent \texttt{TMhelix   2-18}\\
\indent \texttt{outside	    19-19}

Por lo tanto, el puntaje resultante de la evaluación es:

\vspace{0.5cm}
\noindent
\begin{adjustbox}{width=\textwidth}
\begin{tabular}{lllllllllllllllllllll} 
\hline 
Secuencia & \textbf{N} & \textbf{F} & \textbf{V} & \textbf{L} & \textbf{I} & \textbf{G} & \textbf{S} & \textbf{F} & \textbf{V} & \textbf{A} & \textbf{F} & \textbf{F} & \textbf{V} & \textbf{I} & \textbf{T} & \textbf{Y} & \textbf{F} & \textbf{L} & \textbf{E}  \\ \hline
Puntaje TMHMM & 0 & 1 & 1 & 1 & 1 & 1 & 1 & 1 & 1 & 1 & 1 & 1 & 1 & 1 & 1 & 1 & 1 & 1 & 0\\ \hline
\end{tabular}
\end{adjustbox}



 
 
 
 



% ******************************************************************************** 
% ******************************************************************************** 
% 	FORMACION DE AGREGADOS, BORRO COMPLETAMENTE LA INTRO A  ESTA SECCIÓN ...
%	QUEDA TODO DESCRITO EN CADA METODO INDIVIDUAL USADO (TANGO, PASTA,.. )
% ******************************************************************************** 
% ******************************************************************************** 
% 
% 


% 
% \subsection{Formación de agregados}
% \label{agregados}
% 
% 
% % En el capítulo 1 vimos las propiedades más importantes de los agregados formados por proteínas.
% 
% 
% La formación de agregados amorfos y de fibrillas amiloides son procesos diferentes, sin embargo comparten ciertas características similares ya que ambos están asociados con el enriquecimiento de estructuras de hojas-$\beta$ 
% en la forma agregada, por lo que en gran medida pueden analizarse de forma conjunta.
% 
% Como se vió, la estructura de las fibrillas amiloides es altamente estructurada y por lo tanto las preferencias de aminoácidos serán mucho más específicas en relación a la posición comparado con la formación de agregados amorfos.
% La formación de estos agregados amorfos es mucho menos dependiente de los residuos en cada posición y, por lo tanto, su tendencia puede predecirse si se evaluan los parámetros biofísicos generales sobre la comppsición, sin necesidad de 
% evaluar propiedades específicas de secuencia.
% % y, en principio, cualquier secuencia que adopte una conformación extendida, no tenga grupos  y sea suficientemente hidrofóbica 
% % puede agregarse para dar esta conformación
% % Amorphous $\beta$-sheet aggregation, however, is less position-dependent and can, in principle, be achieved by any sequence that can adopt an extended conformation, 
% % is sufficiently hydrophobic and has no unsatisfied hydrogens or electostatic groups. 
% % POR QUE NO USAMOS UN PREDICTOR ESPECIFICO??
% 
% 
% % METODS DISPONIBLES PARA HACER LA DETECCION
% En \cite{hamodrakas2011protein,redler2014computational,agrawal2011aggregation} se describen y evaluan una gran cantidad de software/metodos existentes para predecir agregacion desestructurada y formación de fibras amiloides, 
% Cada método hace sus propias hipótesis e implementa predictores independientes, los cuales varían desde análisis muy simples (por ej. análisis de la composición secuencial) a métodos específicos más complejos.
% La capacidad para formar hojas-$beta$ es una característica central de las evaluaciones ya que es un denominador común de la formación de agregados.
% 
% 
% 
% 
% En primer lugar evaluaremos la tendencia a formar agregados utilizando TANGO(ver sección \ref{tango}). 
% El método utilizado por TANGO, centrado en calcular la función de partición para distintos estados conformacionales, nos permite también evaluar la tendencia a adoptar estructuras nativas como hélices-$\alpha$ y $\beta$-turn,
% además de la tendencia a la formación de agregados. 
% % Este método se basa en el uso de principios fisicoquímicos que gobiernan la formacion de hojas-$\beta$
% % extended by the assumption that the core regions of an aggregate are fully buried
% % , de manera que el método no es específico para la formación de agregados amorfos o amiloides.
% % Like all algorithms that use averaged physicochemical properties to detect aggregation hot spots, TANGO is not specific for amyloid formation or amorphous beta-aggregation. 
% 
% 
% 
% % En cuanto a los agregados amiloides específicamente, 
% En la introduccion se plantearon algunas propiedades fisicoquimicas simples con respecto a la composición de secuencias con tendencia a formar estas estructuras agregados amiloides específicamente.
% En particular, las características mas importantes son la alta hidrofobicidad y baja carga neta, además de tendencia intrinseca a adoptar conformaciones de hoja-$\beta$ en su estructura primaria.
% Se cree que las regiones propensas a formar agregados(APRs como se vió previamente) son segmentos con estas características que normalmente se encuentran ocultos en el núcleo hidrofóbico de la estructura plegada pero
% bajo condiciones de reestructuración del plegamiento o cuando se adoptan estados \textit{misfolded}, estos segmentos pueden quedar expuestos al solvente e iniciar la formación de agregados.
% 
% % \cite: The triple power of D-3: Protein intrinsic disorder in degenerative diseases
% Estos conocimientos forman la base para el desarrollo de una gran variedad de aproximaciones bioinformáticas para predecir tendencias a formar agregados a partir de la secuencia primaria.
% Se debe tener en cuenta, a la hora de analizar los resultados obtenidos con esto métodos, que la existencia de APRs en la secuencia es una condición necesaria pero no suficiente para la formación de agregados
% 
% 
% Algunos métodos intentan distinguir los agregados amorfos de la formación de fibrillas amiloides. 
% Sin embargo, dado que la cantidad de secuencias que generan estructuras amiloides y se han validado experimentalmente es relativamente baja, los algoritmos basados puramente en secuencia no poseen 
% suficiente información como para distinguir los distintos tipos de agregados.
% % They generally also have rather poor predictive capabilities toward amyloid sequences from yeast prions and functional amyloids. 
% Por otro lado, los métodos que se basan en conceptos estructurales obtenidos por homología pueden, en principio, proveer predicciones más específicas aunque también están limitados por la cantidad de estructuras determinadas disponibles.
% 
% Para evaluar específicamente la formación de fibrillas amiloides utilizaremos, en primer lugar, Waltz (descrito en la sección \ref{waltz}),
% el cual combina información secuencial, parámetros fisicoquímicos y aspectos estructurales que, como se dijo, pueden no funcionar muy bien de forma aislada pero permitir una predicción eficiente si se combinan.
% 
% 
% % PASTA
% PASTA es otra de las herramienta que utilizamos para evaluar la tendencia a formación de agregados amiloides. 
% PASTA obtiene potenciales estadisticos para la formación de estructuras de hojas-plegadas-$\beta$ a partir de proteínas globulares y los utiliza para
% predecir cuales son los segmentos de una secuencia que le proveen a la proteína la capacidad de estabilizar la estructura supramolecular cross-$\beta$ característica de las fibrillas amiloides.
% % predicts which interacting portions of a given protein are stabilizing the cross-beta structure by using an energy function. 
% % r las energias asociadas a distintos emparejamientos entre secuencias formando una estructura tipo fibrilla amiloide. 
% 
% Por último, en el capítulo 1 se dejó planteada la idea sobre la existencia de subsecuencias que podrían promover y guiar la agregación para dar estructuras amiloides.
% En el trabajo realizado en \cite{de2004sequence} se analiza esta idea y en \ref{determinantesSecuenciales} explicamos los resultados obtenidos y como los utilizamos para la detección 
% de la tendencia a formar este tipo de agregados dentro de nuestro método.
% 




























\subsection{Tango}\label{tango}


TANGO \cite{fernandez2004prediction} es un método utilizado principalmente para predecir regiones de agregación $\beta$, desarrollado a partir de un modelo de mecánica estadística que define 
un espacio de fases incluyendo, además de estos agregados, conformaciones de $\beta$-turn, hélices-$\alpha$ y hebras-$\beta$.
En nuestro sistema de evaluación, deseamos evitar tanto las conformaciones agregadas como cualquier otra estructura ordenada, por lo tanto utilizaremos TANGO para detectar las posiciones que tienen tendencia a encontrarse formando
alguna de estas conformaciones.
% el estado nativo de la proteína??.

De acuerdo con el modelo definido por TANGO, cada segmento de un péptido/proteína puede encontrarse en alguno de estos estados de acuerdo a una distribución de Boltzmann. 
Es decir, la frecuencia con la que el segmento se encuentra en alguno de los posibles estados, es relativa a la energía asociada a este, la cual se deriva de consideraciones empíricas y estadísticas. 
% Para cada conformación posible se realizan consideraciones y evaluaciones energéticas con respecto al estado desplegado.
De esta forma, el método consiste de un algoritmo que simplemente calcula la función de partición y en base a eso predice las regiones con tendencia a agregación $\beta$

Puntualmente, para el cálculo de la energía asociada a la conformación de hélice-$\alpha$, se tienen en cuenta los parámetros definidos previamente en el desarrollo del método AGADIR \cite{lacroix1998elucidating}. Dentro de estos parámetros, principalmente se tiene en cuenta la interacción entre cadenas laterales de los pares de residuos i,i+3 e i,i+4.
%tienen en cuenta varios componentes que estabilizan la estructura de hélice-$\alpha$, principalmente 
% AGREGAR QUE POSICIONES SE TIENEN EN CUENTA PARA ESTA EVALUACION 

Para el cálculo de la energía asociada a las conformaciones $\beta$, sólo se tiene en cuenta la energía de interacción entre las cadenas laterales de los residuos i,i+1 e i,i+2.
Esta energía se deriva a partir de una base de datos de estructuras, calculando la relación entre frecuencias observadas y esperadas para cada par de aminoácidos que se encuentra en este tipo de conformación. 
% MARCAR QUE POSICIONES SE TIENEN EN CUENTA PARA ESTA EVALUACIÓN
%Side chain-side chain interactions have been analyzed in stretches of two
%(i,i+1 interactions) or three (i,i+2 interactions) consecutive residues with β-strand phi and psi dihedral angles 7 .
% (identificados en conformación $\beta$ de acuerdo a sus ángulos phi y psi).
% Para evaluar esta energía entre cualquier par de aminoácidos en conformación de $\beta$-strands, se calcula un potencial estadístico a partir de la relación 
% entre la frecuencia observada y la frecuencia esperada para los pares encontrados dentro de una base de datos(identificados en conformación $\beta$ de acuerdo a sus ángulos phi y psi).
%En el cPara calcular la energía de interacción entre cadenas laterales (\Delta G_{interacc-cadenas}), 
Asumiendo que la base de datos representa un sistema termodinámico en equilibrio, las frecuencias se relacionan con la energía de interacción de acuerdo a la siguiente ecuación:

\begin{equation}
\large
 \Delta G_{interacc-cadenas}=-RTln(\frac{f_{observada}}{f_{esperada}})
\end{equation}

% ESTA ACLARACION NO SE SI ES NECESARIA, CONFIRMAR SI ESTA ESCRITA CORRECTAMENTE
% \noindent donde $\Delta G_{interacc-cadenas}$ representa la diferencia en energía libre entre la conformación desplegada del polipéptido y la interacción formando la conformación definida.

Para el caso en que la conformación sea de $\beta$-turn, dado que los residuos no están fijos sino que pueden adoptar diferentes conformaciones, se agrega una penalización por el aumento de entropía.
% Este valor empírico es de 0.3 Kcal/mol
Por otro lado, en el caso de agregados $\beta$ se asume que los residuos que adoptan esta conformación se encuentran completamente insertos en el núcleo y pagan el costo energético correspondiente. 
En este caso se calculan los diferentes parámetros energéticos, de hidrofobicidad, de solvatación, las interacciones electrostáticas y las interacciones por formación de puentes de hidrógeno.

A partir de las energías definidas para estas cuatro conformaciones, se calcula la función de partición resultante que permite definir las tendencias de cada residuo a poblar los distintos estados conformacionales evaluados.
La salida que muestra TANGO se compone de un archivo con las siguientes columnas:
Número de posición, residuo en esa posición, porcentaje en conformación hebra-$\beta$, porcentaje en conformación $\beta$-turn, porcentaje en conformación hélice-$\alpha$, porcentaje de agregados $\beta$ y
porcentaje en agregación de hélices-$\alpha$.
Esta última columna no forma parte del modelo inicial que describimos y se calcula aparte, no representando un resultado confiable.
% , por lo tanto la suma total puede dar mayor a 1.

El método puede ser utilizado a través del servidor web en \cite{tangoWeb}. En nuestro caso solicitamos a los creadores especialmente una versión ejecutable para poder usarla de manera local.
Una vez ejecutados y obtenidos los resultados para una secuencia, es necesario definir un valor de corte que determine la significancia de la tendencia obtenida.
En \cite{fernandez2004prediction} se definen dos intervalos de confianza para la predicción de regiones de agregación: segmentos con residuos que posean más de 5\% de tendencia agregación $\beta$ (resultando en una alta certeza de predicción),
y segmentos con residuos entre 0.2\% y 5\%. En nuestro caso utilizamos 1\% como valor de corte en esta primera implementación de la herramienta.
% Si bien el objetivo principal de TANGO, y de la aplicación de este en nuestra herramienta, es buscar segmentos con potencial de agregación, estamos interesados en eliminar cualquier región estructurada y, dado que TANGO incorpora como parte
% de la función de partición a distintos estados conformacionales nativos, el puntaje reflejará también todas estas tendencias en la secuencia. 

Dado que TANGO fue desarrollado con el objetivo de evaluar tendencias a formar agregados, no se evaluaron las predicciones asociadas a otros estados conformacionales. 
Para utilizar estas evaluaciones en nuestra implementación utilizaremos el mismo valor de corte que se obtuvo para la predicción de agregación $\beta$. 
Por lo tanto, el puntaje será igual a 1 cuando el residuo tenga una tendencia superior al valor de corte para la formación de agregados, hélices-$\alpha$, hebras-$\beta$ o $\beta$-turns, 
y un puntaje igual a 0 en caso contrario.
Por ejemplo, si evaluamos usando TANGO la secuencia \texttt{AMAPVLYLQDKSS}, obtenemos la siguiente tabla de valores:

\vspace{0.3cm}
\begin{center}
\begin{tabular}{ccccccc}
Pos & Residuo & Beta & Turn & 	Hélice & Beta Aggregation & Helical Aggregation\\
01 &          A & 0.2 & 0.0 &  0.000  & 0.000 &  0.000\\
02 &          M  &     0.2 &    0.0 &  0.000 &  0.000 &  0.000\\
03 &          A &      0.2  &      0.0 &  0.000 &  0.000  & 0.000\\
04  &         P &      0.0   &     0.0 &  0.000 &  0.544  & 0.544\\
05  &         V &      1.6   &     0.0 &  0.000 &  10.620 & 10.620\\
06   &        L &      2.3   &     0.0 &  0.000 &  10.620 & 10.620\\
07      &     Y &      3.1   &     0.0 &  0.000 &  10.620 & 10.620\\
08        &   L &      1.7   &     0.2 &  0.000 &  10.620 & 10.620\\
09         &  Q &      1.0   &     0.2 &  0.000 &  10.203 & 10.203\\
10   &        D &      0.3    &    1.8 &  0.000 &  0.000 &  0.000\\
11  &         K &      0.3   &     1.8 &  0.000 &  0.000 &  0.000\\
12  &         S &      0.2   &     1.6 &  0.000 &  0.000 &  0.000\\
13  &         S  &     0.2  &      1.6 &  0.000 &  0.000 &  0.000\\
\end{tabular}
\end{center}

\vspace{0.5cm}
El segmento en las posiciones 5-9 supera el porcentaje de agregación que utilizamos como valor de corte, mientras que el segmento 10-13 supera este valor en la tendencia a encontrarse en conformaciones de $\beta$-turn.
Por lo tanto, el puntaje resultante de la evaluación es:

\vspace{0.5cm}
% \noindent
\begin{center}
\begin{tabular}{llllllllllllll} 
\hline      		
Secuencia & \textbf{A} & \textbf{M} & \textbf{A} & \textbf{P} & \textbf{V} & \textbf{L} & \textbf{Y} & \textbf{L} & \textbf{Q} & \textbf{D} & \textbf{K} & \textbf{S} & \textbf{S} \\ \hline
Evaluación con TANGO & 0 & 0 & 0 & 0 & 1 & 1 & 1 & 1 & 1 & 1 & 1 & 1 & 1 \\ \hline
\end{tabular}
\end{center}





% 
% ****la inclusion del estado plegado dentro de la funcion de particion tiene como objetivo ayudar a predecir efectos de agregacion debido a mutaciones puntuales en proteínas que naturalmente adquiren un estado plegado. 
% La inclusion de este estado permite ver la competencia entre este estado natural de plegado y otros estados estructurales incluidos en la particion. 
% De esta forma se puede predecir la tendencia a la agregacion del estado desnaturalizado y tambien las mutaciones que aumentan la tendencia a la agregacion de la proteína desestabilizando el estado plegado.








\subsection{PASTA}\label{pasta}

En nuestra herramienta, utilizamos PASTA \cite{trovato2006insight} para identificar cuales son las regiones dentro de la secuencia que estamos evaluando, que podrían estabilizar una estructura de fibrillas amiloides.
% PASTA \cite{trovato2006insight} es una herramienta que permite, principalmente, predecir las regiones de un polipéptido que pueden estabilizar la estructura de las fibrillas amiloides.
Para realizar esta predicción, el método realiza un cálculo de las energías de interacción entre los distintos segmentos, asumiendo que el mecanismo de las interaccion entre aminoácidos que lleva a la formación 
de láminas-$\beta$ en proteínas globulares es el mismo que lleva a la formación de apilamientos de hebras-$\beta$ en fibrillas amiloides.
% (estructura asociada a fibras amiloides).

% En primer lugar, Pasta deriva una funcion energetica a partir de un conjunto de datos de estructuras globulares (potenciales estadisticos).
La evaluación de la energía de interacción se realiza a partir de un potencial estadístico derivado de una base de datos de proteínas globulares.
% (selección de proteínas no-redundantes y con estructuras resuletas con alta definición)
Para esto, se dividen las instancias encontradas de cada par $a$-$b$ de residuos en 4 categorías, según estén interaccionando formando una hoja plegada-$\beta$ en forma paralela ($n_{ab}^p$) o antiparalela ($n_{ab}^a$), 
o si no están participando de una estructura $\beta$ y sus carbonos-$\alpha$ están a menos de $6.5$\AA~($n_{ab}^c$ contactos genéricos), o a más de $6.5$\AA~($n_{ab}^d$ , pares desordenados sin contacto). 
A partir de las frecuencias obtenidas se pueden derivar valores de energía asociados a las interacciones de a pares para los distintos estados, asumiendo que la base de datos analizada es un sistema en equilibrio termodinámico 
a temperatura constante para todas las proteínas.
De esta forma, la probabilidad de cada par $a$-$b$ de encontrarse en un estado $x$ se relaciona con el valor de la energía mediante el factor de Boltzman $p_{ab}(x)= e^{-E_{ab}^x}$. 

Si podemos obtener una aproximación para $p_{ab}(x)$ se puede despejar el valor del potencial de interacción estadístico asociado ($E_{ab}^x$) para cada estado $x$,
el cual corresponde a la diferencia en energía entre el estado $x$ y el estado que se toma como referencia.
Definiendo la probabilidad $p_{ab}(x)$ como la relación entre la frecuencia de interacciones observada para cada par y la esperada en el estado de referencia, 
y aproximando esta última como la frecuencia observada para todos los pares, se obtiene:
% Utilizando los valores definidos previamente, se despeja el 
% Despejando el valor de $E_{ab}^x$ para cada estado $x$, el cual corresponde a la diferencia en energía entre el estado $x$ y el estado que se toma como referencia, se obtiene:
% El método asume que la forma soluble(estado de referencia) es nativamente desestructurada, por lo que hay que tener cuidado cuando se lo usa para predecir proteínas que nativamente adquieren una estructura globular.
% El valor de energía resultante para cada estado $x$ queda definido por:

\begin{equation}
{E_{ab}^x = -log\left(\dfrac{\dfrac{n_{ab}^x}{n_{ab}}} {\dfrac{\sum\limits_{ab} n_{ab}^x}{\sum\limits_{ab} n_{ab}}}\right)}
\end{equation}


% Para derivar este valor, lo que hace es ver cual es 
% la probabilidad de encontrar cierto par de residuos enfrentados en hebras vecinas dentro de una lámina-$\beta$. 
% Se extraen valores de potencial según estén interaccionando en sentido paralelo o antiparalelo.
% El valor de este potencial estadistico para cada clase(paralela,antiparalela, etc) se calcula según la relación entre la frecuencia observada y la frecuencia esperada:
% La frecuencia esperada se aproxima como: el número de pares ab (para cualquiera ab) que está en una clase dada / el número de pares ab (para cualquier ab) que hay.
% la frecuencia observada es el numero de pares ab que estan en esa clase / el total de pares ab encontrados.
% los cuales pueden ser usados 
% para calcular el valor energético para cualquier emparejamiento de dos secuencias con la misma longitud(sumando los scores de cada par que interaccione) 
% El cálculo de los valores de $E_{ab}$ para los estados paralelos y antiparalelos, resulta en un par de matrices con un valor asociado(score) para cada par $a$-$b$ de residuos posibles.

% Usando los valores de $E_{ab}$ resultantes es posible asignar una energía total a cada emparejamiento especifico entre subsecuencias de la misma longitud, simplemente sumando los valores para correspondientes a cada par involucrado. 
% El método permite evaluar emparejamientos entre subsecuencias correspondientes a una misma o proteína o a distintas. 
Para predecir los segmentos que pueden estar involucrados en la agregación de una proteína
se prueban todos los emparejamientos posibles entre subsecuencias de ésta, en sentido paralelo y antiparalelo, y para cada uno 
se calcula la energía asociada, resultante de sumar todos los potenciales de interacción de a pares ($E_{ab}$) involucrados.
% El método devuelve los valores predecidos en unidades PEU (PASTA Energy Units), donde 1 PEU es equivalente a aprox. 1,192 Kcal/mol 
% The server predicts aggregation in energy units where 1 PASTA Energy Unit (PEU) is equivalent to 2 KBT at room temperature, that is 1.192 Kcal/mol (see Supplementary Material). 

% En nuestra evaluación intentamos conocer cuales son los residuos que pueden formar estructuras de fibra amiloide con una energia asociada
% suficientemente baja, es decir, conformaciones suficientemente estables como para que esta estructura se adopte realmente en un \% considerable.
% Para lograr esto debemos obviamente definir un valor de \textit{cut-off} ya que el termino ``suficientemente bajo'' es algo totalmente subjetivo. 
% % El valor de cutoff se debera elegir de forma tal que se balancee la sensibilidad y la especificidad en la deteccion.
% Para balancear la sensibilidad y la especificidad el método permite, entonces, modificar dos paráemtros en el análisis. 
% En primer lugar el valor de \textit{cut-off} que se utilizará y en segundo lugar el número de emparejamientos que se tendrán en cuenta dentro de ese \textit{cut-off}.

Sobre los valores de energía resultantes de estas sumas se aplica un punto de corte, siguiendo las recomendaciones analizadas en \cite{walsh2014pasta}.
% Si bien la herramienta permite definir estos valores libremente, para aplicarla nos basamos en las recomendaciones analizadas en \cite{walsh2014pasta}.
Puntualmente, utilizamos el esquema de valores que se describe como más específico, con un valor de \textit{cut-off}$=-5$ y teniendo en cuenta 
sólo el emparejamiento de menor energía aun cuando existan otros con valores menores al \textit{cut-off}.
% *****ESTAS LINEAS QUE SIGUEN LAS PUEDO SACAR SI HACE FALTA RECORTAR
Sólo usamos el primero de los emparejamientos porque, al implementar un esquema iterativo, si los emparejamientos que se encuentran por debajo  del valor de \textit{cut-off} persisten, se incrementan o se modifican,
igualmente los tendremos en cuenta a todos en las próximas iteraciones. 
% Otra forma de implementarlo sería ``uniendo'' todos los segmentos que entran en el rango de \textit{cut-off}.
% , de forma que en cada iteración se apunte a mutar dentro de un conjunto mas amplio de segmentos. 
% El \textit{cut-off} utilizado puede ser muy específico pero es sólo el punto de inicio para la incorporación de PASTA en nuestra herramienta y es uno de los parámetros que quizás sea mejor optimizar, 
% buscando un valor acorde probablemente más cercano a $-2.8$ que es el próximo valor analizado en \cite{walsh2014pasta}, para tener una mayor sensibilidad. 



El método PASTA es ejecutado de forma local mediante un script escrito en lenguaje Perl, obtenido directamente de sus desarrolladores, junto con las matrices de energías.
Si los resultados indican que algún segmento de la secuencia tiene una tendencia considerable a la formación de agregados, 
simplemente se asigna un puntaje de 1 a cada posición de este.
La mayoría de las veces los emparejamientos de menor energía se corresponden con emparejamientos en forma paralela e \textit{in-register} (PIRA), es decir, involucran el mismo segmento en dos moléculas que interactúan. En otros casos, el emparejamiento corresponde con dos regiones distintas de la misma molécula. En ese caso PATENA asigna el puntaje 1 a las posiciones de ambas regiones. 

Para ejemplificar el proceso de evaluación, se muestra el resultado del evaluar la secuencia \texttt{VTNVGGAVVTGVTAV}.
La ejecución de PASTA sobre esta secuencia devuelve: \\
\noindent
\begin{adjustbox}{width=\textwidth}
\texttt{pairing 0  PASTA energy -5.735704  length 13  between segments 1-13 and 1-13  parallel}
\end{adjustbox}

Este resultado indica que existe un emparejamiento de forma paralela, con un score de $-5.735704$, entre dos segmentos correspondientes a la subsecuencia \texttt{VTNVGGAVVTGVT}. 
El puntaje resultante, por lo tanto, es:

% \vspace{0.5cm}
% \noindent
\begin{center}
\begin{adjustbox}{width=\textwidth}
\begin{tabular}{lllllllllllllllll} 
\hline    
Secuencia & \textbf{V} & \textbf{T} & \textbf{N} & \textbf{V} & \textbf{G} & \textbf{G} & \textbf{A} & \textbf{V} & \textbf{V} &\textbf{T} & \textbf{G} & \textbf{V} & \textbf{T} & \textbf{A} & \textbf{V} \\ \hline
Evaluación con PASTA & 1 & 1 & 1 & 1 & 1 & 1 & 1 & 1 & 1 & 1 & 1 & 1 & 1 & 0 & 0 \\ \hline
\end{tabular}
\end{adjustbox}
\end{center}


















\subsection{Waltz}\label{waltz}

Utilizamos Waltz \cite{maurer2010exploring} dentro de las evaluaciones de nuestra herramienta para predecir la tendencia de la secuencia a formar agregados amiloides.
% Específicamente, el método Waltz combina información secuencial, fisicoquímica y estructural para generar una matriz específica de posición (PSSM),
% en función de la cual se evalúan las secuencias.
% para predecir la tendencia a formar agregados amiloides.
% we explored the sequence diversity of amyloid hexa-
% peptides by inspecting more than 200 peptides using various
% structural and biophysical methods,
% and used the derived data to build Waltz, a
% web-based tool that uses a position-specific scoring matrix to
% determine amyloid-forming sequences. Waltz allows users to
% identify and better distinguish between amyloid sequences
% and amorphous beta-sheet aggregates
Específicamente, el método Waltz determina la tendencia a formar agregados amiloides en base a 
% El puntaje que determina la tendencia para cada posición se calcula, entonces, en base a 
tres contribuciones: un componente derivado de información secuencial, un componente derivado de un conjunto de 19 propiedades fisicoquímicas, 
y un componente resultante de evaluar distintos residuos sobre un modelo estructural de la cadena carbonada de fibras amiloides.

El primer componente se obtiene a partir del análisis secuencial de una bases de datos de hexapéptidos generadores de fibrillas amiloides, los cuales han sido evaluados experimentalmente (AmylHex).
Esta base de datos y el estudio realizado están basados en hexapéptidos porque la mayoría de las secuencias amiloides disponibles hasta el momento tienen esta longitud.
De esta forma, se ha tomado como una longitud representativa del núcleo que genera la formación de amiloides, asumiendo que
la inserción de estos segmentos es suficiente para inducir la conversión de todo el dominio de la proteína hacia una estructura agregada. 

La base de datos utilizada, sin embargo, posee una alta redundancia secuencial ya que tiene una gran cantidad de entradas correspondientes a mutaciones puntuales del péptido \texttt{STVIIE}.
Para reducir esto, el conjunto inicial de la base de datos se utiliza para obtener una PSSM inicial e identificar nuevas instancias de hexapéptidos, los cuales fueron evaluados experimentalmente para el desarrollo de la herramienta.
A partir de este conjunto expandido de datos se realiza un alineamiento y se usa para generar una nueva PSSM, utilizando el método de puntuación por log-probabilidades (log-odd score).
Este es el primer componente de la función de scoring de Waltz.

El segundo componente se deriva de analizar un conjunto de propiedades físicas sobre el conjunto de secuencias alineadas.
Estas 19 propiedades se incluyen en la función como un score que se deriva de sumar, para cada posición y cada aminoácido, el producto entre la frecuencia de ese residuo
y el valor normalizado de la cada propiedad. De esta forma, se obtiene un perfil de las propiedades físicas para cada posición.


El último componente de la función de scoring es una PSSM derivada del modelado estructural. 
Para obtener esta matriz se utiliza el campo de fuerzas FoldX \cite{schymkowitz2005foldx}, y la estructura tridimensional del péptido \texttt{GNNQQNY} en una fibrilla amiloide (conocido como formador de agregados amiloides, proveniente de Sup35).
En primer lugar se mutan todas las posiciones a Alanina calculando el valor de $\Delta$G para obtener finalmente el hexapéptido poli-A.
Sobre este se comienzan a realizar todas las combinaciones posibles de mutaciones utilizando los 20 aminoácidos naturales, calculando ahora el valor de $\Delta$G con respecto a este hexapéptido poli-A de referencia y 
el $\Delta\Delta$G con respecto al péptido original. 
El valor del score específico para cada posición y cada aminoácido se obtiene de promediar los valores de energía fijando este aminoácido a la posición y combinando todos los demás aminoácidos posibles en el resto de las posiciones.


La función de scoring final resulta de la combinación lineal de los tres scores detallados:

{
\large
% \centering
\begin{equation}
S_{total}= \alpha S_{secuencial} + \beta S_{propiedadesFisicas} + \gamma S_{estructural}
\end{equation}
}
% DE DONDE SALEN LOS TERMINOS INDEPENDIENTES??


% Aplicando esta función, es posible obtener los valores de score asociados a un hexapéptido(la PSSM se deriva del análisis posicional de cada una de las 6 posiciones de un hexapéptido).
El resultado de aplicar esta función puede obtenerse mediante un servidor web \cite{waltzWeb}, el cual permite detectar todas las posiciones de una secuencia que forman parte de algún polipéptido y que supera cierto
valor de \textit{cut-off}. En el mismo servidor se definen dos opciones estándar para este valor de corte: alta especificidad, con un \textit{cut-off} de 97, o alta sensibilidad, con un valor de \textit{cut-off} de 79.

Para nuestro sistema de evaluaciones obtuvimos, de los desarrolladores del método, una versión de la PSSM correspondiente junto con un script en lenguaje Perl que permite evaluar el score de cada posición.
% detectar las posiciones que superen el \textit{cut-off} indicado.
Ejecutando el script sobre la secuencia en cada evaluación, usando un valor de \textit{cut-off} = 79 (alta sensibilidad), asignamos un puntaje igual a 1 en todas las posiciones que superen este valor.
Por ejemplo, ejecutando Waltz sobre la secuencia \texttt{VTNVGGAVVTGVT}, se obtiene que el segmento 6-13 forma parte de hexapéptidos con score mayor al \textit{cut-off}. Por lo tanto, el puntaje de la evaluación será:


\vspace{0.3cm}
\noindent
\begin{center}
\begin{tabular}{lllllllllllllll} 
\hline    
Secuencia & \textbf{V} & \textbf{T} & \textbf{N} & \textbf{V} & \textbf{G} & \textbf{G} & \textbf{A} & \textbf{V} & \textbf{V} &\textbf{T} & \textbf{G} & \textbf{V} & \textbf{T}  \\ \hline
Evaluación con Waltz & 0 & 0 & 0 & 0 & 0 & 1 & 1 & 1 & 1 & 1 & 1 & 1 & 1 \\ \hline
\end{tabular}
\end{center}







\subsection{Determinantes secuenciales de fibras amiloides}\label{determinantesSecuenciales}

En esta etapa de la evaluación utilizaremos un patrón secuencial extraído a partir de evaluaciones experimentales \cite{de2004sequence} 
con el fin de encontrar determinantes secuenciales para la formación de fibrillas amiloides sobre la secuencia linker que estamos evaluando.

% Como parte de tal trabajo se deriva un patrón secuencial a partir de los resultados cuali/cuantitativos experimentales, el cual que permitiría identificar tramos de secuencias formadores de fibrillas amiloides.
El trabajo realizado para obtener el patrón secuencial que determina la formación de fibrillas amiloides se basa en un experimento de mutagénesis a partir de un péptido formador de fibrillas amiloides diseñado \textit{de novo} en \cite{de2002novo}. 
En este proceso se reemplazan
% El ensayo desarrollado para obtener el patrón consiste en reemplazar 
sistemáticamente los residuos del péptido diseñado (STVIIE) por todos 
los aminoácidos naturales, excepto Cisteína el cual no fue utilizado bajo la suposición que, por su similitud con Serina, las restricciones secuenciales para ambas serían similares.
% Este estudio sigue la linea de utilizar péptidos cortos para investigar elementos de la secuencia que favorecen la agregación, 
% basándose en la idea que este proceso es conducido por fragmentos cortos de proteínas mal(o parcialmente) plegadas.

El trabajo implica, luego, la evaluación experimental de los péptidos resultantes, para lo cual se monitorea la polimerización de hojas-$\beta$ utilizando la técnica de dicroísmo circular (CD) y la detección de las fibrillas formadas mediante 
microcopía electrónica. Un análisis descriptivo se provee en \cite{de2004sequence}, encontrándose una dependencia posicional con la formación de este tipo de estructuras de agregación, existiendo tanto posiciones muy tolarantes como 
restrictivas a las mutaciones.
% Como parte de tal trabajo se deriva un patrón secuencial a partir de los resultados cuali/cuantitativos experimentales, el cual que permitiría identificar tramos de secuencias formadores de fibrillas amiloides.
Dado que existen diferencias en los resultados experimentales según el estado de ionización de algunos residuos, se obtienen dos patrones distintos según el pH en el que se encuentran los péptidos.
Los patrones resultantes son:
\vspace{0.2cm}

\begin{adjustbox}{width=\textwidth}
\noindent \textbf{A pH ácido:} $\{P\}_1 -\{PKRHW\}_2 -[VLS(C)WFNQE]_3 -[ILTYWFNE]_4 -[FIY]_5- \{PKRH\}_6 $
\end{adjustbox}

\begin{adjustbox}{width=\textwidth}
\noindent \textbf{A pH neutro:}  $\{P\}_1 -\{PKRHW\}_2 -[VLS(C)WFNQ]_3 -[ILTYWFN]_4 -[FIY]_5- \{PKRH\}_6 $\\
\end{adjustbox}
\textit{\noindent Donde los \{\} indican residuos ``prohibidos'' en esa posición, y los} [ ] \textit{representan aquellos que son ``aceptados'', con respecto a la formación de amiloides.
Los subíndices indican las posiciones correspondientes en el hexapéptido.}


\vspace{0.2cm}
% 
% DISCUSION DEL PATRON ENCONTRADO
% In the light of the knowledge acquired in this study, we believe that, in same way that it has been shown for globular proteins, there are general rules governing the amyloidogenicity of a polypeptide chain.
% Because a good resolution structural model of the fibrils formed by these peptides is still lacking, any discussion about the thermodynamic origin of the effect of mutation on amyloid formation would be difficult and rather speculative. Therefore, we
% have provided only a descriptive analysis of the sequential dependence found.

% EL ESTUDIO EXPERIMENTAL SE BASA EN MEDICIONES A DISTINTOS TIEMPOS(ENTRE T=0 TIEMPO=T=1mes). EN BASE A ESTO SE OBTIENE CUALES SON LAS MUTACIONES QUE MAS ACELERAN LA FORMACION DE ESTE TIPO DE FIBRAS, Y TAMBIEN CUALES MUTACIONES 
% SON LAS QUE DAN LA MAYOR FORMACION DE FIBRAS FINAL (A TIEMPO=t), LO QUE SE ENCUENTRA ES QUE ESTOS 2 EFECTOS NO SIEMPRE COINCIDEN EN LA MISMA MUTACION
% At the most tolerant positions (1, 2, and 6), one can find many substitutions that accelerate beta-sheet polymerization dramatically
% Interestingly, the more restrictive the position is, the less the number of amino acid that are capable of accelerating the process. At position 5, no substitution accelerates the process at all.
% Amino acid replacements producing abundant amyloid products are not always the substitutions that allow for a faster beta-sheet polymerization
% *************


% Para validar experimentalmente el patrón, se seleccionarion algunas combinaciones de mutaciones que, de forma individual, daban altas capacidades de formación de fibras amiloides. 
% VER COMO SE HACE LA EVALUACION EXPERIMENTAL 

% Este patrón fue también validado \textit{in silico}. En este análisis, se encontró que las secuencias de una base de datos que matcheaban el patron eran menos frecuentes en proteínas que las combinaciones innocuas y que, en caso de encontrarse, estaban rodeadas de aminoácidos que 
% rompen esta capacidad de agregación(conocidos como amyloid breakers).


% *************

% En nuestra aplicación, implementamos la búsqueda de este patrón como parte del conjunto de evaluaciones que se hacen sobre la secuencia.
El patrón secuencial resultante no será capaz de detectar, por si sólo, todos los motivos asociados con la generación de fibrillas amiloides, ya que es el resultado de un análisis sobre un espacio muy reducido de secuencias.
En nuestro caso, sin embargo, es un elemento útil ya que es aplicado en conjunto con una serie de recursos adicionales dentro de un análisis secuencial exhaustivo.

Utilizaremos el patrón correspondiente a pH ácido por ser el más general de los dos.
Para implementar la evaluación de la secuencia usando este patrón, se buscan instancias de la expresión regular asociada a éste dentro de la secuencia..
% El resultado de buscar la expresión regular asociada a este patron secuencial podría aportar subsecuencias que generen una tendencia a formar este tipo de fibras amiloides.
Para buscarlas se utiliza el módulo re de Python que permite, justamente, buscar instancias de una expresión regular sobre una secuencia. 
Todas las posiciones que pertenecen a instancias de este patrón tendrán un puntaje igual a 1.
Por ejemplo, al evaluar la secuencia \texttt{HPALFTIWHP} se encuentra una instancia del patrón buscado en la subsecuencia \texttt{ALFTIW}, por lo tanto el puntaje correspondiente es:

\vspace{0.3cm}
\noindent
\begin{center}
\begin{adjustbox}{width=\textwidth}
\begin{tabular}{llllllllllll} 
\hline    
Secuencia & \textbf{H} & \textbf{P} & \textbf{A} & \textbf{L} & \textbf{F} & \textbf{T} & \textbf{I} & \textbf{W} & \textbf{H} &\textbf{P}  \\ \hline
Evaluación en busca de patrón secuencial & 0 & 0 & 1 & 1 & 1 & 1 & 1 & 1 & 0 & 0 \\ \hline
\end{tabular}
\end{adjustbox}
\end{center}





































































%     REESCRIBIR/RESUMIR ESTA SECCION, AGREGANDO ANCHOR Y LIMBO(QUE LOS SAQUE DE LA PARTE DE PROPIEDADES CONFORMACIONALES)

\section{Elementos biológicamente funcionales}\label{evaluacionFuncional}

% POR QUE QUEREMOS SACAR TODAS LAS PROPIEDADES FUNCIONALES
% ¿QUE DESEAMOS PARA EL LINKER? ¿POR QUE?

% El objetivo de este trabajo es lograr una herramienta que provea una secuencia capaz ser utilizada experimentalmente como linker en el proceso de ingeniería de proteínas. 
% Esto implica no sólo probar con cierta certeza que tendrá la funcionalidad deseada, sino también evaluar el comportamiento en todos los pasos del proceso de ingeniería. 
% propiedades de la secuencia no afectarán este proceso de ninguna forma.
% De forma general, esto se traduce en tener una secuencia 
Para asegurar que el linker diseñado funcione únicamente como conector flexible entre dominios, debemos evaluar que sea biológicamente inerte.
Esto implica que su secuencia no posea interacción alguna, evitando así cualquier interferencia con la expresión, utilización y actividad biológica de la proteína quimérica.
% , producto final de la ingeniería de proteínas.
% Esto implica que no tenga regiones target para clivaje,fosforilación, glicosilación, regiones de unión a otras proteínas, etc.
% Si bien es posible simplificar este análisis conociendo en detalle las condiciones experimentales con las que se trabajará, el objetivo de la herramienta es que pueda obtener 
% un resultado genérico, cuyo resultado cumpla con el principio de permanecer inerte en cualquier contexto experimental. 
Realizamos la evaluación de posibles funcionalidades biológicas existentes en la secuencia a través de distintos métodos.

% Durante el capítulo 1(sección \ref{functionalLandscape}) describimos una gran cantidad de elementos funcionales que integran las proteína naturales. 
% Siguiendo estos conocimientos, en esta etapa de la evaluación secuencial describiremos un conjunto de recursos para poder detectar la presencia de estas funcionalidades en nuestra secuencia, 
% con el objetivo de eliminarlas mediante mutaciones.



% There are many methods, such as SMART (Simple Modul	ar
% Architecture Research Tool) (3), PRODOM (4) , Pfam (5,6),
% PROSITE (7) and ELM (Eukaryotic Linear Motif, http://
% elm.eu.org) (8), available for finding globular domains (e.g.
% SH3, TyrKc, active sites) and linear motifs (e.g. SH3 ligands,
% LXXLL nuclear receptor ligands, tyrosine phosphorylation
% sites, post-translational modification sites) within a protein
% sequence. These methods typically rely on sequence similarity
% models, looking for recurrence of known domains or motifs by
% such means as HMMs (Hidden Markov Models) (9), pattern
% discovery
% (http://www.cs.ucr.edu/~stelo/pattern.html)
% or
% SW(Smith-Waterman)-profiles (10). Although these methods
% are of great value in annotating protein sequences, they are
% limited in their ability to uncover new features not yet
% discovered.




Utilizando la herramienta BLAST (descrita en \ref{blast}) intentaremos detectar posibles regiones biológicamente funcionales infiriéndolas a partir de la similitud con proteínas naturales.
% Esta herramienta permite hacer un alineamiento secuencial general frente a una base de datos de proteínas anotadas(en nuestro caso SwissProt).
% Sin embargo, algunas funcionalidades en las proteínas no imponen suficientes restricciones, o no lo hacen homogéneamente a lo largo de toda la secuencia y, por lo tanto, 
% la similitud global entre dos secuencias puede no ser suficientemente considerable, aunque efectivamente tengan una similitud funcional. 
% Dado que, al trabajar con secuencias de longitudes muy cortas, la detección de funcionalidades mediante similitud secuencial puede no resultar exhaustiva, 
En muchos casos, sin embargo, los determinantes secuenciales para la función biológica no están homogéneamente distribuidos a los largo de la 
secuencia o son limitados en tamaño, por lo tanto no serán detectados por métodos de similitud secuencial.
Para detectar este tipo de elementos funcionales utilizamos el recurso PROSITE (descrito en \ref{prosite}).
% En muchos casos en que las proteínas cumplen una misma función puede haber regiones sobre 
% las cuales se imponen restricciones evolutivas muy fuertes, ya que contienen determinantes secuenciales importantes para la función.
% Estos casos puedan no ser detectados por una búsqueda de similitud secuencial ya que no están homogéneamente distribuidos a los largo de la secuencia y son limitados en tamaño. 
% Para detectarlos, utilizamos el recurso PROSITE (descrito en \ref{prosite}).
% para buscar motivos en la secuencia. 
% búsqueda de motivos en la secuencia a través del recurso PROSITE(descrito en \ref{prosite}).
% Utilizando el recurso PROSITE(descrito en \ref{prosite}) buscaremos la presencia de estos en la secuencia. 
% , lo que podría una funcionalidad biológica.

% contiene extracciones de estas agrupaciones de residuos, conocidos como patrones, motivos, marcas o huellas.
% Usando esta herramienta se pueden buscar distintos motivos en la secuencia, permitiendo detectar nuevas regiones con propiedades funcionales. 
% Estos motivos pueden representar funcionalidades simples que en la naturaleza no suelen componer una proteína por sí solos, sino que la funcionalidad global de las proteínas emerge de la unión de distintos elementos en forma modular,
% tal como se vió en el capítulo 1. En nuestro caso, es fundamental poder detectar estos elementos modulares individuales.



% FUNCIONALIDAD EN IDPs

% Predicting function and/or functional sites of IDPs is a task even more difficult(que predecir la conformacion desordenada). 
% Recently, significant advance has been made in this direction, based on the observation that interactions of IDPs are often mediated by short linear motifs [11]. 
% Because linear motifs are 3–15 residues in length, they contain very little sequence information and their prediction from sequence alone is fraught with very
% high false positive rates. A critical advance in this direction has been made by applying context-filters, which significantly increase prediction accuracy by taking into
% consideration motif enrichment in proteins that share the same binding partner or evolutionary history (e.g. SLiM-Finder [12]). 
% Completely different logic forms the basis of ANCHOR, which predicts disordered binding sites by estimating their interaction energy with a general partner [13], and of the molecular recognition feature predictor
% (a-MoRF-PredII  



% Estas búsquedas de similitud secuencial serán importantes en la detección de elementos funcionales de distintas longitudes. 
Más allá de estos, en la sección \ref{idpFunction} describimos la existencia de motivos lineales cortos (short linear motifs, o SLiMs) como elementos funcionales con propiedades particulares.  
% distinguibles por su ubicación característica, frecuentemente insertos en regiones desordenadas, y por estar representados por unos pocos residuos contiguos.  
% Estos motivos lineales cortos (short linear motifs, o SLiMs) representan, entonces, una clase de módulos de interacción compactos, degenerados y evolutivamente convergentes.
% Por lo tanto, dado que nuestro algoritmo guiará la búsqueda hacia conformaciones desordenadas, y teniendo en cuenta que el proceso de mutaciones iterativas podría fácilmente hacer (re)surgir este tipo de elementos de forma similar 
% a como lo hacen en la naturaleza, es necesario analizar la existencia de estos elementos en la secuencia con mayor detalle.
% Dado que nuestro algoritmo guiara la búsqueda hacia conformaciones desordenadas, y que una propiedad característica de los LMs es su relación con un contexto de estructura desordenada\cite{fuxreiter2007local}.
% será relevante para nuestro trabajo buscar este tipo de elementos con mayor detenimiento. Más aún, teniendo en cuenta las caracteristicas secuenciales de este tipo de motivos, que le 
% podía esperarse que, luego de ser detectados y mutados puedan resurgir facilmente dentro de una misma ejecución, con un comportamiento similar al que ocurre en la naturaleza.
%  COMO VAMOS A DETECTARLOS
Para detectar SLiMs se utilizará principalmente el recurso ELM, cuya aplicación en nuestro método se describe en \ref{elm}.
% DIFERENCIAS CON PROSITE
% La base de datos de PROSITE probablemente tenga representados entre sus patrones un número considerable de SLiMs, de esta forma, podremos encontrar cierto solapamiento en los resultados de ambos(PROSITE y ELM). 
% Encontrar los elementos funcionales conocidos en la naturaleza dentro de nuestra secuencia es, entonces, una tarea más dificil.

% The PROSITE database has collected a number of linear protein motifs, representing them as regular expression patterns. 
% PROSITE patterns have been very useful, but also suffer from severe overprediction problems and more recently the database has emphasised globular domain annotation at the expense of linear motifs.



% SOLAPAMIENTOS ANCHOR, BLAST, SLiMs
Además de los motivos lineales, en la sección \ref{idpFunction} se describieron otros módulos funcionales que suelen estar contenidos en IDRs/IDPs.
Dentro de estos, los MoREs son elementos que intervienen en procesos de señalización y reconocimiento entre proteínas y pueden distinguirse a partir de sus propiedades conformacionales características.  
% que le permiten     un proceso de binding & folding.
Estos elementos son detectados en nuestra evaluación mediante la herramienta ANCHOR (descrita en \ref{anchor})
% se pueden detectar motivos de reconocimiento a partir de sus propiedades estructurales(MoREs), lo cual, como se menciona en la sección \ref{propiedadesConformacionales}, 

% es importante para restringir las caracteristicas estructurales del linker resultante. Sin embargo, dado que estos elementos son muy importantes para la funcionalidad de reconocimiento en IDRs/IDPs, la detección y 
% eliminación de estos con objetivos estructurales también persiguie implícitamente objetivos funcionales, evitando la ourrencia de funcionalidades no deseadas en el linker resultante.
% 
% El método de busqueda secuencial BLAST puede detectar dominios que son total o parcialmente desordenados, y teniendo en cuenta la hipótesis desarrollada en la sección \ref{continumm} acerca 
% del continuo de elementos funcionales,
% % (solapamiento entre los MoREs y SLiMs, e incluso con los dominios intrínsecamente desordenados), 
% las herramientas utilizadas para detectar estos elementos funcionales podrían tener cierto solapamiento en los resultados. 
% % Sin embargo, como ya se describió previamente, esto no representa una desventaja.


% Ademas, most protein domains that are identified using sequence-based approaches(como el utilizado en \ref{blast}) are structured, but some can be fully or largely disordered or contain conserved disordered regions,
% es decir, intrinsically disordered domains (IDDs).
% ESTO ES CORRECTO, LO SAQUE DE    \cite{tompa2009close}
% Podria entonces decir que BLAST tambien ayuda a eliminar secuencias que puuedan adoptar estructuras plegadas?? 



Otro tipo de interacciones biológicas relevantes son aquellas mediadas por chaperonas.
Dentro del complejo mecanismo de proteostasis celular, las chaperonas son elementos fundamentales para el correcto funcionamiento y calidad de las proteínas.
El reconocimiento por parte de chaperonas implica la unión a ésta lo cual puede, además, interferir en la flexibilidad de la secuencia linker diseñada.
% % QUE HERRAMIENTAS VAMOS A USAR
Para intentar detectar en la secuencia de trabajo motivos asociados con el reconocimiento por parte de chaperonas utilizamos la herramienta Limbo, detallada en la sección \ref{limbo}. 


% 
% interviniendo en el plegado, activación, la posible translocación, replegamiento y/o degradación de diversas proteínas clientes.
% % *****************
% % CHAPERONAS
% % *********************
% % POR QUE LO QUEREMOS SACAR
% Otro tipo de interacciones que puede afectar a la flexibilidad estructural es el reconocimiento y unión de chaperonas.
% En la sección \ref{proteostasis} se describió cómo, dentro del complejo mecanismo de proteostasis celular, las chaperonas son elementos fundamentales para el correcto funcionamiento y calidad de las proteínas, 
% interviniendo en el plegado, activación, la posible translocación, replegamiento y/o degradación de diversas proteínas clientes.
% Distintas chaperonas reconocen distintos motivos secuenciales expuestos por las proteínas y esta gama de chaperonas llevará a la proteína unida a un final distinto.
% Existen distintas posibilidades reales en el proceso de ingenieria de proteínas en las cuales la secuencia diseñada artificialmente se encuentre con chaperonas en alguna situación experimental. 
% Sin dudas la degradación de nuestra secuencia linker es el peor final en esta situación, 
% pero incluso el simple proceso de reconocimiento y unión puede afectar la flexibilidad, y por lo tanto la funcionalidad, del linker. 
% % pero todas las posibilidades tienen un impacto negativo en la funcionalidad de linker que se quiere asignar(imponer) a la secuencia. 
% % El reconocimiento por parte de la chaperona implica la unión a esta, lo cual puede interferir en la flexibilidad natural que (idealmente) ha adquirido la secuencia linker diseñada. 
% De esta forma, dado que apuntamos a abarcar todas las posibles situaciones experimentales con las que uno se puede encontrar, es necesario tener en cuenta 
% la posibilidad de interacción con proteínas chaperonas. 
% % en algun paso del uso experimental de la secuencia. 
% % Por otro lado, es importante este paso porque las secuencias intrinsecamente desordenadas tales como la secuencia linker que estamos diseñando suelen 
% % tener propiedades que forman targets comunes para las chaperonas (exposicion de sitios hidrofobicos???)?????????????????????????























% ********************************************************
% ********************************************************
% 	DEJO ACÁ LA VERSIÓN ANTERIOR DE LA INTRO 
% 	A LA SECCION DE ELEMENTOSN FUNCIONALES
% ********************************************************
% ********************************************************
 

% \section{Elementos funcionales}
% 
% % POR QUE QUEREMOS SACAR TODAS LAS PROPIEDADES FUNCIONALES
% % ¿QUE DESEAMOS PARA EL LINKER? ¿POR QUE?
% 
% El objetivo de este trabajo es lograr una herramienta que provea una secuencia capaz ser utilizada experimentalmente como linker en el proceso de ingeniería de proteínas. 
% Esto implica no sólo probar con cierta certeza que tendrá la funcionalidad deseada, sino también evaluar el comportamiento en todos los pasos del proceso de ingeniería. 
% Es necesario saber que las propiedades de la secuencia no afectarán este proceso de ninguna forma.
% De forma general, esto se traduce en tener una secuencia que posea interacción ni funcionalidad alguna, evitando así cualquier interferencia con la expresión, utilización y actividad biológica del producto de ingeniería.
% Esto implica que no tenga regiones target para clivaje,fosforilación, glicosilación, regiones de unión a otras proteínas, etc.
% Si bien es posible reducir este análisis conociendo más detalladamente las condiciones experimentales con las que se trabajará, el objetivo de esta herramienta es que se pueda obtener 
% un resultado genérico, que cumpla con este principio de permanecer inerte en cualquier contexto experimental. 
% 
% Durante el capítulo 1(sección \ref{functionalLandscape}) describimos una gran cantidad de elementos funcionales que integran las proteína naturales. 
% Siguiendo estos conocimientos, en esta etapa de la evaluación secuencial describiremos un conjunto de recursos para poder detectar la presencia de estas funcionalidades en nuestra secuencia, 
% con el objetivo de eliminarlas mediante mutaciones.
% 
% 
% 
% % There are many methods, such as SMART (Simple Modular
% % Architecture Research Tool) (3), PRODOM (4) , Pfam (5,6),
% % PROSITE (7) and ELM (Eukaryotic Linear Motif, http://
% % elm.eu.org) (8), available for finding globular domains (e.g.
% % SH3, TyrKc, active sites) and linear motifs (e.g. SH3 ligands,
% % LXXLL nuclear receptor ligands, tyrosine phosphorylation
% % sites, post-translational modification sites) within a protein
% % sequence. These methods typically rely on sequence similarity
% % models, looking for recurrence of known domains or motifs by
% % such means as HMMs (Hidden Markov Models) (9), pattern
% % discovery
% % (http://www.cs.ucr.edu/~stelo/pattern.html)
% % or
% % SW(Smith-Waterman)-profiles (10). Although these methods
% % are of great value in annotating protein sequences, they are
% % limited in their ability to uncover new features not yet
% % discovered.
% 
% 
% 
% % Búsqueda de secuencias homólogas
% % Como se desarrollo en la introducción,
% En particular, al principio de dicha sección vimos cómo las secuencias de proteínas pueden tener ciertas restricciones en su evolución, principalmente asociadas 
% con la funcionalidad que deben proveer y que pueden ser más estrictas(o aplicarse a un mayor porcentaje de la secuencia) si la función está fuertemente asociada a una estructura.
% % principalmente, en secuencias que codifican proteínas globulares, cuyos requerimientos estructurales son mas estrictos. 
% Estos requerimientos funcionales y estructurales en la evolución resultan en una similitud secuencial entre las diversas proteínas homólogas.
% Una forma de evaluar la posible funcionalidad asociada a una proteína, entonces, es buscando la existencia de secuencias proteicas similares en la naturaleza, que hayan sido detectadas y se encuentren anotadas en una base de datos.
% 
% El descubrimiento de una nueva proteína con gran similitud frente a alguna secuencia conocida, permite presumir que se trata de secuencias homólogas y por lo tanto muy probablemente tendrán la misma función o una función similar. 
% En el contexto de la herramienta que estamos desarrollando, las secuencias pueden no pertenecer a proteínas naturales y por lo tanto, no representan casos de homología, sin embargo la similitud secuencial podría dar indicios de elementos funcionales emergentes de nuestra secuencia. La información resultante de la búsqueda sirve como indicador de qué posiciones puntuales son las que conforman esta similitud secuencial, las cuales , por lo tanto, podrían inducir la función. 
% 
% % Domains are now readily detectable with sequence searching programs (e.g., Blast [9] or HMMer [10]) and readily alignable by standard methods (e.g., ClustalW [11] or MUSCLE [12]). 
% % Known domains are now stored in a number of databases including Pfam [1], SMART [13], CDD [14] and
% % InterPro [15] and remain a critical component of genome annotation procedures. 
% 
% Para realizar la búsqueda de similitud secuencial utilizaremos, en primer lugar, la herramienta BLAST. 
% Esta herramienta permite hacer un alineamiento secuencial general frente a una base de datos de proteínas anotadas(en nuestro caso SwissProt).
% % , y por lo tanto se encontraran secuencias 
% Sin embargo, algunas funcionalidades en las proteínas no imponen suficientes restricciones, o no lo hacen homogéneamente a lo largo de toda la secuencia y, por lo tanto, 
% la similitud global entre dos secuencias puede no ser suficientemente considerable, aunque efectivamente tengan una similitud funcional. 
% Estos casos quizás no puedan ser detectados por una búsqueda BLAST. 
% 
% 
% En muchos caso en que las proteínas(o resgiones de estas) cumplen una misma función, aún cuando no existan restricciones impuestas homogéneamente a lo largo de la secuencia, puede haber regiones de tamaño limitado sobre 
% las cuales se imponen restricciones evolutivas muy fuertes, ya que contienen determinantes secuenciales y/o funcionales importantes.
% El recurso PROSITE contiene extracciones de estas agrupaciones de residuos, conocidos como patrones, motivos, marcas o huellas.
% Usando esta herramienta se pueden buscar distintos motivos en la secuencia, permitiendo detectar nuevas regiones con propiedades funcionales. 
% Estos motivos pueden representar funcionalidades simples que en la naturaleza no suelen componer una proteína por sí solos, sino que la funcionalidad global de las proteínas emerge de la unión de distintos elementos en forma modular,
% tal como se vió en el capítulo 1. En nuestro caso, es fundamental poder detectar estos elementos modulares individuales.
% % Sin embargo, en nuestro caso, donde estamos trabajando con secuencias artificiales, o que han sido iterativamente mutadas, 
% % esta herramienta nos permite detectar(principalmente en los pasos finales???) la aparición(o la existencia/que no se haya eliminado) de elementos funcionales que igualmente queremos eliminar.  
% 
% 
% 
% % FUNCIONALIDAD EN IDPs
% 
% % Predicting function and/or functional sites of IDPs is a task even more difficult(que predecir la conformacion desordenada). 
% % Recently, significant advance has been made in this direction, based on the observation that interactions of IDPs are often mediated by short linear motifs [11]. 
% % Because linear motifs are 3–15 residues in length, they contain very little sequence information and their prediction from sequence alone is fraught with very
% % high false positive rates. A critical advance in this direction has been made by applying context-filters, which significantly increase prediction accuracy by taking into
% % consideration motif enrichment in proteins that share the same binding partner or evolutionary history (e.g. SLiM-Finder [12]). 
% % Completely different logic forms the basis of ANCHOR, which predicts disordered binding sites by estimating their interaction energy with a general partner [13], and of the molecular recognition feature predictor
% % (a-MoRF-PredII  
% 
% 
% 
% Estas búsquedas de similitud secuencial serán importantes en la detección de elementos funcionales de distintas longitudes. 
% Más allá de estos, en la sección \ref{functionalLandscape} descubrimos la existencia de elementos modulares de interacción con algunas propiedades particulares,  
% distinguibles por su ubicación característica, frecuentemente insertos en regiones desordenadas, y por estar representados por unos pocos residuos contiguos.  
% Estos motivos lineales cortos (short linear motifs, o SLiMs) representan, entonces, una clase de módulos de interacción compactos, degenerados y evolutivamente convergentes.
% Por lo tanto, dado que nuestro algoritmo guiará la búsqueda hacia conformaciones desordenadas, y teniendo en cuenta que el proceso de mutaciones iterativas podría fácilmente hacer (re)surgir este tipo de elementos de forma similar 
% a como lo hacen en la naturaleza, es necesario analizar la existencia de estos elementos en la secuencia con mayor detalle.
% % Dado que nuestro algoritmo guiara la búsqueda hacia conformaciones desordenadas, y que una propiedad característica de los LMs es su relación con un contexto de estructura desordenada\cite{fuxreiter2007local}.
% % será relevante para nuestro trabajo buscar este tipo de elementos con mayor detenimiento. Más aún, teniendo en cuenta las caracteristicas secuenciales de este tipo de motivos, que le 
% % podía esperarse que, luego de ser detectados y mutados puedan resurgir facilmente dentro de una misma ejecución, con un comportamiento similar al que ocurre en la naturaleza.
% %  COMO VAMOS A DETECTARLOS
% Para detectar SLiMs se utilizará principalmente el recurso ELM, cuya aplicación en nuestro método se describe en \ref{elm}
% % DIFERENCIAS CON PROSITE
% La base de datos de PROSITE probablemente tenga representados entre sus patrones un número considerable de SLiMs, de esta forma, podremos encontrar cierto solapamiento en los resultados de ambos(PROSITE y ELM). 
% % Encontrar los elementos funcionales conocidos en la naturaleza dentro de nuestra secuencia es, entonces, una tarea más dificil.
% 
% % The PROSITE database has collected a number of linear protein motifs, representing them as regular expression patterns. 
% % PROSITE patterns have been very useful, but also suffer from severe overprediction problems and more recently the database has emphasised globular domain annotation at the expense of linear motifs.
% 
% 
% % SOLAPAMIENTOS ANCHOR, BLAST, SLiMs
% Además de los motivos lineales, en la introducción se describieron otros módulos funcionales que pueden estar contenidos en IDRs/IDPs.
% Utilizando la herramienta ANCHOR(ver \ref{anchor}) se pueden detectar motivos de reconocimiento a partir de sus propiedades estructurales(MoREs), lo cual, como se menciona en la sección \ref{propiedadesConformacionales}, 
% es importante para restringir las caracteristicas estructurales del linker resultante. Sin embargo, dado que estos elementos son muy importantes para la funcionalidad de reconocimiento en IDRs/IDPs, la detección y 
% eliminación de estos con objetivos estructurales también persiguie implícitamente objetivos funcionales, evitando la ourrencia de funcionalidades no deseadas en el linker resultante.
% Además, el método de busqueda secuencial BLAST puede detectar dominios que son total o parcialmente desordenados, y teniendo en cuenta la hipótesis desarrollada en la introduccion acerca 
% del solapamiento entre los MoREs y SLiMs, e incluso con los dominios intrínsecamente desordenados, las herramientas utilizadas para detectar estos tres pueden también solaparse en los resultados que proveen.. 
% Sin embargo, como ya se describió previamente, esto no representa una desventaja.
% 
% 
% % Ademas, most protein domains that are identified using sequence-based approaches(como el utilizado en \ref{blast}) are structured, but some can be fully or largely disordered or contain conserved disordered regions,
% % es decir, intrinsically disordered domains (IDDs).
% % ESTO ES CORRECTO, LO SAQUE DE    \cite{tompa2009close}
% % Podria entonces decir que BLAST tambien ayuda a eliminar secuencias que puuedan adoptar estructuras plegadas?? 
% 
% 
% 
% % % *****************
% % % CHAPERONAS
% % % *********************
% % % POR QUE LO QUEREMOS SACAR
% % Otro tipo de interacciones que puede afectar a la flexibilidad estructural es el reconocimiento y unión de chaperonas.
% % En la sección \ref{proteostasis} se describió cómo, dentro del complejo mecanismo de proteostasis celular, las chaperonas son elementos fundamentales para el correcto funcionamiento y calidad de las proteínas, 
% % interviniendo en el plegado, activación, la posible translocación, replegamiento y/o degradación de diversas proteínas clientes.
% % Distintas chaperonas reconocen distintos motivos secuenciales expuestos por las proteínas y esta gama de chaperonas llevará a la proteína unida a un final distinto.
% % Existen distintas posibilidades reales en el proceso de ingenieria de proteínas en las cuales la secuencia diseñada artificialmente se encuentre con chaperonas en alguna situación experimental. 
% % Sin dudas la degradación de nuestra secuencia linker es el peor final en esta situación, 
% % pero incluso el simple proceso de reconocimiento y unión puede afectar la flexibilidad, y por lo tanto la funcionalidad, del linker. 
% % % pero todas las posibilidades tienen un impacto negativo en la funcionalidad de linker que se quiere asignar(imponer) a la secuencia. 
% % % El reconocimiento por parte de la chaperona implica la unión a esta, lo cual puede interferir en la flexibilidad natural que (idealmente) ha adquirido la secuencia linker diseñada. 
% % De esta forma, dado que apuntamos a abarcar todas las posibles situaciones experimentales con las que uno se puede encontrar, es necesario tener en cuenta 
% % la posibilidad de interacción con proteínas chaperonas. 
% % % en algun paso del uso experimental de la secuencia. 
% % % Por otro lado, es importante este paso porque las secuencias intrinsecamente desordenadas tales como la secuencia linker que estamos diseñando suelen 
% % % tener propiedades que forman targets comunes para las chaperonas (exposicion de sitios hidrofobicos???)?????????????????????????
% % % QUE HERRAMIENTAS VAMOS A USAR
% % Para intentar detectar en la secuencia de trabajo motivos asociados con el reconocimiento por parte de chaperonas utilizamos la herramienta Limbo(\ref{limbo}). 
% % 

































\subsection{BLAST}\label{blast}

% OBJETIVO: QUE TRATAMOS DE DETECTAR
El método BLAST \cite{altschul1990basic,mcginnis2004blast} permite evaluar la similitud entre dos secuencias biológicas, tales como cadenas de aminoácidos correspondientes a proteínas o secuencias nucleotídicas.
En nuestra herramienta utilizaremos BLAST para inferir la existencia de elementos funcionales en la secuencia linker que estamos evaluando a partir de una similitud considerable con secuencias naturales, 
las cuales asumimos que poseen una funcionalidad biológica.
% En nuestro caso lo utilizaremos, obviamente, sobre secuencias peptídicas.


% 
% La búsqueda BLAST se basa en los cambios que pueden ocurrir entre 2 secuencias homólogas durante la evolución. 
% Las mutaciones pueden dar como resultado distintos residuos en las secuencias de proteínas, también pueden ocurrir inserciones y deleciones de residuos. 
% Cada uno de los posibles eventos tiene una frecuencia de ocurrencia asociada. El método de alineamiento utiliza un esquema de valores para asignar puntaje a cada uno de estos eventos y,
% utilizando una estrategia de optimización, se exploran formas alternativas de alinear los residuos de ambas secuencias para logar sumar el máximo score.
% Realizando una búsqueda optimizada se pueden recuperar secuencias similares de una base de datos con gran cantidad de entradas, haciendo un alineamiento de la secuencia de búsqueda frente a todas las almacenadas en la base de datos. Para analizar los resultados de esta búsqueda no es posible evaluar solamente el score obtenido en cada alineamiento, sino que se debe tener en cuenta la probabilidad de que la similitud encontrada sea solo al azar. En este punto se deben aplicar métodos estadísticos para evaluar la significancia del resultado dada la longitud de la secuencia consultada y el tamaño de la base de datos.
% El resultado que devuelve la búsqueda BLAST es, en caso de éxito, un conjunto de secuencias de la base de datos ordenadas por un score de alineamiento. Además de realizar el alineamiento, BLAST provee información estadística que ayuda a descifrar la significancia biológica del alineamiento, este es el valor "expect" o e-value. Para utilizar los resultados de BLAST en nuestra herramienta utilizaremos un cutoff sobre el valor de e-value que nos de cierta certeza de que la similitud secuencial es significativa y, por lo tanto, la secuencia podría tener una función similar. El valor de cutoff utilizado es de 0.01, todas las secuencias encontradas con un e-value menor que este valor serán significativamente similares. En caso de encontrarse muchas secuencias dentro del cutoff se utilizará la primera de éstas.
% Dado que el objetivo de este paso es evaluar que posiciones tienden a una similitud con alguna secuencia natural, el resultado obtenido se utilizará para identificar estas posiciones y marcarlas para una nueva ronda de mutaciones.
% 


% En particular, al principio de dicha sección vimos cómo las secuencias de proteínas pueden tener ciertas restricciones en su evolución, principalmente asociadas 
% con la funcionalidad que deben proveer y que pueden ser más estrictas(o aplicarse a un mayor porcentaje de la secuencia) si la función está fuertemente asociada a una estructura.
% Estos requerimientos funcionales y estructurales en la evolución resultan en una similitud secuencial entre las diversas proteínas homólogas.
% Una forma de evaluar la posible funcionalidad asociada a una proteína, entonces, es buscando la existencia de secuencias proteicas similares en la naturaleza, que hayan sido detectadas y se encuentren anotadas en una base de datos.
% El descubrimiento de una nueva proteína con gran similitud frente a alguna secuencia conocida, permite presumir que se trata de secuencias homólogas y por lo tanto muy probablemente tendrán la misma función o una función similar. 
% En el contexto de la herramienta que estamos desarrollando, las secuencias pueden no pertenecer a proteínas naturales y por lo tanto, no representan casos de homología, sin embargo la similitud secuencial podría dar indicios de elementos funcionales emergentes de nuestra secuencia. La información resultante de la búsqueda sirve como indicador de qué posiciones puntuales son las que conforman esta similitud secuencial, las cuales , por lo tanto, podrían inducir la función. 
% Domains are now readily detectable with sequence searching programs (e.g., Blast [9] or HMMer [10]) and readily alignable by standard methods (e.g., ClustalW [11] or MUSCLE [12]). 
% Known domains are now stored in a number of databases including Pfam [1], SMART [13], CDD [14] and
% InterPro [15] and remain a critical component of genome annotation procedures. 







% METODO
Para realizar la comparación, BLAST requiere una secuencia de búsqueda (query) y una secuencia contra la cual comparar.
% , o alternativamente, una base de datos conteniendo múltiples secuencias.
% To run the software, BLAST requires a query sequence to search for, and a sequence to search against (also called the target sequence) or a sequence database containing multiple such sequences.
La ventaja principal del método es que permite realizar, de forma muy eficiente, la comparación de una misma secuencia contra gran cantidad de secuencias contenidas en una base de datos.
% Esta comparación entre dos secuencias puede extenderse para comparar una secuencia 
Para esto, BLAST utiliza un método heuristico no exhaustivo llamado word method o método de k-tuplas.
Al ser un método heurístico, no está garantizado que encuentre los alineamientos óptimos con las secuencias de la base de datos, lo que sí ocurriría si se utilizara el algoritmo de Smith-Waterman \cite{smith1981identification}.
Este último permite encontrar el alineamiento óptimo a expensas de un gran costo computacional.
% it cannot "guarantee the optimal alignments of the query and database sequences" as Smith-Waterman does.
% para hallar alineamientos locales, los cuales son extendidos usando una matriz de sustitución. 

% Consiste en armar una tabla de look-up con subsecuencias y luego realizar búsquedas a gran escala en diversas bases de datos
En primer lugar, BLAST extrae de la secuencia query todas las subsecuencias de largo k (valor definido según el tipo de secuencias que se están comparando) y busca ocurrencias de éstas en la base de datos.
A partir de estos alineamientos locales, se seleccionan aquellos con mayor valor de score y se extiende el alineamiento hacia la derecha e izquierda de las secuencias, 
hasta que ocurre una disminución en el score correspondiente.
Por último, si existen extensiones de alineamiento que se encuentran en una misma secuencia estas pueden unirse.
Finalmente, se muestra un alineamiento completo entre las secuencias con mayor score, junto con 
la significancia estadística del alineamiento, dependiente del largo de la secuencia query y el tamaño de la base de datos.




% UTILIZACION
La herramienta BLAST puede ser ejecutada de dos maneras. La opción más simple es a través del servidor web \cite{blastWeb}, lo cual implica ejecutar la búsqueda remotamente con retardos importantes para obtener los resultados. 
% %     hacer una llamada al servidor remoto que ejecuta la búsqueda (http://blast.ncbi.nlm.nih.gov/). 
% %     Se puede hacer simplemente desde Python utilizando el módulo BioPython(http://biopython.org/). (http://www.biotnet.org/sites/biotnet.org/files/documents/25/biopython_blast.pdf ) 
La segunda opción es realizar la búsqueda de forma local, para lo cual es necesario tener disponible el paquete de software provisto por NCBI \cite{blastLocal}, junto con la base de datos sobre la cual se quiere hacer la comparación.
Esta es la opción que utilizamos en nuestra herramienta y, en nuestro caso, realizamos la búsqueda sobre la base de datos UniProtKB \cite{bairoch2000swiss}.
Utilizamos 0.01 como valor de \textit{cutoff}, es decir, todas las secuencias encontradas que tengan un e-value menor a 0.01 serán consideradas como significativamente similares.
Todas las posiciones que, en el alineamiento final, sean idénticas en ambas secuencias, tendrán un puntaje igual a 1 en nuestro sistema de evaluación.

Como ejemplo de aplicación realizamos la búsqueda de la secuencia \\ \texttt{MVLSPADKTNVKGGWGKV}, encontrando la secuencia \mbox{\texttt{MVLSPADKTNVKAAWGKV}} con un e-value de 7e-09. 
El alineamiento entre estas dos secuencias y el puntaje resultante asignado por nuestro método es:
\vspace{0.5cm}

%\noindent
\begin{adjustbox}{width=\textwidth}
\begin{tabular}{lllllllllllllllllll} 
\hline
Secuencia 		& \textbf{M} & \textbf{V} & \textbf{L} & \textbf{S} & \textbf{P} & \textbf{A} & \textbf{D} & \textbf{K} & \textbf{T} & \textbf{N} & \textbf{V} & \textbf{K} & \textbf{G} & \textbf{G} & \textbf{W} & \textbf{G} & \textbf{K} & \textbf{V}\\ \hline
Alineamiento Hit	& \textbf{M} & \textbf{V} & \textbf{L} & \textbf{S} & \textbf{P} & \textbf{A} & \textbf{D} & \textbf{K} & \textbf{T} & \textbf{N} & \textbf{V} & \textbf{K} & \textbf{-} & \textbf{-} & \textbf{W} & \textbf{G} & \textbf{K} & \textbf{V}\\ \hline
Puntaje BLAST 		& 1 & 1 & 1 & 1 & 1 & 1 & 1 & 1 & 1 & 1 & 1 & 1 & 0 & 0 & 1 & 1 & 1 & 1 \\ \hline
\end{tabular}
\end{adjustbox}
\vspace{0.5cm}

% Es importante aclarar que l
La búsqueda BLAST, aun siendo un método heuristico y altamente optimizado, tiene un tiempo de ejecución considerablemente alto con respecto al resto de las herramientas descritas.
% aproximadamente 10 veces superior
Por otro lado, para encontrar un \textit{hit} estadísticamente significativo la secuencia debe tener, generalmente, un largo considerable.
Esto se debe a que el valor de \textit{e-value} depende, en parte, del largo de la secuencia \textit{query}. 
% Este no es el caso típicolo cual no es el principal objetivo del método que estamos desarrollando.

Gran parte de las evaluaciones realizadas en nuestro método no cumplen esta característica de longitud. 
De esta forma, para reducir el tiempo de ejecución global de nuestro método, la evaluación usando BLAST se realiza de forma separada. 
% De esta forma, dado que el tiempo de ejecución es relativamente alto y que probablemente sólo se encuentren resultados en casos puntuales, 
Para esto, en primer lugar se realiza el proceso iterativo de mutaciones utilizando todo el conjunto de herramientas de evaluación excepto BLAST.
Una vez que se alcanza una secuencia con score = 0 según ese esquema de evaluación, se realiza el mismo proceso utilizando exclusivamente BLAST para las evaluaciones.
La iteración global termina cuando, usando el total de las herramientas aqui descritas, la secuencia alcanzada tiene un score = 0. 














\subsection{Prosite}\label{prosite}

PROSITE \cite{sigrist2002prosite,prositeWeb} es un recurso que agrupa una gran cantidad de motivos secuenciales de diversas características,
permitiendo anotar e identificar regiones conservadas en secuencias de proteínas.
En nuestra implementación utilizamos este recurso para extender la búsqueda de elementos funcionales en la secuencia.  

De forma resumida, se puede definir a PROSITE como una colección anotada de motivos biológicamente significativos, dedicada a la identificación de familias y dominios de proteínas.
Esta base de datos contiene información derivada de alineamientos de múltiples secuencias homólogas. 
Los motivos resultantes se describen usando dos métodos distintos, cada uno con sus ventajas y desventajas.

La primera forma de describir los motivos es a través de patrones secuenciales (utilizando expresiones regulares como se mostró para ELM, sección \ref{elm}),
en los cuales se tiene en cuenta solo la información de los residuos más significativos, descartando el resto. 
La búsqueda de un patrón en una secuencia da un resultado cualitativo: hay una coincidencia o no la hay. 
Si hay una sustitución en alguna de las posiciones de la secuencia el patrón no coincide, independientemente del tipo de sustitución que ocurrió.

Otra forma para describir los motivos es mediante perfiles (o matrices de pesos). Estos pesos proveen valores numéricos para cada posible coincidencia o sustitución cuando se busca el motivo en una secuencia. 
De esta forma, al utilizarlos en la búsqueda de un motivo, funcionan como descriptores cualitativos que consideran la similitud global en toda la longitud secuencial de un dominio o proteína. 
Un motivo puede ser encontrado en una secuencia que posee una sustitución en una posición conservada si el resto de la secuencia tiene un nivel de similitud suficientemente alto.
Estas propiedades dan una mayor sensibilidad a los perfiles con respecto a los patrones, permitiendo encontrar dominios o familias con alta divergencia que solo tienen unas pocas posiciones muy conservadas.

Diversas búsquedas relacionadas con patrones anotados en PROSITE se pueden hacer a través de la herramienta ScanProsite \cite{de2006scanprosite,scanprositeWeb}, 
la cual permite escanear secuencias para buscar ocurrencias de los motivos, buscar motivos en una base de datos entera de secuencias, o buscar motivos propios del usuario en una secuencia.
Esta herramienta se encuentra disponible para descargar, junto con la base de datos completa de motivos secuenciales, lo que permite realizar la búsqueda de forma local.

% El objetivo de nuestra herramienta es poder encontrar cualquier ocurrencia de motivos en la secuencia sobre la que estamos trabajando. 
Para encontrar motivos secuenciales en la secuencia que evaluamos, es posible escanearla utilizando patrones y/o perfiles, y variar también los límites usados en la detección de los perfiles. 
En nuestro caso utilizaremos patrones para realizar la búsqueda, principalmente porque el tiempo de búsqueda utilizando perfiles es de aproximadamente 10 veces el que demora la búsqueda mediante patrones 
y, si bien esta diferencia es despreciable cuando se hacen búsquedas sobre una sola secuencia, al realizar evaluaciones en una gran cantidad de iteraciones la diferencia se vuelve significativa.
En segundo lugar, consideramos que la sensibilidad provista por los patrones es suficiente para los fines buscados en nuestra herramienta, al menos inicialmente.
De todas formas, es uno de los aspectos a evaluar con mayor profundidad a futuro, por lo que no se descarta extender la búsqueda para poder utilizar perfiles, al menos de forma opcional para el usuario. 
% describir si hay superposición entre elm y prosite. Se justifica usar los dos? qué hipótesis subyacente estamos usando?

Un aspecto relevante de la base de datos PROSITE es que, si bien se ha orientado hacia la anotación de dominios globulares por sobre motivos lineales, existen anotaciones que representan motivos lineales cortos, 
lo cual trae dos consecuencias. En primer lugar, para cualquier búsqueda, pueden ocurrir una gran cantidad de falsos positivos debido a las propiedades intrínsecas de estos.
En segundo lugar, debido al contexto en el cual la estamos aplicando, donde también se buscan SLiMs mediante el recurso ELM, puede haber un solapamiento con los resultados obtenidos entre ambas herramientas.
% utilizando la herramienta ELM.

Usando la secuencia \texttt{VKTCLALGVDINTCD} ejemplificaremos el proceso de evaluación.
% , analizaremos la secuencia \texttt{VKTCLALGVDINTCD}. 
Mediante la ejecución de ScanProsite se identifica que esta secuencia contiene el patrón PS00008 (MYRISTYL N-myristoylation site 
\url{http://prosite.expasy.org/cgi-bin/prosite/nicedoc.pl?PS00008}) ubicado en la subsecuencia \texttt{GVDINT} (posiciones 8-13), por lo tanto el puntaje correspondiente es:

\vspace{0.5cm}
\begin{adjustbox}{width=\textwidth}
\begin{tabular}{llllllllllllllll} 
\hline
Secuencia & \textbf{V} & \textbf{K} & \textbf{T} & \textbf{C} & \textbf{L} & \textbf{A} & \textbf{L} & \textbf{G} & \textbf{V} & \textbf{D} & \textbf{I} & \textbf{N} & \textbf{T} & \textbf{C} & \textbf{D}\\ \hline
Evaluación global ELM & 0 & 0 & 0 & 0 & 0 & 0 & 0 & 1 & 1 & 1 & 1 & 1 & 1 & 0 & 0 \\ \hline
\end{tabular}
\end{adjustbox}






















\subsection{ELM}\label{elm}

El recurso de motivos lineales eucariotas (ELM) \cite{puntervoll2003elm,dinkel2013eukaryotic} fue establecido con la misión de recolectar, anotar y clasificar motivos lineales cortos 
(conocidos como LMs, ELMs, SLiMs o MiniMotifs y detallados en la sección \ref{idpFunction}). 
Lo utilizaremos para detectar instancias de estos elementos dentro de la secuencia linker que estamos evaluando, algo que, debido a las propiedades intrínsecas de los SLiMs, no sería posible sin un recurso 
con datos curados manualmente de la literatura y que están completamente a disposición de la comunidad científica, como es ELM.
% Debido a las características de estos elementos funcionales, no es nada simple detectar instancias de estos en la secuencia linker que estamos evaluando sin las consecuencias de tener gran cantidad de falsos positivos.
% Sin embargo, los datos almacenados en este recursos son curados manualmente a partir de la literatura y están a disposición de la comunidad científica, permitiendo detectarlos con 
% instancias de SLiMs en la secuencia linker que estamos evaluando, algo que, por las características propias de estos elementos funcionales, no es posible de 

% The aim(del recurso) is to cover the set of functional sites that can be defined by the local peptide sequence, operating essentially independently of protein tertiary structure. 
% Este recurso provee actualmente una completa base de datos con motivos conocidos validados experimentalmente, con datos curados manualmente a partir de la literatura.
% además de una herramienta para descubrir instancias de estos sobre secuencias provistas por el usuario. 
% segmentos posiblemente correspondientes a motivos lineales 

Este recurso provee actualmente una completa base de datos con motivos organizada jerárquicamente: en el nivel superior se tiene un conjunto de tipos (actualmente hay un total de 6 tipos diferentes). 
% los tipos son: 
% 	Proteolytic cleavage sites (CLV)
% 	general ligand binding sites (LIG),
% 	sites for post-translational modification (MOD)
% 	sub-cellular targeting sites (TRG)
% 	Ligand binding classes describing docking sites (DOC): can be described as motifs that recruit a modifying enzyme using a site that is distinct from the active site
% 	destruction motifs (DEG) is a specific region of a protein sequence that directs protein polyubiquitylation and targets the protein to the proteasome for degradation
%  ESTOS ULTIMOS 2 TIPOS FUERON AGREGADOS EN EL ULTIMO TIEMPO: 
%           Technically, all docking sites and destruction motifs belong to the ‘ligand binding sites (LIG)’ type; however, grouping together motif classes of similar function adds an additional level of discrimination.
Cada tipo agrupa un conjunto de clases, y cada clase define la especificación de un dominio o familia de dominios de péptidos, 
los cuales se describen mediante una expresión regular (cadena de carácteres para describir un patrón secuencial \cite{regex}) representativa de la secuencia que los compone.
Cada clase contiene al menos una instancia anotada, donde cada instancia representa una secuencia determinada experimentalmente que se ajusta a la expresión definida para la clase.
El énfasis está puesto en la validación experimental que ha sido realizada sobre estas secuencias, logrando un proceso de curación manual a partir de la literatura con las instancias que son ingresadas en la base de datos.

Dado que los motivos suelen tener solo un pequeño número de posiciones fijas, es normal que las búsquedas resulten en una gran cantidad de falsos positivos.  
Es por esto que el recurso también provee opciones para filtrar los resultados según la especie, el compartimento en el cual se va a encontrar la secuencia, etc. 
De todas formas, la validación final de la funcionalidad del motivo debe ser siempre realizada experimentalmente.

% The resource suffers from the overprediction problem inherent to small protein motifs, but we are developing context filters such as cell compartment, taxonomy and globular domain clash that can partly reduce the severity
% of the problem. 
% In this resource, we use the term ELM to denote our bioinformatical representation of a functional site including the sequence motif and its context.
% For any such analysis, the user should be aware that many matches to ELM regular expressions are false positives. 
% Before conducting experiments based on ELM results, it is strongly advisable to check if a motif match is conserved, exposed in a cell compartment in which the motif is known to be functional. 
% The ELM resource applies several filters to provide the user with such information that should ideally also be supported by the experimental evidence.


Este recurso puede ser utilizado directamente a través del servidor web \cite{elmweb}, el cual provee una herramienta para encontrar, en una secuencia ingresada por el usuario, instancias de los motivos contenidos en la base de datos.
% Esta herramienta permite encontrar en una secuencia ingresada por el usuario, instancias de los motivos contenidos en la base de datos.
Otra forma de realizar esto es haciendo una búsqueda local. Para ésto es necesario descargar la base de datos de expresiones regulares correspondientes a los motivos y, 
a partir de estos datos, realizar la búsqueda de cada expresión regular sobre la secuencia con la que estamos trabajando.
Esta última forma de detección es la que utilizaremos para nuestra evaluación, mediante el módulo re de Python que permite buscar instancias de una expresión regular sobre una secuencia.
La base de datos utilizada corresponde a la versión con fecha 1-3-2015 obtenida de \cite{elmweb}.
% ya que nos permite independizarnos de la disponibilidad de la herramienta en el momento en que la requerimos.

La búsqueda de motivos sobre la secuencia resulta en un conjunto de subsecuencias correspondientes a cada instancia encontrada las cuales pueden estar solapadas, es decir, 
cada posición de la secuencia puede estar contenida en 0,1 o más instancias encontradas. 
Cada una de estas subsecuencias se trata de forma individual y, por cada posición de ésta, se suma 1 al puntaje asociado.
% El algoritmo toma cada una de las subsecuencias resultantes 
A modo de ejemplo, si estamos trabajando con la secuencia \texttt{PSKPLRGNAMVGL}, el resultado de la búsqueda indica 2 motivos encontrados:
% es un conjunto de 2 motivos encontrados en las subsecuencias:
% INCLUIR EXPRESIONES REGULARES, A SER POSIBLE TOMADAS DE ELM
% 
\vspace{0.5cm}

% \rule[-0.8\baselineskip]{0pt}{\baselineskip}
\noindent 
\begin{adjustbox}{width=\textwidth}
\begin{tabular}{c|c|c} 
% \hline
\textbf{Clase ELM} & \textbf{Expresión regular} & \textbf{Instancia (ubicación)}\\ \hline
LIG\_SH3\_2 & P..P.[KR] & PSKPLR (1-6)\\ 
% LIG\_PDZ\_Class\_2 & ...[VLIFY].[ACVILF]\$ & NAMVGL (8-13)  \\
DOC\_MAPK\_1 MAPK  & [KR]\{0,2\}[KR].\{0,2\}[KR].\{2,4\}[ILVM].[ILVF] & KPLRGNAMVGL(3-13)
\end{tabular}
\end{adjustbox}

\vspace{0.5cm}

% \noindent 
% \begin{tabular}{lll} 
% \hline
% Clase ELM & Instancia & Descripción \\ \hline
% LIG\_SH3\_2 & PSKPLR (posiciones 1-6) &  This is the motif recognized by class II SH3 domains\\ 
% LIG\_PDZ\_Class\_2 & NAMVGL (posiciones 8-13) & The C-terminal class 2 PDZ-binding motif \\
% DOC\_MAPK\_1 MAPK  & KPLRGNAMVGL(posiciones 3-13) &interacting molecules (e.g. MAPKKs, substrates, phosphatases) carry docking motif that help to regulate specific interaction in the MAPK cascade. \\
% \end{tabular}
% The classic motif approximates (R/K)xxxx#x# where # is a hydrophobic residue.\\
% \vspace{0.4cm}
Cada instancia encontrada implica una posible funcionalidad mediada por un motivo lineal, por lo tanto, se aumenta en 1 el valor del puntaje a las posiciones involucradas \underline{para cada instancia} por separado.
El puntaje resultante de este paso es:

\vspace{0.5cm}
% \rule[-0.8\baselineskip]{0pt}{\baselineskip}
\begin{adjustbox}{width=\textwidth}
\begin{tabular}{llllllllllllll} 
\hline
Secuencia & \textbf{P} & \textbf{S} & \textbf{K} & \textbf{P} & \textbf{L} & \textbf{R} & \textbf{G} & \textbf{N} & \textbf{A} & \textbf{M} & \textbf{V} & \textbf{G} & \textbf{L} \\ \hline
LIG\_SH3\_2 (1-6) 		& 1 & 1 & 1 & 1 & 1 & 1 & 0 & 0 & 0 & 0 & 0 & 0 & 0\\ \hline
% LIG\_PDZ\_Class\_2 (8-13) & 0 & 0 & 0 & 0 & 0 & 0 & 0 & 1 & 1 & 1 & 1 & 1 & 1 \\ \hline
DOC\_MAPK\_1 MAPK (3-13)  	& 0 & 0 & 1 & 1 & 1 & 1 & 1 & 1 & 1 & 1 & 1 & 1 & 1 \\ \hline
Evaluación global ELM 		& 1 & 1 & 2 & 2 & 2 & 2 & 1 & 1 & 1 & 1 & 1 & 1 & 1\\ \hline
\end{tabular}
\end{adjustbox}

























\subsection{Limbo: Interacción con chaperonas} \label{limbo}
% QUE DETECTA
El método Limbo \cite{van2009accurate} es un predictor de sitios de unión a la chaperona DnaK, representante de la extensa familia Hsp70, que se especializa en la unión de regiones hidrofóbicas expuestas, generalmente
presentes en polipéptidos desplegados. La unión de chaperonas a una secuencia gobierna una gran variedad de procesos tales como translocación, replegamiento y degradación de la proteína reconocida,
además de la activación de un amplio rango de proteínas asociadas.
El objetivo de aplicar esta herramienta en nuestro esquema de evaluación es detectar la presencia sitios de unión a chaperonas en la secuencia linker, 
evitando el reconocimiento y la interacción con proteínas en el sistema biológico de expresión o aplicación.

% METODO
El método Limbo se desarrolló utilizando una combinación de información secuencial y estructural para analizar el perfil de secuencias que se unen a la DnaK.
El primer paso en el desarrollo del método es crear un predictor basado exclusivamente en información secuencial.
Para armar el set de aprendizaje usado, se realizaron una serie de ensayos experimentales en los cuales un conjunto de péptidos se inmoviliza en membranas de celulosa 
% de unión a péptidos, los cuales se inmobilizaron unidos a una membrana de celulosa, 
y luego de la incubación con DnaK se detectó la unión a estos mediante un anticuerpo específico.

Los péptidos usados en los ensayos de unión se obtienen de 3 conjuntos distintos. El set inicial se obtiene de predicciones basadas en el uso de TANGO \cite{fernandez2004prediction} (ver sección \ref{tango}),
asumiendo que la DnaK se une a secuencias propensas a formar agregados.
Usando los resultados obtenidos, se desarrolla un predictor simple de unión a DnaK, el cual es utilizado para la búsqueda de sitios de unión a chaperonas en el proteoma de \textit{E.coli} 
% y separando las proteínas cortas con al menos dos de estos sitios de unión
, las cuales conforman el segundo set. El último conjunto corresponde a dos péptidos supuestamente reconocidos por la DnaK, ubicados en el factor $\sigma$ de la RNA-polimerasa.
Estos péptidos, luego de ser sometidos al ensayo de binding, se dividen en dos conjuntos según los resultados obtenidos:
% Los resultados de los ensayos fueron filtrados para separar aquellos que resultaban en señal positiva aún cuando no se realizaba el paso de incubación con DnaK (falsos positivos). 
% La señal de todo el conjunto resultante se corrigió usando esta señal de control negativo. Finalmente los resultados se dividieron, según dos valores de cutoff(alto y bajo), para dar 2 conjuntos de péptidos:
los que son reconocidos por la DnaK y lo que no. 

El set de aprendizaje final se construye obteniendo primero todas las subsecuencias de 7 residuos posibles. 
Del set de péptidos de no-unión todos los posibles heptapéptidos formados son seleccionados, para dar el set de resultados negativos.
Por su parte, del set de péptidos de unión sólo se incluyen en el set de péptidos positovos a aquellos que dan una mayor energía de unión a DnaK, evaluada mediante el campo de fuerzas de FoldX.
A partir de los conjuntos de heptapéptidos positivos y negativos se obtienen dos matrices PSSM y una matriz final se obtiene restando los valores de score de la PSSM de no-unión a los valores encontrados en la PSSM positiva.

De manera independiente se contruyó un perfil secuencial a partir de un análisis puramente estructural de unión a DnaK.
Para esto se hizo primero un análisis de mutaciones posicionales sobre el heptapéptido unido a la DnaK utilizando la estructura de ésta (cristalizada junto con un hepapéptido) y el campo de fuerzas FoldX. 
En primer lugar se mutaron todas las posiciones a Alanina y luego cada posición fue mutada individualmente a cada uno de los restantes 19 aminoácidos.
Se obtuvieron así los valores de $\Delta\Delta$G para cada residuo y cada posición, donde los valores más negativos indican una mejor unión a la DnaK.
Por lo tanto, la PSSM se obtiene del negativo correspondiente a cada valor de $\Delta\Delta$G.
% The more
% negative the DDG, the better the residue fits DnaK binding. To
% convert each DDG into a PSSM score, we took the negative of
% each DDG and filled the structure-based PSSM accordingly.

Los valores de todas las matrices son optimizados mediante un algoritmo de validación cruzada que permite eliminar algunos péptidos de los conjuntos de aprendizaje.
Los perfiles representados en las PSSMs obtenidas del análisis secuencial y del análisis estructural por separado se combinan para dar el predictor final. 


% UTILIZACION
La versión final de la matriz y un script implementado en Python para realizar el análisis fue provisto por los desarrolladores de Limbo.
El valor de corte utilizado es 11.08 (valor por defecto). Para utilizar el método de manera local se corre el script provisto y se pasa como parámetro el archivo conteniendo las secuencias a analizar en formato fasta.
El resultado contiene una lista de heptapéptidos cuyo score calculado supera el valor de corte.
Por ejemplo, analizando la secuencia \texttt{DLWKLLPENNVLSP}, el resultado obtenido es:

% \noindent
	\texttt{1 DLWKLLP 11.7190683353}   \\
\indent \texttt{3 WKLLPEN 23.0607028837} 

Este resultado indica que los heptapéptidos DLWKLLP y el WKLLPEN, poseen valores de score superior al valor de corte (11.7190683353 y 23.0607028837, respectivamente).
Utilizando nuestro sistema de evaluación, se aumenta en 1 el puntaje de todas las posiciones asociadas a cada (debemos tener en cuenta que algunas posiciones pueden sumar más de 1 ya que los heptapéptidos pueden estar solapados).
A partir de los resultados anteriores, el puntaje de la evaluación es:

\vspace{0.5cm}
%\noindent
\begin{center}
\begin{adjustbox}{width=\textwidth}
\begin{tabular}{llllllllllllllll} 
\hline      		
Secuencia & \textbf{D} & \textbf{L} & \textbf{W} & \textbf{K} & \textbf{L} & \textbf{L} & \textbf{P} & \textbf{E} & \textbf{N} & \textbf{N} & \textbf{V} & \textbf{L} & \textbf{S} & \textbf{P} \\ \hline
Evaluación con Limbo & 1 & 1 & 2 & 2 & 2 & 2 & 2 & 1 & 1 & 0 & 0 & 0 & 0 & 0 \\ \hline
\end{tabular}
\end{adjustbox}
\end{center}






% Para desarrollar el algoritmo predictor, lo que se realizó fue:
% 
% ARMAN 3 GRUPOS DE PEPTIDOS (ver de donde salen los 3 grupos)
% PRUEBAN TODOS LOS PEPTIDOS MEDIANTE ENSAYO DE UNION A DnaK: sintetizan los peptidos unidos a placa de celulosa, los lavan, los incuban con DnaK y los revelan con un anticuerpo anti-DnaK. 
% (hacen ademas controles negativos de donde vuelan un par de peptidos que se unen directamente al anticuerpo, ademas restan el valor de fluorecencia que aparece cuando no ponen peptidos)
% Del total de peptidos se dividieron en conjuntos de peptidos binders y peptidos no binders(usando dos valores de cutoff - un valor alto y un bajo). 
% De cada uno de estos conjuntos se separo un pequeño % como conjunto de prueba(para probar luego que tal funciona el predictor que se va a hacer) y un gran % es el que luego se usa para el set de aprendizaje? (conjunto benchmark)
% Para poder armar una matriz de score especifica de posicion(PSSM) es necesario que todos los peptidos del conjunto de aprendizaje tengan la misma longitud. El conjunto de aprendizaje en si se obtiene dividiendo los peptidos binders y los no binders en heptapéptidos. Para el conjunto de no binders se tomaron todas las posibles subsecuencias de longitud 7 como negativos(y se agregaron al conjunto negativo de aprendizaje). Para el conjunto de binders no es tan simple, entonces se utilizo el campo de fuerzas FoldX: se evaluo la energia de union para cada heptapeptido posible(subsecuencias) y se agrega al conjunto de aprendizaje el mejor heptapeptido.(tambien se agregaron aquellos que tenian una enegia de union en un rango de 0.5kcal/mol menor)
% Construccion de la PSSM basada en datos de secuencias: inicialmente se construyeron 2 PSSMs separadas basandose en los conjuntos de aprendizaje positivos y negativos. La frecuencia observada se calculo normalizando el numero de ocurrencias de un dado residuo por el numero de secuencias totales en el conjunto de aprendizaje. La frecuencia esperada es la ocurrencia de residuos obtenida de la base de datos SwissProt. El valor que se usa en la PSSM resultante es el logaritmo de la relacion entre la frecuencia observada y la frecuencia esperada. Se generaron asi una PSSM que representa el perfil de secuencia favorable para la union a DnaK(obtenida a partir del conjunto de binders) y otra PSSM que representa el perfil desfavorable(obtenida a partir del conjunto no binder). Estos datos se integran en una misma PSSM cuyos valores estan dados por la resta de ambos valores(valor binder - valor no binder).
% Construccion de la PSSM basada en datos estructurales: se uso como template la estructura cristalizada de DnaK, junto con la cual estaba co-cristalizado un peptido con la secuencia NRLLLTG, el cual muestra el motivo(estructural?) reconocido por la DnaK, constituido por un minimo de 7 residuos en una conformacion extendida. 
% Para conocer el aporte energetico de cada posible residuo en cada una de las 7
% posiciones se utilizo nuevamente FoldX para hacer un scan posicion por posicion.
% En primer lugar se pusieron todas alaninas. Después se fueron mutando cada
% posicion por los 19 residuos restantes. Para cada uno se calcula el valor de la 
% diferencia energética con el valor del de alanina $\Delta\Delta$G (cuanto mas negativo es este 
% valor mejor es el binding). El valor que se usó para llenar la PSSM es el negativo de 
% este valor $\Delta\Delta$G.
% Dado que los valores se evaluaron mutando las posiciones sobre un backbone fijo,
% este backbone va a influenciar la PSSM resultante. Lo que se hizo entonces fue
% generar distintas PSSMs utilizando múltiples conformaciones de backbones de 
% toda la estructura de la DnaK, obtenidas de un ensamble de conformaciones 
% resueltas por NMR (para cada una se hizo una PSSM y se hizo la evaluación de 
% la ROC). Los resultados de las estructuras del ensamble NMR fueron mucho peores
% que el de la estructura cristalizada y resuelta por rayos X, por lo tanto se uso esta
% solamente
% 
% 
% La evaluación de la performance se hizo mediante 3? tests que se aplican sobre las dos PSSM: la PSSM basada solo en información secuencial y los mismos tests sobre la PSSM que combina información secuencial y estructural. Los tests consisten en calcular el MCC para la evaluación del set de entrenamiento, calcular el MCC mediante una cross-validation(se separan distintos grupos -***EN BASE A QUE???** del set de entrenamiento inicial y se generan nuevos PSSM en base a esto y evaluándolo sobre el resto del conjunto. Se calcula el MCC para cada combinación de grupos y se saca el valor medio), el otro test es calcular el MCC resultante de evaluar contra el conjunto independiente(separado al principio) para la PSSM entrenada con todo el conjunto de pruebas.
% El resultado da que las 2 primeras pruebas son un poquito mejores para la PSSM hecha solo en base a secuencias, pero en la prueba sobre el conjunto independiente la PSSM basada solo en secuencias da muy mal y la PSSM con informacion secuencial es considerablemente mejor. Esto indicaría que la informacion estructural ayuda a hacer el predictor mas general.
% 
% 
% % *****************************************************
% % FALTA DESDE DONDE DICE:  Although the heptapeptides in the learning sets were selected on a methodologically acceptable basis, inconsistencies in the learning set selection could not be excluded.
% 




















\subsection{ANCHOR: predicción de MoREs} \label{anchor}

La herramienta ANCHOR\cite{meszaros2009prediction} busca identificar una clase especial de segmentos desordenados que son capaces de experimentar un proceso de \textit{binding \& folding}, 
propio de la unión a una proteína globular. El objetivo de aplicar esta herramienta como parte de nuestra evaluación se debe, principalmente, a que estos procesos de reconocimiento molecular 
suelen estar asociados a la señalización de distintos procesos biológicos, algo que destacamos como negativo para un linker ya que podría afectar el normal funcionamiendo del sistema biológico en el que se encuentra. 
De esta forma, intentamos detectar cuales son las posiciones que componen estos elementos en la secuencia que estamos evaluando.
% reconocimiento molecular en proteínas. 


Para identificar estos elementos, el método reutiliza el modelo definido para implementar la herramienta IUPred (ver sección \ref{iupred}), en la cual se logra estimar la energía de interacción asociada a cada posición basándose en 
el tipo de aminoácido y la composición del contexto más próximo. Usando esta modelo se pueden calcular, no solo la capacidad de interacción intramolecular, 
sino también la energía de interacción en el contexto de la unión a una proteína globular.
Mediante la diferencia entre estos valores, se pueden identificar cuales son los segmentos que experimentan interacciones favorables en este nuevo contexto y que podrían estabilizar el plegamiento y unión con respecto a la conformación desordenada. 

% The goal of the present work was to recognize a special class of
% disordered segments from the amino acid sequence, namely those
% that are capable of undergoing a disorder-to-order transition upon
% binding to a globular protein partner
% Dado que el método busca identificar una clase especial de segmentos desordenados que son capaces de experimentar un proceso de \textit{binding \& folding}   , 
% se identifican las siguientes propiedades que los segmentos buscados deben cumplir:
% Para esto se plantea, en primer lugar, un conjunto de propiedades que debe cumplir el segmento a identificar:
Los segmentos buscados deben poseer residuos con propiedades específicas:
\begin{enumerate}
 \item El residuo debe pertenecer a una región larga desordenada, es decir, fuera de cualquier dominio globular.
 \item En el estado aislado, el residuo no debe ser capaz de formar uniones favorables con sus vecinos cercanos que le permitan plegarse.
%  (sin unirse a ninguna molécula), asegurándose que el residuo no es capaz de formar uniones suficientemente favorables con sus vecinos cercanos como para plegarse por sí mismo.
 \item El residuo tiene una ganancia neta de energía proveniente de la interacción con proteínas globulares.
%  El tercer criterio tiene en cuenta la capacidad del residuo para interaccionar con proteínas globulares durante la unión a éstas. 
 \end{enumerate}

Cada uno de estos criterios está asociado a un valor numérico. La predicción de los segmentos buscados se basa en una combinación derivada directamente de estos.
% tres criterio, reutilizando los conceptos y parámetros definidos para la herramienta IUPred.
La ecuación asociada tendrá entonces tres componentes que se combinan linealmente:
\begin{enumerate}
 \item El primer componente resulta de promediar los valores de \textit{score} obtenidos directamente de IUPred, en una ventana de tamaño $w_1$ (que deberá definirse como parte de los ajustes de este predictor) alrededor de cada residuo. 
Esto evalúa la tendencia al desorden que tiene el entorno de cada residuo, separando regiones desordenadas de posiciones puntuales que puedan tener cierta tendencia al desorden.
\begin{equation}\label{score1}
 S_k = \frac{1}{N} \sum_{j=b_{lower}}^{b_{upper}} score_j
\end{equation}

donde $N$ es el número de aminoácidos efectivamente contenidos en la ventana para obtener el promedio del residuo con índice $k$, y $b_{lower}$ - $b_{upper}$ los límites de esta.
$score_j$ es el valor obtenido directamente de IUPred para el residuo en la posición $j$.

\item El segundo componente evalúa la ganancia de energía que tendrá el residuo al formar interacciones de a pares con los vecinos contenidos dentro de una ventana de tamaño $w_2$. 
La ecuación asociada es idéntica a la obtenida para IUPred (ecuación \ref{modelo2}), pero el valor de la ventana se redefine como parámetro, el valor del cual será ajustado 
al nuevo predictor de segmentos ANCHOR.
% \begin{equation}\label{score2}
% ddd 
% \end{equation}

\item El tercer componente evalúa la ganancia de energía que tendrá el residuo al formar interacciones de a pares con una proteína globular, 
con respecto a la formación de contactos únicamente entre los vecinos (componente 2). Para hacer esta evaluación, se reutiliza el modelo de IUPred, pero ahora, 
la composición del contexto con el cual se dan las interacciones estará dado por la composición de una proteína globular hipotética. 
Para esto se utiliza la frecuencia de aminoácidos estándar en estas proteínas.
La diferencia resultante entre la interacción con los vecinos propios de la secuencia y este nuevo valor calculado será:
\begin{equation}\label{score3}
E_i^{ganancia,k} = E_i^{intra,k} - E_i^{globular,k}
\end{equation}

\noindent donde $E_i^{intra,k}$ y $E_i^{globular,k}$ representan la energía de interacción de a pares asociada a cada posición, y se calculan usando nuevamente la ecuación \ref{modelo2}, 
con las frecuencias de aminoácidos del contexto de la posición $k$ (en el primer caso) y las frecuencias estándar de proteínas globulares (en el segundo caso).
\end{enumerate}

La ecuación para cada posición $k$ de la secuencia, resultante de la combinación de criterios es:

\begin{equation}\label{scorefinal}
I_k = p_1S_k + p_2E_i^{intra,k} + E_i^{ganancia,k}
\end{equation}


Varios de los parámetros de este nuevo modelo ya fueron determinados previamente utilizando datos conocidos de estructuras de proteínas globulares (durante el desarrollo de IUPred).
Queda determinar cuál es el peso de cada uno en el valor total (coeficientes de la combinación lineal) y los tamaños de la ventanas ($w_1$ y $w_2$, correspondientes a los componentes 1 y 2).
% (cantidad de vecinos) que se tienen en cuenta para hacer el promedio en el componente 1 (w1).
% Valor de la ventana que se tiene en cuenta para evaluar cada el componente 2 (w2).

Para determinar los valores óptimos de estos parámetros se utilizaron dos conjuntos de datos: un conjunto negativo compuesto por cadenas de proteínas globulares y un conjunto de resultados positivos compuesto por complejos formados por segmentos desordenados unidos a proteínas globulares..
% Este último conjunto de datos, representa una seria limitación ya que la cantidad de elementos que se conocen es muy limitada. 
% Dada esta condición, se considera que una ventaja de este método el reducido número de parámetros(5 en total) que se deben evaluar en base a este conjunto de datos.
Cabe destacar que no es posible entrenar el predictor utilizando un conjunto de proteínas desordenadas que se sepa que no forman uniones con proteínas globulares, 
principalmente porque no existe método preciso para comprobar que esto no ocurre. 
A pesar de esta condición, y que la cantidad de datos del conjunto de resultados positivos es considerablemente limitada, una ventaja del método es que contiene sólo 5 parámetros para los cuales se deben evaluar sus valores óptimos. 

% Dado que el resultado de ANCHOR es una combinación de varios aportes, principalmente la tendencia al desorden y la sensibilidad al estar en un entorno estructurado, 
% el resultado obtenido es relativamente independiente del score obtenido únicamente con IUPred. 

Como se define en \cite{dosztanyi2009anchor}, las regiones que tienen un $score>0.5$ se puede tomar como potenciales segmentos de unión y, por lo tanto, asignaremos a cada residuos en estas un puntaje asociado = 1.
% como resultado de la evaluación.
% Dentro de nuestro esquema de evaluación, los segmentos de unión representan posibles pérdidas de flexiblidad en la región linker y, por lo tanto, se toman como propiedades negativas.
% De esta forma, los residuos que posean $score>0.5$ tendrán un puntaje asociado = 1 como resultado de la evaluación.
% la herramienta ANCHOR se aplica utilizando un punto de corte igual a 0.5, como se define en el servidor . 
% Los residuos de la secuencia que tengan un valor asociado mayor a éste se considerarán como posiblemente pertenecientes a segmentos desordenados de unión a proteínas.   
Por ejemplo, la evaluación de la secuencia \texttt{TFSLWKPENMLSPDD} con ANCHOR devuelve los valores de \textit{score} mostrados en la figura \ref{anchorResults}. 

\begin{figure}[ht]
% {\linewidth}
\centering
\includegraphics[width=0.36\textwidth]{img/anchorTabla.png} 
\caption{}
\label{anchorResults}
\end{figure}

Las posiciones 1-6 tienen asociados valores $>0.5$. Aplicando nuestro esquema de evaluación, estos resultados se traducen en la siguiente tabla de puntajes:

\vspace{0.2cm}
% \noindent
\begin{center}
\begin{adjustbox}{width=\textwidth}
\begin{tabular}{llllllllllllllll} 
% \centering
\hline      
Secuencia & \textbf{T} & \textbf{F} & \textbf{S} & \textbf{L} & \textbf{W} & \textbf{K} & \textbf{P} & \textbf{E} & \textbf{N} & \textbf{M} & \textbf{L} & \textbf{S} & \textbf{P} & \textbf{D} & \textbf{D} \\ \hline
Evaluación con ANCHOR & 1 & 1 & 1 & 1 & 1 & 1 & 0 & 0 & 0 & 0 & 0 & 0 & 0 & 0 & 0\\ \hline
\end{tabular}
\end{adjustbox}
\end{center}

























\section{Otros propiedades evaluadas}

Como parte de la herramienta se provee la opción de realizar evaluaciones adicionales sobre la secuencia que permiten restringir la búsqueda más allá de las propiedades intrínsecas que debe poseer una secuencia linker.
Las distintas características ``extra'' que se permiten definir sobre la secuencia complementan el objetivo principal de la herramienta que es proveer secuencias que puedan ser utilizadas 
en el contexto de desarrollos experimentales de biología molecular.
% distintas propiedades de la secuencia que, si bien no hacen directamente a su funcionalidad como linker, 
% complementan las características de la herramienta y 
% el objetivo de las secuencias resultantes como productos que serán utilizados 

Al no formar parte del conjunto de propiedades evaluadas de forma estándar, se debe indicar la aplicación de estas opciones mediante parámetros al iniciar la ejecución, tal como se indica en la sección \ref{evaluacion} del manual.
% Estas opciones, al no formar parte del conjunto de propiedades evaluadas de forma estándar, se deben indicar su aplicación mediante parámetros al iniciar la ejecución, tal como se indica en la sección \ref{evaluacion} del manual.

\subsection{Carga neta de la secuencia}

% Otro de los motivos por las cuales se podría querer eliminar aminoácidos cargados es cuando se esta diseñando una proteína que se une a dna porque los residuos del linker podrían 
% entonces formar puentes salinos con los fosfatos del dna
Esta opción permite que el usuario defina una carga neta deseada para la secuencia linker resultante lo que implica que, en cada paso, se evalúe la secuencia con respecto a la carga. 
Para esta evaluación se tienen en cuenta tres categorias de aminoácidos: con carga positiva (K,R,H), carga negativa (E,D) y neutros.
Al evaluar una secuencia con respecto a la carga neta, puede ocurrir alguno de los siguientes casos:

\begin{itemize}
  \item Si la carga neta deseada y la actual son iguales, todos los puntajes serán 0.
 
\item Si la carga neta a la que se está apuntando es más positiva que la actual, los puntajes resultantes son: 
  \begin{itemize}
   \item  aminoácidos con carga negativa = 2
  \item aminoácidos con carga positiva = 0
  \item aminoácidos neutros = 1
  \end{itemize}
Por ejemplo, si se está evaluando la secuencia \texttt{VKTCLALGVDI} (carga neta=0), y el usuario indicó como requerimiento una carga neta objetivo igual a +2, 
   
 
%  \vspace{0.2cm}
\begin{center}
  \begin{tabular}{llllllllllllllll} 
\hline
Secuencia & \textbf{V} & \textbf{K} & \textbf{T} & \textbf{C} & \textbf{L} & \textbf{A} & \textbf{L} & \textbf{G} & \textbf{V} & \textbf{D} & \textbf{I} \\ \hline
Evaluación carga neta & 1 & 0 & 1 & 1 & 1 & 1 & 1 & 1 & 1 & 2 & 1  \\ \hline
\end{tabular}
\end{center}

 
 \item Si la carga neta a la que se apunta es más negativa que la actual, los puntajes resultantes son:
  \begin{itemize}
   \item  aminoácidos con carga negativa = 0
  \item aminoácidos con carga positiva = 2
  \item aminoácidos neutros = 1
  \end{itemize}

   Por ejemplo, si se está evaluando la secuencia \texttt{VKTCLALGVDI} (carga neta=0), y el usuario indicó como requerimiento una carga neta objetivo igual a -2:
   
%    \vspace{0.3cm}

   \begin{center}
\begin{tabular}{llllllllllllllll} 
\hline
Secuencia & \textbf{V} & \textbf{K} & \textbf{T} & \textbf{C} & \textbf{L} & \textbf{A} & \textbf{L} & \textbf{G} & \textbf{V} & \textbf{D} & \textbf{I}\\ \hline
Evaluación carga neta & 1 & 2 & 1 & 1 & 1 & 1 & 1 & 1 & 1 & 0 & 1  \\ \hline
\end{tabular}

   \end{center}
   
\end{itemize}





\subsection{Absorción UV}

El objetivo de esta opción es que la secuencia resultante no posea ningún aminoácido que absorba en el rango del UV (W,Y,F).
En este caso, ya que se espera que el resultado final no posea ninguno de estos residuos, directamente se eliminan de la composición de aminoácidos que se utiliza para reemplazar las posiciones mutadas. 
% Es decir, al momento de hacer las mutaciones, los residuos T,Y,F poseerán una frecuencia de s
Para eliminar los residuos que se encuentran en la secuencia, se asigna un puntaje de 1 a aquellas posiciones que contengan residuos absorbentes en el rango UV. 
Por ejemplo, si estamos evaluando la secuencia \texttt{VYTCLALGWDI}:

%   \vspace{0.3cm}

\begin{center}
\begin{tabular}{lcccccccccccccc} 
\hline
Secuencia 	& \textbf{V} & \textbf{Y} & \textbf{T} & \textbf{C} & \textbf{L} & \textbf{A} & \textbf{L} & \textbf{G} & \textbf{W} & \textbf{D} & \textbf{I}\\ \hline
Evaluación UV 	& 0 & 1 & 0 & 0 & 0 & 0 & 0 & 0 & 1 & 0 & 0  \\ \hline
\end{tabular}
\end{center}











% 
% \section{Ejemplo completo de evaluación}
% 
% Esta secuencia linker (que se encuentra en las posiciones 202-284) une 2 dominios de la Regulatory protein E2 (del Human papillomavirus type 16):
% Un dominio es el TRANSACTIVATION DOMAIN (posiciones 1-201), el otro es el DNA-BINDING DOMAIN (posiciones 285-365),
% 
% 
% \resizebox{1.2\linewidth}{!}{
% \hspace{-1.5cm}
% \begin{tabular}{ccccccccccccccccccccccccccccccccccccccccccccccccccccccccccccccccccccccccccccccccccccc}
% algo&N&E&V&S&S&P&E&I&I&R&Q&H&L&A&N&H&P&A&A&T&H&T&K&A&V&A&L&G&T&E&E&T&Q&T&T&I&Q&R&P&R&S&E&P&D&T&G&N&P&C&H&T&T&K&L&L&H&R&D&S&V&D&S&A&P&I&L&T&A&F&N&S&S&H&K&G&R&I&N&C&N&S&N \\ \hline
% \end{tabular}
% }
% 







