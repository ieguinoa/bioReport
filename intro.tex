%% Los cap'itulos inician con \chapter{T'itulo}, estos aparecen numerados y
%% se incluyen en el 'indice general.
%%
%% Recuerda que aqu'i ya puedes escribir acentos como: 'a, 'e, 'i, etc.
%% La letra n con tilde es: 'n.

\chapter{Introducci'on}
% 
% Existen dos tipos de citas bibliograf'icas: usa \verb|\citep{..}| para
% citas en \emph{par'entesis} y \verb|\citet{..}| para citas
% en el \emph{texto}. Por ejemplo, estudios reciente han mostrado nuevos e
% interesantes modelos que se pueden aplicar para reformular teor'ias
% f'isicas~\citep{NewCam97}. Mientras que, el trabajo de \citet{Rofl06} fue
% considerado muy divertido por una significativa fracci'on de la comunidad
% de investigadores. Tambi'en es posible citar a varios trabajos en una sola
% referencia \citep{Lamport86,Knuth84}.


% LA INTRODUCCION DEBERIA TENER 2 PARTES:
% 1-UNA PARTE DONDE SE EXPLICA EL PROBLEMA GENERAL DEL CUAL NACE EL ALGORITMO (SENSORES FLUORECENTES, LINKERS, ETC)
% 2- PARTE PODRIA PONER ALGO MUUUUY GENERAL DE METODOS BIOINFORMATICOS
%   EN ESTA PARTE DEBERIA TENER UNA SUBSECCION DE METODOS/ALGORITMOS PROBABILISTICOS

\section{Secuencias linker}

Algunos linkers en la naturaleza se caracterizan por poseer estructuras rígidas que permiten mantener a los dominios funcionales a una distancia mínima(por lo cual se conocen como molecular rulers).



\section{Fundamentos}

