\chapter{Introducci'on}
% 
% Existen dos tipos de citas bibliograf'icas: usa \verb|\citep{..}| para
% citas en \emph{par'entesis} y \verb|\citet{..}| para citas
% en el \emph{texto}. Por ejemplo, estudios reciente han mostrado nuevos e
% interesantes modelos que se pueden aplicar para reformular teor'ias
% f'isicas~\citep{NewCam97}. Mientras que, el trabajo de \citet{Rofl06} fue
% considerado muy divertido por una significativa fracci'on de la comunidad
% de investigadores. Tambi'en es posible citar a varios trabajos en una sola
% referencia \citep{Lamport86,Knuth84}.



\section{Conformación de proteinas}
Breve historia estructura-función, dominios globulares, folding, estados intermedios, IDPs, misfolding, aggregation 

Función?

\section{Arquitectura de proteínas}
dominios funcionales, proteinas multidominio, linkers 

\subsection{Secuencias linker naturales}
Estructura - Composición -  Estudios sobre linkers naturales
% Algunos linkers en la naturaleza se caracterizan por poseer estructuras rígidas que permiten mantener a los dominios funcionales a una distancia mínima(por lo cual se conocen como molecular rulers).

\subsection{Ingeniería de proteínas}
Objetivos, fundamentos, origen, creacion de proteinas multidominio



\subsubsection{Diseño de linkers artificiales}
Metodos y herramientas disponibles




