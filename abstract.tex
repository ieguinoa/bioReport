\prefacesection{Abstract}

The successful construction of multidomain fusion proteins requires a linker sequence to covalently join the
selected domains. Usual requirements for this sequence are to lack any interfering functional feature and the
adoption of an extended conformation, allowing globular domains to move freely
Natural linkers are not always flexible and cannot be considered inert. Furthermore, the diversity of sequences
available in linker databases does not usually fill the properties required by the protein engineering process.
Hence, a rational approach could aid linker design.

This work presents PATENA, a novel method that allows the user to generate linkers de novo from a random 
sequence or starting from a user input sequence.
The initial sequence is evaluated for structured regions using the algorithms IUPRED, ANCHOR, TANGO, PASTA
and WALTZ. Putative functional sites are searched using BLAST and sequence patterns in the ELM and
PROSITE databases. Net charge and UV absorption can also be evaluated at the user’s request. Undesired
structure and functional features are mapped to each sequence position, and the total number of undesired
features is calculated.
Point mutations are iteratively proposed in order to remove all structural and functional features. The mutation is
accepted if the total number of undesired features decreases. If the mutation results in an increased number of
undesired features, the decision is based on a Monte Carlo approach. This method uses a beta parameter to
define the probability of acceptance for a given change in the number of undesired features, where higher beta
values are associated with higher probability of acceptance.

We tested beta values ranging from 0.1 to 2.5, using random (n=3) and natural (n=3) starting sequences of length
30. The algorithm found a suitable linker in every case. The execution time was shorter for smaller beta values,
with a plateau below beta = 1.5 in the minutes timescale. We chose a beta value of 1 for PATENA.
PATENA runs starting from an input set comprising a total of 36 random and natural sequences, with lengths
varying from 5 to 50 residues also found a suitable linker in every case. The execution time increased with
sequence length in an approximately linear manner.

PATENA can find suitable protein linkers in a short execution time. We interpret that the space of suitable linker
sequences is a large fraction of the whole sequence space, while the space of sequences with predicted
structural or functional features is relatively small.
