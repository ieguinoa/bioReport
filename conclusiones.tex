\chapter{Conclusiones y trabajo a futuro} \label{conclusiones}


\begin{itemize}

 \item Hemos implementado un nuevo método que permite obtener secuencias linker flexibles, sin estructura residual, sin interacciones conocidas con otros componentes celulares, sin secuencias repetitivas difíciles de clonar y expresar y con un costo metabólico por aminoácido similar al de las proteínas naturales.

\item El método permite obtener secuencias linker en un tiempo de minutos, adecuado para el trabajo de diseño de proteínas multidominio. Los diseños resultantes son diversos en secuencia, aun partiendo de la misma secuencia inicial. Estos dos resultados son compatibles con la hipótesis inicial de que existe un gran número de secuencias con las características que definimos como deseables para un linker.

 \item El método desarrollado se provee en forma estandarizada y evaluada incluyendo, además, diversas funcionalidades que permiten personalizar la ejecución. De esta forma, se da la posibilidad al usuario de obtener resultados específicos de acuerdo al contexto de aplicación que tendrá el diseño resultante.

 \item En el futuro inmediato se espera poder desarrollar un servidor web que provea la posibilidad de obtener fácilmente secuencias linker. Este paso implica la implementación de una interfaz simple para que pueda ser usada por cualquier usuario experimental, adaptando las funcionalidades del método desarrollado. 
\end{itemize}








% 
% \begin{itemize}
%  \item Hemos implementado un nuevo método que permite obtener secuencias linker en un tiempo totalmente aceptable.
% %  \item El método desarrollado permite ampliar o reducir los requerimientos 
%  \item El análisis de los diseños resultantes muestra una gran diversidad secuencial, a la vez que se permite minimizar el costo metabólico asociado. 
%  Estas propiedades proveen a los diseños una gran aplicabilidad en diversos contextos experimentales. 
%   \item A partir de estas características en la ejecución y los resultados del método 
%   podemos interpretar que el espacio de secuencias que describe a los linkers, tal como los definimos en este desarrollo, es efectivamente un espacio amplio y diverso en relación al espacio total de posibles secuencias.
% %   Al conformar una fracción relativamente grande del espacio total de posibles secuencias, 
% %   Al comprender un buscarlo 
%  \item El método desarrollado se provee en forma estandarizada y evaluada incluyendo, además, diversas funcionalidades que permiten personalizar la ejecución.
% %  de funcionalidades que permiten adaptar .......la ejecución. 
%   De esta forma, se da la posibilidad al usuario de obtener resultados específicos de acuerdo al contexto de aplicación que tendrá el diseño resultante.
% %  El método desarrollado permite al usuario modificar los requerimientos impuestos sobre los diseños resultantes, mediante la evaluación de conjuntos reducidos de propiedades. 
% %   Además el esquema de evaluación permite extender las propiedades evaluadas utilizando nuevas herramientas o modificando la aplicación de los métodos de evaluación incluidos.
%  \item En el futuro inmediato se espera poder desarrollar un servidor web que provea la posibilidad de obtener fácilmente secuencias linker.  
%  Este paso implica la implementación de una interfaz simple para que pueda ser usada por cualquier usuario experimental, adaptando las funcionalidades del método desarrollado. 
% %  aca puedo poner, tambien con respecto al futuro, que el metodo es facil de ampliar con respecto a las propiedades que se evaluan
% \end{itemize}
% 
