\chapter{Conclusiones y trabajo a futuro} \label{conclusiones}




% A FUTURO
% PRUEBAS EXPERIMENTALES DE LOS RESULTADOS

% DESARROLLO DE LA INTERFAZ WEB, IMPLICA SELECCIONAR LOS PARAMETROS QUE SE PONDRAN A DISPOSICION DEL USUARIO









% 
% 
% \section{Nuevas propiedades/herramientas evaluadas}

% LA MODIFICACION DEL CONJUNTO DE EVALUACIONES PUEDE REQUERIR LA REEVALUACION DE LOS PARAMETROS DE EJECUCION(beta)

% 

% Homologias secuenciales mas distantes? ej mediante HMM (usando HMMer por ej.)
% 
% Busqueda de dominios en BBDD: Known domains are now stored in a number of databases including Pfam [1], SMART [13], CDD [14] and InterPro [15]. % ESTAS BBDD ALMACENAN SOLO DOMINIOS GLOBULARES????????
% Particular domains and domain architectures are well conserved over the course of evolution. The sequences diverge, but the overall domain architecture remains the same.
% 
% 
% Protein interaction databases such as BIND and DIP. More informative protein interaction databases that store known instances of linear motifs (36) include MINT (37), Phosphobase (20) and ASC (38). 
% Databases of instances are not directly useful for prediction but provide valuable data-mining resources.
% 
% 
% 
% Analisis de otras condiciones experimentales tales como pH y temperatura:
% For a number of extendedly-disordered proteins it has been shown that a decrease (or increase) in pH induces partial folding of intrinsically unordered proteins due to the minimization of
% their large net charge present at neutral pH, thereby decreasing charge/charge intramolecular repulsion and permitting hydrophobic-driven collapse to the partially-folded conformation
% 
% Importantly, this high temperature and extreme pH stability of ID proteins can be used isolate them from cell extracts
% 
% An increase in temperature often induces the partial folding of intrinsically unstructured
% proteins (i.e., proteins with extended disorder), rather than the unfolding that is typically
% observed for globular proteins. The effects of elevated temperature may be attributed to the
% increased strength of the hydrophobic interaction at higher temperatures, leading to a stronger
% hydrophobic driving force for partial folding
% 
% 
% 
% 
% ELM is already the largest collection of linear motifs, followed
% by PROSITE and Scansite (32). There are other sites that
% specialise on one or a few motifs for which they may provide
% better prediction quality than ELM and should be utilised
% where appropriate. Many functional sites reside in unstructured
% polypeptide regions and the GlobPlot server (http://globplot.
% embl.de/) is useful for revealing sequence segments of
% non-globular character (33), the inverse of the SMART and
% Pfam domain servers. Some useful motif servers are listed in
% Table 2 and the ELM and ExPASy servers list more. Also of
% note are protein interaction databases such as BIND (34) and
% DIP (35). More informative protein interaction databases that
% store known instances of linear motifs (36) include MINT (37),
% Phosphobase (20) and ASC (38). Databases of instances are
% not directly useful for prediction but provide valuable
% data-mining resources.
