\chapter{Conclusiones y trabajo a futuro} \label{conclusiones}

% 
% \begin{itemize}
%  \item Hemos implementado un nuevo método que permite obtener secuencias linker en un tiempo totalmente aceptable.
% %  \item El método desarrollado permite ampliar o reducir los requerimientos 
%  \item El análisis de los diseños resultantes muestra una gran diversidad secuencial, a la vez que se permite minimizar el costo metabólico asociado. 
%  Estas propiedades proveen a los diseños una gran aplicabilidad en diversos contextos experimentales. 
%   \item A partir de estas características en la ejecución y los resultados del método 
%   podemos interpretar que el espacio de secuencias que describe a los linkers, tal como los definimos en este desarrollo, es efectivamente un espacio amplio y diverso en relación al espacio total de posibles secuencias.
% %   Al conformar una fracción relativamente grande del espacio total de posibles secuencias, 
% %   Al comprender un buscarlo 
%  \item El método desarrollado se provee en forma estandarizada y evaluada incluyendo, además, diversas funcionalidades que permiten personalizar la ejecución.
% %  de funcionalidades que permiten adaptar .......la ejecución. 
%   De esta forma, se da la posibilidad al usuario de obtener resultados específicos de acuerdo al contexto de aplicación que tendrá el diseño resultante.
% %  El método desarrollado permite al usuario modificar los requerimientos impuestos sobre los diseños resultantes, mediante la evaluación de conjuntos reducidos de propiedades. 
% %   Además el esquema de evaluación permite extender las propiedades evaluadas utilizando nuevas herramientas o modificando la aplicación de los métodos de evaluación incluidos.
%  \item En el futuro inmediato se espera poder desarrollar un servidor web que provea la posibilidad de obtener fácilmente secuencias linker.  
%  Este paso implica la implementación de una interfaz simple para que pueda ser usada por cualquier usuario experimental, adaptando las funcionalidades del método desarrollado. 
% %  aca puedo poner, tambien con respecto al futuro, que el metodo es facil de ampliar con respecto a las propiedades que se evaluan
% \end{itemize}
% 


\begin{itemize}

 \item Hemos implementado un nuevo método que permite obtener secuencias linker flexibles, sin estructura residual, sin interacciones conocidas con otros componentes celulares, sin secuencias repetitivas difíciles de clonar y expresar y con un costo metabólico por aminoácido similar al de las proteínas naturales.

\item El método permite obtener secuencias linker en un tiempo de minutos, adecuado para el trabajo de diseño de proteínas multidominio. Los diseños resultantes son diversos en secuencia, aun partiendo de la misma secuencia inicial. Estos dos resultados son compatibles con la hipótesis inicial de que existe un gran número de secuencias con las características que definimos como deseables para un linker.

 \item El método desarrollado se provee en forma estandarizada y evaluada incluyendo, además, diversas funcionalidades que permiten personalizar la ejecución. De esta forma, se da la posibilidad al usuario de obtener resultados específicos de acuerdo al contexto de aplicación que tendrá el diseño resultante.

 \item En el futuro inmediato se espera poder desarrollar un servidor web que provea la posibilidad de obtener fácilmente secuencias linker. Este paso implica la implementación de una interfaz simple para que pueda ser usada por cualquier usuario experimental, adaptando las funcionalidades del método desarrollado. 
\end{itemize}











% 
% 
% \section{Nuevas propiedades/herramientas evaluadas}

% LA MODIFICACION DEL CONJUNTO DE EVALUACIONES PUEDE REQUERIR LA REEVALUACION DE LOS PARAMETROS DE EJECUCION(beta)

% 

% Homologias secuenciales mas distantes? ej mediante HMM (usando HMMer por ej.)
% 
% Busqueda de dominios en BBDD: Known domains are now stored in a number of databases including Pfam [1], SMART [13], CDD [14] and InterPro [15]. % ESTAS BBDD ALMACENAN SOLO DOMINIOS GLOBULARES????????
% Particular domains and domain architectures are well conserved over the course of evolution. The sequences diverge, but the overall domain architecture remains the same.
% 
% 
% Protein interaction databases such as BIND and DIP. More informative protein interaction databases that store known instances of linear motifs (36) include MINT (37), Phosphobase (20) and ASC (38). 
% Databases of instances are not directly useful for prediction but provide valuable data-mining resources.
% 
% 
% 
% Analisis de otras condiciones experimentales tales como pH y temperatura:
% For a number of extendedly-disordered proteins it has been shown that a decrease (or increase) in pH induces partial folding of intrinsically unordered proteins due to the minimization of
% their large net charge present at neutral pH, thereby decreasing charge/charge intramolecular repulsion and permitting hydrophobic-driven collapse to the partially-folded conformation
% 
% Importantly, this high temperature and extreme pH stability of ID proteins can be used isolate them from cell extracts
% 
% An increase in temperature often induces the partial folding of intrinsically unstructured
% proteins (i.e., proteins with extended disorder), rather than the unfolding that is typically
% observed for globular proteins. The effects of elevated temperature may be attributed to the
% increased strength of the hydrophobic interaction at higher temperatures, leading to a stronger
% hydrophobic driving force for partial folding
% 
% 
% 
% 
% ELM is already the largest collection of linear motifs, followed
% by PROSITE and Scansite (32). There are other sites that
% specialise on one or a few motifs for which they may provide
% better prediction quality than ELM and should be utilised
% where appropriate. Many functional sites reside in unstructured
% polypeptide regions and the GlobPlot server (http://globplot.
% embl.de/) is useful for revealing sequence segments of
% non-globular character (33), the inverse of the SMART and
% Pfam domain servers. Some useful motif servers are listed in
% Table 2 and the ELM and ExPASy servers list more. Also of
% note are protein interaction databases such as BIND (34) and
% DIP (35). More informative protein interaction databases that
% store known instances of linear motifs (36) include MINT (37),
% Phosphobase (20) and ASC (38). Databases of instances are
% not directly useful for prediction but provide valuable
% data-mining resources.
