\documentclass[a4paper,10pt]{article}
%\documentclass[a4paper,10pt]{scrartcl}
\usepackage[margin=2cm]{geometry}

\usepackage[utf8]{inputenc}

% \title{}
% \author{}
% \date{}
% \pdfinfo{%
%   /Title    ()
%   /Author   ()
%   /Creator  ()
%   /Producer ()
%   /Subject  ()
%   /Keywords ()
% }





\begin{document}
% \maketitle

\textbf{Slide 1: Presentation}\\
My name is Ignacio Eguinoa...\\
I am from protein physiology lab\\
and I will present PATENA, which is an algorithm for the design of protein linker sequences\\
...lets first see what is and why we need a linker sequence *************SWITCH \\



\textbf{Slide 2: Intro}\\
% APROX TIME: 1 min\\
lets say we want to tag a protein using Green fluorescent protein\\ %can provide this (fluorescence or functionality)  \\
we know GFP is (relatively) small and it is inert so, by itself, it will probably not interfere with any biological process\\
we think(design) this construction.... a chimeric protein made of target protein(in this case, tubulin), using a random sequence to connect to GFP.\\
% we know GFP is a relatively small and inert molecule, so, by itself, it will probably not interfere with any biological processes of interest \\
% aca hay dos opciones: \\
% (primera opcion)-so the idea SEEMS simple(o directa, o posible), but the addition of GFP can affect, FOR EXAMPLE, the localization or function of the tagged protein\\
% so, the idea seems simple, but then....when we try to get it working, we find out that, the addition of GFP affected, for example, the function of the target(tagged) protein\\
% so....this simple construction is not working as we thought\\
% .... yes, the problem is in the linker...\\
so, the idea seems simple..... but, in practice, it may not work.....\\
the addition of GFP can affect, for example, the functionality of our target protein IF we dont use a correct linker \\
(OPTIONAL)....so....when designing a chimeric protein, it is important to define the correct units, but it is also relevant to use the CORRECT linker \\
% to have the function we are looking for\\
% (if we want our construction to have the desired function)\\
% we find then, that it is important to use the correct linker to get the results we want\\
% we find, then, that not any sequence can work as a linker but its is important to use the correct sequence to get
% if we want that the resulting construction has the function we want\\   
we know we need a sequence to keep the units together but....what makes a sequence, a correct(good) linker?\textit{*************SWITCH } \\


% GFP is inert by itself, but the addition of it can still
% so, besides selecting 




\textbf{Slide 3: Linker properties}\\
% APROX TIME: 1.15 min\\
% So, linker design is an important step when building chimeric proteins, becuase it has important functions\\
we define a sequence as a good linker when...\\
... allows domains to fold independently, move and interact freely\\
Based on this concept we can define a set of relevant properties:\\
First: the sequence length should allow sufficient distance between units\\
Second: the sequence needs to adopt a flexible conformation and it needs to remain so.\\
...that means, a disordered structure and no aggregation\\
% (OPTIONAL) This property will allow the units to move and interact freely. \\
% (OPTIONAL)To provide this flexibility, the sequence needs to remain in a disordered conformation. \\
% What is more, this property needs to remain constant, so it is important that the linker 
It is also very important that the linker sequence remains biologically inert.....\\
we dont want any kind of interaction with molecules that could interfere with normal processes of biological system.\\
Finally, the linkers will be part of an experimental process, so there can be some other preferences, specifically related to sequence composition \\
% (AAs frequencies, UV silent, net charge.)\\
so, how can we get such sequences? ****SWITCH\\




\textbf{4: current aporoaches}\\
% APROX TIME: 1.15 min\\
At the moment we have 3 different approaches to linker design\\
First we can, of course, use linkers extracted from natural multidomain proteins\\
We can also propose novel designs containing small and disorder-favoring(disorder-promoting) amino acids.\\
Another option is to use different servers available. These servers collect linkers(natural and aritificial) and provide a search method using specific parameters\\
% (length,    ....)

but....\\
we dont know if natural linkers will really have the required conformational propertiess\\
we dont know if the designs we propose will remain biologically inert\\
finally, even if we can find some correct designs, the number of options is very restricted (we will get a very small diversity).\\

Clearly, we dont have a method to design linkers with ALL the desired properties****SWITCH\\


\textbf{5: PATENA scoring function}\\
% APROX TIME: \\
But............... if we have a candidate linker, it is possible to evaluate the presence of each unwanted property\\
% the first step to introduce PATENA is this set of tools to evaluate undesired features in a linker sequence\\

Using IUPRED we can predict tendencies to adopt a globular structure\\
using tango, pasta and waltz we can predict tendencies to aggregate\\
the functional features can be found by means of sequence similarity using blast, also searchin for sequence motifs using elm and prosite, and anchor to search for recognition elements.
other preferences can be evaluated identifieng specific aminoacids\\

this idea of evaluating underisred features is one of the most important parts of PATENA\\
but......how do we use this set of tools?\\
% what do we do with these tools?? ......\\
All these methods provide information (of the presence of undesired features) for each position of the sequence\\
so, it is possible to build a scoring function to quantify the number of(how many) undesired features in the sequence\\
............\\
to do this, we interpret(use) the presence of these features as a binary value*****SWITCH\\ 



% OPTIONAL ...That is, we can evaluate (get an idea, get an estimation of) how good or how bad(quality) is a sequence to work as a linker.

% ACA NO DIGO NADA DEL NOMBRE DEL METODO, TOTAL ESTA EN LA DIAPO
% The main idea behind our method is to define a set of tools to check each of the undesired features.
% This evaluation set is composed of: 
% IUPred, to evalute if the sequence will adopt a globular structure
% we have 3 different methods to evaluate sequence propensity to form insluble aggregates
% functional features are evaluated by means of: BLAST(to try to infer function through sequence similarity), also  ELM resource(to find instances of linear motifs), PROSITE(to find functional sites), and ANCHOR(to find recognition elements)
% Finally, some optional evaluations are:
%      check the presence of ultraviolet aboserbers aminoacids W(Tryptophan), Y(Tyrosine), F(Phenylalanine)
% 	and check the presence of charged residues.


% so....what do we do with this evaluation set?








\textbf{Slide 6: Scoring example}\\
% APROX TIME: 1min \\
% viene de antes.....
....here is an example of application of this scoring function\\
for each method, a value of 1 means the presence of undesired features\\
% each tool provides a value of 1 for each position that has undesired feature. 
if we sum up all the values for each position we have a position score, defining the total number of undesired features.\\
and finally, if we add all the position scores we can get a global score.\\
.....so we have a scoring function for the sequence, where a higher valuer means a worse linker\\
 what can we do with this? ..... ****SWITCH\\



\textbf{Slide 7: PATENA input-output}\\
% APROX TIME: \\
% now, lets get a bit moe into our method.....\\
% given this scoring function the idea behind PATENA is very (simple)intuitive\\
first, we need to get an initial sequence, that is the input... \\
this can be a candidate linker defined by the user, or we can use a random sequence created using just the length....\\
% This can be done starting from a user defined sequence, or a random sequence generated at the start of the method.\\
the method will then try to optimize the score of this sequence to reach a value of zero....\\
that is, we are looking after a sequence with no undesired features\\
% we expect that the output sequence can be used as a linker\\
So, we have here the structure of this method...we will now see how this is done... ****SWITCH\\


\textbf{Slide 8: PATENA schema}\\
% APROX TIME: \\
Once we have an initial sequence\\
we make use of the scoring function to see if the score is zero\\
if NOT...we propose a point mutation, aiming at the higher scoring positions\\
we then reevaluate the sequence, including this mutation to see if it decreses the score\\
if so, we incorporate that mutation and move on\\
if not, it might not be a good idea to discard that mutation\\
so the method uses an heuristic decision, based on how much the mutation incremented the score and also it is based on a parameter beta\\
the result of this decision can be...to incorporate the mutation or to discard it\\
after incorporating a mutation we check if the new score is zero...if not, we repeat the same schema\\
as i said, the final sequence will have score zero , and thus, it is defined as a correct linker\\
just as a general review, we have an initial sequence, we implement an optimization schema proposing point mutations and get the resulting sequence once we reach a minimal value...\\
% What we have here is, again, an optimization of the scoring function.\\
Once we have this algorithm implemented we need to see if we can find an effective beta value that will allow the execution to efficiently find a design.\\


\textbf{Slide 9: Effective beta range}\\
% APROX TIME: 45seg\\
We evaluate the method starting from natural(defined) sequences and also from random sequences, using different beta values\\
First, it is important that the method can find designs in all cases with variable execution times\\
then we can see that the effective range of beta is located below beta value of 2.0\\
This is the range where the execution time is lower\\
Just to move on with evaluation we choose a beta of 1.0.\\
so, from now on, all executions will be made using a beta of 1.0. This is just a decision we made.\\


\textbf{Slide 10: Execution example}\\
% APROX TIME: 45seg\\
we know it works and we have one effective value for beta.\\
We have here an example of execution\\
we see the scrore decreasing after the aplication of mutations\\
how many mutations are required? in this particular execution we have a bit more than two hundred mutations\\ 
but we have a non deterministic method behind so the execution profile can change for different executions\\
one importante aspect(variable) is the length of the initial sequence, this can change the number of mutations required\\
so we will see now, what happens if we variate the length of the initial sequence?? ***** SWITCH\\


\textbf{11: Dependence with length}\\
% APROX TIME: 30seg\\
to evaluate this we ran several executions starting from natural and random sequences of diffferent lengths ....totalling 36 executions\\
what we find is an approximate linear dependence of execution time with sequence length\\
but...we still dont know what kind of results we are getting\\
...what sequences\\
********SWITCH\\


\textbf{12: divergence}\\
% APROX TIME: \\
% So, we have an effective method to obtain linkers and we can use it for different lengths of sequence\\
% another ideal goal is to get a diverse set of results....lets see how different are the results ?
Here we have an histogram of identity between a set of results(a total of 74 designs) obtained from running PATENA using the same initial sequence(these are the green bars).\\
We have also blue bars indicating the identity that can be found between random sequences of the same length\\
Conclusion: we can see, that, the identity between resulting designs is usually higher than that between random sequences.\\
.....it is great to have such diversity but, if the user defines a sequence to start, the idea is to get a at least some similarity in the resulting design. \\
how similar are these results with the initial sequence??  ******** SWITCH DIAPO\\


\textbf{12: divergence 2}\\
% APROX TIME: \\
Using the same set of results we show here(in green bars) an histogram of the identity between the results and the initial sequence.\\
if we compare this with the identity found for a set of random sequences(that is in blue bars) we can see that....\\
although the resulting set is quite diverse(is clearly heterogeneous), still maintain certain similarity with initial sequence.\\



\textbf{13: conclusions}



\end{document}
