\documentclass{beamer}
\usepackage[utf8]{inputenc}
% \usepackage{default}
% \usepackage{paralist}
% \usepackage{enumitem}
\usepackage{changepage}
\usepackage{color, colortbl}
\usepackage{enumerate}
% \setlist{align=left}
\usetheme{Warsaw}
\usecolortheme{beaver}
% \usetheme{Antibes}
\geometry{paper=a4}
\def\Put(#1,#2)#3{\leavevmode\makebox(0,0){\put(#1,#2){#3}}}
\setbeamertemplate{navigation symbols}{}   
\setbeamertemplate{footline}{}   

\definecolor{Gray}{gray}{0.9}
\definecolor{GrayOscuro}{gray}{0.71}
\definecolor{celestito}{rgb}{0.88,1,1}

\title{PATENA: an algorithm for the design of protein linker sequences}
\date{}

\author{Ignacio Eguinoa, Ignacio Enrique Sánchez}
\institute[VFU] % (optional)
{ Protein Physiology Laboratory \\
Departamento de Química Biológica \\
Facultad de Ciencias Exactas y Naturales and IQUIBICEN-CONICET\\
Universidad de Buenos Aires - Buenos Aires, Argentina}

\begin{document}











% *********************************************
%            START 
% *********************************************
\begin{frame}
 \titlepage
% \includegraphics[width=70px]{../img/logofcen.pdf} 
\hspace{4px}
\includegraphics[width=55px,height=50px]{../img/logouba.png}
\hspace{50.7px}
% \centering
\includegraphics[width=70px,height=50px]{../img/logoCONICET.jpg} 
\hspace{15px}
\includegraphics[width=100px,height=50px]{../img/logoLFP.jpeg}
\end{frame}






\begin{frame}[plain]{Chimeric Proteins}
\begin{adjustwidth}{-2.0em}{-2.0em}


Use of GFP as fluorescent tag 


% \vspace{30px}
% Adition of GFP to target proteins\\ % amonitoring cellular processes in living systems\\
\Put(60,-300){ \includegraphics[width=260px]{../img/proteinFusion-GFP-tubulinTransparent.png}}

% \vspace{70px}
Construction: GFP + Linker + Tubulin

\vspace{145px}
Addition of GFP can affect the localization or function of the tagged protein $\rightarrow$ \textbf{importance of correct linker}
\end{adjustwidth}
\end{frame}

% There are several relevant aspects to decide when designing this construction (for example, if the GFP is added in C or N terminal, that is , the order of the elements), 
% one of these aspects is the sequence that will be used to link both proteins
%  in this case is quite relevant that the naturally expressed proten(i.e tubulin) 
%  it has been studied that GFP is a relatively small and inert molecule, that doesn't seem to interfere with any biological processes of interest....
%  but what about the new sequence that function as a linker? these and other considerations build of different uses help to build a definition of a linker sequence...







% \begin{frame}[plain]{Chimeric Proteins}
% \begin{adjustwidth}{-2.0em}{-2.0em}
% 
% \Put(0,-200){\includegraphics[width=190px]{../img/gfp.jpg}}
% % \vspace{-1.5\baselineskip}
% \hspace{180px} Use of GFP as fluorescent tag 
% % \vspace{-20px}
% 
% \vspace{50px}
% % Adition of GFP to target proteins\\ % amonitoring cellular processes in living systems\\
% \Put(130,-240){ \includegraphics[width=220px]{../img/proteinFusion-GFP-tubulinTransparent.png}}
% 
% \vspace{70px}
% GFP + Linker + Tubulin
% 
% \vspace{35px}
% Addition of GFP can affect the localization or function of the tagged protein $\rightarrow$ \textbf{importance of correct linker}
% \end{adjustwidth}
% \end{frame}





% EJEMPLO VIEJO

% \begin{frame}[plain]{Chimeric proteins}
% %  Many proteins in nature are multidomain \\
% %  use of protein engineering process to covalently join 2+ proteins/domains
% %  New construction made of 2+ proteins/domains\\
%  %  \Put(0,10)
% %  \vspace{-3.5\baselineskip}
% 
% 
% % for example, we have here 2 fluorescent domains(Cyan fluorescent protein and yellow fluorescent protein)
% % When 2 fluorescent proteins/domains) are sufficiently separated, they retain their spectral properties.
% 
% % FRET dependence with distance\\
% Cyan Fluorescent Protein(CFP) - Yellow Fluorescent Protein(YFP)\\
%  \begin{adjustwidth}{-1.5em}{-3.5em}
%  \includegraphics[width=170px]{../img/fretDomainsA.png}
%  \includegraphics[width=170px]{../img/fretDomainsB.png}
% \end{adjustwidth}
% Distance-dependent spectral properties\\
%   \vspace{10px}
% Macromolecular crowding sensor:  CFP + Linker + YFP \\
% \vspace{10px}
% % Multiple molecular sensors:  YFP -  
% \includegraphics[width=125px,height=60px]{../img/fretSeparadosCFP.png}
% \includegraphics[width=80px]{../img/crowdingSign.png}
% \includegraphics[width=125px]{../img/fretJuntosCFP.png}
% % one construction(one macrmolecule) function as a sensor
% %   \centering
% %  \includegraphics[width=330px]{../img/fret.png}
% \end{frame}






% These construcions are 

% QUE SON
% ejemplos:
% 	Formación de proteinas bifuncionales mediante union covalente de dominios.
% 	Promover la unión entre proteinas que forman complejos, generando una unión covalente entre ellos
% 
% proceso de sintesis:
% 	-unir los dominios de interes
%       -clone into expression system










\begin{frame}{Relevant linker properties}
% From this example application(fret sensors) we see that, besides defining the domains to be linked, we need to define a linker sequence.
%  a linker is a sequence with a very important function....it allows the domains to remain covalently linker while folding in an independent manner, moving and interacting freely
\vspace{-20px}
\begin{adjustwidth}{-0.5em}{-1.5em}
\textbf{Allows domains to fold independently, move and interact freely}
 \end{adjustwidth}

 \vspace{10px}

\begin{itemize}
  \item Length
  \item Remain in a flexible conformation  %incluye prevencion de aggregation
    \begin{itemize}
     \item Disordered structure, Non-aggregating
    \end{itemize}

  \item Biologically inert  %lack of functional features
    \begin{itemize}
     \item Does not interact with other molecules
    \end{itemize}

  \item Other desirable aspects  %experimental aspects
    \begin{itemize}
     \item AAs frequencies, UV silent, net charge
%   Composition
    \end{itemize}
\end{itemize}

% \Large{....linker design is not a trivial problem}

\end{frame}











\begin{frame}{Current approaches to linker design}
\begin{itemize}

\item Use linkers from natural multi-domain proteins


\item Intuitive design.
%  there are lots of chimeric constructions that were experimentally tested....reuse. may involve small engineering process to adapt the sequence(lenght, composition, etc)
 \begin{itemize}
%   \item Reuse from literature or propose novel sequence. 
  \item Usually (G/S/P)-rich 
%   \begin{itemize}
%    \item  i.e $(GGGS)_n$ 
  \end{itemize}
%     \end{itemize}

\item Pseudo-Rational design. 
\begin{itemize}
 \item Servers: LINKER, Linkerdb, SynLinker.
\end{itemize}

%      DB (natural/previously used designs)
%     these include from different collections obtained in different projects	

%   \item Very limited set of sequences and composition
%  \subitem Requires experimental test (prueba y error) 
% \begin{itemize}
%  \item 
\end{itemize}
\vspace{20px}
\LARGE{Structure? Inert? Diversity?}
% \end{itemize}
\end{frame}















% \section{Method}

\begin{frame}{Rational approach}
% No method can provide a clean design\\ % a design clean of unwanted properties
 
% If we have a candidate linker sequence...
PATENA uses different applications to evaluate a candidate linker.
\vspace{15px}
% all apsects associated with linker properties can be assses by means of different bioinformatic tools available, that can analyze the sequence
% a pesar que los requerimientos no son muuchos, es decir, podemos asumir que gran cantidad de secuencias podrian funcionar como linkers

% We can evaluate (get an idea, get an estimation of) how good or how bad(quality) is a sequence to work as a linker.
% We can predict all unwanted properties.
\begin{tabular}{l|l}
\textbf{Undesired feature} & \textbf{Prediction method} \\ \hline \hline
\rowcolor{Gray} Globular structure & IUPred  \\
Insoluble aggregates & TANGO, PASTA, WALTZ   \\
\rowcolor{Gray}Functional features & BLAST, ELM, PROSITE, ANCHOR \\  
UV absorption(OPTIONAL) & W, Y, F \\ %Presence of UV-absorbers residues \\
\rowcolor{Gray}Net charge(OPTIONAL)& Presence of charged residues\\
\end{tabular}
 
% \Large{Use this information to aid in linker design process}
\vspace{20px} 
% DECIR QUE TODOS LOS MÉTODOS DAN INFORMACION A NIVEL DE POSICION DENTRO DE LA SECUENCIA
%  any of these methods allows to define if a particular position contains a specific negative aspect 
% each of these methods allows to asses if a particular position has that specific negative aspect we are evaluating
% so we can define a scoring function
Each of these methods allows to predict the presence of undesired features at specific position $\rightarrow$ define a scoring function
% Each method predicts the presence of undesired features at position level $\rightarrow$ we can define a scoring function
% \vspace{10px}
 
% once we have a scoring function, we add a method to 

% Use heuristic approach(Monte Carlo): \\
% \textbf{Input}
% (User's sequence / Random sequence) $\rightarrow$ (Sequence with score=0)




\end{frame}






% ESTO ES BASICAMENTE EL FUNDAMENTO DEL MÉTODO
% if we want to use this aspects to evaluate
% para usar esta informacion necesitamos 2 cosas
% 1 - formalize this estimation idea...i mean, we can get automatize and quantify the evaluation 
% AND
% 2 - find a good way to search for candidate sequences , to evaluate if they have the desired properties.
%	 we (probably) cant just start testing random sequences and see if they work...it would be highly inefficient
%	 we cant use natural sequences because, as we said, they dont always fulfill our needs
% 
% if we can find these 2 things we will have a design method
% 
% what is more, in our application these 2 aspects are connected !!!



% ACA EXPLICO EL PUNTO 1 - CUANTIFICACION DE LA CALIDAD 
% \begin{frame}{Linker evaluation}
% \begin{itemize}
% 
% %  Turn the evaluation schema into a function $f(sequence) -> score$
% % quantify using discrete values
%  \item Turn the estimation into a scoring function by calculating the undesired features\\
% % Penalize unwanted properties\\
% % Quantify how good(or bad) is a sequence
% 
% % quantify de quality of the sequence
% % es decir...if we define a function f(face)
% %   
% %   \item Higher score $\rightarrow$ unwanted properties $\rightarrow$ worse linker
%   
%  \item Score $\geqslant 0$ (Higher score $\rightarrow$ bad linker)
% % we dont really care how high or low is the score, we want a linker with score = 0
% \end{itemize}
% 
% % ES IMPORTANTE DESTACAR QUE EL PROCESO DE EVALUACIÓN CAN BE ADAPTED TO ANY RELEVANT ASPECT THAT WANT TO BE ASSESED
% % IF YOU THINK THAT ITS GONNA BE RELEVANT THAT ASSES 
% % \vspace{0.3cm}
% % \begin{adjustwidth}{-1.5em}{-2.5em}
%  
% 
% % \end{adjustwidth}
% \end{frame}





% 
% % dijimos que queriamos usar esta informacion (el score) to aid in linker design
% \begin{frame}{Linker design}
% % the second thing we need is 
% % ahora que definimos un score
% \begin{itemize}
%  \item Aiming at sequence with Global score = 0.
%   
%  \item We have information for each position and the whole sequence
% %  \item We have information of quality(score) at sequence level (detalied for each position).
%  
% \end{itemize}
% 
% %  TENEMOS INFORMACION SOBRE DETALLADA A NIVEL DE RESIDUO(NO SOLO EL SCORE GLOBAL)
% % we can use this information to guide the search
% 
% % so, we propose a search method 
% % We propose a search method: 
% % to search for (optimal) sequence 
% \begin{enumerate}
%  \item Propose tentative linker as starting sequence 
% %     \begin{itemize}
% %     \item Random sequence works too! (\textit{de novo} design)
% %     \end{itemize}
%    
%  \item Propose point mutations to high scoring positions
% %  we will get back to this later
%       \begin{itemize}
%       \item If score decreases $\rightarrow$ incorporate mutation to design.  
%       \item If NOT $\rightarrow$ heuristic decision (Score difference $\rightarrow$  $A_{rate}$).  
%      \end{itemize}
%       
%  \item Repeat step 2 until score=0
% \end{enumerate}
% 
% 
% % $\beta=k_b Temp$  POR LO TANTO, A MAYOR Temp, MAYOR BETA Y POR LO TANTO MAYOR FRECUENCIA DE ACEPTACION
%  
% \Large{Random sequences can work too! (\textit{de novo} design)}\\
% \vspace{7px}
%  
% \Large{Obtaining a design is now an optimization problem}
% 
% 
% % and we are trying to apply an heuristic search method
% 
% % SOLVING AN OPTIMIZATION PROBLEM MIGHT HAVE DIFFERENT SOLUTIONS
% % una aproximacion mediante mutaciones permitiria buscar soluciones similares a una secuencia inicial definida por el usuario
% % ademas, mediante el control de la frecuencia de aminoacidos podriamos controlar la composicion
% 
% \end{frame}


% *********************************************
%      EJEMPLO DE EVALUACION DE SCORE
% *********************************************

% ACA DIGO: YA QUE TENEMOS LAS HERRAMIENTAS DISPONIBLES PARA LA EVALUACION Y UN METODO GENERAL PARA ENCONTRAR UNA SECUENCIA FINAL, DETALLAMOS CUAL ES LA FUNCION DE SCORING
% 
\begin{frame}{Scoring example}
 \begin{tabular}{llllllllllllll} 
\hline
\rowcolor{GrayOscuro}Sequence & \textbf{M} & \textbf{V} & \textbf{L} & \textbf{S} & \textbf{P} & \textbf{A} & \textbf{D} & \textbf{K} & \textbf{T} & \textbf{N} & \textbf{P} & \textbf{D} \\ \hline \hline
 
% Puntaje Inicial & 0 & 0 & 0 & 0 & 0 & 0 & 0 & 0 & 0 & 0 & 0 & 0\\ \hline
\rowcolor{Gray}IUPred           	& 0 & 1 & 1 & 1 & 1 & 0 & 0 & 1 & 1 & 1 & 0 & 0\\ \hline  
TANGO 		       			& 1 & 1 & 1 & 1 & 0 & 0 & 0 & 0 & 0 & 0 & 0 & 0\\ \hline  
\rowcolor{Gray}PASTA			& 1 & 1 & 0 & 0 & 0 & 0 & 0 & 1 & 0 & 1 & 1 & 1\\ \hline  
ELM          	      			& 0 & 1 & 1 & 0 & 0 & 0 & 0 & 1 & 1 & 1 & 1 & 1\\ \hline 
\rowcolor{Gray}BLAST			& 1 & 0 & 1 & 1 & 1 & 0 & 0 & 0 & 0 & 1 & 1 & 1\\ \hline 
PROSITE 	      			& 0 & 0 & 1 & 0 & 0 & 0 & 0 & 0 & 0 & 1 & 1 & 1\\ \hline 
\rowcolor{Gray}ANCHOR	        	& 1 & 1 & 1 & 0 & 0 & 0 & 0 & 0 & 0 & 1 & 0 & 1\\ \hline \hline
\rowcolor{GrayOscuro}Position score     & 4 & 5 & 6 & 3 & 2 & 0 & 4 & 3 & 2 & 6 & 4 & 5\\ \hline
\rowcolor{GrayOscuro}Global Score  & 44 &&&&&&&&&&& \\ \hline
\end{tabular}
\end{frame}






% *********************************************
%      ESQUEMA GRAFICO DE PATENA
% *********************************************
\begin{frame}[plain]{PATENA method}
% \vspace{-0.5\baselineskip}
\begin{adjustwidth}{-2.0em}{-2.0em}
% \begin{adjustheight}{-1.5em}{-1.5em}
% \includegraphics[width=350px,height=250px]{../img/patenaReduced.png} 
\includegraphics[width=350px,height=250px]{../img/patenaReduced-Beta.png} 
% \end{adjustheight}
\end{adjustwidth}
% \vspace{-\baselineskip}
\end{frame}

























% *********************************************
%      BETA vs TIME
% *********************************************
\begin{frame}[plain]{Effective $\beta$ range $\approx$ [0.1 - 2.0]}
\centering
% Optimal $\beta$ range  
[Length = 50] - [Random (n=3) and Natural (n=3) seq. / each $\beta$] \\
\begin{adjustwidth}{-1.5em}{-2.5em}
\includegraphics[width=330px,height=210px]{../img/beta-vs-time-length50-300dpi.png} 
\end{adjustwidth}
\end{frame}





% ***************************************************************
%      SCORE vs MUTACION  PARA 1 SOLA EJECUCION USANDO BETA=1.0
% ****************************************************************
\begin{frame}[plain]{Sample execution profile using defined $\beta$ (1.0)}
% \vspace{-0.5\baselineskip}
\begin{adjustwidth}{-1.5em}{-2.5em}
\centering
% \begin{Verbatim}
% How many mutations are required?  - \hspace{5px} Length = 30   
% Length = 30 residues \hspace{5px} - \hspace{5px}   Effect of $\beta$  
% \end{Verbatim}

\includegraphics[width=330px,height=210px]{../img/iterationVsScore-individualBeta1-EXAMPLE.png}
% \includegraphics[width=330px,height=210px]{../img/scoreVsMutation-individual.png} 

\end{adjustwidth}
% \vspace{-\baselineskip}
\end{frame}







% *********************************************
%      TIEMPO EJECUCION vs LENGTH SECUENCIA
% *********************************************
\begin{frame}[plain]{Approximate linear dependence with length}
\centering
% Optimal $\beta$ range  
% [Length = 30] - [Random (n=3) and Natural (n=3) seq. / each $\beta$] \\
\begin{adjustwidth}{-1.5em}{-2.5em}
\includegraphics[width=330px,height=210px]{../img/lengthVsTime.png} 
\end{adjustwidth}
\end{frame}





% *********************************************
%      DIVERGENCIA: dividido en 2 slides
% *********************************************% 
% 
% 	DECIR SIEMPRE IDENTITY , PORQUE SI DIGO SIMILARITY SE CONFUNDE CON ALINEAMIENTO , ETC, ETC
% 		ACLARAR QUE COMO LOS RESULTADOS TIENEN LA MISMA LONGITUD, LA IDENTIDAD SE EVALUA SIMPLEMENTE COMPARANDO QUE POSICIONES TIENEN EXACTAMENTE EL MISMO RESIDUO
% 
\begin{frame}[plain]{Nondeterministic algorithm: same input $\rightarrow$ different results}
\centering
\vspace{10px}
\begin{adjustwidth}{-2.5em}{-2.5em}
\hspace{10px} 
Fixed starting sequence $\rightarrow$ 74 designs - Identity \textbf{between final sequences}
% mayor que random pero claramente heterogeneas
\end{adjustwidth}
\hspace{10px} 
\includegraphics[width=275px,height=215px]{../img/againstAll-random.png}
% \includegraphics[width=175px,height=145px]{../img/againstInitial-random.png}
\end{frame}


\begin{frame}[plain]{Nondeterministic algorithm: same input $\rightarrow$ different results}
\centering
\vspace{10px}
\begin{adjustwidth}{-2.5em}{-2.5em}
\hspace{10px}
Fixed starting sequence $\rightarrow$ 74 designs - Identity \textbf{with initial sequence}
% mayor que random pero claramente hubo bastantes mutaciones
\end{adjustwidth}
\hspace{10px}
% \includegraphics[width=175px,height=145px]{../img/againstAll-random.png}
\includegraphics[width=275px,height=215px]{../img/againstInitial-random.png}
\end{frame}





\begin{frame}{Conclusion}
\begin{itemize}
 \item PATENA can find suitable protein linkers in a short execution time.
 \item The set of designs that can be obtained from the same sequence shows high diversity. 
 \item Allows for development of server to design linker sequences.
%  \item We interpret that the space of suitable linker sequences is a large fraction of the whole sequence space.
\end{itemize}
\end{frame}

















% *************************************************************
% *************************************************************
% *************************************************************
% 			EXTRA
% *************************************************************
% *************************************************************
% *************************************************************







% *************************************************************
% 		 LOGO  
% *************************************************************

\begin{frame}
\centering
Fixed starting sequence $\rightarrow$ 74 designs\\
\begin{adjustwidth}{-1.5em}{-2.5em}
\includegraphics[width=340px,height=150px]{../img/logo.png}\\ 
\vspace{10px}
\hspace{18px}\includegraphics[width=325px,height=15px]{../img/sequence.png}
\end{adjustwidth}
\end{frame}

\begin{frame}
\begin{adjustwidth}{-1.5em}{-2.5em}
\includegraphics[width=340px,height=150px]{../img/logo.png}\\ 
% \vspace{10px}
\hspace{18px}\includegraphics[width=325px,height=25px]{../img/sequence2.png}
\end{adjustwidth}
\end{frame}










% *************************************************************
%      OBSERVED FREQUENCY vs EXPECTED FREQUENCY
% *************************************************************
\begin{frame}
% {Standard composition deviation}
\centering
Standard composition
% Fixed starting sequence $\rightarrow$ 74 designs\\
\begin{adjustwidth}{-1.5em}{-2.5em}
\includegraphics[width=340px,height=250px]{../img/frequenciesComparison.png}\\ 
\end{adjustwidth}
\end{frame}












% *************************************************************
% 		BETA  vs  MUT. ATTEMPTS + ITERATIONS
% *************************************************************
\begin{frame}
\begin{adjustwidth}{-1.5em}{-2.5em}
% \includegraphics[width=340px,height=250px]{../img/betaVsIterations-MutAttempts.png}\\
\includegraphics[width=340px,height=250px]{../img/beta-vs-Mut-iterations}\\
\end{adjustwidth}
\end{frame}











% *************************************************************
%     ITERATION vs MUT. ATTEMPTS(MEAN)
% *************************************************************
\begin{frame}
\begin{adjustwidth}{-2.0em}{-2.0em}
\includegraphics[width=340px,height=250px]{../img/iterationVsMutAttempts-mean.png} 
\end{adjustwidth}
\end{frame}









% 
% \begin{frame}
% \centering
% Fixed starting sequence $\rightarrow$ 74 designs \\
% \begin{adjustwidth}{-1.5em}{-2.5em}
% \includegraphics[width=340px,height=150px]{../img/mutationsPerPosition.png}\\ 
% % \vspace{10px}
% \hspace{25px}\includegraphics[width=310px,height=10px]{../img/sequence.png}
% \end{adjustwidth}
% \end{frame}
% 





\begin{frame}
\centering
Fixed starting sequence $\rightarrow$ 74 designs - Beta 0.5 , 0.1\\

\begin{adjustwidth}{-1.5em}{-2.5em}
\includegraphics[width=340px,height=80px]{../img/logo.png}\\ 
\includegraphics[width=345px,height=80px]{../img/logoBeta0-1.png}\\ 
\vspace{5px}
\hspace{18px}\includegraphics[width=324px,height=15px]{../img/sequence.png}
\end{adjustwidth}
\end{frame}

\end{document}
