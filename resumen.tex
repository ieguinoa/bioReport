\prefacesection{Resumen}


La construcción exitosa de proteínas multidominio de fusión requiere una secuencia linker para unir covalentemente los dominios elegidos.
Usualmente se requiere que esta secuencia no posea características funcionales que interfieran 
y que la misma adopte una conformación flexible, permitiendo a los dominios globulares moverse libremente.
Los linkers naturales no siempre son flexibles y no pueden ser considerados inertes. 
Además, la diversidad de secuencias disponibles en las bases de datos de linkers no suele cubrir las propiedades requeridas por el proceso de ingeniería de proteínas.
De esta forma, una aproximación racional podría ayudar en el diseño de linkers.



Este trabajo presenta un método que permite al usuario generar linkers \textit{de novo} a partir de una secuencia random o de una secuencia impuesta por el usuario.
La secuencia inicial se evalúa en busca de regiones estructuradas usando IUPRED, TMHMM, TANGO, PASTA, WALTZ, 
y también analizando determinantes secuenciales para la formación de fibrillas amiloides.


Se buscan posibles sitios funcionales utilizando BLAST, patrones secuenciales de las bases de datos ELM y PROSITE, y usando ANCHOR para detectar elementos de reconocimiento molecular.
La carga neta y la absorción UV también pueden ser evaluadas según los requerimientos del usuario.
Las características estructurales y funcionales no deseadas se mapean a cada posición de la secuencia y se calcula el total de características no deseadas.
Se proponen mutaciones puntuales de forma iterativa para quitar todas las características estructurales y funcionales.
Las mutaciones son aceptadas si decrece el número total de características no deseadas.
Si la mutación resulta en un incremento del número de características no deseadas, la decisión se basa en una aproximación de Monte Carlo.
Este método utiliza un parámetro beta para definir la probabilidad de aceptación de un determinado cambio en el número de caracteristicas no deseadas, 
donde un valor mayor de beta se asocia a una mayor probabilidad de aceptación.



Se probaron valores de beta en el rango 0.1 a 2.5, usando secuencias iniciales random (n=3) y naturales (n=3) de largo 30.
El algoritmo pudo encontrar linkers apropiados en cada caso. El tiempo de ejecución fue más corto para valores menores de beta,
con un estancamiento en el orden de los minutos por debajo de beta = 2.0.
De este rango, el valor 1.0 se eligió como estándar para el método.



Diferentes ejecuciones iniciadas a partir de un conjunto compuesto por 36 secuencias random y naturales en total, 
con longitudes variables entre 5 y 50 residuos, también pudieron encontrar linkers apropiados en cada caso.
El tiempo de ejecución se incrementó con la longitud de secuencia en una forma aproximadamente lineal.




Finalmente, se analizaron un total de 74 resultados obtenidos usando la composición de aminoácidos de UniProtKB/Swiss-Prot para las mutaciones, e iniciando a partir una única secuencia inicial.
Se encontró una alta diversidad en el conjunto de resultados. Sin embargo, este mostró una identidad remanente con la secuencia inicial.
Las frecuencias de aminoácidos encontradas en el conjunto de resultados no muestran diferencias significativas con la composición aplicada durante las mutaciones.
Esto permite apreciar la capacidad del método para proveer un conjunto diverso de diseños, a la vez que se minimiza el costo metabólico asociado.
La composición es entonces incorporada por defecto en el método.




Usando los parámetros evaluados, el método puede encontrar linkers apropiados en un corto tiempo de ejecución.
Se interpreta que el espacio de secuencias linkers apropiadas es una gran fracción del espacio completo de secuencias, 
mientras que el espacio de secuencias que se predicen con características estructurales y funcionales es relativamente pequeño.
Como un paso hacia el desarrollo de un servidor web, el método se pone a disposición bajo el nombre de PATENA incluyendo el conjunto de parámetros estándar evaluados.