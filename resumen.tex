\prefacesection{Resumen}


% \prefacesection{Abstract}

% The successful construction of multidomain fusion proteins requires a linker sequence to covalently join the
% selected domains. Usual requirements for this sequence are to lack any interfering functional feature and the
% adoption of an extended conformation, allowing globular domains to move freely
% Natural linkers are not always flexible and cannot be considered inert. Furthermore, the diversity of sequences
% available in linker databases does not usually fill the properties required by the protein engineering process.
% Hence, a rational approach could aid linker design.


La construcción exitosa de proteínas multidominio de fusión requiere una secuencia linker para unir covalentemente los dominios elegidos.
Usualmente se requiere que esta secuencia no posea características funcionales que interfieran 
y que la misma adopte una conformación extendida, permitiendo a los dominios globulares moverse libremente.
Los linkers naturales no siempre son flexibles y no pueden ser considerados inertes. 
Además, la diversidad de secuencias disponibles en las bases de datos de linkers no suele cubrir las propiedades requeridas por el proceso de ingeniería de proteínas.
De esta forma, una aproximación racional podría ayudar en el diseño de linkers.





% This work presents a method that allows the user to generate linkers de novo from a random 
% sequence or starting from a user input sequence.
% The initial sequence is evaluated for structured regions using the algorithms IUPRED, TMHMM, TANGO, PASTA, WALTZ, 
% and also scanning for sequence determinants of amyloid fibril formation. 

Este trabajo presenta un método que permite al usuario generar linkers \textit{de novo} a partir de una secuencia random o de una secuencia impuesta por el usuario.
La secuencia inicial se evalúa en busca de regiones estructuradas usando IUPRED, TMHMM, TANGO, PASTA, WALTZ, 
y también analizando determinantes secuenciales para la formación de fibrillas amiloides.



% Putative functional sites are searched using BLAST, sequence patterns in the ELM and
% PROSITE databases, and using ANCHOR to detect molecular recognition elements. 
% Net charge and UV absorption can also be evaluated at the user’s request. 
% Undesired structure and functional features are mapped to each sequence position, and the total number of undesired
% features is calculated.

Se buscan posibles sitios funcionales utilizando BLAST, patrones secuenciales de las bases de datos ELM y PROSITE, y usando ANCHOR para detectar elementos de reconocimiento molecular.
La carga neta y la absorción UV también pueden ser evaluadas según los requerimientos del usuario.
Las características estructurales y funcionales no deseadas se mapean a cada posición de la secuencia, y se calcula el total de características no deseadas.
% Point mutations are iteratively proposed in order to remove all structural and functional features. The mutation is
% accepted if the total number of undesired features decreases. If the mutation results in an increased number of
% undesired features, the decision is based on a Monte Carlo approach. This method uses a beta parameter to
% define the probability of acceptance for a given change in the number of undesired features, where higher beta
% values are associated with higher probability of acceptance.
Se proponen mutaciones puntuales de forma iterativa para quitar todas las características estructurales y funcionales.
Las mutaciones son aceptadas si decrece el número total de características no deseadas.
Si la mutación resulta en un incremento del número de características no deseadas, la decisión se basa en una aproximación de Monte Carlo.
Este método utiliza un parámetro beta para definir la probabilidad de aceptación de un determinado cambio en el número de caracteristicas no deseadas, 
donde un valor mayor de beta se asocia a una mayor probabilidad de aceptación.





% We tested beta values ranging from 0.1 to 2.5, using random (n=3) and natural (n=3) starting sequences of length
% 30. The algorithm found a suitable linker in every case. The execution time was shorter for smaller beta values,
% with a plateau below beta = 2.0 in the minutes timescale. 
% From this range, the value 1.0 was chosen as standard for the method.

Se probaron valores de beta en el rango 0.1 a 2.5, usando secuencias iniciales random (n=3) y naturales (n=3) de largo 30.
El algoritmo pudo encontrar linkers apropiados en cada caso. El tiempo de ejecución fue más corto para valores menores de beta,
con un estancamiento en el orden de los minutos por debajo de beta = 2.0.
De este rango, el valor 1.0 se eligió como estándar para el método.


% Different executions starting from an input set comprising a total of 36 random and natural sequences, with lengths
% varying from 5 to 50 residues also found a suitable linker in every case. 
% The execution time increased with sequence length in an approximately linear manner.


Diferentes ejecuciones iniciadas a partir de un conjunto compuesto por 36 secuencias random y naturales en total, 
con longitudes variables entre 5 y 50 residuos, también pudieron encontrar linkers apropiados en cada caso.
El tiempo de ejecución se incrementó con la longitud de secuencia en una forma aproximadamente lineal.





% Finally, we analyzed a set of 74 results obtained after using UniProtKB/Swiss-Prot amino acid composition for mutations and starting from a unique input sequence.
% A high degree of diversity is found in the set of results. Nevertheless, it still shows a remaining identity with the initial sequence.
% The amino acid frequencies found in this set of results show no significant difference with the composition applied during mutations. 
% This allows to assess the capacity of the method to provide a diverse set of designs, while minimizing the associated metabolic cost.
% The composition is then incorporated to the method as default.

Finalmente, se analizaron un total de 74 resultados obtenidos usando la composición de aminoácidos de UniProtKB/Swiss-Prot para las mutaciones, e iniciando a partir una única secuencia inicial.
Se encontró una alta diversidad en el conjunto de resultados. Sin embargo, este mostró una identidad remanente con la secuencia inicial.
Las frecuencias de aminoácidos encontradas en el conjunto de resultados no muestran diferencias significativas con la composición aplicada durante las mutaciones.
Esto permite apreciar la capacidad del método para proveer un conjunto diverso de diseños, a la vez que se minimiza el costo metabólico asociado.
La composición es entonces incorporada por defecto en el método.


% Using the evaluated parameters, the method can find suitable protein linkers in a short execution time. 
% We interpret that the space of suitable linker sequences is a large fraction of the whole sequence space, 
% while the space of sequences with predicted structural or functional features is relatively small.
% As a step towards the development of a web server, the method is made available under the name of PATENA including the standard set of evaluated parameters.


Usando los parámetros evaluados, el método puede encontrar linkers apropiados en un corto tiempo de ejecución.
Se interpreta que el espacio de secuencias linkers apropiadas es una gran fracción del espacio completo de secuencias, 
mientras que el espacio de secuencias que se predicen con características estructurales y funcionales es relativamente pequeño.
Como un paso hacia el desarrollo de un servidor web, el método se pone a disposición bajo el nombre de PATENA incluyendo el conjunto de parámetros estandar evaluados.