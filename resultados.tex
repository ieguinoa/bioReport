\chapter{Evaluaciones y análisis de resultados}
\label{results}

\section{Estimación del parámetro Beta}
% PRIMERO EXPLICAR COMO DEPENDE LA EJECUCION CON EL PARÁMETRO BETA
% PARA VALORES MAS CHICOS ...
% PARA VALORES MAS GRANDES...
% DAR UNA IDEA DEL RANGO DE VALORES BETA DE ACUERDO A LA ECUACION, PARA DIFERENCIAS DE SCORE = 1
% MOSTRAR GRAFICO/S DE RELACION ENTRE 
% POR LO TANTO, UN PARAMETRO QUE TIENE EN CUENTA ESTOS ASPECTOS, QUE PODRIA EVALUAR CORRECTAMENTE LA EJECUCION EN FUNCION DE BETA , Y QUE ADEMAS ES EL DATO MAS IMPORTANTE EN LA EJECUCION ES EL TIEMPO DE EJECUCION
En una búsqueda estocástica cómo la que 
El valor(o el rango) óptimo para este parámetro, por lo tanto, depende de la forma que tiene la función sobre la cual estamos realizando la búsqueda.
Una superficie de búsqueda con más cantidad de mínimos locales(y más profundos) requerirá un valor de Beta mayor para poder sobrepasar fácilmente estos durante la búsqueda, 
mientras que una superficie mas lisa con un mínimo global más marcado, probablemente aprovechará la explotación local que puede proveer un valor de Beta más chico. 
Dadas las características estocásticas de la búsqueda, la definición de valores/rangos óptimos se refiere al promedio del tiempo de búsqueda y no a la comparación entre dos ejecuciones puntuales.

En nuestro caso, las propiedades de la superficie de búsqueda están dadas por la función que relaciona el espacio de secuencias y el puntaje resultante del conjunto de evaluaciones detallado previamente.
Esta función, y por lo tanto la superficie resultante, no sólo no se conocen sino que, además, variarán al modificar el subconjunto de herramientas que se utilizan durante las evaluaciones y/o al cambiar los parámetros de éstas.
% En primer lugar, entonces, se debe hacer una estimación del rango óptimo de 

El valor de beta, como se describió en las secciones previas, relaciona el porcentaje de aceptacion con una diferencia en el score evaluado.
Para una misma diferencia
Por ejemplo, para una diferencia de 1 en el puntaje, el porcentaje de aceptacion ....varia entre .....

El valor de beta, entonces, permite balancear entre el número de iteraciones(es decir, las mutaciones aceptadas) y el número de intentos de mutación.  

Para evaluar correctamente cómo la ejecución global depende del valor de beta se utiliza el tiempo total de ejecución. 
El correcto balance entre mutaciones evaluadas y aceptadas dara un menor número total de secuencias evaluadas hasta llegar al mínimo y, por lo tanto, un menor tiempo de ejecución 
% Como variable para evaluar la relación entre los diferentes valores de Beta se analiza el tiempo de ejecución. 
% Este valor es proporcional al total de evaluaciones que se realizaron sobre la secuencia


\section{Análisis del proceso de búsqueda}

% RESULTANDOS DE TIEMPO EN FUNCION DE LONGITUD, NUMERO DE ITERACIONES, INTENTOS DE MUTACION 




\section{Análisis de diseños resultantes}

% DIVERGENCIA EN EL CONJUNTO DE RESULTADOS