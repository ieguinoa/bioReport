\documentclass[oneside,numbers,spanish]{ezthesis}
\usepackage[utf8]{inputenc}
% \usepackage{caratula}
% \usepackage{cite}
\usepackage{comment}
\usepackage{hyperref}
\usepackage{enumerate}
\hypersetup{
    colorlinks,
    linkcolor={red!50!black},
    citecolor={blue!50!black},
    urlcolor={blue!80!black}
}


%% # Opciones disponibles para el documento #
%%
%% Las opciones con un (*) son las opciones predeterminadas.
%%
%% Modo de compilar:
%%   draft            - borrador con marcas de fecha y sin im'agenes
%%   draftmarks       - borrador con marcas de fecha y con im'agenes
%%   final (*)        - version final de la tesis
%%
%% Tama'no de papel:
%%   letterpaper (*)  - tama'no carta (Am'erica)
%%   a4paper          - tama'no A4    (Europa)
%%
%% Formato de impresi'on:
%%   oneside          - hojas impresas por un solo lado
%%   twoside (*)      - hijas impresas por ambos lados
%%
%% Tama'no de letra:
%%   10pt, 11pt, o 12pt (*)
%%
%% Espaciado entre renglones:
%%   singlespace      - espacio sencillo
%%   onehalfspace (*) - espacio de 1.5
%%   doublespace      - a doble espacio
%%
%% Formato de las referencias bibliogr'aficas:
%%   numbers          - numeradas, p.e. [1]
%%   authoryear (*)   - por autor y a'no, p.e. (Newton, 1997)
%%
%% Opciones adicionales:
%%   spanish         - tesis escrita en espa'nol
%%
%% Desactivar opciones especiales:
%%   nobibtoc   - no incluir la bibiolgraf'ia en el 'Indice general
%%   nofancyhdr - no incluir "fancyhdr" para producir los encabezados
%%   nocolors   - no incluir "xcolor" para producir ligas con colores
%%   nographicx - no incluir "graphicx" para insertar gr'aficos
%%   nonatbib   - no incluir "natbib" para administrar la bibliograf'ia

%% Paquetes adicionales requeridos se pueden agregar tambi'en aqu'i.
%% Por ejemplo:
%\usepackage{subfig}
%\usepackage{multirow}

%% # Datos del documento #
%% Nota que los acentos se deben escribir: \'a, \'e, \'i, etc.
%% La letra n con tilde es: \~n.
% 
% \author{Ignacio Eguinoa}
% \title{Implementación de una herramienta bioinformática para el diseño de secuencias linker}
% % \degree{Li}
% \supervisor{Ignacio E. Sánchez}
% \institution{Universidad Nacional de La Plata}
% \faculty{Facultad de Ciencias Exactas}
% 
% 
% \author{Ignacio Eguinoa}
% \tituloTesis{Implementación de una herramienta bioinformática para el diseño de secuencias linker}
% % \degree{Li}
% \director{Ignacio E. Sánchez}
% \institution{Universidad Nacional de La Plata}
% \faculty{Facultad de Ciencias Exactas}
% \lugar{La Plata, 2015}
% 


% \department{Departamento de Sistemas Computacionales}

%% # M'argenes del documento #
%% 
%% Quitar el comentario en la siguiente linea para austar los m'argenes del
%% documento. Leer la documentaci'on de "geometry" para m'as informaci'on.

%\geometry{top=40mm,bottom=33mm,inner=40mm,outer=25mm}

%% El siguiente comando agrega ligas activas en el documento para las
%% referencias cruzadas y citas bibliogr'aficas. Tiene que ser *la 'ultima*
%% instrucci'on antes de \begin{document}.
% \hyperlinking
\begin{document}

\def\titulo{Licenciado }
%\def\titulo{Licenciado }

\def\autor{Ignacio Eguinoa}
\def\tituloTesis{Implementación de una herramienta bioinformática para el diseño de secuencias linker}
\def\director{Dr. Ignacio E. Sánchez}
% \def\codirector{Dr. Mariano Camilo Gonz\'alez Lebrero}
\def\lugar{La Plata, 2015}
% \input{caratula}
%% En esta secci'on se describe la estructura del documento de la tesis.
%% Consulta los reglamentos de tu universidad para determinar el orden
%% y la cantidad de secciones que debes de incluir.

%% # Portada de la tesis #
%% Mirar el archivo "titlepage.tex" para los detalles.
% \include{titlepage}

% \include{caratula}
%% # Prefacios #
%% Por cada prefacio (p.e. agradecimientos, resumen, etc.) crear
%% un nuevo archivo e incluirlo aqu'i.
%% Para m'as detalles y un ejemplo mirar el archivo "gracias.tex".

\prefacesection{Resumen}


La construcción exitosa de proteínas multidominio de fusión requiere una secuencia linker para unir covalentemente los dominios elegidos.
Usualmente se requiere que esta secuencia no posea características funcionales que interfieran 
y que la misma adopte una conformación flexible, permitiendo a los dominios globulares moverse libremente.
Los linkers naturales no siempre son flexibles y no pueden ser considerados inertes. 
Además, la diversidad de secuencias disponibles en las bases de datos de linkers no suele cubrir las propiedades requeridas por el proceso de ingeniería de proteínas.
De esta forma, una aproximación racional podría ayudar en el diseño de linkers.



Este trabajo presenta un método que permite al usuario generar linkers \textit{de novo} a partir de una secuencia random o de una secuencia impuesta por el usuario.
La secuencia inicial se evalúa en busca de regiones estructuradas usando IUPRED, TMHMM, TANGO, PASTA, WALTZ, 
y también analizando determinantes secuenciales para la formación de fibrillas amiloides.


Se buscan posibles sitios funcionales utilizando BLAST, patrones secuenciales de las bases de datos ELM y PROSITE, y usando ANCHOR para detectar elementos de reconocimiento molecular.
La carga neta y la absorción UV también pueden ser evaluadas según los requerimientos del usuario.
Las características estructurales y funcionales no deseadas se mapean a cada posición de la secuencia y se calcula el total de características no deseadas.
Se proponen mutaciones puntuales de forma iterativa para quitar todas las características estructurales y funcionales.
Las mutaciones son aceptadas si decrece el número total de características no deseadas.
Si la mutación resulta en un incremento del número de características no deseadas, la decisión se basa en una aproximación de Monte Carlo.
Este método utiliza un parámetro beta para definir la probabilidad de aceptación de un determinado cambio en el número de caracteristicas no deseadas, 
donde un valor mayor de beta se asocia a una mayor probabilidad de aceptación.



Se probaron valores de beta en el rango 0.1 a 2.5, usando secuencias iniciales random (n=3) y naturales (n=3) de largo 30.
El algoritmo pudo encontrar linkers apropiados en cada caso. El tiempo de ejecución fue más corto para valores menores de beta,
con un estancamiento en el orden de los minutos por debajo de beta = 2.0.
De este rango, el valor 1.0 se eligió como estándar para el método.



Diferentes ejecuciones iniciadas a partir de un conjunto compuesto por 36 secuencias random y naturales en total, 
con longitudes variables entre 5 y 50 residuos, también pudieron encontrar linkers apropiados en cada caso.
El tiempo de ejecución se incrementó con la longitud de secuencia en una forma aproximadamente lineal.




Finalmente, se analizaron un total de 74 resultados obtenidos usando la composición de aminoácidos de UniProtKB/Swiss-Prot para las mutaciones, e iniciando a partir una única secuencia inicial.
Se encontró una alta diversidad en el conjunto de resultados. Sin embargo, este mostró una identidad remanente con la secuencia inicial.
Las frecuencias de aminoácidos encontradas en el conjunto de resultados no muestran diferencias significativas con la composición aplicada durante las mutaciones.
Esto permite apreciar la capacidad del método para proveer un conjunto diverso de diseños, a la vez que se minimiza el costo metabólico asociado.
La composición es entonces incorporada por defecto en el método.




Usando los parámetros evaluados, el método puede encontrar linkers apropiados en un corto tiempo de ejecución.
Se interpreta que el espacio de secuencias linkers apropiadas es una gran fracción del espacio completo de secuencias, 
mientras que el espacio de secuencias que se predicen con características estructurales y funcionales es relativamente pequeño.
Como un paso hacia el desarrollo de un servidor web, el método se pone a disposición bajo el nombre de PATENA incluyendo el conjunto de parámetros estándar evaluados.
\prefacesection{Abstract}


The successful construction of multidomain fusion proteins requires a linker sequence to covalently join the
selected domains. Usual requirements for this sequence are to lack any interfering functional feature and the
adoption of an extended conformation, allowing globular domains to move freely.
Natural linkers are not always flexible and cannot be considered inert. Furthermore, the diversity of sequences
available in linker databases does not usually fill the properties required by the protein engineering process.
Hence, a rational approach could aid linker design.

This work presents a method that allows the user to generate linkers de novo from a random 
sequence or starting from a user input sequence.
The initial sequence is evaluated for structured regions using the algorithms IUPRED, TMHMM, TANGO, PASTA, WALTZ, 
and also scanning for sequence determinants of amyloid fibril formation. 

Putative functional sites are searched using BLAST, sequence patterns in the ELM and
PROSITE databases, and using ANCHOR to detect molecular recognition elements. 
Net charge and UV absorption can also be evaluated at the user’s request. 
Undesired structure and functional features are mapped to each sequence position, and the total number of undesired
features is calculated.
Point mutations are iteratively proposed in order to remove all structural and functional features. The mutation is
accepted if the total number of undesired features decreases. If the mutation results in an increased number of
undesired features, the decision is based on a Monte Carlo approach. This method uses a beta parameter to
define the probability of acceptance for a given change in the number of undesired features, where higher beta
values are associated with higher probability of acceptance.

We tested beta values ranging from 0.1 to 2.5, using random (n=3) and natural (n=3) starting sequences of length
30. The algorithm found a suitable linker in every case. The execution time was shorter for smaller beta values,
with a plateau below beta = 2.0 in the minutes timescale. 
From this range, the value 1.0 was chosen as standard for the method.
% We chose a beta value of 1 for .

Different executions starting from an input set comprising a total of 36 random and natural sequences, with lengths
varying from 5 to 50 residues also found a suitable linker in every case. 
The execution time increased with sequence length in an approximately linear manner.
% PATENA runs starting from an input set comprising a total of 36 random and natural sequences, with lengths
% varying from 5 to 50 residues also found a suitable linker in every case. The execution time increased with
% sequence length in an approximately linear manner.



Finally, we analyzed a set of 74 results obtained after using UniProtKB/Swiss-Prot amino acid composition for mutations and starting from a unique input sequence.
A high degree of diversity is found in the set of results. Nevertheless, it still shows a remaining identity with the initial sequence.
The amino acid frequencies found in this set of results show no significant difference with the composition applied during mutations. 
This allows to assess the capacity of the method to provide a diverse set of designs, while minimizing the associated metabolic cost.
The composition is then incorporated to the method as default.


%Finally, we analyzed a set of 74 results obtained after using UniProtKB/Swiss-Prot amino acid composition for mutations and starting from a unique input sequence.
%The corresponding results allow to assess the capacity of the method to provide a diverse set of designs, while minimizing the associated metabolic cost.
%This composition is then incorporated to the method as default.

%Finally, we analyzed a set of 74 results obtained after using UniProtKB/Swiss-Prot amino acid composition for mutations and starting from a unique input sequence.
%The amino acid frequencies found in this set of results show no significant difference with the composition applied during mutations.  
%This allows to assess the capacity of the method to provide a diverse set of designs, while minimizing the associated metabolic cost.
%This composition is then incorporated to the method as default.

Using the evaluated parameters, the method can find suitable protein linkers in a short execution time. 
We interpret that the space of suitable linker sequences is a large fraction of the whole sequence space, 
while the space of sequences with predicted structural or functional features is relatively small.
As a step towards the development of a web server, the method is made available under the name of PATENA including the standard set of evaluated parameters.

% \prefacesection{Agradecimientos}
Prometo ser breve...

A Nacho por aceptar la dirección de esta tesis, pero más que nada por enseñarme todo para dar mis primeros pasos como investigador. 
Me llevo mucho más que lo que se puede leer en este trabajo.

A toda la gente de LFP por los consejos y por bancarse las miles de exposiciones que hice presentando este trabajo.

A Silvina y Gustavo, no sólo por aceptar ser jurados de este trabajo sino también por todo lo que me enseñaron de bioinformática. 
En cada problema que encaro me doy cuenta lo valioso que es todo lo aprendido en su materia. 

A mis viejos y mis hermanos por bancarme en todo, aunque todavía no entiendan muy bien que es lo que hago y no se acuerden el nombre de la carrera.

A mis amigos del colegio, que durante la carrera se aguantaron que muchas veces no esté presente porque ``estoy estudiando, tengo que rendir!'' o ``estoy escribiendo la tesis les juro que falta poco para recibirme!''.

A mis amigos de la facu, porque sin ellos el paso por la facu no hubiera sido lo mismo.  

A todos los profesores que me dejaron algo.

A esta Universidad y en particular a esta facultad por la calidad de educación recibida.



%% # 'Indices y listas de contenido #
%% Quitar los comentarios en las lineas siguientes para obtener listas de
%% figuras y cuadros/tablas.
\setcounter{secnumdepth}{4}
\setcounter{tocdepth}{4}
\tableofcontents
%\listoffigures
%\listoftables

%% # Cap'itulos #
%% Por cada cap'itulo hay que crear un nuevo archivo e incluirlo aqu'i.
%% Mirar el archivo "intro.tex" para un ejemplo y recomendaciones para
%% escribir.

% CAPITULO 1: INTRODUCCION
\chapter{Introducci'on}
% 
% Existen dos tipos de citas bibliograf'icas: usa \verb|\citep{..}| para
% citas en \emph{par'entesis} y \verb|\citet{..}| para citas
% en el \emph{texto}. Por ejemplo, estudios reciente han mostrado nuevos e
% interesantes modelos que se pueden aplicar para reformular teor'ias
% f'isicas~\citep{NewCam97}. Mientras que, el trabajo de \citet{Rofl06} fue
% considerado muy divertido por una significativa fracci'on de la comunidad
% de investigadores. Tambi'en es posible citar a varios trabajos en una sola
% referencia \citep{Lamport86,Knuth84}.



\section{Conformación de proteinas}
\begin{itemize}
 \item proteinas globulares
 \item folding
 \item estados intermedios de plegamiento
 \item IDPs
 \item misfolding + aggregation + amyloid formation
\end{itemize}


The structure-function paradigm claims that a specific function of a protein is determined by its unique and rigid three-dimensional (3D) structure. 
Thus, following its biosynthesis on the ribosome, a protein must fold to a native, defined 3D structure to be functional. 
The primary origin of this structure-function paradigm is the “lock and key” hypothesis formulated in 1894 by Emil Fischer to explain the astonishing specificity of the enzymatic hydrolysis of glucoside multimers by different types
of similar enzymes. For a long period of time, the validity of “lock and key” model and its associated sequence-structure-function paradigm was unquestioned, especially after the crystal structures of proteins started to be solved by X-ray diffraction

Este paradigma se deriva de los primeros avances en el estudio de estructuras de proteinas. Over the past century evidence steadily accumulated that a well-defined structure is the prerequisite of protein function.



Basic biology and biochemistry textbooks that explain biological phenomena at the molecular level exquisitely rely on this notion, the ‘structure–function paradigm’.

Although deviations from this norm were always apparent, they had been invariably neglected or ignored.
Mas adelante en el tiempo se fue encontrando que, además de las estructuras globulares funcionales mas estudiadas, proteins also populate various other states under native, functional conditions, including disordered and partially ordered conformations, and different aggregated assemblies.
Although it was evident from early studies that both soluble and fibrillar forms of proteins could exist (33 ????) , most attention was focused on the soluble states that were found to possess demonstrable biological function.




En el grafico (*** poner grafico de proteostasis que tiene un estado 'nativo' , un estado 'misfolded' y los demas estados de agregacion, etc) se ve que al sintetizarse, 
la proteina naturalmente tiende a ir adquiriendo una conformacion estructural (conjunto de conformaciones mas restringida).
Lo que se fue desarrollando en los ultimos años es que el estado funcional no pertenece(siempre) a una estructura altamente definida sino que puede pertenecer a un conjunto de conformaciones que van desde
estado plegado hasta estados desestructurados, que pueden confundirse con estados totalmente desordenados. 
% fall onto a structural continuum, from tightly folded single domains, to multidomain proteins that might have flexible or disordered regions, to compact but disordered MOLTEN GLOBULES and, finally, to highly extended, heterogeneous unstructured states 
Thus, not just the ordered state but any of the known polypeptide conformations can be the native state of a protein.

Las sintesis de proteínas en el ribosoma puede verse mejor como el punto de inicio para que cada proteina tome una ruta particular a medida que emerge y se acopla al entorno celular. 
Entre las posibles rutas esta la adquisicion de un continuo de estructuras conformacionales funcionales, la adquisicion de una estructura distinta a la funcional(misfolding), y una variedad de estados de agregacion.

% adquiera un continuo de estructuras conformacionales funcionales a medida q emergen y se acoplan al entorno celular.% adoptando la estructura que la caracteriza .

Es por esto que las propiedades conformacionales de una proteína quedan mejor definidas si se declara el conjunto de multiples estados accesibles por sus estructuras.
% A este continuo de conformaciones que pueden adoptar las proteinas individuales, se agregan distintos tipos de estados agregados, conformados a partir de uniones entre proteinas en distintas conformaciones iniciales(conformaciones desestructuradas, intermedios o plegados).
De acuerdo con esta forma de ver, esta accesibilidad estará dada por la estabilidad termodinámica de cada conformacion accesible y la cinética de interconversión entre estas.
Estos requerimientos pueden indicar por ejemplo q ciertos estados altamente estables como puede ser las fibras amiloides no sean estados naturales comunes debido a los requerimientos cinéticos, a pesar de ser termodinamicamente estables.
% De acuerdo con esta forma de describir las posibles conformaciones de las proteinas, the various fates awaiting a polypeptide chain once it has been synthesized in the cell 
% will depend on the kinetics and thermodynamics of the various equilibria between di¡erent possible states.

Las propiedades cinéticas y termodinámicas están dadas por el contexto celular (pH, iones, concentracion de proteinas, presencia de otras moleculas con las que interactúa, etc) y, obviamente, por los posibles cambios que puede tener una proteina,
ya sean mutaciones o modificaciones postraduccionales.
% Las propiedades estructurales de una proteina, descritas de esta forma, pueden modificarse para una proteina a partir de cambios en el contexto(pH, iones, concentracion de proteinas, presencia de otras moleculas con las que interactúa, etc) 
% y, obviamente, también a partir de mutaciones o modificaciones quimicas sobre ésta. 
Esta forma de ver las propiedades estructurales permite ver los origenes de los cambios conformacionales desde el punto de vista de las propiedades fisicoquimicas de las proteinas.
If the stability or cooperativity of the native state of a protein is reduced, for example by a mutation, the population of non-native states will increase.


Es importante destacar, entonces que, for a given polypeptide chain a chosen fate is not a final one, and a choice may be further modulated by environmental pressure. 
Thus, intrinsically unstructured proteins may be forced to fold or misfold via modification of their environment (addition of natural binding partners, changes in properties of solvent and so on), whereas a destabilizing environment may push a natively
folded protein to the misfolding route. Alternatively, the
presence of chaperones may reverse the misfolding route
and effectively dissolve small aggregates




Durante la última década, uno de los mayores cambios en el campo de las proteínas ha sido el reconocimiento de la ubicuidad e importancia de las proteínas intrínsecamente desordenadas. 
% \cite(INTRINSICALLY UNSTRUCTURED PROTEINS AND THEIR FUNCTIONS)
Las proteinas en solucion fall onto a structural continuum, from tightly folded single domains, to multidomain proteins that might have flexible or disordered regions, to compact but disordered MOLTEN GLOBULES and, finally, to highly extended, heterogeneous unstructured states (FIG. 1) 
% This continuum has been interpreted in terms of a ‘PROTEIN TRINITY’ (ordered, molten globule and RANDOM COIL 20 ) or ‘PROTEIN QUARTET ’ 
As the main criterion of a native protein is its ability to perform a biological function, these partially or completely disordered proteins must be regarded as native entities.

% Si bien decimos aca que las proteinas pueden adquirir estructuras correspondientes un ensamble de estructuras accesibles , es posible hacer una clasificación entre proteinas que adoptan estructuras plegadas(con un estado nativo definido por un pequeño conjunto de estructuras) 
% y proteínas intrínsecamente desestructuradas (que adoptan estructuras correspondientes a un gran ensamble continuo de conformaciones). Luego se tratará las conformaciones agregadas de proteínas como una dimensión conformacional distinta.







\subsection{Proteinas globulares}
% En el caso de proteinas altamente estructuradas, la búsqueda de la estructura nativa no es un proceso trivial, y ha promovido la investigación de estructura de proteínas durante muchos años
El proceso mediante el cual una proteína adquiere (rápidamente) 









\subsection{Proteinas intrínsecamente desestructuradas}

Varios reviews en \cite{uversky2010understanding,dyson2005intrinsically}.

The suggestion that the native state of many proteins is intrinsically disordered (or, as originally termed, unstructured) is now integral to our general view of protein structure and function.
Como se dijo en la introduccion, a little more than 10 years ago,however, such challenge to the almost dogmatic ‘structure–function paradigm’ was pure heresy due to the overwhelming evidence that structure determines function.
A decade of steady progress turned skepticism around.
Como parte de estos avances, en los ultimos años se han encontrado gran cantidad de proteinas que adoptan estructuras cuyas conformaciones son total o parcialmente desordenadas(segmentos globulares), soportando esta definicion, formalizada a principios de milenio. 
the transition in paradigm was enforced by scattered experimental observations of disorder in a few dozen proteins. 
% una de las dudas que se tenia era si este estado conformacional existia solo in vitro y en realidad in-vivo lo que ocurria era que el crowding generaba el plegamiento
Evidence added to that obtained by other techniques [mostly NMR and circular dichroism (CD)]..... and the evidence seems overwhelming now that structural disorder also exists in vivo and it is truly the native, functional state of these proteins.


Estas secuencias presentan en solución un conjunto heterogéneo de conformaciones fácilmente maleable por cambios en el medio. 
En lugar del análisis estructural clásico basado en dominios globulares, podemos describir estos conjuntos conformacionales en términos de los promedios y desviaciones estándar de la distancia entre los extremos de la secuencia, el radio hidrodinámico, el contenido en estructura secundaria y otros parámetros (20).

Ten years ago, only predictor of natural disordered regions (PONDR) was available [14]; today, one can use any of about 50 predictors, which are based on several different principles \cite{he2009predicting}.
Como parte de estos ultimos años de investigacion se encontro una gran cantidad de funcionalidades y mecanismos asociados a proteins IDP(ver seccion funciones).
Se desarrollaron tambien bases de datos específicas (ver disprot, y IDEAL: Intrinsically Disordered proteins with Extensive Annotations and Literature. ) y aplicaciones bioinformáticas capaces de identificar el desorden intrínseco (19).

% The identification of many IDPs  enabled the development of sophisticated bioinformatic algorithms for predicting disorder from sequence, which further advanced the field.
The availability of such predictors further advanced the field.
Based on predictions, we know that structural disorder is abundant in all species, and due to its strong correlation with regulatory and signaling functions, its level is significantly higher in eukaryotes than in prokaryotes
Although it has become almost commonplace in the field that structural disorder increases with the complexity of the organism, the highest levels are not witnessed in the most complex metazoan eukaryotes (e.g., in humans), but in single-celled eukar-
yotes that lead a host-changing lifestyle

% TERMINOLOGIA
En un principio se comenzo a definir a este tipo de secuencias como desestructuradas, indicando que prescindian completamente de estructura, aun sabiendo que they have potentially function-related
short- and long-range structural organization, which eventually called upon a change in terminology.
At that time, high-resolution data were rather limited, thus the concept was mostly phrased from the global structural level, which suggested that IDPs fall into coil-like, pre-molten globuletype and molten-globule types
% aca deberia ir algo sobre estudios estructurales que se han hecho
These and many other studies made the term unstructured obsolete. 
% IMPORTANTE
The distinguishing and unifying feature of these proteins – if any – is their inability to fold into a unique and stable tertiary structure
The terms intrinsically unstructured and natively unfolded may be also be suitable for extended random coils and even those that are collapsed, but these terms don't seem to appropriately describe proteins that form transient or
stable secondary structure. The term disorder suffers because of its negative connotation and its possible confusion with a pathological state, yet, on the other hand, disorder can be used for proteins like the molten globule that form substantial secondary structure but that
nevertheless are highly dynamic and non-uniform. For this last reason, herein we will call these proteins “intrinsically disordered” (ID).
By “intrinsic disorder” we mean that the protein exists as a structural ensemble, either at the secondary or at the tertiary level.

% DISORDER vs FLEXIBILITY:
Disorder and flexibility are often used synonymously, but the two terms are quite distinct\cite{radivojac2004protein}. With regard to an ordered protein, flexibility refers to the magnitudes of the excursions of the atoms from their equilibrium positions.
For a disordered region, variation in flexibility refers to differences in the speed of interconversion among the various members of the structural ensemble. A variety of methods have been
used to investigate the flexibilities of disordered regions and proteins, including NMR.


% PROPIEDADES ESTRUCTURALES
A combination of experimental and theoretical studies has clarified how given sequences define specific free energy surfaces that enable folding.
A diferencia de estas superficies energeticas, caracteristicas de las estructuras con plegamiento definido, in some cases the native state of a given peptide or protein may not be structured in a globular form but disordered.

In general, proteins with intrinsically disordered sequences cannot bury sufficient hydrophobic core to fold spontaneously into the highly organized 3D structures that characterize the proteins that are represented in the Protein Data Bank (see the online links
box). In some cases, compact but disordered molten-globule-like states can be formed, or local regions of the sequence can have a propensity to adopt isolated and fluctuating elements of secondary structure (which is equivalent to the ‘pre-molten globule’ proposed by Uversky 2 ). 

Structural disorder can now be studied in great detail by several dozen experimental techniques [21], and the most spectacular advance has been achieved through the application of multidimensional NMR. This approach is often
complemented by other structural techniques, such as small-angle X-ray scattering (SAXS), which is combined with advanced computational data integration based upon molecular dynamics (MD) simulations (Figure 1). These new
approaches [22] enabled the characterization of the full structural ensemble of several dozen IDPs

It is probable that proteins rarely, if ever, behave as true random coils, especially in non-denaturing media: even in their most highly unfolded states, proteins show a propensity to form local elements of secondary structure or hydrophobic clusters






%%% 
% .............

Todas estas propiedades hacen que parezca razonable que este tipo de proteínas este altamente regulada dentro de la célula \cite{gsponer2008tight}.

















\subsection{Misfolding \& Aggregation}

% The sequences of proteins have evolved in such a way that their unique native states can be found very efficiently even in the complex environment inside a living cell. 

contrary to the process of productive protein folding, leading to the appearance of rigid conformation with specific function, the end products of misfolding may have a different appearance. 
The morphology of these end products depends on the particular experimental conditions, and misfolded product may appear as soluble oligomers, amorphous aggregates or amyloid-like fibrils. 
Any of these three species could be cytotoxic, thus giving rise to the development of pathological conditions. 
The reason for such a morphological difference is potentially connected with the diversity of the partially folded intermediates favoring protein self-association. 
In fact, multiple environmental factors, such as point mutations, decrease in pH, increase in temperature, the presence of small organic molecules or metal ions, and other charged molecules, might induce structural rearrangements within a protein molecule, shifting equilirium toward the partially folded conformation(s). As different factors may stabilize slightly different partially folded intermediates, the formation of morphologically
different aggregates is expected.


Como se dijo en las secciones cuando se habló sobre el proceso de plegamiento, las secuencias de proteinas con estructuras plegadas have evolved in such a way that their unique native states 
can be found very efficiently even in the complex environment inside a living cell.

Sin embargo, as the size and complexity of proteins increase, therefore, the folding process becomes more complex. 
Intermediates with only partially formed structures can be populated and have significant lifetimes. 
Under some conditions, then, proteins fail to fold properly or to remain correctly folded; this misfolding can lead to the development of different pathological conditions.
In addition, events that may be termed 'misfolding' may take place during the search for the stable native-like contacts between residues. 
That such complexities are seen even in the benign environment of a dilute solution of a pure protein suggests that they are even more likely to occur in the crowded environment of the cell.

Protein misfolding is a wide-spread phenomenon. Any protein with changes in native structure which affect its normal function is misfolded. 
The terms 'misfolded' and 'aggregated' are not equivalent. Como se verá en las proximas secciones, la habilidad de una proteina formar agregados desestructurados o fibras depende de muchos factores, including protein sequence and environment.
De otra forma, las proteinas que adquieran conformaciones no estructuradas o que se encuentren en estados de plegamiento intermedios semi-estables tendrían una gran tendencia a la agregación, sin embargo esto no es así. 
De hecho, sólo una pequeña fraccion de tales conformaciones tiene tendencia a formar agregados







% **************** AGGREGATION

Proteins might aggregate into ordered(fibrillar amyloids) or amorphous aggregates structures, utilizing relatively short sequence stretches, usually organized in b-sheet-like assemblies. 
These differences in structure also reflect biological differences; amyloid and amorphous beta-sheet aggregates have different chaperone affinities, accumulate in different cellular locations and are degraded by different mechanisms.

La mayoria de las proteínas forma agregados desordenados, caracterizados por la falta de una estructura tridimensional regular(amorfos).
En condiciones fisiologicas, sin embargo, varias proteinas pueden agregarse formando estructuras altamente ordenadas conocidas como fibras amiloides.
Si bien no se van a dar detalles puntuales de la estructura, se sabe que estas diferencias que se ven en la estructura macromolecular son el reflejo de diferencias en las interacciones y estructura a nivel atómico.


De forma gráfica, los equilibrios de agregacion pueden verse en la figura \ref{aggregationDiagram}

\begin{figure}[h!,centered]
\includegraphics[width=\textwidth]{img/aggregationDiagram.png} 
\caption{Equilibrio de los estados de agregación} \label{aggregationDiagram}
\end{figure}



De la figura puede verse también, que pueden ocurrir distintos procesos de agregación, que resultan en disintas estructuras agregadas, a partir de distintas estructuras adoptadas por las proteinas.

The critical step in the aggregation process is the unfolding of the native structure. 
En la mayoria de las proteinas, excepto las mas pequeñas, el 'unfolding' que ocurre en condiciones fisiologicas no lleva a una estructura totalmente desplegada sino que la proteina adquiere una estructura semi-estable parcialmente collapsada,
donde las interacciones no son las mismas que en el estado nativo estructurado, estas son caracteristicas propias de los intermediarios del plegamiento.
La formacion de estos intermediarios es importante porque generalmente son mucho mas solubles que si se formaran conformaciones altamente desplegadas. 
Esta solubilidad permite alcanzar las concentraciones requeridas para la nucleacion propia de la formacion de amyloids(y de agregados en general????)





% ***********************************************************
% ******* PROPIEDADES DE FORMACION DE AMYLOIDS
% ***********************************************************

Amyloidogenic proteins are quite diverse, with little similarity in sequence and native three-dimensional structure.
Additionally, several proteins and peptides not related to amyloidoses have the potential to form amyloid fibrils in vitro, suggesting that this ability for structural rearrangement and aggregation may be inherent to proteins.
Even though the ability to form amyloid fibrils seems to be generic(is a general property of the polypeptide backbone), the propensity to do so under given circumstances can vary markedly between different sequences(depends enormously on amino acid composition.).

This picture enables us to speculate on the origins of
the amyloid diseases from the point of view of the
physico-chemical properties of the protein molecules. If
the stability or cooperativity of the native state of a
protein is reduced, for example by a mutation, the popu-
lation of non-native states will increase.
This rise will increase the probability of aggregation, as the
concentration of polypeptide chains with at least partial
exposure to the external environment will be greater.
Whether or not aggregation does occur will depend on
the concentration of protein molecules, the intrinsic
propensity for a given sequence to aggregate when
unfolded, and on the rate of the aggregation process. The
fact that formation of ordered amyloid fibrils can be
seeded, like the well-studied processes of crystallization
and gelation, means that once the aggregation process is
initiated it often proceeds very much more rapidly.
In the absence of seeding there can be
long 'lag' phases before aggregation occurs .
This lag can be thought of as arising because the growth
of a fibril cannot occur until a 'nucleus' of a small number
of aggregated molecules is formed. Such a nucleus can be
formed by the local fluctuations in concentration that
occur in solution as a result of random molecular motion.
When such fluctuations result in a local concentration of
molecules above a critical value, the molecules associate
with one other to form a species that is suficiently large
to have intrinsic stability, and hence to grow in size by
interacting with other molecules in the solution. The act
of seeding provides such nuclei to the solution and hence
reduces or abolishes the lag phase







% **********************
% AGREGADOS AMORFOS
% **************************

Amorphous beta-sheet aggregation, is less position-dependent(than amyloid aggregation) and can, in principle, be achieved by any sequence that can adopt an extended conformation, is sufficiently hydrophobic and has no unsatisfied hydrogens or electostatic groups. 
Thus beta-sheet aggregation can be relatively easily predicted by methods that evaluate aggregation by evaluating biophysical parameters over a sequence segment, without the need for considering position-dependent values of these parameters.
















\subsection{Herramientas y mecanismos de proteostasis}
En secciones previas(en la introduccion) se dijo que el espacio de conformaciones accesibles por las proteinas dependia de las estabilidades termodinamicas de los estados y la cinetica de interconversión entre estos. 
Estas propiedades, a su vez, dependen del contexto celular en el que se encuentran(ademas de la secuencia de la proteina en si misma), concentraciones de proteinas, localizacion, interacciones, etc.
El balance correcto entre todas estas condiciones está regulado a traves del proceso de proteostasis.
Proteostasis refers to controlling the concentration, conformation, binding interactions (quaternary structure), and location of individual proteins making up the proteome by readapting the innate biology of the cell, 
often through transcriptional and translational changes.
The protein components of eukaryotic cells face acute and chronic challenges to their integrity. Eukaryotic protein homeostasis, or proteostasis, enables healthy cell and organismal development and aging and protects against disease
Los mecanismos dentro del proceso de proteostasis permiten successful organismal development and aging in the face of constant intrinsic and environmental challenges %previniendo el desarrollo de enfernedades.

La proteostasis es influenciada por todos los mecanismos que controlan los aspectis vistos en las secciones previas..... by the chemistry of protein folding/misfolding and by numerous regulated networks of interacting and competing biological pathways 
that influence protein synthesis, folding, trafficking, disaggregation, and degradation.

La celula posee distintos mecanismos para regular estos procesos .....chaperonas para regular el proceso de folding/misfonding (y tambien de aggregation), 
Nature developed very sophisticated protection mechanisms (chaperones, proteasome, etc.) for the effective regulation of the folding process, y por lo tanto prevenir las peligrosas consecuencias de misfolding. 
In other words, in the cell, any given protein is not acting in the isolation, is never alone and is
constantly "watched" by the protective machinery. This machinery is rather robust and can
tolerate significant loads. Obviously, factors that affect these protective mechanisms will
contribute to the probability of disease development


-Cells possess a complex proteostasis network (PN) to ensure protein homeostasis.
-Aggregates permanently engage molecular chaperones and other PN components.
-The PN is challenged by chronic stress in protein-aggregation diseases and aging.
-Overtaxing the PN drives a vicious cycle of disease progression with eventual proteostasis collapse.



Las diferencias estructurales vistas entre los distintos estados de agregación tambien se ven reflejada en diferencias biológicas 
Whereas misfolded and aggregated proteins are found in perinuclear locations and are generally degraded by the 
proteasomal system, amyloids preferentially accumulate in perivacuolar inclusions, where they are degraded by the autophagosome.
% This segregation directly results from a differential recognition by the protein quality control system. 
Esta segregación resulta en un tratamiento distinto por parte de los mecanismos encargados de la proteostasis.
A reduced affinity of amyloids for the protein quality control system is probably also at the root of their higher toxicity 11 as artificial overexpression of chaperones
usually leads to decreased toxicity and removal of amyloids from he cell.
Contrary to amorphous aggregates, amyloid structures can fulfill biological functions, and functional amyloids are found in organisms from prokaryotes to humans \cite{fowler2007functional}.





% PREVENCION DE AGREGACION EN FOLDED PROTEINS
the normal folding process may pass through
partially folded states on the route to the fully native state,
but the aggregation of these species will be minimized by
the presence of molecular chaperones. In addition, if the
protein is able to fold rapidly, any partially folded species
will have a short lifetime, reducing the probability of inter-
molecular interactions occurring. Moreover, once folded,
the native state is generally a highly compact structure
that conceals the polypeptide main chain within its
interior. Such a state is protected from aggregation except
through the interactions of surface side chains (as is the
case, for example, in protein crystals) and is unable to
form the strong intermolecular hydrogen bonds associated
with the polypeptide backbone. Provided that the native
state is maintained under conditions where it remains
folded, aggregation to amyloid fibrils will be resisted by
the kinetic barrier associated with unfolding, even if the
aggregated state is thermodynamically more stable.
Importantly, the cooperative nature of protein structures
means that virtually none of the polypeptide chain in indi-
vidual molecules is locally unfolded, and that virtually no
molecules in an ensemble are globally unfolded, even
though native proteins are only marginally stable relative
to denatured ones under normal physiological conditions


% ACA EXPLICO POR QUE LAS IDPs NO TIENEN TANTA TENDENCIA A FORMAR AGREGADOS
In contrast to the globular proteins, which have to unfold prior to aggregation (Jahn and Radford, 2005), IDPs are always ready for intermolecular interactions. 
An unbound fragment of an IDP possesses a strong ability to interact, and therefore can bind either to natural partners forming native complexes or to similar molecules forming various aggregates. 
This raises the question of why IDPs do not always form aggregates in the norm. One of the potential answers is the fact that inside the cell, the IDPs typically form complexes with natural partners.
Los mecanismos de proteccion (chaperonas, proteasoma, etc) tienen una gran capacidad para mantener la solubilidad de proteinas.
Además, based on the analysis of the IDP amino acid composition, it would be clearly a mistake to assume that an averaged IDP possesses higher propensity towards aggregation than an averaged ordered
protein. In fact, many IDPs contain large number of charged and polar residues. In addition to the hydrophobic interactions, the net charge is one of the major factors determining aggregation behavior of a protein.













\section{Características funcionales}
Breve historia estructura-función, motivos secuenciales, funcionalidad en proteinas IDPs


This means that the structure-function paradigm, which emphasizes that ordered 3D structures represent an indispensable prerequisite to effective protein functioning, should be redefined to include intrinsically unstructured proteins [159]. According to this redefined paradigm, native proteins (or their functional regions) can exist in any of the known conformational states, ordered, molten globule, premolten globule and coil. Function can arise from any of these conformations
and transitions between them. Thus, not just the ordered state but any of the known polypeptide conformations can be the native state of a protein.




% FUNCIONALIDAD EN IDPs
The most important question of the field, therefore, is the physiological function and functional mode of IDPs/IDRs.
It is now clear that structural disorder provides multiple functional advantages\cite{gunasekaran2003extended,dyson2005intrinsically},
and IDP functions either directly stem from their disorder (entropic chains) or from molecular recognition, when they undergo induced folding (disorder-to-order transition) upon binding to a partner molecule
The functional role of structural disorder from a biological process view addresses what type of cellular functions benefit most from the lack of a stable structure. This
question was addressed in several large bioinformatics studies. As a result, IDPs are generally thought to be involved in processes of signaling and regulation, and in depth correlation analysis [52] of 710 Swiss-Prot functional keywords suggested significant positive correlation with 238 functions and negative correlation with 302 functions (Table 2). Most of the functions that correlate with the
presence of long disordered regions are related to regulation via transcription and translation, whereas functions
that correlate with the lack of disorder are dominated by
enzymatic catalysis.

Algunas propiedades interesantes de las proteinas IDPs se revelan a partir de estudios a nivel de proteoma y sistema.
As suggested, IDPs/IDRs often function by binding accompanied by induced folding (molecular recognition) [6–8] mediated by SLiMs/ELMs, and in several recent
works it has been shown that the functional and evolutionary agility of IDPs can be ascribed to the inclusion or exclusion of such motifs in RNA maturation; that is, by alternative splicing, alternative promoter usage, and RNA
editing, the alternative isoforms thus generated promote the functional diversification of the proteome
This
mechanism can result in a change in diverse functional attributes, such as subcellular localization, protein–protein interaction, phase transitions and even opposing (dominant negative) function, as also demonstrated by
studying tissue-specific forms of alternative splicing. These protein isoforms tend to occupy central positions in interaction networks and their pattern of interaction partners tend to significantly differ [63]; that is, structural disorder
and encoded motifs have a strong potential to define and redefine wiring of cellular signaling pathways.













\section{Arquitectura de proteínas}
dominios funcionales, proteinas multidominio, linkers 


% Many cellular processes involve proteins with multiple domains.
Many eukaryotic proteins are modular — that is,
they contain independently folded globular domains
that are separated by flexible linker regions. 

The modular nature of proteins has many advantages,
providing increased stability and new cooperative functions.
Other advantages include the protection of intermediates within
inter-domain clefts that may otherwise be unstable in aqueous
environments and the fixed stoichiometric ratio of enzymatic
activity necessary for a sequential set of reactions


Protein domains can be defined as segmented por-
tions of a polypeptide sequence that assume stable
three-dimensional structure. 6 Such recurring protein
motifs are significant because it is increasingly recog-
nized that there are only a limited number of domain
families in nature.

These domains are duplicated and
combined in different ways to form the set of proteins
in genomes. 10 The importance of domains is further
exemplified by the fact that multidomain proteins
play a major role in many cellular processes.

Although a consensus in detail is still lacking, various
effective criteria have been proposed to detect and
define protein domains. These criteria rely mainly on
the existence of local structural compactness arising
from beta-sheets or hydrophobic cores. 11,12 Based on
such compactness, computational algorithms to detect
structural domains have been proposed








\subsection{Secuencias linker naturales}
Estructura - Composición -  Estudios sobre linkers naturales
% Algunos linkers en la naturaleza se caracterizan por poseer estructuras rígidas que permiten mantener a los dominios funcionales a una distancia mínima(por lo cual se conocen como molecular rulers).


Linker
sequences vary greatly in length and composition, but
many are rich in polar, uncharged amino acids (such as
Ser, Thr, Gln and Asn), in the small residues Ala and Gly,
and in Pro residues. Many of these residues tend to bias
the polypeptide chain towards the polyproline-II region
of the RAMACHANDRAN PLOT 27,28 .This means that such
linkers, although flexible, have a propensity to be highly
extended. Compositionally biased linker sequences of
significant length are found mainly in eukaryotic pro-
teins 1,29 , but short linker sequences of similar composi-
tion, known as Q-linkers, are also found in a number of
bacterial regulatory proteins 30 .
In the absence of their targets, modular proteins
often behave as ‘beads on a flexible string’, where the
function of the linker is, primarily, to enable a relatively
unhindered spatial search by the attached domains 31 .
However, binding can induce structure formation in
linkers, which can have significant functional conse-
quences. For example, the sequence-specific binding of
CYS HIS ZINC-FINGER PROTEINS to DNA causes the linker to
fold, cap and thereby stabilize the preceding helix in the
protein, and to orientate the next zinc finger correctly
for binding in the major groove of DNA



\subsection{Ingeniería de proteínas}
Objetivos, fundamentos, origen, creacion de proteinas multidominio



\subsubsection{Diseño de linkers artificiales}
Metodos y herramientas disponibles






% CAPITULO 2: ESQUEMA GENERAL DEL ALGORITMO
% CAPITULO 1: EXPLICO EL ESQUEMA GENERAL DEL ALGORITMO

\chapter{Herramienta desarrollada}
\label{method}


\section{Fundamentos del método utilizado}
\label{fundamentos}

\subsection{Detección de propiedades a partir del análisis secuencial}

% Mapeo de los objetivos con existencia de herramientas disponibles.

% PRIMERO QUE BUSCAMOS EN REALIDAD
% Para entender los fundamentos del método implementado primero debemos analizar mejor los detalles del objetivo.
% El requerimiento fundamental de la secuencia linker resultante es que esta posea la flexibilidad necesaria para que los dominios que conecta se muevan libremente.
% Para poder lograr esto la secuencia debe adoptar y mantener \textit{in-vivo} las características de un amplio ensamble conformacional que no restrinja la estructura, 
% lo cual está asociado a las propiedades conformacionales de las proteínas intrínsecamente desordenadas. 
% % donde las diferentes conformaciones desordenadas se intercambian rápidamente de manera continua 
% % Por lo tanto, tenemos un panorama más claro acerca de cuales serán las características conformacionales de la secuencia que esperamos obtener.
% Como se detalló en la introducción,  han permitido comprender cuales son las características que distinguen a estos elementos proteicos. 
% Las propiedades estudiadas sirven para implementar una gran cantidad de herramientas bioinformáticas capaces de distinguir es



% 
% 
% % Como vimos en la sección \ref{linkerDesign} la flexibilidad no es todo y, por eso, 
% Otra parte fundamental del objetivo es que el linker obtenido se mantenga inerte frente a cualquier actividad biológica que pueda 
% interferir en el proceso se expresión o en la correcto funcionamiento de la proteína resultante del diseño.
% % Una parte relevante de esta secuencia objetivo es su inercia frente al pipeline de 
% En la sección \ref{functionalLandscape} se describieron distintos elementos funcionales que pueden estar presentes en una secuencia.
% Clasificando, aislando y analizando estos elementos hemos podido comprender mejor cuales son sus propiedades. 
% Lo que lo objetivo requiere, entonces, es que la secuencia resultante este libre de cualquiera de estos elementos permaneciendo, preferentemente, inerte ante cualquier actividad o interacción posible \textit{in-vivo}.
% 
% % Teniendo en claro los objetivos en términos de los posibles elementos estructurales y funcionales, lo que el método requiere es, entonces, obtener una secuencia que adopte ,que esté limpia de elementos estructurales 
% A lo largo del capítulo 1, hemos presentado una gran cantidad de conocimientos obtenidos acerca de las propiedades secuenciales y su relación con la conformación y función resultante.
% Los avances mostrados en este campo se trasladan a una gran cantidad(y variedad) de métodos que permiten predecir la ocurrencia de estas propiedades sobre cualquier secuencia polipeptídica.





En la sección \ref{aspectosDiseno} se desarrollaron distintos aspectos estucturales y funcionales que se deben tener en cuenta en el diseño de un linker.
A partir de esta descripción podemos distinguir diferentes características que afectan positiva y negativamente a la funcionalidad del diseño resultante.
Esto permite definir una secuencia linker como aquella que posea todas las características positivas y no posea ninguna de las característica negativas buscadas.
Es importante recordar que, como se describió previamente, algunas de las características están en función de consideraciones hechas por el usuario experimental.

En la sección \ref{modulosProteicos} describimos distintos módulos proteicos encontrados en la naturaleza, identificando los elementos conformacionales y funcionales asociados.
El estudio de estos elementos generalmente apunta, no sólo a modelizar sus propiedades asociadas (mecanismos funcionales, estructuras, propiedades secuenciales asociadas, etc.), 
sino también a poder trasladar los conocimientos a herramientas bioinformáticas que permitan predecir la ocurrencia de estos.
Estas propiedades están directamente asociadas a la secuencia de la proteína. Además, la secuencia es lo primero que se conoce cuando se encuentra o diseña una nueva proteína, por lo tanto,  
como objetivo general se intenta poder obtener la mayor cantidad de información a partir de ésta.
Como consecuencia, existen cada vez más herramientas, que utilizando aproximaciones distintas permiten detectar diversas características a partir de la secuencia.

% Como vimos en la sección \ref{modulosProteicos} estas características están asociadas a distintos elementos proteicos estudiados a partir de su ocurrencia en proteínas naturales.
% Los elementos estructurales y funcionales y su  
% Todas estas características estructurales y funcionales han sido estudiadas en el contexto de los módulos proteicos encontrados en proteinas naturales.
% los cuales se muestran en las primeras secciones de este capitulo.
% Como vimos en la sección \ref{modulosProteicos}, los distintos elementos conocimientos obtenidos permiten comprender los aspectos      y cómo 
% Los conocimientos obtenidos(desarrollados en la sección \ref{modulosProteicos}) permiten, no sólo entender estas propiedades, sino también desarrollar 
% diferentes herramientas bioinformáticas utilizadas luego para predecir la ocurrencia de estas propiedades a partir de la secuencia polipeptídica.
% Los estudios han resultado en un amplio conocimiento de las propiedades secuenciales y 
% Los avances en este campo se trasladan a una gran cantidad de métodos que, a través de diferentes aproximaciones, permiten predecir estas propiedades a partir de la secuencia polipeptídica.

Una parte fundamental del método implementado en este trabajo es que actualmente podemos predecir, con cierta confianza, 
la ocurrencia de los distintos aspectos positivos y negativos asociados a un linker conociendo solamente la secuencia de aminoácidos de una proteína.
% Esto tiene como corolario que podremos decir si una secuencia resultará o no en un buen candidato para funcionar como linker, usando estas Usando la definición de linker que recien vimos parametrizadas con las características que , 
Lo que es más, las predicciones pueden ser mapeadas a posiciones puntuales dentro de la secuencia. 
Por ejemplo, podemos localizar dentro de la secuencia cuales serán las posiciones 
que favoreceran la adopción de una conformación intrínsecamente desordenada, lo cual es una característica deseable para el diseño de un linker.







\subsection{Espacio de secuencias buscadas}\label{espacioSecuencial}

% *****************PROPIEDADES DEL ESPACIO DE BUSQUEDA 

% las propiedades conformacionales pueden ser un especto muy importante si se intenta diseñar proteínas con estructura plegada definida
De todos los requerimientos de diseño vistos en la sección \ref{aspectosDiseno} el más restrictivo es, en principio, que el diseño resultante tenga la flexibilidad necesaria. 
Esto implica que la secuencia se ajuste a las propiedades de una proteína intrínsecamente desordenada, las cuales identificamos previamente como un conjunto de secuencias 
claramente sesgadas en su composición, con baja complejidad y generalmente con baja abundancia de varios tipos de aminoácidos.
Esto parece indicar que el espacio de secuencias comprendido por éstas debe ser significativamente pequeño con respecto al espacio total de posibles soluciones(todas las posibles combinaciones de aminoácidos),
y por lo tanto obtener un diseño que encaje dentro de este espacio deja pocas opciones y hace difícil la búsqueda de la secuencia resultante.

Sin embargo, la realidad es más compleja de lo que parece.
Las IDPs no requieren que su secuencia posea un código definiendo el plegamiento y la estructura nativa final. 
Además, vimos que las IDPs se caracterizan por tener secuencias conteniendo múltiples elementos estructurales y funcionales relativamente cortos, formando un mosaico. 
De esta forma, el espacio abarcado por las secuencias que poseen las propiedades conformacionales que buscamos, puede no ser tan restringido como se pensabamos y, de hecho,
puede tener un tamaño considerable con respecto al espacio total de posibles secuencias.
Dentro de este espacio, el resto de las propiedades positivas que buscamos y de las propiedades negativas que queremos evitar, no restringe demasiado el número de soluciones
ya que, como vimos, probablemente esten limitadas a segmentos muy cortos sobre la secuencia.

Asumimos, entonces, que el conjunto de diseños posibles, constituido por todas las secuencias que poseen las propiedades positivas y 
no poseen las negativas, tiene un tamaño considerable con respecto al espacio de posibles soluciones compuesto por todas las combinaciones de aminoácidos.
Este conjunto, sin embargo, pude reducirse considerablemente si se tienen en cuenta algunos aspectos definidos por el usuario 
como una composición muy especifica, una carga neta elevada, o una combinación de cualquiera de éstos. 















\section{Esquema general de la implementación}




% De esta forma, estamos en una situación en la que: tenemos un conjunto de herramientas que nos permiten mapear cada secuencia con distintos comportamientos (deseado o no deseados) de acuerdo a nuestros objetivos y, además, 
% podemos asumir que el conjunto de secuencias que cumple con los principales comportamientos buscados es considerablemente grande con respecto al espacio de soluciones posibles (cualquier combinación de AAs).



A partir de los aspectos detallados en los fundamentos, una primera aproximación obvia sería realizar una búsqueda sistemática dentro del espacio 
de secuencias hasta obtener una que resulte favorable de acuerdo a nuestros parámetros de evaluación.  
El problema radica en que es una forma ineficiente de búsqueda, tiene un alto costo computacional y el resultado(asumiendo que lo encontramos) será una composición al azar.
Una aproximación más adecuada,  sería utilizar todo el conocimiento que tenemos para poder evaluar la secuencia y usar esto a nuestro favor para guiar la búsqueda. 
Al guiar la búsqueda utilizando los resultados de las evaluaciones, esta sera mucho más efectiva y el costo computacional sera mucho menor.
Además, según el punto de inicio que utilicemos para la búsqueda, podremos intentar obtener secuencias resultantes con cierta similitud a la secuencia inicial.

% La existencia de una función ideal implicaría que esta represente la dependencia de cada posicion con el contexto de la secuencia, con respecto a la propiedad analizada.
% Si bien las herramientas disponibles de predicción nos son las ideales, generalmente brindan información sobre cada posición y no sobre la secuencia como un todo.

% Uilizando esta información podríamos analizar que posiciones no son acordes con el comportamiento/propiedades deseado/a y utilizar esta información para guiar la búsqueda, modificando solo aquellas posiciones que no nos resultan favorables.
 
% Si bien asumimos que el conjunto de secuencia que cumplen las condiciones estandar del resultado es grande con respecto al especio total de soluciones, no conocemos la forma funcional que no esta guiando la busqueda,
% es decir, como esta depende con la secuencia.  Para implementar un mecanismo eficiente de búsqueda debemos tener en cuenta este desconocimiento.

El procedimiento general del algoritmo consiste en aplicar mutaciones puntuales iterativamente a partir de una secuencia inicial, en busca de una secuencia final con las características deseadas.
En cada iteración se proponen y analizan posibles mutaciones de acuerdo a un sistema de evaluación de secuencias.
La selección de posibles posiciones a mutar se hace de acuerdo a un muestreo ponderado en función de los resultados de la evaluación secuencial.
Además, la aceptación de cada mutación está asociada a un método heurístico de decisión.
De esta forma, la secuencia de mutaciones aceptadas para una determinada secuencia inicial no necesariamente será siempre la misma. La búsqueda será guiada por el análisis realizado sobre las secuencias pero no quedará completamente determinada por éste. 

En la figura \ref{fig:esquema-algoritmo} se ve un esquema general del método completo. 
En la seccion \ref{seqInicial} se detalla como se obtiene la secuencia inicial y luego en las secciones \ref{evaluacion} y \ref{mutacion} se describen y ejemplifican los pasos para evaluar la secuencia y obtener una mutación puntual 
en cada iteración.


\begin{figure}[h!]
\centering
   \includegraphics[width=\textwidth]{img/diagrama-algoritmo-2.png}
 \caption{\textbf{Esquema general del método}}
 \label{fig:esquema-algoritmo}
\end{figure}

El resultado principal de la ejecución es una secuencia que cumple con todas las características positivas que se indicaron y no posee ninguno de los aspectos negativos.
En un archivo asociado se refleja el proceso de mutación indicando, para cada paso, cual es la posición mutada y la secuencia resultante.
Otras opciones de salida son posibles. En la sección \ref{output} del manual se describen opciones que permiten, entre otras cosas, realizar una ejecución paso a paso con detalles de cada evaluación realizada, 
o limitar la salida mostrando únicamente la secuencia resultante.

\subsection{Secuencia inicial}\label{seqInicial}

El método comienza a iterar a partir de una secuencia inicial. 
Esta secuencia puede ser creada de forma aleatoria como primer paso del 
algoritmo(ver \ref{secuenciaInicialRandom}) o puede ser pasada como parámetro por el usuario \ref{secuenciaInicialDefinida}. 

En el caso que se defina una secuencia inicial esta definirá también, de manera implícita, la logitud de la secuencia resultante. Por lo tanto, no es necesario definir la longitud.
Sin embargo, en este caso se da la posibilidad al usuario de identificar segmentos flanqueantes en uno o ambos extremos de esta secuencia inicial que, si bien serán tenidos en cuenta a la hora de la evaluación,
no serán mutados como parte del proceso y por lo tanto formarán parte del diseño resultante. 
La utilización de este parámetro opcional se detalla en la sección \ref{flanqueantes}.



La generación de una secuencia aleatoria no es un problema trivial. 
En la naturaleza las proteínas están compuestas de un conjunto de 20 aminoácidos, los cuales se encuentran con diferentes abundancias relativas. 
La frecuencia de cada aminoácido dentro de un proteoma está dada por un balance entre el costo metabólico de este y la necesidad de contar con un conjunto de secuencias diversas que darán proteínas funcionales \cite{krick2014amino}. 
Siguiendo los mismos principios, definiremos estas frecuencias como la composición estándar para nuestra herramienta. 
De esta forma, podremos obtener la diversidad buscada a la vez que minimizamos el costo metabólico de las secuencias linker generadas.
Las frecuencias estándar resultan del cálculo de la composición de aminoácidos de todas las proteínas de la base de datos SwissProt \cite{compositionAA}.  
% De esta forma, la composición total representa un consenso de frecuencias de aminoácidos entre todas las secuencias documentadas hasta el momento en la base de datos UniProtKB/Swiss-Prot.

% De esta forma, utilizando la 
% Esta frecuencia 
% Los valores de la composición estándar se encuentran en http://web.expasy.org/protscale/pscale/A.A.Swiss-Prot.html

Una vez definida la frecuencia que tendrá cada aminoácido, la aplicación que se utiliza para generar una secuencia aleatoria es RandSeq \cite{randseq}.
% Por default, la herramienta utiliza la composición estándar obtenida de SwissProt \cite{compositionAA}.  
% Para generar una secuencia aleatoria, entonces, es necesario definir primero la frecuencia que tendrá cada aminoácido. 
% Esta frecuencia se obtiene a partir de la frecuencia global que tiene cada aminoácido en la base de datos de secuencias proteicas.
Entre otras que existen, esta herramienta ofrece la posibilidad de definir las frecuencias que desea para cada aminoácido e, incluso,
indicar solo la frecuencia de algunos residuos en particular, dejando el restos con la frecuencia estándar. 
Nuestra herramienta también ofrece al usuario esta flexibilidad para definir la composición que tendrá la secuencia (ver \ref{composicion})
Esta funcionalidad permitiría, por ejemplo, indicar que cierto aminoácido no esté presente en la secuencia (asignándole una frecuencia igual a 0). 
Para el fin que tiene la herramienta, es útil tener este tipo de funcionalidades ya que permite adaptar los requerimientos a las capacidades(limitaciones) del laboratorio experimental. 
El nuevo linker diseñado deberá poder ser sintetizado eficientemente junto con la nueva proteína, lo cual implica una carga metabólica para el sistema biológico en el cual está siendo expresado. 
De esta forma, se intentará adaptar las propiedades de la secuencia diseñada para aumentar la capacidad de síntesis reduciendo, por ejemplo, los aminoácidos que implican un gran gasto energético
y que limitarán la producción de la proteína final.





% \subsection{Método de evaluación de la secuencia}


% Para poder implementar la busqueda guiada por información resultante de evaluaciones secuenciales, primero debemos poder representar esta información de forma concreta/cuantitativa.
% Para esto, de cada evaluación realizada sobre la secuencia se extrae un valor binario para cada posición. Un valor = 1 representa una evaluación negativa con respecto al comportamiento deseado, 
% mientras que un valor = 0 implica una evaluación positiva.
% Por ej: .........
% ARMAR EJEMPLOS CON HERRAMIENTAS SIN ESPECIFICAR EL NOMBRE DE ESTA,ej:  LA EVALUACION DE LA HERRAMIENTA A RESULTA EN..... Y LA HERRAMIENTA B RESULTA EN ...
% Los valores correspondientes a distintas
% Por ej. si al analisis previo le agregamos una evaluacion para....

% Usando este esquema, se deduce simplemente que la posicion que tenga el mayor valor será la que más se aleja de nuestro objetivo.

% Una de las desventajas(quizás no), de este esquema , es que el metodo esta orientado a realizar evaluaciones que permitan derivar un valor binario para cada posicion de la secuencia. 
% Esto, al parecer, no seria un problema, ya que el resultado deberia poder adaptarse a este esquema.


% Ya que no conocemos la función exacta que determina la relacion entre la secuencia y los valores de puntajes en las distintas posiciones(ni tampoco su superficie funcional),
% no podemos conocer como varía el puntaje asociado si realizamos una modificacion sobre la secuencia
% Ademas, esta variacion del puntaje con la secuencia depende de distintos factores, no solo del conjunto de herramientas/evaluaciones realizadas, sino también de los parámetros utilizados para éstas.
% La implementación de la búsqueda deberá tener esto en cuenta para realizar una busqueda guiada por estos valores.





\subsection{Evaluación}

% TENIENDO EN CUENTA ESTAS CONSIDERACIONES, EL ESQUEMA QUE USAMOS NOS PERMITE REALIZAR UN NUMERO REDUCIDO DE MUTACIONES YA QUE SE APUNTA A LAS POSICIONES CON más CONFLICTO, 
% ADEMAS EL PARAMETRO BETA PERMITE ADAPTAR LA BUSQEUDA A DISTINTAS FORMAS DE LA SUPERFICIE DEL ESPACIO DE SOLUCIONES, LAS CUAL DEPENDE DE LAS CONDICIONES DE LOS PARAMETROS QUE USAMOS(COMPOSICION, THRESHOLDS, HERRAMIENTAS).
% ES DECIR SI LA SUPERFICIE TIENE MINIMOS LOCALES QUE DEBAN SER SUPERADOS, UN VALOR DE BETA más ALTO RELAJA LAS RESTRICCIONES PARA
% ACEPTAR MUTACIONES Y POR LO TANTO PERMITE . CUANDO LA SUPERFICIE ES más 'PLANA', UN VALOR DE BETA MENOR PERMITE ENCONTRAR LA SOLUCION CON UN MENOR NUMERO DE MUTACIONES(LO QUE NO SIGNIFICA QUE SEA más RAPIDO)
% En este caso, la función a optimizar está dada por el resultado de todas las evaluaciones que queremos realizar.


En cada iteración del método se analiza la secuencia utilizando un conjunto de herramientas de evaluación.
Los resultados de cada evaluación serán reflejados en un valor numérico o puntuación asociado a cada posición
(el capítulo \ref{tools} se centra en las propiedades evaluadas y cómo se aplican las herramientas utilizadas para dar un valor numérico a cada posición). 

El puntaje es inicializado en 0 para todas las posiciones y cada evaluación ejecutada puede aumentar o mantener el valor de manera tal que, 
al finalizar todas las evaluaciones correspondientes, el puntaje de cada posición tendrá un valor mayor o igual a cero.
% El aumento de este puntaje depende de la concordancia que tiene cada posición con las características evaluadas.
En cada evaluación \textbf{se aumenta el puntaje si el residuo en esa posición NO favorece las propiedades deseadas}.
Definimos el puntaje total de la secuencia como la suma de puntajes de cada posición.
Un ejemplo de análisis de la secuencia mediante dos herramientas de evaluación hipotéticas A y B sería: % PONER EJEMPLO PARA QUE QUEDE CLARO
% \rule[0.9cm]{0pt}{0.9cm}

\vspace{0.3cm}
\begin{center}
\begin{tabular}{llllllllllllll} 
\hline
Secuencia & \textbf{M} & \textbf{V} & \textbf{L} & \textbf{S} & \textbf{P} & \textbf{A} & \textbf{D} & \textbf{K} & \textbf{T} & \textbf{N} & \textbf{P} & \textbf{D} \\ \hline
Puntaje Inicial & 0 & 0 & 0 & 0 & 0 & 0 & 0 & 0 & 0 & 0 & 0 & 0\\ \hline
Evaluación con herramienta A & 0 & 1 & 1 & 1 & 1 & 0 & 3 & 2 & 5 & 4 & 1 & 0\\ \hline
Evaluación con herramienta B & 0 & 0 & 1 & 1 & 3 & 5 & 1 & 1 & 0 & 2 & 1 & 2\\ \hline
Puntaje global & 0 & 0 & 2 & 2 & 4 & 5 & 4 & 3 & 5 & 6 & 2 & 2\\ \hline
Puntaje total  & 35 \\ \hline
\end{tabular}
\end{center}

\vspace{0.5cm}

% La secuencia linker es solo una parte de la construccion final correspondiente a la proteína quimérica. 
% De esta forma, y dado que muchos aspectos relevantes del diseño dependen del contexto, el algoritmo permite 


\subsection{Mutación}\label{mutacion}

Al terminar todas las evaluaciones sobre la secuencia, el puntaje resultante se utiliza para el proceso de mutación.
% POR QUE HACEMOS UNA MUTACION DE A 1 PASO, ES DECIR 1 SOLA MUTACION POR VEZ? 



El proceso de mutación sigue los siguientes pasos:
\begin{enumerate}
 \item En primer lugar se selecciona uno de los residuos de la secuencia como objetivo para realizar la mutación. 
 Esta selección será ponderada. El factor de ponderación utilizado será el valor resultante de sumar 1 al puntaje global asociado a la posición.
 Siguiendo el ejemplo anterior:
 
 \begin{center}
 \begin{tabular}{lllllllllllll} 
 \hline
Secuencia &  \textbf{M} & \textbf{V} & \textbf{L} & \textbf{S} & \textbf{P} & \textbf{A} & \textbf{D} & \textbf{K} & \textbf{T} & \textbf{N} & \textbf{P} & \textbf{D}\\  \hline
Puntaje global & 0 & 0 & 2 & 2 & 4 & 5 & 4 & 3 & 5 & 6 & 2 & 2\\  \hline
Factor de ponderación & 1 & 1 & 3 & 3 & 5 & 6 & 5 & 4 & 6 & 7 & 3 & 3\\  \hline
\end{tabular}
 \end{center}
 
\vspace{0.5cm}
Esta forma de calcular el factor de ponderación implica que nuncá se tendrán factores iguales a cero y, por lo tanto, siempre se podrá seleccionar alguna posición para mutar.

%  Se debe tener en cuenta que, como se mencionó previamente, el valor del puntaje resultante puede ser igual a 0 y, si esto se repite para todas las posiciones, puede ocurrir que ningún residuo sea factible de ser seleccionado, en cuyo caso no se continúa el proceso de mutación. 
   \item El segundo paso, consiste en seleccionar con que aminoácido se sustituirá la posición seleccionada. 
   Esta selección se realiza siguiendo la misma composición que se definió al iniciar la ejecución \ref{composicion}. 
   Dado que la selección del sustituto es independiente del tipo de residuo seleccionado para mutar,
   es posible que el mutado y su reemplazo seleccionado sean iguales. 
   En este caso simplemente se vuelve a seleccionar un sustituto hasta que el resultado sea un residuo distinto al anterior.     
    \item El tercer paso consiste en aceptar o rechazar la mutación propuesta.
    Para realizar esto, primero se vuelve a evaluar el puntaje, analizando todas las características deseadas de la secuencia pero, esta vez, sobre la secuencia que contiene la mutación propuesta.
    Este resultado nos permite saber como se modificarían los puntajes si aceptáramos la mutación. 
    La decisión se toma en base al puntaje total de la secuencia, resultante de sumar los puntajes de cada posición. Si este  \underline{puntaje total disminuye}, entonces \underline{la mutación es aceptada}. 
    Por ejemplo, si evaluamos la posibilidad de mutar la Leucina en la posición 3 por Fenilalanina:
%     PONER EJEMPLO DONDE DISMINUYA EL PUNTAJE TOTAL

\vspace{0.3cm}
\begin{center}
       \begin{tabular}{ccccccccccccc}
       \hline
       
	Secuencia previa &  \textbf{M} & \textbf{V} & \textcolor{blue}{L} & \textbf{S} & \textbf{P} & \textbf{T} & \textbf{D} & \textbf{K} & \textbf{T} & \textbf{N} & \textbf{P} & \textbf{D}\\  \hline
	Puntaje global & 0 & 0 & 2 & 2 & 4 & 5 & 4 & 3 & 5 & 6 & 2 & 2\\  \hline
       Puntaje total(previo) & 35 \\ \hline \hline
       Secuencia mutada & \textbf{M} & \textbf{V} & \textcolor{blue}{F} & \textbf{S} & \textbf{P} & \textbf{T} & \textbf{D} & \textbf{K} & \textbf{T} & \textbf{N} & \textbf{P} & \textbf{D}\\  \hline
	Puntaje global  & 0 & 0 & 3 & 1 & 3 & 4 & 4 & 3 & 5 & 6 & 2 & 2\\  \hline
       Puntaje total(posterior) & 33 \\  \hline
      \end{tabular}\\
\end{center}      
%       \vspace{0.2cm}
    En este caso, la mutación disminuye el puntaje total(33$<$35) y por lo tanto es aceptada, finalizando la iteración.
%     Dado que las evaluaciones pueden depender del contexto en el que se encuentra cada posición, 
    Una mutación puntual puede cambiar el valor del puntaje en esa posición y/o el valor del puntaje correspondiente a otras posiciones. 
    Por lo tanto, como se ve en el ejemplo, para evaluar el cambio se tiene en cuenta el puntaje total y no exclusivamente el de la posición mutada. 
%     Dado que el puntaje es el resultado de un conjunto de evaluaciones diferentes que analizan la secuencia como un todo, 
    
    
    Si la mutación propuesta \textbf{NO} disminuye el puntaje total de la secuencia, entonces la mutación se acepta o rechaza siguiendo el método de decisión de Monte Carlo. 
    
%     \subitem 
    Este método deriva una probabilidad de aceptación($P_{aceptar}$) a partir de la diferencia entre los puntajes totales resultantes de la secuencia antes($Puntaje_{previo}$) y después($Puntaje_{posterior}$) de la mutación propuesta.
    La probabilidad de aceptación resulta de calcular:
  
 {
    \begin{equation}\label{monteCarlo}
    \Large
    P_{aceptar} =  e^{\Big(\frac{Puntaje_{previo}  - Puntaje_{posterior} } {\beta}\Big)} 
   \end{equation}
  }
    
    El valor del parámetro $\beta$ es propio del método y permite ajustar la probabilidad de aceptación según el perfil de evaluaciones realizadas. 
    En el capítulo \ref{results} se analizan en detalle los efectos globales de este parámetro sobre la herramienta y se evalúan los valores óptimos en función del conjunto de herramientas aplicadas(detalladas en el capítulo \ref{tools}).
    Para ejemplificar el método de decisión de Monte Carlo, tomemos el siguiente caso de mutación en donde se propone mutar el ácido aspártico en la posición 7 por ácido glutámico :
    
    %     EJEMPLO DONDE *NO* DISMINUYE EL PUNTAJE TOTAL
      \vspace{0.3cm}
      \begin{center}
       \begin{tabular}{ccccccccccccc}\hline
	Secuencia previa &  \textbf{M} & \textbf{V} & \textbf{L} & \textbf{S} & \textbf{P} & \textbf{A} & \textcolor{blue}{D} & \textbf{K} & \textbf{T} & \textbf{N} & \textbf{P} & \textbf{D} \\ \hline
	Puntaje global & 0 & 0 & 2 & 2 & 4 & 5 & 4 & 3 & 5 & 6 & 2 & 2\\  \hline
       Puntaje total(previo) & 35 \\ \hline \hline 
       Secuencia mutada  &  \textbf{M} & \textbf{V} & \textbf{L} & \textbf{S} & \textbf{P} & \textbf{A} & \textcolor{blue}{E}& \textbf{K} & \textbf{T} & \textbf{N} & \textbf{P} & \textbf{D}\\  \hline
	Puntaje global & 0 & 0 & 3 & 3 & 4 & 5 & 4 & 3 & 5 & 6 & 2 & 2\\  \hline
       Puntaje total(posterior) & 37 \\ \hline
      \end{tabular}\\
    \end{center}
    
    Para este caso $Puntaje_{previo}=35$ y $Puntaje_{posterior}=37$. Usando un valor de $\beta=1.5$ la ecuación \ref{monteCarlo} resulta en $P_{aceptar} = 0.26$.
%     El resultado de la ecuación \ref{monteCarlo} dará un valor de probabilidad en el rango [0,1]. 
    Para aplicar esta probabilidad dentro del método, obtenemos primero un valor \textit{random} en el rango [0,1], si el valor obtenido es menor que el calculado con la ecuación \ref{monteCarlo} entonces se acepta la mutación propuesta, 
    caso contrario esta no es aceptada. 
    En caso de NO aceptarse la mutación propuesta el algoritmo vuelve al paso 1, es decir, se vuelve a intentar una nueva mutación. 
        
\end{enumerate} 



% De forma resumida, solo se incorporan mutaciones que se traduzcan en una disminución del puntaje total de la secuencia.


Dado que el objetivo final es obtener una secuencia que posea, de acuerdo con las herramientas bioinformáticas usadas en la evaluación, todas las características deseadas, 
el algoritmo finaliza cuando el puntaje resultante sea igual a cero, o cuando se cumple alguna de las condiciones de finalización especificadas en \ref{condicionFin}. 
% Esto implica que todas las evaluaciones concuerdan con los objetivos buscados en el diseño de la secuencia linker.








% CAPITULO 3: POPIEDADES EVALUADAS / HERRAMIENTAS UTILIZADAS
\chapter{Propiedades evaluadas}

\section{Búsqueda de similitudes secuenciales}

BÚSQUEDA DE SIMILITUDES SECUENCIALES (BLAST)

Una forma de evaluar la existencia de secuencias similares en la naturaleza es haciendo un alineamiento secuencial frente a una base de datos de secuencias proteicas conocidas. El descubrimiento de una secuencia homóloga provee un primer indicio acerca de la función que está puede cumplir. En el contexto de la herramienta que estamos desarrollando, la información resultante de la búsqueda sirve como indicador de que posiciones puntuales son las que conforman esta similitud secuencial y que, por lo tanto, podrían inducir la función. 

¿QUE DESEAMOS PARA EL LINKER? ¿POR QUE?

Para realizar la búsqueda de similitud secuencial utilizamos la herramienta BLAST (ref. 1 ,2 ), la cual puede ser ejecutada de dos maneras:

%     -Una opción es hacer una llamada al servidor remoto que ejecuta la búsqueda (http://blast.ncbi.nlm.nih.gov/). Esta opción es simple pero implica ejecutar la búsqueda remotamente y retardos para obtener los resultados. Se puede hacer simplemente desde Python utilizando el módulo BioPython(http://biopython.org/). (http://www.biotnet.org/sites/biotnet.org/files/documents/25/biopython_blast.pdf ) 
 
    -Otra opción es instalar localmente el software para realizar la búsqueda, lo cual requiere también disponer de las bases de datos sobre las cuales se realizarán. (http://www.ncbi.nlm.nih.gov/books/NBK52640/).

Para la implementación de nuestra herramienta se utilizará una instalación local del software BLAST junto con las bases de datos asociadas, contra las cuales se realizará la búsqueda de similitud. En nuestro caso utilizaremos la base de datos SwissProt (Ref. 3 )

(Descripción BLAST??)
DESCRIBIR BREVEMENTE BLAST COMO PARA UN EXPERIMENTALISTA
INCLUIR REFERENCIA BLAST
¿QUE OPCION TOMAMOS NOSOTROS?
QUE SCORE TOMAMOS COMO CUTOFF?


La búsqueda BLAST se basa en los cambios que pueden ocurrir entre 2 secuencias homólogas durante la evolución. Las mutaciones pueden dar como resultado distintos residuos en las secuencias de proteínas, también pueden ocurrir inserciones y deleciones de residuos. Cada uno de los posibles eventos tiene una frecuencia de ocurrencia asociada. El método de alineamiento utiliza un esquema de valores para asignar puntaje a cada uno de estos eventos y,  utilizando una estrategia de optimización, se exploran formas alternativas de alinear los residuos de ambas secuencias para logar sumar el máximo score.
Realizando una búsqueda optimizada se pueden recuperar secuencias similares de una base de datos con gran cantidad de entradas, haciendo un alineamiento de la secuencia de búsqueda frente a todas las almacenadas en la base de datos. Para analizar los resultados de esta búsqueda no es posible evaluar solamente el score obtenido en cada alineamiento, sino que se debe tener en cuenta la probabilidad de que la similitud encontrada sea solo al azar. En este punto se deben aplicar métodos estadísticos para evaluar la significancia del resultado dada la longitud de la secuencia consultada y el tamaño de la base de datos.
El resultado que devuelve la búsqueda BLAST es, en caso de éxito, un conjunto de secuencias de la base de datos ordenadas por un score de alineamiento. Además de realizar el alineamiento, BLAST provee información estadística que ayuda a descifrar la significancia biológica del alineamiento, este es el valor "expect" o e-value. Para utilizar los resultados de BLAST en nuestra herramienta utilizaremos un cutoff sobre el valor de e-value que nos de cierta certeza de que la similitud secuencial es significativa y, por lo tanto, la secuencia podría tener una función similar. El valor de cutoff utilizado es de 0.01, todas las secuencias encontradas con un e-value menor que este valor serán significativamente similares. En caso de encontrarse muchas secuencias dentro del cutoff se utilizará la primera de éstas.
Dado que el objetivo de este paso es evaluar que posiciones tienden a una similitud con alguna secuencia natural, el resultado obtenido se utilizará para identificar estas posiciones y marcarlas para una nueva ronda de mutaciones.

EJEMPLO DE RESULTADO
DESCRIBIR BIEN QUE VALORES SE OBTIENEN

Ejemplo:
Si realizamos la búsqueda de la secuencia MVLSPADKTNVKGGWGKV, se encontrarán varias secuencias dentro del cutoff. La primer similitud es MVLSPADKTNVKAAWGKV, con un e-value de 7e-09. El alineamiento correspondiente es:

Query  1   MVLSPADKTNVKGGWGKV  18
           MVLSPADKTNVK  WGKV
Sbjct  1   MVLSPADKTNVKAAWGKV  18

USAR FUENTE COURIER PARA ALINEAMIENTOS

Dentro de nuestra herramienta se utiliza este resultado para marcar los posibles targets en la mutación. En el ejemplo dado, excepto las posiciones 13 y 14, todos los residuos tienen una similitud puntual con la secuencia encontrada y, por lo tanto, son targets para una próxima mutación.
En este paso, a cada residuo que se alinea correctamente con el hit se le sumará 1 punto al valor de score asociado.





\section{IUPred (analisis de propiedades de flexibilidad/estrucutales)}
% ANÁLISIS ESTRUCTURAL (IUPred)

Otro de los aspectos a considerar acerca de la secuencia es el grado de estructuración o desorden que puede adoptar. 
Una de las principales propiedades que buscamos al diseñar una secuencia linker es la existencia de ésta en un ensamble
de estructuras desordenadas que proveen la flexibilidad necesaria para que los dominios que conecta puedan realizar sus funciones libremente.
Si bien el paradigma clásico, que asume la necesidad de adoptar una estructura tridimensional estable para que la proteína posea una función,
ha sido desafiado por diversas proteínas intrínsecamente desordenadas que llevan a cabo funciones celulares básicas, 
evaluar la tendencia de la secuencia a formar una estructura nos da una pauta acerca de la posibilidad de tener una función asociada. 
De esta forma, además de asegurarnos obtener una estructura flexible que permita cumplir la función de linker, 
estamos descartando que la secuencia tenga asociada alguna función que requiera la existencia de una conformación tridimensional.



La herramienta que utilizaremos es IUPred, disponible a través de un servidor web (http://iupred.enzim.hu/) o descargando la implementación 
y ejecutándola localmente (http://iupred.enzim.hu/Downloads.php).   
Tal como se menciona en la información de la aplicación (ref. 1), los residuos que tengan un valor asociado resultante mayor a 0.5 pueden ser tomados como desordenados.
Utilizando esta herramienta se identifican los residuos que favorecen la formación de una estructura estable 
y se los marca como posibles objetivos para la siguiente ronda de mutaciones del algoritmo, llevando a la secuencia 
global hacia un perfil estructuralmente flexible y desordenado, con pocas posibilidades de desarrollar una función.


Fundamento
Utilizando bases de datos con estructuras conformacionales de proteínas globulares, 
es posible derivar campos de fuerzas empíricos(potenciales estadísticos) de las interacciones que ocurren en las conformaciones nativas. 
Estos modelos de campos de fuerzas han resultado muy útiles para evaluar la contribución de cada interacción entre pares a la energía total,
lo que permitió aplicarlos para predecir la estabilidad de una estructura, evaluar la calidad de diferentes modelos estructurales, aplicar técnicas de fold recognition o threading, etc. 
Sin embargo, obviamente, se requiere conocer una estructura a partir de la cual extraer los pares que interaccionan y poder así evaluar el aporte energético de éstos. 
La ecuación asociada tendrá la forma:





Donde Cij es el número de interacciones entre residuos de tipo i y de tipo j que se encuentran en la estructura conformacional. 
El valor Mij es un parámetro experimental derivado de un onjunto de datos de estructuras conocidas, que representa la energía de interacción entre residuos de tipo i y residuos de tipo j.

Partiendo del concepto general que la conformación nativa está determinada por la estructura primaria de la proteína, y que esta conformación se corresponde con el mínimo global del
espacio conformacional, es posible parametrizar un modelo que permita predecir este mínimo de energía sin asumir ninguna conformación estructural. (Ref. 2)
Esta aproximación es posible ya que la contribución energética de un residuo depende, no solo del tipo de aminoácido, sino también de los potenciales parejas de interacción en la secuencia. 
El aporte de un residuo será más favorable si la secuencia en la que se encuentra contiene más residuos que pueden formar interacciones favorables con este. 
La forma de plantear este modelo es mediante una expresión cuadrática sobre la composición de aminoácidos de la secuencia:
Los valores de n representan las frecuencias de aminoácidos i y j en la secuencia.
El valor de P es el parámetro a estimar, el cual se deriva a partir del modelo mencionado previamente, que evalúa la energía a partir de las interacciones que ocurren en la estructura. 
El ajuste se realiza minimizando la diferencia entre ambas ecuaciones.
De esta forma, mediante un ajuste de mínimos cuadrados se puede parametrizar el modelo a partir de datos de estructuras pertenecientes a proteínas globulares.
Dado que las proteínas globulares forman un gran número de interacciones entre los residuos (lo que les provee la energía estabilizante para superar la pérdida de entropía), 
y las proteínas IU/desordenadas tienen secuencias especiales que no poseen esta capacidad de formación de interacciones, la estimación del potencial de interacción permite diferenciar entre 
regiones de proteínas ordenadas y desordenadas. Esto transforma el modelo de predicción en un eficiente método para diferenciar secciones desordenadas de secciones con estructura definida.

Como se mencionó previamente, la parametrización de este modelo se realiza a partir de estructuras contenidas en una base de datos de proteínas globulares, 
por lo que son datos suficientes para realizar una buena parametrización, además de ser consistentes y curadas. Esto diferencia el método de otros, que se basan en adaptar 
un modelo a datos de estructuras correspondientes a proteínas intrínsecamente desordenadas, agrupados en bases de datos chicas, con datos obtenidos usando diversas técnicas 
y con distintos significados del término desordenado.
  

  
  
\section{Prediccion de union a proteinas (ANCHOR)}
Los valores de n representan las frecuencias de aminoácidos i y j en la secuencia.
El valor de P es el parámetro a estimar, el cual se deriva a partir del modelo mencionado previamente, que evalúa la energía a partir de las interacciones que ocurren en la estructura. 
El ajuste se realiza minimizando la diferencia entre ambas ecuaciones.
De esta forma, mediante un ajuste de mínimos cuadrados se puede parametrizar el modelo a partir de datos de estructuras pertenecientes a proteínas globulares.
Dado que las proteínas globulares forman un gran número de interacciones entre los residuos (lo que les provee la energía estabilizante para superar la pérdida de entropía), 
y las proteínas IU/desordenadas tienen secuencias especiales que no poseen esta capacidad de formación de interacciones, la estimación del potencial de interacción permite diferenciar entre
regiones de proteínas ordenadas y desordenadas. Esto transforma el modelo de predicción en un eficiente método para diferenciar secciones desordenadas de secciones con estructura definida.

Como se mencionó previamente, la parametrización de este modelo se realiza a partir de estructuras contenidas en una base de datos de proteínas globulares, por lo que son datos suficientes
para realizar una buena parametrización, además de ser consistentes y curadas. Esto diferencia el método de otros, que se basan en adaptar un modelo a datos de estructuras correspondientes
a proteínas intrínsecamente desordenadas, agrupados en bases de datos chicas, con datos obtenidos usando diversas técnicas y con distintos significados del término desordenado.

El primer componente resulta de promediar los valores obtenidos directamente de IUPred en una ventana de tamaño w1 alrededor de cada residuo. Esto evalúa la tendencia al desorden
que tiene el entorno de cada residuo.

El segundo componente evalúa la ganancia de energía que tendrá el residuo al formar interacciones de a pares con los vecinos contenidos dentro de una ventana de tamaño w2. 
La ecuación asociada es idéntica a la obtenida para IUPred.

El tercer componente evalúa la ganancia de energía que tendrá el residuo al formar interacciones de a pares con los una proteína globular, con respecto a la formación de contactos únicamente entre los vecinos(componente 2). Para hacer esta evaluación, se reutiliza el modelo de IUPred, pero ahora, la composición del contexto con el cual se dan las interacciones estará dado por la composición de una proteína globular hipotética. Para esto se utiliza la frecuencia de AAs estándar en estas proteínas. Este valor se calcula como:
La diferencia resultante entre la interacción con los vecinos propios de la secuencia y este nuevo valor calculado será:
    

La ecuación resultante que combina los tres criterios es:

  
  
Varios de los parámetros de este nuevo modelo ya fueron determinados previamente utilizando datos conocidos de estructuras de proteínas globulares. Queda determinar:
Cual es la proporción de cada uno al aporte total (coeficientes de la combinación lineal) 
Tamaño de la ventana(cantidad de vecinos) que se tienen en cuenta para hacer el promedio en el componente 1 (w1).
Valor de la ventana que se tiene en cuenta para evaluar cada el componente 2 (w2).

Para determinar los valores óptimos de estos parámetros se utilizaron dos conjuntos de datos: un conjunto negativo compuesto por cadenas de proteínas globulares y un conjunto de resultados
positivos, compuesto por complejos cortos desordenados. Este último conjunto de datos, compuesto por proteínas desordenadas que se unen a proteínas ordenadas formando complejos,
representa una seria limitación ya que la cantidad de elementos que se conocen es muy limitada. Dada esta condición, se considera que una ventaja de este método el reducido número de 
parámetros(5 en total) que se deben evaluar en base a este conjunto de datos.
Cabe destacar que no es posible entrenar el predictor utilizando un conjunto de proteínas desordenadas que se sepa que no forman uniones con proteínas globulares, principalmente 
porque no existe método preciso para comprobar que esto NO ocurre.

Dado que el resultado de Anchor es una combinación de varios aportes, principalmente la tendencia al desorden y la sensibilidad al estar en un entorno estructurado, 
el resultado obtenido es relativamente independiente del score obtenido únicamente con IUPred. 

Dentro de nuestro algoritmo, la herramienta ANCHOR se aplica utilizando un punto de corte igual a 0.5. Los residuos de la secuencia que tengan un valor asociado mayor a 
éste se considerarán como posiblemente pertenecientes a segmentos desordenados de unión a proteínas.   



\section{Deteccion de motivos lineales}


% QUE SON ?
Short linear motifs (SLiMs, LMs or MiniMotifs) are
regulatory protein modules characterized by their
compact interaction interfaces (the affinity and specificity
determining residues are usually encoded between 3 and
11 contiguous amino acids (1)) and their enrichment in
natively unstructured, or disordered, regions of proteins
(2). As a result of limited intermolecular contacts with
their interaction partners, SLiMs bind with relatively
ow affinity (in the low-micromolar range), an advan-
tageous attribute for use as transient, conditional and
tunable interactions necessary for many regulatory
processes. 
Due to the limited number of mutations
necessary for the genesis of a novel motif, SLiMs are
amenable to convergent evolution, functioning as a
driver of network evolution by adding novel interaction
interfaces, and thereby new functionality, to proteins. 
Novel SLiMs can readily evolve de novo, adding functionality to a protein.
This evolutionary plasticity facilitates the rapid proliferation
within a proteome, and as a result, motif use is ubiquitous
in higher eukaryotes.
The ease of evolution of motifs has resulted in the proliferation of
SLiMs that encode functions of broad utility and many
motif classes are ubiquitous, occurring in tens or hundreds
of proteins. Many pathogens have also taken advantage
of the intrinsic evolutionary plasticity of SLiMs by
mimicking host motifs to deregulate and repurpose host
pathways.
On a higher regulatory level, short linear motifs often
exhibit complex switching behavior by co-operating with
each other and with post-translational modifications to
facilitate switching between different functional states of
a protein, and thus, SLiMs function as key regulatory
modules that allow for context-dependent, integrative
regulatory decision-making

% En los últimos años, sin embargo, ha cambiado drásticamente el concepto que se tenía sobre las interacciones entre proteínas. En este paso, se han establecido regiones intrínsecamente
% desordenadas como factores claves en la funcionalidad de proteínas. Se puede ver a partir de esto que, independientemente de la eficiencia que tengan los métodos usados para lograr una 
% estructura desordenada, no es correcto asegurar que la secuencia no tendrá ninguna función asociada solamente por haber logrado restringir la formación de una estructura conformacional definida.
% 
% Los principales modulos funcionales encontrados en las regiones intrínsecamente desordenadas son los motivos lineales cortos (short linear motifs, o SLiMs), que se refieren a una clase de modulos de interacción compactos, degenerados y evolutivamente convergentes.
% Se ha demostrado que la interacción mediada por SLiMs participa en diversos procesos biológicos tales como el control de progresión del ciclo celular, marcado de proteínas para la degradación en el proteasoma, modulación de la eficiencia en la traducción, y localización de proteínas en compartimientos subcelulares específicos. Se espera que en un futuro se develen más funciones asociadas a instancias de SLiMs.



% FUNCIONES:
Interactions mediated by SLiMs, also referred to as linear motifs or MiniMotifs, have been shown to direct many diverse processes, such as controlling cell cycle progression, 
tagging proteins for proteasomal degradation, modulating the efficiency of translation, targeting proteins to specific sub-cellular localizations and stabilizing scaffolding complexes.
Undoubtedly, more functions will be revealed in the future as additional SLiM instances are characterized.

Short linear peptide motifs are used for regulation by phosphorylation, acetylation, glycosylation and a host of other post-translational modifications.
The functionality of linear motifs can be modulated by modifications such as phosphorylation

% funcionamiento como switches
As key regulatory interaction modules, linear motifs are tightly controlled and many motifs are conditionally turned ‘on’ and ‘off’ depending on cell state. Pre-translational addition or removal of a SLiM-containing exon,
post-translational modification of the SLiM-containing peptide, allosteric SLiM inhibition or activation and SLiM binding site competition are amongst the most common mechanisms to regulate linear motifs.
% El descubrimiento de estas propiedades de ser activados o desactivados mediante modifications pre-traduccionales llevo al desarrollo de una BBDD especifica:
% The switches.ELM  database is a resource dedicated to the curation and representation of experimentally validated motif-based molecular switches.
% (ver paper The switches.ELMresource:a compendium of conditional regulatory interaction interfaces)



% POR QUE LOS QUITAMOS?
El objetivo de este trabajo es lograr una herramienta que provea una secuencia capaz ser utilizada experimentalmente como linker en el proceso de ingeniería de proteínas. 
Esto implica no sólo probar con cierta certeza que tendrá la funcionalidad deseada, sino también evaluar el comportamiento en todos los pasos del proceso de ingeniería. 
Es necesario saber que las propiedades de la secuencia no afectarán este proceso de ninguna forma.
De forma general, se debe apuntar a tener una secuencia que no tenga ninguna interacción ni funcionalidad, evitando así cualquier interferencia con la expresión y utilización del producto 
de ingeniería. Esto implica que no tenga regiones target para clivaje (fosforilación, glicosilación,etc.), regiones de unión a otras proteínas, etc.
Si bien es posible reducir este análisis conociendo más detalladamente las condiciones experimentales con las que se trabajará, el objetivo de esta herramienta es que se pueda obtener 
un resultado que cumpla con el objetivo en cualquier contexto experimental. 

En este algoritmo utilizamos distintas herramientas para evaluar la estructura que adopta la secuencia, con el fin de poder llevarla hacia una composición que provea la flexibilidad
necesaria para la función de linker. En ese proceso podemos asumir también que la secuencia no podrá cumplir ninguna función que requiera de una estructura en una conformación definida.


Una propiedad característica de los LMs es su relación con un contexto de estructura desordenada\cite{fuxreiter2007local}.
% ACA DESCRIBIR UN POCO LAS CONCLUSIONES DE ESTE PAPER

Dadas estas propiedades estructurales, será relevante para nuestro trabajo . Más aún, teniendo en cuenta las caracteristicas secuenciales de este tipo de motivos, que le 
podía esperarse que, luego de ser detectados y mutados puedan resurgir facilmente dentro de una misma ejecución, con un comportamiento similar al que ocurre en la naturaleza.
% es relevante detectarlos en cada paso...., y se espera que, dadas las caracteristicas evolutivas??? de este tipo de motivos, 
% resurgan luego de una cierta cantidad de pasos de mutacion.





% COMO VAMOS A DETECTARLOS
En este paso del algoritmo se intentará identificar este tipo de motivos en la secuencia, con el fin de eliminarlos, reduciendo las posibilidades que el resultado final tenga funcionalidades no deseadas. 
Para detectarlos se utilizará el recurso ELM.
El recurso de motivos lineales eucariotas(ELM) \cite{puntervoll2003elm,dinkel2013eukaryotic} fue establecido con la misión de recolectar, anotar y clasificar motivos lineales cortos. 
Provides the biological community with a comprehensive database of known experimentally validated motifs, and an exploratory tool to discover putative linear motifs in user-submitted protein sequences.
Es relevante destacar que todos los datos anotados son curados manualmente a partir de la literatura y están a disposición de la comunidad científica.
La base de datos se organiza jerárquicamente: en el nivel superior se tiene un conjunto de tipos (actualmente hay un total de 6 tipos diferentes), que agrupan clases de motivos. 
% los tipos son: 
% 	Proteolytic cleavage sites (CLV)
% 	general ligand binding sites (LIG),
% 	sites for post-translational modification (MOD)
% 	sub-cellular targeting sites (TRG)
% 	Ligand binding classes describing docking sites (DOC): can be described as motifs that recruit a modifying enzyme using a site that is distinct from the active site
% 	destruction motifs (DEG) is a specific region of a protein sequence that directs protein polyubiquitylation and targets the protein to the proteasome for degradation
%  ESTOS ULTIMOS 2 TIPOS FUERON AGREGADOS EN EL ULTIMO TIEMPO: 
%           Technically, all docking sites and destruction motifs belong to the ‘ligand binding sites (LIG)’ type; however, grouping together motif classes of similar function adds an additional level of discrimination.

Cada clase define la especificación de un dominio o familia de dominios de péptidos, los cuales se describen generalmente mediante una expresión regular sobre la secuencia que los compone.
Cada clase contiene al menos una instancia, donde cada instancia representa una secuencia encontrada experimentalmente que se ajusta a la expresión definida para la clase del motivo.
El énfasis está puesto en la validación experimental que ha sido realizada sobre estas secuencias, logrando un proceso de curación manual con las instancias que son ingresadas en la base de datos. 


The aim(del recurso) is to cover the set of functional sites that can be defined by the local peptide sequence, operating essentially independently of protein tertiary structure. 
The resource suffers from the overprediction problem inherent to small protein motifs, but we are developing context filters such as cell compartment, taxonomy and globular domain clash that can partly reduce the severity
of the problem. In this resource, we use the term ELM to denote our bioinformatical representation of a functional site including the sequence motif and its context.
For any such analysis, the user should be aware that many matches to ELM regular expressions are false positives. 
Before conducting experiments based on ELM results, it is strongly advisable to check if a motif match is conserved, exposed in a cell compartment in which the motif is known to be functional. 
The ELM resource applies several filters to provide the user with such information that should ideally also be supported by the experimental evidence.
% ACA PODRIA DECIR QUE LA CANTIDAD DE FALSOS POSITIVOS(OVERPREDICTION) NO ES UN PROBLEMA PARA EL USO QUE LE VAMOS A DAR YA QUE ASUMIMOS INICIALMENTE Q EL ESPACIO DE SOLUCIONES ES SUFICIENTEMENTE GRANDE, Y CON EL METODO DE BUSQUEDA
% QUE UTILIZAREMOS, ESTA CONDICION PODRA, COMO MUCHO, DISMINUIR UN POCO ESTE ESPACIO DE SOLUCIONES RETRASANDO MINIMAMENTE LA OBTENCION DE UN RESULTADO


Para poder utilizar este recurso es posible hacerlo directamente a traves de la herramienta web que se provee en http://elm.eu.org/. Esta herramienta permite encontrar en una secuencia ingresada por el usuario, instancias de los motivos contenidos en la base de datos.
Otra forma de realizar esto es haciendo una búsqueda local. Para ésto es necesario descargar las expresiones correspondientes a todas las clases de la base de datos. 
A partir de estos datos,sólo queda realizar la búsqueda de la expresión regular sobre la secuencia con la que estamos trabajando.
En la BBDD de ELM, the motif patterns are currently represented as POSIX regular expressions (usable in the Python and Perl languages), analogous to PROSITE, but with a different syntax.
Esta última forma es la que utilizamos en nuestra herramienta, ya que nos permite independizarnos de la disponibilidad de la herramienta en el momento en que la requerimos.
Para realizar la detección de los motivos no se utiliza ningún punto de corte, todos los motivos se tienen en cuenta, independientemente de la probabilidad de ocurrencia por azar que tengan (en la herramienta online es posible restringir la búsqueda usando un valor de corte límite).
% DEBERIA ACLARAR QUE TAMPOCO SE UTILIZA NINGUN FILTRO: LOS MOTIVOS ENCONTRADOS NO SE FILTRAN SEGUN LA UBICACION CELULAR EN LA QUE SUPUESTAMENTE ESTAN(USANDO GO), NI PARA UNA ESPECIE EN PARTICULAR,..

La búsqueda de motivos sobre la secuencia resulta en un conjunto de subsecuencias que representan una instancia encontrada, las cuales pueden estar solapadas. Cada una de estas subsecuencias se trata de forma individual.
El algoritmo toma cada una de las subsecuencias resultantes y, por cada posición de éste, se suma 1 al valor de la posición en el puntaje global.
Se puede ver ésto en un ejemplo sencillo:
Si estamos trabajando con la secuencia PSKPLRGNAMVGL , el resultado de la búsqueda da un conjunto de 3 motivos encontrados en las subsecuencias: INCLUIR EXPRESIONES REGULARES, A SER POSIBLE TOMADAS DE ELM
PSKPLR (posiciones 1-6), NAMVGL (posiciones 8-13), KPLRGNAMVGL(posiciones 3-13).
Por lo tanto, el puntaje resultante de este paso es:

Secuencia global             =   PSKPLRGNAMVGL
PSKPLR (posiciones 1-6)      =   1111110000000 
NAMVGL (posiciones 8-13)     =   0000000111111  
KPLRGNAMVGL(posiciones 3-13) =   0011111111111
 -------------
Total                        =   1122221222222


% DIFERENCIAS CON PROSITE
The PROSITE database has collected a number of linear protein motifs, representing them as regular expression patterns. 
PROSITE patterns have been very useful, but also suffer from severe overprediction problems and more recently the database has emphasised globular domain annotation at the expense of linear motifs.


\section{Motivos Secuenciales (Prosite)}

Como ya se ha mencionado, uno de los objetivos de la herramienta desarrollada es eliminar cualquier posible funcionalidad asociada a la secuencia resultante. Para esto se realiza una búsqueda exhaustiva de motivos secuenciales que puedan relacionar la composición secuencial con alguna función biológica.
Como parte de esta búsqueda, en este paso se utilizará un recurso que agrupa una gran cantidad de motivos secuenciales de diversas características, se trata de PROSITE (http://prosite.expasy.org/), el cual permite anotar e identificar regiones conservadas en secuencias de proteínas.
De forma resumida, se puede definir como una colección anotada de motivos biológicamente significativos, dedicada a la identificación de familias y dominios de proteínas.
Este tipo de bases de datos contiene información derivada de alineamiento de múltiples secuencias homólogas. Los motivos resultantes se describen usando dos métodos distintos, cada uno con sus ventajas y desventajas que definen la utilidad que tendrán:

La primera forma de describir los motivos es a través de patrones (expresiones regulares) en los cuales se tiene en cuenta solo la información de los residuos más significativos, descartando el resto. La búsqueda de un patrón en una secuencia da un resultado cualitativo: hay una coincidencia o no la hay. Si hay una sustitución en alguna de las posiciones de la secuencia el patrón no coincide, independientemente del tipo de sustitución que ocurrió.

Otra forma de describir los motivos es mediante perfiles (o matrices de pesos). Estos pesos proveen valores numéricos para cada posible coincidencia o sustitución cuando se busca el motivo en una secuencia. De esta forma, al utilizarlos en la búsqueda de un motivo, funcionan como descriptores cualitativos que consideran la similitud global en toda la longitud secuencial de un dominio o proteína. Un motivo puede ser encontrado en una secuencia que posee una sustitución en una posición conservada si el resto de la secuencia tiene un nivel de similitud suficientemente alto.
Estas propiedades dan una mayor sensibilidad a los perfiles con respecto a los patrones, permitiendo encontrar dominios o familias con alta divergencia que solo tienen unas pocas posiciones muy conservadas.


Se pueden realizar búsquedas relacionadas con motivos secuenciales a través de la herramienta ScanProsite(http://prosite.expasy.org/scanprosite/), la cual permite escanear secuencias para buscar ocurrencias de los motivos, buscar motivos en una base de datos entera de secuencias, o buscar motivos propios del usuario en una secuencia

Esta misma herramienta se encuentra disponible para descargar, junto con la base de datos completa de motivos secuenciales, lo que permite realizar la búsqueda de forma local.
El objetivo de nuestra herramienta es poder encontrar cualquier ocurrencia de motivos en la secuencia sobre la que estamos trabajando. Para esto es posible escanearla utilizando patrones y/o perfiles, y variar también los límites usados en la detección de los perfiles. En nuestro caso solo utilizaremos patrones para realizar la búsqueda por considerar que la sensibilidad provista por estos es suficiente para los fines buscados.

PONER UN EJEMPLO!!! especificar que usamos patrones por la lentitud de los perfiles (justificar con números), pero sin excluir el uso a futuro de perfiles

describir si hay superposición entre elm y prosite. Se justifica usar los dos? qué hipótesis subyacente estamos usando?

\section{Limbo}

Limbo utiliza Tango(deberia explicar este antes) ?????????????

http://www.ncbi.nlm.nih.gov/pmc/articles/PMC2717214/

Las chaperonas son elementos fundamentales para el correcto funcionamiento y calidad de las proteinas, interviniendo en el correcto plegado, activacion, la posible translocacion, replegamiento o degradacion de proteinas incorrectamente plegadas, etc.

Distintas chaperonas reconocen distintos motivos secuenciales expuestos por las proteinas y esta gama de chaperonas llevara a la proteina unida a un final distinto.
Existen distintas posibilidades reales en el proceso de ingenieria de proteinas en las cuales la secuencia diseñada artificialmente se encuentre con chaperonas en alguna situación experimental. Sin dudas la degradacion de nuestra secuencia linker es el peor final en esta situación, pero todas las posibilidades tienen un impacto negativo en la funcionalidad de linker que se quiere asignar(imponer) a la secuencia. El reconocimiento por parte de la chaperona implica la unión a esta, lo cual puede interferir en la flexibilidad natural que (idealmente) ha adquirido la secuencia linker diseñada. 
De esta forma, dado que apuntamos a abarcar todas las posibles situaciones experimentales con las que uno se puede encontrar, es necesario tener en cuenta la posibilidad de que esten presentes proteinas chaperonas en algun paso del uso experimental de la secuencia. 
 
Existen chaperonas cuya funcion implica unirse a proteinas que no esten totalmente plegadas para evitar la agregacion de estas. El objetivo de esto es intervenir en situaciones de shock termico donde la perdida de la estructura nativa puede llevar a uniones entre distintas proteinas dando un agregado no funcional. Si bien uno de los objetivos de este trabajo es lograr una tendencia reducida a la agregacion en la secuencia generada y ciertas chaperonas podrian facilitar esto, la union implica, como se menciono anteriormente una perdida de flexibilidad que es necesaria para la funcion de linker. De esta forma buscamos evitar la agregacion mediante analisis sobre la secuencia que tienen este objetivo particular.

??????????
Por otro lado, es importante este paso porque las secuencias intrinsecamente desordenadas tales como la secuencia linker que estamos diseñando suelen tener propiedades que forman targets comunes para las chaperonas (exposicion de sitios hidrofobicos???)?????????????????????????


A continuación se describe la aproximacion utilizada para intentar detectar en la secuencia de trabajo motivos asociados con el reconocimiento por parte de chaperonas.

El método utiliza una combinacion de informacion secuencial y estructural para analizar el perfil de secuencias que se unen a la proteina chaperona DnaK

Para desarrollar el algoritmo predictor, lo que se realizó fue:

ARMAN 3 GRUPOS DE PEPTIDOS (ver de donde salen los 3 grupos)
PRUEBAN TODOS LOS PEPTIDOS MEDIANTE ENSAYO DE UNION A DnaK: sintetizan los peptidos unidos a placa de celulosa, los lavan, los incuban con DnaK y los revelan con un anticuerpo anti-DnaK. (hacen ademas controles negativos de donde vuelan un par de peptidos que se unen directamente al anticuerpo, ademas restan el valor de fluorecencia que aparece cuando no ponen peptidos)
Del total de peptidos se dividieron en conjuntos de peptidos binders y peptidos no binders(usando dos valores de cutoff - un valor alto y un bajo). De cada uno de estos conjuntos se separo un pequeño % como conjunto de prueba(para probar luego que tal funciona el predictor que se va a hacer) y un gran % es el que luego se usa para el set de aprendizaje? (conjunto benchmark)
Para poder armar una matriz de score especifica de posicion(PSSM) es necesario que todos los peptidos del conjunto de aprendizaje tengan la misma longitud. El conjunto de aprendizaje en si se obtiene dividiendo los peptidos binders y los no binders en heptapéptidos. Para el conjunto de no binders se tomaron todas las posibles subsecuencias de longitud 7 como negativos(y se agregaron al conjunto negativo de aprendizaje). Para el conjunto de binders no es tan simple, entonces se utilizo el campo de fuerzas FoldX: se evaluo la energia de union para cada heptapeptido posible(subsecuencias) y se agrega al conjunto de aprendizaje el mejor heptapeptido.(tambien se agregaron aquellos que tenian una enegia de union en un rango de 0.5kcal/mol menor)
Construccion de la PSSM basada en datos de secuencias: inicialmente se construyeron 2 PSSMs separadas basandose en los conjuntos de aprendizaje positivos y negativos. La frecuencia observada se calculo normalizando el numero de ocurrencias de un dado residuo por el numero de secuencias totales en el conjunto de aprendizaje. La frecuencia esperada es la ocurrencia de residuos obtenida de la base de datos SwissProt. El valor que se usa en la PSSM resultante es el logaritmo de la relacion entre la frecuencia observada y la frecuencia esperada. Se generaron asi una PSSM que representa el perfil de secuencia favorable para la union a DnaK(obtenida a partir del conjunto de binders) y otra PSSM que representa el perfil desfavorable(obtenida a partir del conjunto no binder). Estos datos se integran en una misma PSSM cuyos valores estan dados por la resta de ambos valores(valor binder - valor no binder).
Construccion de la PSSM basada en datos estructurales: se uso como template la estructura cristalizada de DnaK, junto con la cual estaba co-cristalizado un peptido con la secuencia NRLLLTG, el cual muestra el motivo(estructural?) reconocido por la DnaK, constituido por un minimo de 7 residuos en una conformacion extendida. 
Para conocer el aporte energetico de cada posible residuo en cada una de las 7
posiciones se utilizo nuevamente FoldX para hacer un scan posicion por posicion.
En primer lugar se pusieron todas alaninas. Después se fueron mutando cada
posicion por los 19 residuos restantes. Para cada uno se calcula el valor de la 
% diferencia energética con el valor del de alanina ΔΔG (cuanto mas negativo es este 
valor mejor es el binding). El valor que se usó para llenar la PSSM es el negativo de 
% este valor ΔΔG.
Dado que los valores se evaluaron mutando las posiciones sobre un backbone fijo,
este backbone va a influenciar la PSSM resultante. Lo que se hizo entonces fue
generar distintas PSSMs utilizando múltiples conformaciones de backbones de 
toda la estructura de la DnaK, obtenidas de un ensamble de conformaciones 
resueltas por NMR (para cada una se hizo una PSSM y se hizo la evaluación de 
la ROC). Los resultados de las estructuras del ensamble NMR fueron mucho peores
que el de la estructura cristalizada y resuelta por rayos X, por lo tanto se uso esta
solamente


La evaluación de la performance se hizo mediante 3? tests que se aplican sobre las dos PSSM: la PSSM basada solo en información secuencial y los mismos tests sobre la PSSM que combina información secuencial y estructural. Los tests consisten en calcular el MCC para la evaluación del set de entrenamiento, calcular el MCC mediante una cross-validation(se separan distintos grupos -***EN BASE A QUE???** del set de entrenamiento inicial y se generan nuevos PSSM en base a esto y evaluándolo sobre el resto del conjunto. Se calcula el MCC para cada combinación de grupos y se saca el valor medio), el otro test es calcular el MCC resultante de evaluar contra el conjunto independiente(separado al principio) para la PSSM entrenada con todo el conjunto de pruebas.
El resultado da que las 2 primeras pruebas son un poquito mejores para la PSSM hecha solo en base a secuencias, pero en la prueba sobre el conjunto independiente la PSSM basada solo en secuencias da muy mal y la PSSM con informacion secuencial es considerablemente mejor. Esto indicaría que la informacion estructural ayuda a hacer el predictor mas general.


*****************************************************
FALTA DESDE DONDE DICE:  Although the heptapeptides in the learning sets were selected on a methodologically acceptable basis, inconsistencies in the learning set selection could not be excluded.


% \section{Formación de fibras amyloides}

\section{Formación de agregados de proteínas}
En el review \cite{hamodrakas2011protein} hay una gran cantidad de software/metodos existentes para predecir agregacion desestructurada y formacion de fibras amiloides

Each method makes its own assumptions and implements its own predictors, which range from quite simplistic to quite complex. 
The ability to form b-strands is a predominant feature in most works, either in the form of statistical propensities or in the form of structural stability.

% 
% Como conlusion interesante en este paper(ver mas en detalle)
% se tiene que: the introduction of aggregation-disrupting
% amino acid substitutions in the aggregation-
% prone ⁄ amyloidogenic short sequence regions suggests
% the possibility of fine-tuning and controlling the solu-
% bility of proteins, synthesized by recombinant technol-
% ogy in bacterial cell factories.
% Basicamente plantean analizar las secuencias mediante alguno/s de estos software y mutar las posiciones que poseen la tendencia a formar amyloids.
% se mutan por cualquiera o por cuales????
% 


\subsection{Tango}

General sobre agregacion, beta-aggregation, amyloid fybrils, etc
De (1): although aggregation and amyloidosis correlate to a certain extent, they are different
processes and should be treated as such.

b-Aggregation and amyloidosis often co-occur in these disease-associated protein aggregation processes and, when this is the case, the former is frequently observed as a precursor of the latter. It is now also generally accepted that a subgroup of these prefibrillar
aggregates, not the mature fibers themselves, are associated with cytotoxicity.

Some proteins form fibers that are non-toxic and probably even functionally relevant,
whereas other proteins form toxic aggregates without forming fiber

Comparing the sequence space of b-aggregation predicted by TANGO or Zyggregator with the sequence space of amyloidosis derived from experimental studies of the STVIIE amylogenic peptide reveals the similarities, but also interesting differences between both processes:

-As both amyloid formation and amorphous cross-b aggregation require amino acid compositions that are compatible with a b-strand conformation, an overlap in
sequence space is to be expected

-Sin embargo, la estructura de amorphous cross-b aggregates is not
clearly defined and seems to be characterized by a high degree of flexibility. On the other hand, the structure of amyloid fibers is quasi-crystalline. As a consequence, amino acid preferences will be much more position specific in an amyloid fiber than in amorphous cross-b aggregates.

Amylogenic sequences are therefore more position specific, but also more tolerant to polar and charged residues than b-aggregating sequences. This will also have consequences on the kinetics of both processes: Due to its less stringent conformational requirements, b-
aggregation is generally much faster than amyloidosis]. 
As b-aggregates are often observed as precursors on the path to fiber formation, the stability of these precursor aggregates will strongly influence the kinetics of amyloidosis: Stable b-aggregated amyloid precursors will therefore probably slow down amyloidosis

In summary, amorphous cross-b aggregation and amyloidosis can occur in common, and the stability and kinetics of both processes will be determined by the extent to which the structural requirements of both processes are fulfilled.


It can be considered that aggregation-sensitive protein sequences are the price to be paid for the existence of globular protein structures: as tertiary sidechain interactions mainly occur in the hydrophobic core, protein stretches spanning this region generally have a propensity to aggregate. Accordingly, intrinsically disordered proteins that lack tertiary structure are much less hydrophobic and thus have a much lower aggregation propensity [41]. However, for native 

globular proteins, aggregation is generally not an issue, as aggregation- prone protein stretches are generally sequestered bythe protein structure and thereby protected from self-
association [45]. On the other hand, during proteintranslation and folding, or in the case of cellular stress or destabilizing mutations, partially unfolded states aremuch more likely to self-associate and induce aggregation and amyloidosis



As aggregation cannot be completely eliminated, because of structural constraints from the native state, aggregation is further contained by placing not only charged residues but also prolines and glycines at the flanks of aggregating sequence segments. These effectively act as gatekeeper residues, opposing aggregation and thereby promoting the native folding reaction. It has also been reported on several occasions that the introduction of charged residues, prolines or glycines in aggregation-prone sequences reduces aggregation.
The aggregation-opposing properties of proline and glycine originate primarily from their
structure-breaking properties. Identically charged residues are also very effective at opposing aggregation, because of the huge repulsive force generated upon self-assembly.


El método se deriva a partir de aplicar la mecanica estadistica a un conjunto de datos experimentales obtenidos de los residuos dentro de proteinas que promueven el 'agregamiento ordenado' o la formacion de amyloids. El algoritmo resultante permite identificar las regiones de agregacion beta dentro de una secuencia.
El algoritmo tiene en cuenta un conjunto de estados(conformaciones) posibles: alfa helice, hojas beta, beta-turns, estado plegado(nativo??*****ver abajo) y agregado de hojas beta. Ademas, se asume incialmente que la region en el estado de agregado de hojas beta esta totalmente internalizada y tiende a satisfacer el potencial de puentes de hidrogeno. 
El metodo TANGO tiene tambien en cuenta la estabilidad de la proteinas y los parametros fisicoquimicos tales como el pH, la concentracion, la fuerza ionica y la concentracion de trifluoroetanol.

Cada segmento de una proteina puede estar en alguno de los estados conformacionales de acuerdo con una distribucion de Boltzman. Es decir, la frecuencia con la que ese segmento adopta un estado en una poblacion es relativa a su energia, la cual se deriva de consideraciones estadisticas a partir de datos empiricos.  
Para predecir la agregado beta de un segmento, Tango calcula la funcion de particion del espacio de fases conformacionales.


****la inclusion del estado plegado dentro de la funcion de particion tiene como objetivo ayudar a predecir efectos de agregacion debido a mutaciones puntuales en proteinas que naturalmente adquiren un estado plegado. La inclusion de este estado permite ver la competencia entre este estado natural de plegado y otros estados estructurales incluidos en la particion. De esta forma se puede predecir la tendencia a la agregacion del estado desnaturalizado y tambien las mutaciones que aumentan la tendencia a la agregacion de la proteina desestabilizando el estado plegado.


Like all algorithms that use averaged physicochemical properties to detect aggregation hot spots, TANGO is not specific for amyloid formation or amorphous beta-aggregation. However, amyloid structures are a very specific subclass of aggregates formed by sequences that allow the intermolecular beta-sheet arrangements to pack in a well defined three dimensional structure, resulting in the formation of highly stable amyloid fibrils. The biological properties of these fibrils differ critically from those of amorphous aggregates: amyloid fibrils are highly stable nanowires that are used throughout all kingdoms of life as structural scaffolds, adhesives, water tension modulators etc. Also, protein deposits found in association with a range of human diseases are most often enriched in amyloid structure, which is probably also due to their stability. In order to specifically predict amyloid structure, we developed the Waltz algorithm





\subsection{PASTA}
PRIMERO HAGO UNA DESCRIPCION DE LO QUE SON LAS FIBRAS AMILOIDES Y POR QUE LAS QUEREMOS SACAR (ESTO ES GENERAL PARA TODOS LOS METODOS)

fibras amiloides: convierte a la forma soluble de la proteina en agregados fibrilares altamente organizados.

% En nuestro caso, la formacion de amiloides fibrilares entre los linkers ¨arrastraria¨ a los dominios que unen y podria interferir en la funcionalidad de estos. La funcionalidad de cualquier dominio proteico se podria ver limitada si esta se encuentra formando algun tipo de agregado.

Amyloid formation is not restricted, however, to those
polypeptide chains that have recognised links to protein
deposition diseases. Several other proteins that have no such
link have been found to form fibrillar aggregates in vitro with
morphological, structural, and tinctorial properties that
allow them to be classified as amyloid-like fibrils [4,5]. This
finding has led to the idea that the ability to form the amyloid
structure is an inherent property of polypeptide chains,
encoded in main backbone chain interactions.

Esto ultimo es importante porque al ser una caracteristica comun es bastante probable que ocurra este tipo de agregacion. Ademas, no podriamos simplemente descartar las proteinas que se sabe tienen tendencia a formar fibras amiloides, al ser una propiedad intrinseca de las secuencias polipeptidicas es necesario analizar todas las secuencias.


Los metodos que utilizaremos para intentar eliminar la tendencia a formacion de fibras amiloides son: busqueda de determinantes secuenciales, PASTA, TANGO.


**************************


Pasta asume que el mecanismo (las interacciones) que llevan a la formacion de beta-sheets en proteinas globulares es el mismo que el que lleva a la formacion de beta-sheets en estructuras cross-beta(estructura asociada a fibras amiloides).
En primer lugar, Pasta deriva una funcion energetica a partir de un conjunto de datos de estructuras globulares (potenciales estadisticos). Para derivar la funcion, lo que hace es ver cual es la probabilidad de encontrar cierto par de residuos dentro de una hoja beta, enfrentados en hebras vecinas. Esto resulta en una matriz con un valor asociado para cada par de residuos posibles.
El metodo asume que la forma soluble es nativamente desestructurada, por lo que hay que tener cuidado cuando se lo usa para predecir proteinas que nativamente adquieren una estructura globular.
Ademas, se asume que las moleculas involucradas en la formacion de pares de interaccion adoptaran la forma en la que el emparejamiento tenga la menor energia.
el valor de este potencial estadistico para cada clase(paralela,antiparalela, etc) se calcula segun la relacion entre la frecuencia observada y la frecuencia esperada:
la frecuencia observada es el numero de pares ab que estan en esa clase / el total de pares ab encontrados.
La frecuencia esperada se aproxima como: el numero de pares ab (para cualquiera ab) que esta en una clase dada / el numero de pares ab (para cualquier ab) que hay.

Se extraen valores de potencial segun esten interaccionando en sentido paralelo o antiparalelo.

Durante el analisis se usan estos conjuntos para asignar un score a cada emparejamiento especifico entre subsecuencias de la misma longitud (PERTENECIENTES A MOLECULAS DISTINTAS CON LA MISMA SECUENCIA). Durante la evaluacion se prueban todos los emparejamientos posibles (todas las subsecuencias posibles, pero siempre las subsecuencias de ambas moleculas son de la misma longitud).
Para cada posible subsecuencia se analiza tanto la posibilidad de que esten emparejados paralelamente o de forma antiparalela (tienen distinta tendencia por lo que el score final es distinto)

Entonces: dada una longitud L , un par de indices i,j (correspondientes al inicio de la subsequencia de interaccion en las moleculas 1 y 2), se puede deducir un score para el emparejamiento beta paralelo y otro para el antiparalelo. Estos scores se calculan como la suma de todos los potenciales estadisticos asociados a los residuos que se estan emparejando en la conformacion, mas un valor de diferencia en la entropia que corresponde a la perdida de entropia por el emparejamiento de estos L residuos. Esta diferencia se calcula como L*deltaS, donde deltaS es la perdida de entropia promedio por residuo.


Estos scores energeticos permiten saber cual de todos los emparejamientos posibles tendra una menor energia, y por lo tanto(segun lo asumido previamente) sera el que adoptaran las secuencias EN CASO QUE HAYA UNA AGREGACION FORMANDO FIBRAS AMILOIDES.



Para conocer mejor cual es la dependencia que tiene la secuencia con la tendencia a formar agregados, es posible calcular un valor de propensity(tendencia) a formar un agregado amiloide que tiene cada residuo de la secuencia. Para esto se usa una aproximacion de mecanica estadistica, que relaciona el peso de la enegia asociada a todos los emparejamientos en lss que participa cierto residuo(usando una distribucion de boltzman) con respecto al peso total de todos los emparejamientos posibles de una conformacion de agregado amiloide. 
ESTE VALOR DA LA TENDENCIA QUE TIENE CADA RESIDUO A FORMAR PARTE DEL EMPAREJAMIENTO EN CASO QUE LA SECUENCIA ADQUIERA UNA CONFORMACION DE AMILOIDE, NO RELACIONA EL VALOR ESTADISTICO DE LA ENERGIA CONFORMACIONAL CON RESPECTO A CUALQUIER OTRA CONFORMACION (ESTRUCTURA GLOBULAR, ALFA-HELICE, ETC)

El valor de propensity es util para saber, en caso de formar agregados amiloides, cuales serian los segmentos que formaran las uniones.
El valor de la tendencia por residuo es un valor relativo de la tendencia que tiene cada residuo a formar conformaciones amiloides estables, con respecto a la tendencia que tienen los demas residuos. La suma total de los valores de toda la secuencia da 1?? de esta forma, el valor de la tendencia serviria para saber, en caso de formar una fibra amiloide, cuales son los residuos involucrados.


Lo que nosotros queremos conocer son los residuos que podrian formar conformaciones de fibra amiloide(interacciones beta) con una energia potencial(score) suficientemente chico, es decir, conformaciones suficientemente estables como para que esta estructura se adopte realmente en un % considerable.
Para lograr esto debemos obviamente definir un valor de cutoff ya que el termino "suficientemente chica" es algo relativo. De esta forma, todos los residuos que tengan un valor de perfil energetico (masomenos este valor es la suma de score de todas las conformaciones en las que esta involucrado el residuo) menor que el valor de cut-off, seran considerados como los posibles estabilizadores de la estructura y por lo tanto se marcan como targets para la mutacion en nuestro algoritmo.
El valor de cutoff se debera elegir de forma tal que se balancee la sensibilidad y la especificidad en la deteccion.


In this version of the server we included methods for secondary structure and intrinsic disorder, which provide additional reinforcement to the fibril assignment. Briefly, a new machine learning algorithm was constructed to detect secondary structure while our previously developed disorder predictor was also included.


Basicamente, lo que dicen es que para formar las fibras amiloides, la secuencia tiene que tener tendencia a interaccionar estando en una estructura beta. Si analizamos cuales son los pares de residuos que mas interaccionan(estadisticamente) entre diferentes hebras dentro de una hoja plegada beta, entonces podemos predecir la tendencia que tendra una secuencia, analizando solamente su composicion.



Otra caracteristica del servidor PASTA 2.0 es que implementa un mecanismo basado en redes neuronales(ESpritz: accurate and fast prediction of protein disorder) para predecir la tendencia a formar estructuras secundarias o a permanecer en una conformacion desordenada. 





\subsection{Waltz}


Otra de las herramientas que se utilizó para analisis de tendencias a formar fibras amiloides es Waltz \cite{maurer2010exploring}

Waltz allows users to identify and better distinguish between amyloid sequences and amorphous $\beta$-sheet aggregates and allowed us to identify
amyloid-forming regions in functional amyloids.

% RESUMEN DEL METODO **************************
EL METODO SE BASA EN UN PROFILE DE SCORES(SCORES POR AA POR POSICION). ESTE SCORE TIENE 3 COMPONENTES:
-UN COMPONENTE ES DERIVADO DE INFORMACION SECUENCIAL: A PARTIR DE UNA BASE DE DATOS DE HEAXPEPTIDOS POSITIVOS Y NEGATIVOS PARA FORMACION DE AMYLOID (VER COMO SE REFINO ESTE CONJUNTO DE DATOS) 
SE HIZO UN ALINEAMIENTO Y SE SACO UNA PSSM CON LAS PROBABILIDADES PARA CADA AA Y CADA POSICION.
-OTRA PARTE DEL SCORE ESTA DADA POR UN CONJUNTO DE 19 PROPIEDADES FISICAS 
-The final component of the scoring function is the position-specific pseudoenergy matrix from structural modeling using amyloid backbone structures
% ************************************************

% 
% ****************** RESUMEN SACADO DEL REVIEW
% 
%  waltz combines sequence, physicochemical as well as structural information into a composite scoring function.
% web-based tool that uses, mainly, a posi-
% tion-specific scoring matrix (PSSM) to determine amy-
% loid-forming sequences [39]. The PSSM was built
% based on the experimental exploration of the sequence
% space of amyloid hexapeptides. According to its
% authors, waltz allows for identification and better dis-
% tinction between amyloid sequences and amorphous
% b-sheet aggregates, and also allows for identification of
% amyloid-forming regions in functional amyloids. In
% more detail, the waltz algorithm was developed by
% combining specific sequence information with physico-
% chemical as well as structural information.
% The PSSM for amyloid propensity of waltz, was
% constructed from an experimentally defined training set
% comprising 116 ‘positive’ (amyloid-forming) hexapeptides and 162 ‘negative’ (non-forming) hexapeptides
% (http://waltz.switchlab.org/).
% The position-specific score for
% an amino acid was calculated as a standard log-odd
% score in a position-specific scoring matrix (the value
% for each amino acid at each position is the logarithm
% of the ratio of its frequency in the training set and the
% background database).
% 
% As the analysis of the hexapeptide experimental data
% sets (positive and negative) may impose sequence bias
% specific to the available data, the authors of waltz
% estimated the preference or non-preference of amino
% acids at the hexapeptide motif positions on a structural
% basis using the atomic force field foldx. The fibril
% crystal structure of the GNNQQNY peptide from
% Sup35 (PDB 1YJP) was first simplified to polyalanine.
% Then, all possible pair combinations of all 20 natural
% amino acids at all positions were generated and
% energy-optimized using foldx [40]. Energy estimates
% were calculated with foldx as the DG difference
% (DDG) to the reference polyalanine. To retrieve a posi-
% tion-specific pseudoenergy matrix for the prediction
% scoring function (and calculate S struct ), they averaged
% for each amino acid the energies for all its occurrences
% at a certain position in combination with all amino
% acids at other positions [39]. 



To better understand the sequence determinants of amyloid structure, we explored the sequence diversity of amyloid hexapeptides by inspecting more than 200 peptides using various
structural and biophysical methods.
Our analysis highlighted the strong
position-specific tendencies of the different amino acids for form-
ing amyloid structures observed both in disease-related as well as
in functional amyloids.

Como se vio hasta ahora, most of the experimentally characterized amyloid sequences available to this date are hexapeptides.
Community generated, experimentally
verified amyloidogenic hexapeptides are avaailable in the AmylHex 
database consisting of 67 positive and 91 
negative examples that have been used to benchmark new methods.
Although there is no doubt that amyloid nucleating sequences in full- ength proteins are often longer than six resi-
dues 29 and that the full-length protein context can considerably
modulate amyloid propensity 30 , this length of six residues has
been found to represent a minimalistic amyloid nucleating core,
which as an insertion is sufficient to induce amyloid conversion
of an entire protein domain 2 . The use of short peptides therefore
minimizes the risk of alignment errors, whereas for longer amy-
loid sequences it is often not clear whether all residues participate
in the $\beta$-sheet or are just tolerated dangling ends.

The AmylHex database, however, has a high sequence redun-
dancy, as it has a strong overrepresentation (51\%) of point muta-
tions of the amyloidogenic hexapeptide STVIIE 15 . To unbias
this dataset to more generally reflect amyloidogenic 
properties in naturally ocurring peptides, we used the AmylHex set as a
starting point to construct a first position scoring matrix, which
we used to select increasingly divergent hexapeptides (Online
Methods). In this manner, we identified and experimentally
validated 49 new amyloid hexapeptide sequences as well as
71 negative (non-amyloid-forming) hexapeptide sequences
(Supplementary Table 1).
To unequivocally distinguish amyloid
fibrils from nonfibrillar $\beta$-aggregates, we relied on a combination
of electron microscopy (Fig. 1a and Supplementary Fig. 1), cir-
cular dichroism and Fourier transform infrared (FTIR) spectroscopy. 
For selected cases, we also collected X-ray diffraction data
(Fig. 1a), which revealed oriented diffraction patterns, all showing
the characteristic diffraction signals for cross–$\beta$-amyloid fibrils 16 .

When we added the 49 new sequences we identified to the exist-
ing 67 AmylHex peptides, our training set of amyloid forming hexapeptides was comprised of 116 positive and 103 negative
sequence example. 
This increases the amount of learning examples
compared to the AmylHex database by 70%. More importantly,
we increased the number of nonredundant positive sequences
by 370\%, allowing for a better mapping of the position-specific
sequence propensities of amyloidogenic hexapeptides.



We developed the Waltz algorithm by combining the position-
specific sequence information described above with physicochemical as well as structural information described in Online
Methods and Supplementary Notes 1 and 2. A sequence score
S profile was calculated from the log-odd based position-specific
coring matrix (PSSM) (Online Methods). Nineteen selected
physical properties, which best describe amyloid propensity enter
the scoring function as a physical property term S physprop consist-
ing of the sum of the products of the amino acid frequency with
the normalized property value of the respective amino acid for
each position (Supplementary Fig. 2). The final component of
the scoring function is the position-specific pseudoenergy matrix
from structural modeling using amyloid backbone structures
(S struct ). Relative weightings a of individual terms was introduced
for a balanced scoring function
S total = a profile S profile + a physprop S physprop + a struct S stru c t (1)

To test the sensitivity of Waltz, we searched the UniProt data-
base for amyloid-forming hexapeptide sequences and experi-
mentally validated predictions by transmission electron
microscopy (TEM) and spectroscopic techniques.




Given
the low abundance and diversity of experimentally validated
amyloid-forming sequence sets, we did not have sufficient infor-
mation to develop a reliable method to differentiate the specific
amyloid morphology from amorphous $\beta$-sheet aggregates.
To address this issue, we expanded the existing AmylHex hexa-
peptide set to include increasingly diverging amyloid sequences,
increasing the sequence diversity of the existing dataset by almost
400\%. Waltz is unique in that it is, to our knowledge, the first
position-specific amyloid sequence prediction tool that can be
used to distinguish between amyloid and amorphous aggre-
gate sequences.






\subsection{Determinantes secuenciales}
Ya se ha mencionado en varias oportunidades hasta aquí la relación entre la secuencia y la predisposición a formar fibras amiloides in vivo. En el trabajo realizado en \cite{de2004sequence} se analiza esta relación en profundidad mediante un 
analisis de mutagenesis a partir de un péptido amiloide diseñado \textit{de novo} en \cite{de2002novo}. El ensayo consiste en reemplazar sistematicamente los residuos de este péptido(STVIIE) con todos los AAs naturales, excepto Cys.

Este estudio sigue la linea de utilizar péptidos cortos para investigar elementos de la secuencia que favorecen la agregación, 
basándose en la idea que este proceso(agregación) es conducido(generado?) por fragmentos cortos de proteinas mal plegadas.

El trabajo implica la evaluación experimental de los péptidos resultantes de la mutación
El análisis revela una dependencia posicional con la formación de este tipo de estructuras de agregación, existiendo posiciones muy tolearantes y muy restrictivas a las mutaciones.
Como parte de tal trabajo, se deriva a partir de los resultados de esta mutación sistematica, un patrón secuencial que permitiría identificar tramos de secuencias formadores de fibras amiloides.

El patrón resultante es:

$\{P\}_1 -\{PKRHW\}_2 -[VLS(C)WFNQ]_3 -[ILTYWFN]_4 -[FIY]_5- \{PKRH\}_6 $

% Because the sequence space explored in this work is very small, we are conscious that many dangerous motifs cannot be detected unless a more complete pattern is provided. Therefore, similar experiments on unrelated amyloid sequences would be very
% useful to refine this amyloid fingerprint.
% 
% DISCUSION DEL PATRON ENCONTRADO
% In the light of the knowledge acquired in this study, we believe that, in same way that it has been shown for globular proteins, there are general rules governing the amyloidogenicity of a polypeptide chain.
% Because a good resolution structural model of the fibrils formed by these peptides is still lacking, any discussion about the thermodynamic origin of the effect of mutation on amyloid formation would be difficult and rather speculative. Therefore, we
% have provided only a descriptive analysis of the sequential dependence found.

% EL ESTUDIO EXPERIMENTAL SE BASA EN MEDICIONES A DISTINTOS TIEMPOS(ENTRE T=0 TIEMPO=T=1mes). EN BASE A ESTO SE OBTIENE CUALES SON LAS MUTACIONES QUE MAS ACELERAN LA FORMACION DE ESTE TIPO DE FIBRAS, Y TAMBIEN CUALES MUTACIONES 
% SON LAS QUE DAN LA MAYOR FORMACION DE FIBRAS FINAL (A TIEMPO=t), LO QUE SE ENCUENTRA ES QUE ESTOS 2 EFECTOS NO SIEMPRE COINCIDEN EN LA MISMA MUTACION
% At the most tolerant positions (1, 2, and 6), one can find many substitutions that accelerate beta-sheet polymerization dramatically
% Interestingly, the more restrictive the position is, the less the number of amino acid that are capable of accelerating the process. At position 5, no substitution accelerates the process at all.
% Amino acid replacements producing abundant amyloid products are not always the substitutions that allow for a faster beta-sheet polymerization
% *************


Para validar experimentalmente el patrón, se seleccionarion algunas combinaciones de mutaciones que, de forma individual, daban altas capacidades de formación de fibras amiloides. 
% VER COMO SE HACE LA EVALUACION EXPERIMENTAL 

Este patrón fue también validado \textit{in silico}. En este análisis, se encontró que las secuencias de una base de datos que matcheaban el patron eran menos frecuentes en proteinas que las combinaciones innocuas y que, en caso de encontrarse, estaban rodeadas de aminoácidos que 
rompen esta capacidad de agregación(conocidos como amyloid breakers).


% *************

En nuestra aplicación, implementamos la búsqueda de este patrón como parte del conjunto de evaluaciones que se hacen sobre la secuencia. El resultado de buscar la expresión regular asociada a este patron secuencial podría aportar subsecuencias
que generen una tendencia a formar este tipo de fibras amiloides.

\section{Secuencias transmembrana}
\cite{sonnhammer1998hidden}
\cite{krogh2001predicting}


\section{Carga neta de la secuencia}

Otro de los motivos por las cuales se podría querer eliminar aminoácidos cargados es cuando se esta diseñando una proteína que se une a dna porque los residuos del linker podrían 
entonces formar puentes salinos con los fosfatos del dna



\section{Otras posibles herramientas}
Homologias secuenciales mas distantes? ej mediante HMM (usando HMMer por ej.)

Busqueda de dominios en BBDD: Known domains are now stored in a number of databases including Pfam [1], SMART [13], CDD [14] and InterPro [15]. % ESTAS BBDD ALMACENAN SOLO DOMINIOS GLOBULARES????????
Particular domains and domain architectures are well conserved over the course of evolution. The sequences diverge, but the overall domain architecture remains the same.


Protein interaction databases such as BIND and DIP. More informative protein interaction databases that store known instances of linear motifs (36) include MINT (37), Phosphobase (20) and ASC (38). 
Databases of instances are not directly useful for prediction but provide valuable data-mining resources.

% CAPITULO 4: EVALUACION DE LA HERRAMIENTA Y RESULTADOS
\chapter{Evaluaciones y análisis de resultados}
\label{results}

\appendix
%% Cap'itulos incluidos despues del comando \appendix aparecen como ap'endices
%% de la tesis.
%\include{apendiceA}
%\include{apendiceB}
%\include{apendiceC}

%% Incluir la bibliograf'ia. Mirar el archivo "biblio.bib" para m'as detales
%% y un ejemplo.
\bibliography{biblio,linkers,arquitectura}

\end{document}
