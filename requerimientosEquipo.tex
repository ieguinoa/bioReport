\chapter{Instalación y equipo utilizado}\label{equipo}

\section{Instalación}\label{install}

El software fue desarrollado para correr en un entorno Linux.
Para poder instalar PATENA se necesita primero tener instalados los siguientes paquetes de software:

\begin{itemize}
 \item Python $>$2.7 \cite{python}
 \item BioPython \cite{biopython}
 \item Perl \cite{perl}
 \item Blast software command line \cite{blastLocal} (asegurarse que el ejecutable esté en el path del usuario)
 \item Git \cite{git} (necesario sólo para la instalación)
 %  \item Standalone BLAST Setup for Unix \cite{blastLocal} % http://www.ncbi.nlm.nih.gov/books/NBK52640/
\end{itemize}

Una vez que se cumplen todos los requerimientos, el proceso de instalación de la herramienta requiere dos pasos: \\ 

\noindent 1. \hspace{0.5cm} \texttt{git clone https://github.com/ieguinoa/patena}\\
2. \hspace{0.5cm} \texttt{source install.sh}







\section{Equipamiento utilizado para las pruebas}\label{equipamiento}

% Para realizar las pruebas se utilizó el siguiente equipamiento:

\begin{center}
\begin{description}
 \item[Sistema operativo:] Ubuntu 14.04 - Kernel version: 3.13.0
\item[CPU:] Intel(R) Core(TM) i7-4770 CPU @ 3.40GHz
\item[Memoria RAM:] 16GB
\end{description}
\end{center}
% \end{itemize}


