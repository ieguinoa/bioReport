\chapter{Manual de uso}\label{manual}


\section{Parámetros de ejecución}\label{parametros}

% El conjunto de parámetros corresponde al , no todos los parámetros serán parte de la versión publica en el servidor

\subsection{Secuencia inicial} \label{secuenciaInicial}
\subsubsection{Secuencia inicial random}\label{secuenciaInicialRandom}
Por default (si no se agrega ningún parámetro), la secuencia inicial se generará una secuencia random con la composición designada para la ejecución (ver \ref{composicion}).\\
Por default esta secuencia tendrá una longitud inicial = 10 AAs.

La longitud(en número de residuos) de esta secuencia inicial puede ser definida por el usuario mediante el parámetro \texttt{length} de la forma: \\
\indent \texttt{--length [seq-length]}
\\Por ejemplo: \\
\indent \texttt{--length 30} \hspace{0.5cm} Inicia la ejecución con una secuencia random de largo = 30 AAs


\subsubsection{Secuencia inicial definida}\label{secuenciaInicialDefinida}
Para definir una secuencia inicial especifica se debe usar el parámetro \texttt{seq} de la forma: \\
\indent \texttt{--seq [secuencia]} 
\\Por ejemplo: \\
\indent \texttt{--seq AAHHWWWLLLLHHGGG} \hspace{0.5cm} Inicia la ejecución con la secuencia \texttt{AAHHWWWLLLLHHGGG}

En caso de especificarse una secuencia inicial no es necesario definir longitud y, si se define, es ignorada.

\subsection{Composición de la secuencia} \label{composicion}

\subsubsection{Composición estándar}
% Existen dos opciones para definir la composición usada durante la ejecución del método.
Si no se define ningún parametro al respecto, la composición \textit{default} que se usa es la composición estándar obtenida de SwissProt \cite{compositionAA}.  

En caso que se haya seleccionado la opcion de secuencia silente en UV (ver sección \ref{uvsilent}) la composición es igual a la estándar pero los residuos Y, T y F tendrán una frecuencia igual a 0.

\subsubsection{Composición definida por el usuario}
Es posible también definir una composición a medida, para esto se deben definir las frecuencias de los residuos

En caso que se haya seleccionado la opcion de secuencia silente en UV (ver sección \ref{uvsilent}) se debe seleccionar una frecuencia igual a 0 para los residuos Y, T y F. 


\subsection{Modificación del parámetro Beta}


\subsection{Evaluación de la secuencia}\label{evaluacion}


\subsubsection{Selección del set de herramientas a utilizar}
% seleccion de herramientas a evaluar


\subsubsection{Evaluación de carga neta}
% evaluacion de la carga neta

La carga neta deseada para la secuencia resultante puede ser definida mediante la opción \texttt{netcharge} de la forma: \\
\indent \texttt{--netcharge [carga neta deseada]} 
\\Por ejemplo: \\
\indent \texttt{--netcharge +3} \hspace{0.5cm} La secuencia resultante tendrá carga neta = +3

El valor absoluto de la carga buscada debe ser menor o igual que la longitud de la secuencia, de otra forma no podría alcanzarse nunca la carga neta objetivo.
Esta condición se evalúa al inicio de la ejecución y se informa si no es posible continuar


\subsubsection{Evaluación de silente en UV}\label{uvsilent}

Es posible restringir el conjunto de secuencias resultantes a aquellas que no absorban en el rango de UV, es decir, que no posean residuos Y, T o F.
La forma de indicar esto es mediante la opción \texttt{--uvsilent}(sin parámetros)



\subsection{Condición de finalización} \label{condicionFin}

\subsection{Formato y detalles del resultado}\label{output}
Los parámetros que permiten modificar el formato de la salida son:
\vspace{0.2cm}\\
\texttt{--verbose} \\
\indent \indent Devuelve una salida detallada de la ejecución \\
\texttt{--minoutput} \\
\indent \indent Solo devuelve el resultado final y guarda el historial de mutaciones en un log \\
\texttt{--stepped} \\
\indent \indent Espera por input del usuario en cada paso del método. \\
