\chapter{Manual de uso}\label{manual}


El método descrito en los capítulos previos se encuentra disponible bajo el nombre de \mbox{PATENA}. 
El origen del nombre se encuentra en la descripción del método, metafóricamente, como un mecanismo para ``limpiar'' secuencias, eliminando elementos funcionales y estructurales.
De esta forma, las secuencias resultantes podrían describirse usando la expresión popular ``limpias como una patena''.
Esta expresión hace alusión al platillo de metal en el que se pone la hostia durante la celebración eucarística (\url{https://es.wikipedia.org/wiki/Patena}), el cual se mantiene siempre limpio y brillante.

% La metáfora hace alusión a que los resultados se corresponden con secuencias ´´limpias como una patena''. dado que, para el usuario, el método puede verse como un   . 


\section{Instalación}


%Los requerimientos de software y el proceso de instalación se detallan en el apéndice \ref{install}.

El software fue desarrollado para correr en un entorno Linux.
Para poder instalar PATENA se necesita primero tener instalados los siguientes paquetes de software:

\begin{itemize}
 \item Python $>$2.7 \cite{python}
 \item BioPython \cite{biopython}
 \item Perl \cite{perl}
 \item Blast software command line \cite{blastLocal} (asegurarse que el ejecutable esté en el path del usuario)
 \item Git \cite{git} (necesario sólo para la instalación)
 %  \item Standalone BLAST Setup for Unix \cite{blastLocal} % http://www.ncbi.nlm.nih.gov/books/NBK52640/
\end{itemize}

Una vez que se cumplen todos los requerimientos, el proceso de instalación de la herramienta requiere dos pasos: \\ 

\noindent 1. \hspace{0.5cm} \texttt{git clone https://github.com/ieguinoa/patena}\\
2. \hspace{0.5cm} \texttt{source install.sh}


Una vez instalado, la ejecución se realiza mediante: 

\texttt{python patena.py [parámetros de ejecución]}


\section{Parámetros de ejecución}\label{parametros}

% El conjunto de parámetros corresponde al , no todos los parámetros serán parte de la versión publica en el servidor

\subsection{Secuencia inicial} \label{secuenciaInicial}
\subsubsection{Secuencia inicial random}\label{secuenciaInicialRandom}
Por default, la ejecución comenzará a partir de una secuencia random generada utilizando la composición designada para la ejecución (ver \ref{composicion}).
% la secuencia inicial se generará una secuencia random con la composición designada para la ejecución (ver \ref{composicion}).\\
La longitud predeterminada para la secuencia es de 10 aminoácidos. 
El usuario puede modificar este valor mediante el parámetro \texttt{length} de la forma: 

\indent \texttt{--length [longitud deseada]}
\\Por ejemplo: \\
\indent \texttt{--length 30} \hspace{0.5cm} \textit{Inicia la ejecución con una secuencia random de largo = 30}



\subsubsection{Secuencia inicial definida por el usuario}\label{secuenciaInicialDefinida}

Para definir una secuencia inicial específica se debe usar el parámetro \texttt{seq} de la forma:

\indent \texttt{--seq [secuencia]} 
\\Por ejemplo: \\
\indent \texttt{--seq AAHHWWWLLLLHHGGG} \hspace{0.5cm} \textit{Inicia la ejecución con la secuencia \texttt{AAHHWWWLLLLHHGGG}}

En caso de especificarse una secuencia inicial no es necesario definir longitud y, si se define, es ignorada.



% PONER detalles de como definir regiones estáticas que no pueden ser mutadas
% \paragraph{Secuencias flanqueantes}\label{flanqueantes} \hspace{0pt} \\
\subsubsection{Secuencias flanqueantes}\label{flanqueantes} 

Es posible especificar segmentos en uno o ambos extremos que, si bien no serán parte del linker ni serán mutados, si se tendrán en cuenta a la hora evaluar la secuencia.
% es decir, que deben permanecer intactos en el diseño resultante.
% Estas regiones flanqueantes permanecerán intactas en el diseño resultante, sin embargo, serán tenidas en cuenta en la etapa de evaluación de la secuencia.
Para indicar estas secuencias flanqueantes se proveen los parámetros \texttt{left} y \texttt{right} los cuales definen secuencias hacia los extremos N- y C-terminal respectivamente.
El formato para especificarlas es:


\indent \texttt{--left  [secuencia flanqueante N terminal]} \\
\indent \texttt{--right [secuencia flanqueante C terminal]} 
\\Por ejemplo: \\
\indent \texttt{--left AHWLLHHGG} \hspace{0.5cm} \textit{Se asume que a la izquierda del linker se encuentra la secuencia \texttt{AHWLLHHGG}}





\subsection{Composición de la secuencia} \label{composicion}

\subsubsection{Composición estándar}
% Existen dos opciones para definir la composición usada durante la ejecución del método.
Si no se define ningún parámetro al respecto, la composición \textit{default} que se usa corresponde a la composición estándar obtenida de SwissProt \cite{compositionAA}.  

En caso que se haya seleccionado la opción de secuencia silente en UV (ver sección \ref{uvsilent}) la composición es igual a la estándar pero los residuos que absorben en el rango del UV (W,Y,F) tendrán una frecuencia igual a 0.

\subsubsection{Composición definida por el usuario}

Es posible que el usuario defina una frecuencia específica para uno o más aminoácidos. 
La frecuencia se debe especificar en forma de porcentaje y cada aminoácido tiene un parámetro asociado para definir su frecuencia. 
Esto parámetro se corresponde con la nomenclatura de una sola letra asociada al aminoácido. Ejemplos de esto son:

\indent \texttt{--a  [frecuencia Alanina]} \\
\indent \texttt{--m  [frecuencia Metionina]} \\
\indent \texttt{--s  [frecuencia Serina]} 
% \indent \indent .............

En caso que no se definan las frecuencias para todos los 20 tipos de aminoácidos, el resto tendrá la frecuencia estándar. El único requerimiento es que la suma total de las frecuencias sea menor o igual a 100.
A modo de ejemplo:

\indent \texttt{--a 50 --l 10 --g 13.3 } \\
\textit{Esta definición de la composición generará una secuencia inicial con 50\% de Alanina, 10\% de Leucina y 13.3\% de Glicina. 
El porcentaje restante tendrá la distribución correspondiente a la composición estándar.}



En caso que también se haya seleccionado la opción de secuencia silente en UV (ver sección \ref{uvsilent}), los residuos que absorben en el rango del UV (W,Y,F) tendrán una frecuencia igual a 0, independientemente del valor que se especificó.

\subsection{Definición del parámetro Beta}

Basándose en las evaluaciones realizadas (ver sección \ref{betaResults}), el valor estándar de Beta está definido como \texttt{1.0}. 
Es posible modificar este valor para ejecuciones individuales mediante el parámetro \texttt{beta} de la forma:

\indent \texttt{--beta [valor deseado de Beta]} 
\\Por ejemplo: \\
\indent \texttt{--beta 2.3} \hspace{0.5cm} \textit{Inicia la ejecución con un valor de beta = 2.3}



\subsection{Evaluación de la secuencia}\label{evaluacion}

Si no se especifica ningún parámetro adicional, el set de evaluaciones estará compuesto de aquellas herramientas detalladas las secciones \ref{propiedadesConformacionales} y \ref{evaluacionFuncional}.
El usuario puede quitar, para una ejecución individual, alguna de las herramientas del conjunto de evaluación. Para esto, cada herramienta tiene asociado un parámetro que permite quitarla del esquema de evaluación.
Los parámetros disponibles son: 
\texttt{--noblast, --notango, --noelm, --noiupred, --noanchor, --noprosite, 
--nolimbo, 
--notmhmm, 
--nopasta, 
--nowaltz, 
--noamyloidpattern}

\noindent Por ejemplo: \\
\indent \texttt{--noblast --nowaltz} \\%\hspace{0.5cm} 
\indent \indent \textit{La evaluación se realiza usando todas las herramientas excepto BLAST y Waltz}


% \subsubsection{Selección del set de herramientas a utilizar}
% seleccion de herramientas a evaluar


El usuario puede, además, agregar evaluaciones correspondientes a la carga neta y a la absorción UV de la secuencia, las cuales se detallan a continuación.

\subsubsection{Evaluación de carga neta}
% evaluacion de la carga neta

La carga neta deseada para la secuencia resultante puede ser definida mediante la opción \texttt{netcharge} de la forma: \\
\indent \texttt{--netcharge [carga neta deseada]} 
\\Por ejemplo: \\
\indent \texttt{--netcharge +3} \hspace{0.5cm} \textit{La secuencia resultante tendrá carga neta = +3}

El valor absoluto de la carga buscada debe ser menor o igual que la longitud de la secuencia, de otra forma no podría alcanzarse nunca la carga neta objetivo.
Esta condición se evalúa al inicio de la ejecución y se informa si no es posible continuar.


\subsubsection{Evaluación de silente en UV}\label{uvsilent}

Es posible restringir el conjunto de secuencias resultantes a aquellas que no absorban en el rango de UV, es decir, que no posean residuos Y, W o F.
La forma de indicar esto es mediante la opción \texttt{--uvsilent}(sin parámetros)



\subsection{Condición de finalización} \label{condicionFin}

Por definición, el método tiene un límite máximo de 5000 mutaciones aceptadas.
% iteraciones. Cada una de estas iteraciones implica   aceptadas. 
Este límite representa un valor muy alto donde se considera improductivo seguir intentando mutaciones porque probablemente no sea posible encontrar el resultado deseado.
%, incluso para secuencias linkers
%Este límite representa una cantidad muy alta de mutaciones a partir de las valor muy alto, incluso para secuencias linkers largas (50 residuos), y representa un valor donde se considera improductivo seguir intentando mutaciones porque puede no ser posible encontrar el resultado deseado. 
Es posible modificar este valor para una ejecución en particular mediante el parámetro \texttt{maxiterations}, de la forma:

\indent \texttt{--maxiterations [número máximo de iteraciones]} 
\\Por ejemplo: \\
\indent \texttt{--maxiterations 30} \hspace{0.5cm} \textit{Finaliza la ejecución si se alcanzan 30 mutaciones}


El objetivo de la modificación puede ser disminuir el número máximo de mutaciones, para obtener un resultado más rapidamente, aunque de menor calidad. También se puede incrementar el número máximo de mutaciones para linkers especialmente largos o con restricciones muy fuertes para el diseño.
En todos los casos donde se alcanza el límite de mutaciones se informa lo sucedido y la secuencia correspondiente a esa iteración. 
%al momento de finalización.

\subsection{Formato y detalles del resultado}\label{output}

Por definición el método sólo devuelve el resultado final de la ejecución informando, además, la condición de finalización.
Es posible extender o modificar el formato de salida mediante las siguientes opciones:
% Los parámetros que permiten modificar el formato de la salida son:
\vspace{0.2cm}\\
\indent \texttt{--logoutput [logpath]} \\
\indent \indent \textit{Guarda el historial de mutaciones en un archivo .log ubicado en logpath.} \\
\indent \texttt{--verbose} \\
\indent \indent \textit{Devuelve por pantalla una salida detallada de la ejecución.} \\
\indent \texttt{--stepped} \\
\indent \indent \textit{Ejecución paso a paso: espera el input del usuario en cada paso del método.} \\
