\chapter{Manual de uso}\label{manual}

\section{Parámetros de ejecución}\label{parametros}
\subsection{Secuencia inicial} \label{secuenciaInicial}
\subsubsection{Secuencia inicial random}\label{secuenciaInicialRandom}
Por default (si no se agrega ningún parámetro), la secuencia inicial se generará una secuencia random con la composición designada para la ejecución (ver \ref{composicion}).\\
Por default esta secuencia tendrá una longitud inicial = 10 AAs.

La longitud(en número de residuos) de esta secuencia inicial puede ser definida por el usuario mediante el parámetro \texttt{length} de la forma: \\
\indent \texttt{--length [seq-length]}
\\Por ejemplo: \\
\indent \texttt{--length 30} \hspace{0.5cm} Inicia la ejecución con una secuencia random de largo = 30 AAs


\subsubsection{Secuencia inicial definida}\label{secuenciaInicialDefinida}
Para definir una secuencia inicial especifica se debe usar el parámetro \texttt{seq} de la forma: \\
\indent \texttt{--seq [secuencia]} 
\\Por ejemplo: \\
\indent \texttt{--seq AAHHWWWLLLLHHGGG} \hspace{0.5cm} Inicia la ejecución con la secuencia \texttt{AAHHWWWLLLLHHGGG}

En caso de especificarse una secuencia inicial no es necesario definir longitud y, si se define, es ignorada.

\subsection{Composición de la secuencia} \label{composicion}

% Existen dos opciones para definir la composición usada durante la ejecución del método.
Si no se define ningún parametro al respecto, la composición \textit{default} que se usa es la composición estándar obtenida de SwissProt \cite{compositionAA}.  

Es posible también definir una composición a medida


\subsection{Condición de finalización} \label{condicionFin}

\subsection{Formato y detalles del resultado}\label{output}
Los parámetros que permiten modificar el formato de la salida son:
\vspace{0.2cm}\\
\texttt{--verbose} \\
\indent \indent Devuelve una salida detallada de la ejecución \\
\texttt{--minoutput} \\
\indent \indent Solo devuelve el resultado final y guarda el historial de mutaciones en un log \\
\texttt{--stepped} \\
\indent \indent Espera por input del usuario en cada paso del método. \\
